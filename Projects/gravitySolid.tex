\documentclass{ximera}

\title{Final Project Help: Gravity Between a Solid and a Point Mass}
\author{Zack Reed}

\begin{document}
\begin{abstract}
This page will help you as you work on your final project, exploring how to model the gravitational attraction between a solid and a point mass. We'll begin with the idealized point-mass formula, then extend it to piecewise and continuous approximations using calculus.
\end{abstract}
\maketitle

\section*{Introduction}

Welcome to the introduction to the ``gravity between a wire and a point'' mini project!  
As with all integration problems, we'll begin with a simple situation where multiplication suffices, and then build up toward an integral model that accounts for continuous variation.

\section*{Final Project - Gravity Between a Solid and a Point Mass}

\subsection*{Introduction}
Welcome to your final project topic focusing on gravitational attraction! As before with all instances of using integration in applications, we will:
- first introduce the basic model for the phenomenon (gravity),
- discuss why variation in some aspect of the phenomenon forces us to not use the basic model
- break up the objects involved into many small pieces on which we can use the basic model
- add up the instances of the basic model in the small pieces to get an approximation
- use integrals as the limiting value of the approximations to get an exact measure

\subsection*{Basic Model: Point Masses and Gravity}
In physics, the gravitational force between two \textbf{point masses} $m_1$ and $m_2$ is given by Newton's Law:
\begin{equation}
\vec{F} = G \frac{m_1 m_2}{||\vec{r}||^2} \hat{r}
\end{equation}
where $G$ is the universal gravitational constant, $\vec{r}$ is the vector between the two masses, and $\hat{r}$ is the unit vector pointing in the direction of $\vec{r}$.

In single-variable cases this can be simplified to:
\begin{equation}
F = G \frac{m_1 m_2}{r^2}
\end{equation}
where $r$ is the distance between the points.

\youtube{animation_placeholder}

You'll notice that the inverse square $\frac{1}{r^2}$ determines the gravity force in the case of two points separated by a single variable. We will explore the gravity between a point and a solid, which means moving away from a single-variable case and into a vector scenario.

\subsection*{Challenges to the Basic Model: Point Mass and a Solid}
We want to calculate the force of gravity between a solid floating in space and a nearby point mass. If the solid were itself a lumped point mass, the situation would be easy:
- Compute the distance from the point mass to the solid's center
- Use Newton's formula.

Unfortunately, many solids don't have exploitable symmetry, and so gravity acts between the solid and the point mass differently depending on where you are within the solid.
- Parts of the solid are closer, exerting stronger forces.
- Other parts are farther, exerting weaker forces.
- Variability in the solid shapes means a lack of symmetry.

\youtube{animation}

\textbf{Reflection Questions:}
- The color of each piece signifies the gravity approximation between the piece and the point mass, using the center of the cube for a measure of distance and the mass of the small piece. Given that the cube is uniformly dense, explain (using Newton's Law of Gravity) the color gradient shown in the plot.
- Why is the gravity smallest at the bottom of the cube?
- Why is the gravity greatest in the middle of the top of the cube, and then why does it decrease even along the top of the cube?

\subsection*{Newton's Law}
Let's recreate Newton's law at a single piece before doing so at scale.

\textbf{MATLAB code:}
\begin{verbatim}
% densities, distances, volumes arrays
% G = 10
% gravity for piece 1: G * m_1 * (densities(1) * volumes(1)) / (distances(1)^2)
\end{verbatim}

\textbf{Reflection Question:}
- Calculate the gravity attraction between the first piece and the point mass using the provided values.

\subsection*{Gravity in Pieces: Building the Global Model}
In this final project, you're going to approximate (and then exactly measure) the force of gravity between different shapes and a point mass. At first, to ease our way into the calculations, we're going to work with the \textbf{simplified setup}:
\begin{equation}
F = G \frac{m_1 m_2}{r^2}
\end{equation}
For each piece of the solid, we calculate the distance between the center of the piece and the point mass, treating the mass of that piece as $m_2$ in the formula.

\textbf{MATLAB code:}
\begin{verbatim}
plot_gravity_cube_distances()
\end{verbatim}

\subsection*{Approximating Gravity: Adding Up Small Gravities}
As with all integration problems, we solve the "multiplying doesn't work" problem by:
1. breaking up an object into small pieces
2. making the desired measurement on each small piece
3. adding up all of the small measurements to get a global measurement

\textbf{MATLAB code:}
\begin{verbatim}
gravities = G * m_1 * (densities .* volumes) ./ (distances.^2)
approx_gravity = sum(gravities)
\end{verbatim}

\textbf{Reflection Questions:}
- Write out symbolically what MATLAB has done using summation notation, using $\Delta M_k, \Delta V_k, \Delta F_k$ for the mass, volume, gravity of each piece.
- How do we make this approximation more precise? What do we do to the sum to get an exact measure of gravity and how does that connect to definite integration?

\subsection*{The Final Model: Gravity Between a Solid and a Point Mass}
We now have a model for the gravity between a solid and a point mass. We can use this model to calculate the gravity between different shapes and a point mass.

\textbf{MATLAB code:}
\begin{verbatim}
gravities = G * m_1 * (densities .* volumes) ./ (distances.^2)
approx_gravity = sum(gravities)
\end{verbatim}

\textbf{Reflection Questions:}
- If the sphere had a uniform density, how would the coloring in the gravity plot be different?
- What must be true about the mass for the gravity plot to have the given coloring?
- Write out symbolically how the gravity (under the simplified setup) is approximated. Use $k$ to denote the index of the pieces.

\subsection*{Calculating Exact Gravity (Simple Setup)}
Now you will receive your own personal solid, go through a similar process to calculate the gravity approximation in MATLAB, and then use a triple integral to get the exact measure of gravity between the solid and the point mass.

\textbf{MATLAB code:}
\begin{verbatim}
[densities, distances, volumes] = generate_student_solid(12345)
G = 10;
gravities = G * m_1 * (densities .* volumes) ./ (distances.^2)
approx_gravity = sum(gravities)
\end{verbatim}

\textbf{Reflection Questions:}
- Why can't we just use Newton's Law of Gravity to calculate the exact gravity between the solid and the point mass for these solids?
- Why is there error in our approximation method after we break up the solid into pieces?

\subsection*{Using Integrals to Get an Exact Measure}
Using triple integrals, we can write the exact measure of gravity as:
\begin{equation}
F = \int \int \int \frac{G m_1 \rho(x,y,z)}{r^2} dV
\end{equation}
where $\rho(x,y,z)$ is the density function and $r$ is the distance from the differential piece to the point mass.

\textbf{Reflection Questions:}
- What is the gravity (under the simple setup) between the solid and the point mass?
- Work out by hand the steps to use the FTC to compute the integral measure
- Verify in MATLAB (using three int() commands) the measure you found.
- Does the measure you found using integration closely match the approximation?
- Describe how you should think about integration as it applies to STEM after you leave calculus.

\subsection*{(Optional) The True Setup for Measuring Gravity}
The true way to measure gravity between two point masses is:
\begin{equation}
\vec{F} = G \frac{m_1 m_2}{||\vec{r}||^2} \hat{r}
\end{equation}
To truly approximate gravity, we need to add together all of the force vectors $\vec{F_k}$ to make a single net force vector $\vec{F}$:
\begin{equation}
\vec{F} \approx \sum_k \vec{F_k} = \sum_k G \frac{m_1 m_2}{||\vec{r_k}||^2} \hat{r_k}
\end{equation}
We can approximate this true gravity by summing all of the force vectors together and then taking the magnitude of the resulting vector.

\textbf{MATLAB code:}
\begin{verbatim}
[densities, gravity_vectors, volumes] = generate_solid_gravity('gravity_mode','true','n',100)
sum_vector = sum(gravity_vectors, 1)
norm(sum_vector)
\end{verbatim}

\textbf{Reflection Questions:}
- Why is the top center of the sphere colored blue? Those points are theoretically closer to the point than others, so why is the gravity calculation so low?
- Similarly, why is the top of the cone colored blue?
- Explain the shapes of the solids (Hint: Think about the coordinate system)

\end{document}