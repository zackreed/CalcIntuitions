\documentclass{ximera}

\title{Warm Up Project Help: Mass of a Solid}
\author{Zack Reed}

\begin{document}
\begin{abstract}
This page will help you as you work on a new project, exploring how we compute the mass of a solid with variable density using integration.
\end{abstract}
\maketitle



\section*{Introduction}

Welcome to your next final project warm up activity! This activity should feel familiar. In fact, we will cover essentially the same topic as the first mini project, finding the masses of objects based on varying densities. 

This topic is very important for various engineering applications, where masses and densities can impact various mechanical and physical interactions of objects that we build.

Here, we will focus on calculating the masses of 3D objects, rather than calculating the masses of wires. This is slightly more realistic than what we were able to accomplish in the case of vector curves, as all physical objects have volume measures and we don't need to make simplifying assumptions if we consider all dimensions of variation.

For this mini-project we'll explore how to model the \textbf{mass of a solid with variable density}. This extends the wire mass problem to three dimensions using \textbf{triple integrals}.

As with all integration problems we'll begin with a simple ``ideal'' situation that can be measured using multiplication, and then build up to an integration-based measurement approach that ``adds up tiny pieces'' of approximations across an object.

\section*{Basic Model: Mass and Density}
The mass of an object is essentially ``how much stuff is in an object''. In the case of \textbf{uniform density}, just like in the case of the wire, we say that no matter where within the object we are, if we ask ``how much stuff is there relative to how much space there is'' and divide the mass by the volume, we always get the same constant $\rho$:
\[
\frac{M}{V} = \rho
\]
This is always constant within the solid.

If we know the constant density $\rho$, then we can compute the mass by multiplication:
\[
M = \rho \cdot V
\]
where $\rho$ is the uniform density and $V$ is the volume.

As before, we will consider new scenarios of more complex solids where the density might vary within the volume. 

Consider the following example:
If you have ever seen a liquid sit long enough to separate, it might look something like this:

\begin{center}
  \includegraphics[width=0.4\textwidth]{mixed_fluid_densities.png}
\end{center}

As you move throughout the volume, you'll notice that depending on the location $\left(x, y, z\right)$, the density $\rho$ might vary between $0.91$ and $1$. We will simulate situations like this with continually varying densities $\rho(x, y, z)$ and use integration to calculate the masses of the solids.

Much like simulating the constant density of a wire, the following animation simulates the constant density of a cube.

The following animation will show the mass and volume calculations at various scales within sub-regions of a cube.

\youtube{EjMeVZVn1pU}



\begin{problem}

  Using the measures of mass and volume within the cube, verify that the density is indeed constant (up to a small error) and determine the density of the cube. Assume the density is an integer.

  \begin{enumerate}
    \item At the first scale, the density is approximately: $\frac{\answer[tolerance=.01]{.3750}}{\answer[tolerance=.01]{.1250}}=\answer{3}$
    \item At the second scale, the density is approximately: $\frac{\answer[tolerance=.01]{.2303}}{\answer[tolerance=.01]{.0768}}=\answer[tolerance=.1]{2.9987}$
    \item At the third scale, the density is approximately: $\frac{\answer[tolerance=.01]{.1286}}{\answer[tolerance=.01]{.0429}}=\answer[tolerance=.1]{2.9977}$
    \item At the fourth scale (.2750), the density is approximately: $\frac{\answer[tolerance=.01]{.0624}}{\answer[tolerance=.01]{.0208}}=\answer[tolerance=.1]{3}$
  \end{enumerate}

  Note that small differences between the density calculations are due to rounding errors in MATLAB. You should conclude that at all possible scales and locations the density is constant and equal to $\answer{3}$ kilograms per cubic meter.

\end{problem}

\subsection*{Challenges to the Basic Model: Variable Density in a Solid}

We want to calculate the mass of a solid floating in space with varying density.
If the cube had uniform density, the situation would be easy:
\begin{itemize}
  \item Compute the volume,
  \item Multiply by the constant density.
\end{itemize}
Unfortunately, many solids don't have uniform density, and so mass varies depending on where you are within the object.
\begin{itemize}
  \item Some regions are denser, contributing more mass per unit volume.
  \item Other regions are less dense, contributing less mass per unit volume.
\end{itemize}

\begin{problem}
You can see this clearly in the following animations and paired 3D plots showing the distribution of density within the cubes:

\begin{enumerate}
  \item First view the animation of sample points within the cube:
  \begin{center}
    \youtube{Lb1o1YIYI2o}
  \end{center}

  Now examine the 3D plot of the density distribution within the cube:
  \begin{center}
    \includegraphics{radial_density_cube.png}
  \end{center}

  Based on the plot and animation, the density of this cube is highest \wordChoice{\choice[correct]{at the center of the cube}\choice{at the edges of the cube}\choice{at the corners of the cube}\choice{uniform throughout the cube}} and lowest \wordChoice{\choice{at the center of the cube}\choice[correct]{at the edges of the cube}\choice{at the corners of the cube}\choice{uniform throughout the cube}}.

  \item Next view the animation of sample points within the cube:
  \begin{center}
    \youtube{vxkA3BIChDs}
  \end{center}

  Now examine the 3D plot of the density distribution within the cube:
  \begin{center}
    \includegraphics{left_to_right_density_cube.png}
  \end{center}

  Based on the plot and animation, the density of this cube is highest (with respect to your perspective) \wordChoice{\choice[correct]{on the left side of the cube}\choice{on the right side of the cube}\choice{at the center of the cube}\choice{uniform throughout the cube}} and lowest \wordChoice{\choice{on the left side of the cube}\choice[correct]{on the right side of the cube}\choice{at the center of the cube}\choice{uniform throughout the cube}}.

  \item Finally view the animation of sample points within the cube:
  \begin{center}
    \youtube{83FxD3cCTKI}
  \end{center}

  Now examine the 3D plot of the density distribution within the cube:
  \begin{center}
    \includegraphics{bottom_to_top_density_cube.png}
  \end{center}

  Based on the plot and animation, the density of this cube is highest (from your perspective) \wordChoice{\choice{at the bottom of the cube}\choice[correct]{at the top of the cube}\choice{at the center of the cube}\choice{uniform throughout the cube}} and lowest \wordChoice{\choice[correct]{at the bottom of the cube}\choice{at the top of the cube}\choice{at the center of the cube}\choice{uniform throughout the cube}}.
\end{enumerate}


\end{problem}

\section*{Density, Mass, and Volume}
The density plot only shows the centers of the ``pieces'', not the whole piece of the solid that has the assigned density.

While it is hard to render 3D solids broken up into small cubes, when you see the plots with points you should instead imagine that each point is the center of a small cube that makes up the solid. This would look like the following:

\begin{center}
  \includegraphics{cube_in_pieces.png}
\end{center}

Because of this, even though we will work with plots of points moving forward, remember that the points represent the \textbf{centers} of the boxes, not the actual boxes themselves.

At a more appropriate scale, we can generate very refined variations in the density across the solid, visualized by plots such as this:

\begin{center}
  \includegraphics{cube_small_scale.png}
\end{center}

\section*{Approximating Mass: Adding up Pieces}

You are tasked with approxiating the mass of a solid with a given variable density. 

\begin{example}

The first example is a sphere whose density plot looks like the following:

\begin{center}
  \includegraphics{sphere_approx.png}
\end{center}

The mass of each piece is given by the basic model $\text{mass} = \text{density} \times \text{volume}$, in our notation $\Delta M = \rho \cdot \Delta V$. 
If we again use $k$ to mark the rows of the arrays, then we would say $\Delta M_k = \rho(x_k, y_k, z_k) \cdot \Delta V_k$.

The total mass, then, is the sum of all of the small masses $M \approx \sum_k \Delta M_k$. Since we can write each small mass as a density-volume product, we get the approximation:
\[
M \approx \sum_k \rho(x_k, y_k, z_k) \cdot \Delta V_k
\]

\end{example}

\begin{problem}

You will also potentially approximate the mass of other solids with variable densities, such as the following cylinder a radius $0.5$ and a height $1$ (volume $\pi r^2 h = \pi \cdot 0.5^2 \approx 0.7854$)

\begin{center}
  \includegraphics{cylinder_approx.png}
\end{center}

You'll notice that the density values go from $0$ to $10$, but the mass is less than the volume even though the density has such high values.
This is because \begin{multipleChoice}
  \choice{the high density pieces make up a small percentage of the solid}
  \choice[correct]{the low density pieces make up a large percentage of the solid}
\end{multipleChoice}

\end{problem}


\section*{Your Turn: Approximating and Calculating Exact Mass}
As with the past warm up projects, you will now generate and analyze your own object!

First you will approximate the mass of a personal solid and then use integrals to calculate the exact mass.

The methods you learned in the textbook to compute triple integrals is how you can leverage the Fundamental Theorem of Calculus to find exact formulas that quickly give you the measure calculated by a volume sum. 
That is,
\[
M \approx \sum_k \Delta M_k = \sum_k \rho \Delta V_k
\]
and
\[
M = \iiint dM = \iiint \rho \, dV = \int_{z_1}^{z_2} \int_{y_1}^{y_2} \int_{x_1}^{x_2} \rho \, dx \, dy \, dz
\]

You will complete this assignment by writing the triple integral that computes the mass of the solid, use the FTC to generate the exact formula that gives the measure of mass (by hand) and then use the \texttt{int()} command in MATLAB to confirm the triple integral calculation.

Good luck! And be sure to ask lots of questions as you work on this project.
\end{document}