\documentclass{ximera}

\title{Warm Up Project Help: Mass of a Solid}
\author{Zack Reed}

\begin{document}
\begin{abstract}
This page will help you as you work on your warm up project, exploring how we compute the mass of a solid with variable density using integration.
\end{abstract}
\maketitle



\section*{Introduction}

Welcome to your first project walkthrough!
This walkthrough will give you a good glimpse of some of the main ways we want to think about differentiating and integrating along curves when modeling phenomena using mathematics! 


\section*{Instructor Solution: Mass of a Solid Cone}

\subsection*{Introduction}
****This script is ONLY FOR INSTRUCTORS and presents a fully worked out/explained solution to the "Mass of a Solid Cone" mini project choice. Instructor text will be given in this text format.****

Welcome to your next final project warm up activity! This activity should feel familiar. In fact, we will cover essentially the same topic as the first mini project, finding the masses of objects based on varying densities. 

This topic is very important for various engineering applications, where masses and densities can impact various mechanical and physical interactions of objects that we build.

Here, we will focus on calculating the masses of 3D objects, rather than calculating the masses of wires. This is slightly more realistic than what we were able to accomplish in the case of vector curves, as all physical objects have volume measures and we don't need to make simplifying assumptions if we consider all dimensions of variation.

For this mini-project we'll explore how to model the \textbf{mass of a solid with variable density}. This extends the wire mass problem to three dimensions using \textbf{triple integrals}.

As with all integration problems we'll begin with a simple "ideal" situation that can be measured using multiplication, and then build up to an integration-based measurement approach that "adds up tiny pieces" of approximations across an object.

\section*{Basic Model: Mass and Density}
The mass of an object is essentially "how much stuff is in an object". In the case of \textbf{uniform density}, just like in the case of the wire, we say that no matter where within the object we are, if we ask "how much stuff is there relative to how much space there is" and divide the mass by the volume, we always get the same constant $\rho$:
\[
\frac{M}{V} = \rho
\]
This is always constant within the solid.

If we know the constant density $\rho$, then we can compute the mass by multiplication:
\[
M = \rho \cdot V
\]
where $\rho$ is the uniform density and $V$ is the volume.

As before, we will consider new scenarios of more complex solids where the density might vary within the volume. 

Consider the following example:
If you have ever seen a liquid sit long enough to separate, it might look something like this:

% \begin{center}
% \includegraphics[width=0.4\textwidth]{mixed_fluids.png}
% \end{center}

As you move throughout the volume, you'll notice that depending on the location $\left(x, y, z\right)$, the density $\rho$ might vary between $0.91$ and $1$. We will simulate situations like this with continually varying densities $\rho(x, y, z)$ and use integration to calculate the masses of the solids.

Much like simulating the constant density of a wire, the following animation simulates the constant density of a cube.
The following animation will show the mass and volume calculations at various scales within sub-regions of a cube.

\youtube{Constant Density Cube Animation}

Using the measures of mass and volume within the cube, verify that the density is indeed constant (up to a small error) and determine the density of the cube. Assume the density is an integer.

\subsection*{Challenges to the Basic Model: Variable Density in a Solid}
We want to calculate the mass of a solid floating in space with varying density.
If the cone had uniform density, the situation would be easy:
\begin{itemize}
  \item Compute the volume,
  \item Multiply by the constant density.
\end{itemize}
Unfortunately, many solids don't have uniform density, and so mass varies depending on where you are within the object.
\begin{itemize}
  \item Some regions are denser, contributing more mass per unit volume.
  \item Other regions are less dense, contributing less mass per unit volume.
\end{itemize}
You can see this clearly in the following animation, where again we will walk through a solid but the density will change.

\textbf{MATLAB code:}
\begin{verbatim}
[densities, locations] = animate_volume_variable_density();
\end{verbatim}

The \texttt{animate\_volume\_variable\_density} function also gave us a 2000-long list of densities called \texttt{densities} and a 2000x3 matrix (three 2000-long lists of coordinates) called \texttt{locations}.

\textbf{MATLAB code:}
\begin{verbatim}
densities
locations
\end{verbatim}

This is a more detailed list of density values found at the locations within the cube.
The rows of \texttt{densities} are linked with the rows of \texttt{locations}. That is, the 5th row (entry) in \texttt{densities} is $0.6432$ and the 5th row of \texttt{locations} is the point $[-0.1222, 0.0581, -0.2779]$. This means that the 5th point within the cube has a density of $0.6432$ and is located at $[-0.1222, 0.0581, -0.2779]$.

\youtube{Variable Density Cube Animation}

The animation shows the mass of small regions within the cube, colored by density.

Use the \texttt{plot\_3D\_densities} function to plot the locations within the cube, color-coded by density.

\textbf{MATLAB code:}
\begin{verbatim}
plot_3D_densities(densities, locations)
\end{verbatim}

You can also click and drag and interact with the cube for the following reflection questions.

\section*{Reflection Questions}
\begin{itemize}
  \item Where is the solid most dense (within the cube)?
  \item Where is the solid least dense (within the cube)?
  \item Describe the qualities of the density function $\rho(x, y, z)$? What does the function use to determine higher or lower values?
\end{itemize}

\subsection*{More Variable Densities}
Let's observe a different density function.

\textbf{MATLAB code:}
\begin{verbatim}
[densities, locations] = animate_volume_variable_density(2)
\end{verbatim}

\youtube{Second Variable Density Animation}

Again use the \texttt{plot\_3D\_densities} function to plot the locations within the cube, color-coded by density.
\begin{verbatim}
plot_3D_densities(densities, locations)
\end{verbatim}

\section*{Reflection Questions}
\begin{itemize}
  \item Where is the solid most dense (within the cube)?
  \item Where is the solid least dense (within the cube)?
  \item Describe the qualities of the density function $\rho(x, y, z)$? What does the function use to determine higher or lower values?
\end{itemize}

\subsection*{One More Density}
Let's observe a final density function.

\textbf{MATLAB code:}
\begin{verbatim}
[densities, locations] = animate_volume_variable_density(3)
\end{verbatim}

\youtube{Third Variable Density Animation}

Again use the \texttt{plot\_3D\_densities} function to plot the locations within the cube, color-coded by density.
\begin{verbatim}
plot_3D_densities(densities, locations)
\end{verbatim}

\section*{Reflection Questions}
\begin{itemize}
  \item Where is the solid most dense (within the cube)?
  \item Where is the solid least dense (within the cube)?
  \item Describe the qualities of the density function $\rho(x, y, z)$? What does the function use to determine higher or lower values?
\end{itemize}

\section*{Density, Mass, and Volume}
The density plot only shows the centers of the "pieces", not the whole piece of the solid that has the assigned density.

The following code uses the function \texttt{generate\_variable\_densities} to plot a rectangular grid of densities within a solid.
\begin{verbatim}
generate_variable_densities(5, 'cube');
\end{verbatim}

\subsection*{Points vs Cubes, Breaking up a Solid}
Importantly, you'll notice that rather than a dense cloud of random points, the points on the grid are organized neatly into rows and columns. The locations given here are the centers of small cubes making the whole shape.
\begin{verbatim}
plot_solid_as_cubes(5, 'cube')
\end{verbatim}

We will mostly work with plots such as given in \texttt{generate\_variable\_densities} because it is very easy for the computer to generate many colored points (centers of the cubes) than it is to generate the same number of colored squares.
\begin{verbatim}
generate_variable_densities(30, 'cube');
\end{verbatim}

Because of this, even though we will work with plots of points moving forward, remember that the points represent the \textbf{centers} of the boxes, not the actual boxes themselves.

To account for this in your calculations, the \texttt{generate\_variable\_densities} function gives three outputs:
\begin{itemize}
  \item The densities of the center points
  \item The locations of the center points
  \item The volumes of the small pieces (cubes for now) of the shapes.
\end{itemize}
The rows keep the information connected. So each row of \texttt{densities} is the density for the 3D point at the same row of \texttt{locations}, which has the volume of the same row of \texttt{volumes}.

\textbf{MATLAB code:}
\begin{verbatim}
[densities, locations, volumes] = generate_variable_densities(20, 'sphere')
\end{verbatim}

\section*{Reflection Questions}
\begin{enumerate}
  \item Why do we need to include the volume information for each point? Why can't we simply add up all of the density values to get the mass?
  \item What should you imagine in your head around each point in the \texttt{generate\_variable\_densities} plot? What does the point in the plot represent?
  \item Using the basic model for mass, how do we find the mass of each small piece of the shape? How do we find the total approximate mass of the shape?
\end{enumerate}

\section*{Approximating Mass: Adding up Pieces}
You should have an $N \times 1$ list of density values, an $N \times 3$ list of locations (each row is a point in 3D space), and an $N \times 1$ list of volumes (the volumes of the cubes around the locations). 

As before with the mass of a wire, we can use MATLAB to simply add up all of the masses of the small pieces of the solid together to get an approximation of the mass. 

The basic operations adding arrays ($N \times 1$ lists of numbers) together are addition ($+$), subtraction ($-$), multiplication ($\cdot$), division ($/$), and exponentiation ($^2$). For multiplication, division and exponentiation, the period ($.$) is important.

Using basic operations and the \texttt{sum} MATLAB command, get an approximation of the mass of the object you generated in the previous cell. It should closely match the mass approximation given in the plot.

The shape (a sphere) has volume $\frac{4}{3}\pi \cdot 0.5^3 \approx 0.5236$ because it has a radius of $0.5$. It makes sense that the mass is less than $0.5236$ because the max density is $1$, only at the center of the sphere. So the decimal densities mean the mass of the sphere is less than the volume.
We will confirm this with integrals at the end of the activity.

\section*{Reflection Questions}
If we were to symbolically write out the operations the computer carried out, it would look just like a Riemann sum!

Each volume measure is the volume of a 3D cube with dimensions $\Delta V_k = \Delta x_k \times \Delta y_k \times \Delta z_k$, where $k$ is the index. In terms of our arrays, $k$ would represent the rows. So in the \texttt{volumes} array, the number found at the 10th row would be $\Delta V_{10}$.

Even though it's not used in computation, the \texttt{locations} matrix is the list of 3D points $(x_k, y_k, z_k)$ that are the centers of the cubes that make up the solid. The second row of \texttt{locations} from the sphere example is $[0.3750, 0.4750, 0.0250]$. This is the point $(0.3750, 0.4750, 0.0250)$ which we would denote $(x_2, y_2, z_2)$. 

The \texttt{densities} are the values of a scalar function $\rho(x, y, z)$. The number is the constant density that we're using as the density of the cube that makes up that piece of the solid. The 2nd value in \texttt{densities} is $0.5082$, meaning that for the given density function, $\rho(0.3750, 0.4750, 0.0250) = 0.5082$ kilograms per cubic meter.

The mass of each cube is given by the basic model $\text{mass} = \text{density} \times \text{volume}$, in our notation $\Delta M = \rho \cdot \Delta V$. 
If we again use $k$ to mark the rows of the arrays, then we would say $\Delta M_k = \rho(x_k, y_k, z_k) \cdot \Delta V_k$.

The total mass, then, is the sum of all of the small masses $M \approx \sum_k \Delta M_k$. Since we can write each small mass as a density-volume product, we get the approximation:
\[
M \approx \sum_k \rho(x_k, y_k, z_k) \cdot \Delta V_k
\]

Questions:
\begin{itemize}
  \item Take three rows from the \texttt{densities} and \texttt{volumes} arrays and calculate the approximate masses of those three small pieces of the solid. Write out in the above notation what each of the calculations represent and what formulas we use to calculate them (be sure to notate the rows).
  \item How do we make the approximation more accurate?
\end{itemize}

\subsection*{Approximating Mass: More Shapes}
\textbf{MATLAB code:}
\begin{verbatim}
[densities, ~, volumes] = generate_variable_densities(10, 'cylinder', 4)
\end{verbatim}

The shape is a cylinder with a radius $0.5$ and a height $1$, so its volume is $\pi r^2 h = \pi \cdot 0.5^2 \approx 0.7854$.
You'll notice that the density values go from $0$ to $10$. Describe why the mass is less than the volume in this case (both fractions) even though the density has such high values.
How does the mass of the pieces with high density relate to the mass of the pieces with low density, in terms of the relative percentage of the solid made up by high density vs low density pieces.

Verify the mass approximation of the cylinder using the basic model for mass on all of the small pieces and the \texttt{densities} and \texttt{volumes} arrays.

\subsection*{Reflection Questions}
\begin{itemize}
  \item Where is the solid most dense (within the cylinder)?
  \item Where is the solid least dense (within the cylinder)?
  \item Describe the qualities of the density function $\rho(x, y, z)$? What does the function use to determine higher or lower values?
\end{itemize}

\section*{Your Turn: Approximating and Calculating Exact Mass}
As with the past warm up projects, you will now generate and analyze your own object!

Use the \texttt{generate\_student\_densities} function with your student ID. 
You will see a plot of your personal shape with a variable density function.
You will also see printed out (above the plot) the $x$, $y$, and $z$ boundaries of your shape and the formula for the density function.

\textbf{MATLAB code:}
\begin{verbatim}
[densities_1, ~, volumes_1] = generate_student_densities(1);
\end{verbatim}

You will first use the \texttt{densities} and \texttt{volumes} information to confirm the mass approximation given in the plot. Then, you will use triple integrals to get an exact measure of the mass.

First, as before, use the densities and volumes information to approximate the mass by "adding up" the pieces of mass from your personally generated approximation of the shape.
Remember that even though the plot gives the centers of the pieces of the shape, in reality the shape is broken up into many small cubes with a uniform density given by the density function at the center point.

\section*{Reflection Questions}
After you've confirmed that the mass approximation from the densities and volumes arrays matches the approximation in the plot, write out in general summation notation how that mass approximation is calculated (using $\rho$ to symbolize the density) and then below the notation write out your personal formula for the density, your personal solid shape and its parameters, and the center point of your solid.

\section*{Using Integrals to Get an Exact Measure}
The methods you learned in the textbook to compute triple integrals is how you can leverage the Fundamental Theorem of Calculus to find exact formulas that quickly give you the measure calculated by a volume sum. 
That is,
\[
M \approx \sum_k \Delta M_k = \sum_k \rho \Delta V_k
\]
and
\[
M = \iiint dM = \iiint \rho \, dV = \int_{z_1}^{z_2} \int_{y_1}^{y_2} \int_{x_1}^{x_2} \rho \, dx \, dy \, dz
\]
You will complete this assignment by writing the triple integral that computes the mass of the solid, use the FTC to generate the exact formula that gives the measure of mass (by hand) and then use the \texttt{int()} command in MATLAB to confirm the triple integral calculation.

\section*{Reflection Questions}
Using your personal density function and the parametric information about your surface:
\begin{itemize}
  \item Write the triple integral that computes the mass of your personal solid.
  \item Use the Fundamental Theorem of Calculus to find a function that produces the value of the mass.
  \item Calculate the mass of the solid.
  \item The mass you calculate using the FTC should be very close to the approximation found from the MATLAB sum.
  \item Then, using MATLAB:
    \begin{itemize}
      \item Define your density function.
      \item Use the \texttt{int} command (three times) to calculate the triple integral that calculates mass.
      \item Check that the mass from the triple integral matches the other two masses you computed.
    \end{itemize}
\end{itemize}

Here is an example of a triple integral computed in MATLAB. These commands find the triple integral of the function $x+y+z$ in the unit cube.
First, the commands integrate each variable one at a time, then the final command calculates the triple integral all at once.

\textbf{MATLAB code:}
\begin{verbatim}
syms x y z
f = x + y + z;
int(f, x)
int(f, y)
int(f, z)
int(int(f, x), y, z)
\end{verbatim}

Good luck! And be sure to ask lots of questions as you work on this project.
\end{document}