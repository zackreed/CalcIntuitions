\documentclass{ximera}

\title{Final Project Help: Gravity Between a Wire and a Point Mass}
\author{Zack Reed}

\begin{document}
\begin{abstract}
This page will help you as you work on your final project, exploring how to model the gravitational attraction between a wire and a point mass. We'll begin with the idealized point-mass formula, then extend it to piecewise and continuous approximations using calculus.
\end{abstract}
\maketitle

\section*{Final Project - Gravity Between a Wire and a Point Mass}

\subsection*{Introduction}
Welcome to your final project topic focusing on gravitational attraction! As before with all instances of using integration in applications, we will:
- first introduce the basic model for the phenomenon (gravity),
- discuss why variation in some aspect of the phenomenon forces us to not use the basic model
- break up the objects involved into many small pieces on which we can use the basic model
- add up the instances of the basic model in the small pieces to get an approximation
- use integrals as the limiting value of the approximations to get an exact measure

\subsection*{Basic Model: Point Masses and Gravity}
In physics, the gravitational force between two \textbf{point masses} $m_1$ and $m_2$ is given by Newton's Law of Gravitation:
\begin{equation}
\vec{F} = G \frac{m_1 m_2}{||\vec{r}||^2} \hat{r}
\end{equation}
where $G$ is the universal gravitational constant, $\vec{r}$ is the vector between the two masses, and $\hat{r}$ is the unit vector pointing in the direction of $\vec{r}$.

In single-variable cases this can be simplified to:
\begin{equation}
F = G \frac{m_1 m_2}{r^2}
\end{equation}
where $r$ is the distance between the points.

The following is an animation illustrating variation in the gravitational force between two point masses as the distance between them changes.

\youtube{in3Kb2c1Xtw}

You'll notice that the inverse square $\frac{1}{r^2}$ determines the gravity force in the case of two points separated by a single variable. We will explore the gravity between a point and a wire, which means moving away from a single-variable case and into a vector scenario.

\subsection*{Challenges to the Basic Model: Point Mass and a Wire}
We want to calculate the force of gravity between a wire floating in space and a nearby point mass. 

If the wire were itself a lumped point mass and was uniformly symmetric, the situation would be easy:
\begin{enumerate}
    \item Compute the distance from the point mass to the wire's center
    \item Use Newton's formula
\end{enumerate}

Unfortunately, many wires don't have exploitable symmetry, and so gravity acts between the wire and the point mass differently depending on where you are along the wire.
\begin{enumerate}
    \item Parts of the wire are closer, exerting stronger forces.
    \item Other parts are farther, exerting weaker forces.
    \item Forces don't all point in the same direction.
\end{enumerate}

You can see this clearly in the following plot, which breaks up the wire into many small segments like in the Module 3 mass example.

\begin{center}
\includegraphics{gravity_wire_pieces_1.png}
\end{center}

\subsection*{Gravity in Pieces: Building the Global Model}
In this final project, you're going to approximate (and then exactly measure) the force of gravity between different wires and a point mass. To ease our way into the calculations, we're going to work with a \textbf{simplified setup} for gravity:
\begin{equation}
F = G \frac{m_1 m_2}{D^2}
\end{equation}
For each segment of the wire, we calculate the distance between the midpoint of the wire and the point mass, treating the mass of that segment as $m_2$ in the formula.

\subsection*{Step 1: Approximate with Pieces}
Suppose the wire has total mass $M$ and length $L$. If we cut it into $N$ pieces, each piece has approximate mass $\Delta M = \frac{M}{N}$. We treat each piece of wire as if it were a point mass and use Newton's formula to approximate the gravitational attraction between the wire piece and the point mass. If $D$ represents the distance between the segment piece and the point mass, we use Newton's law to measure the small addition of force for that section:

\begin{equation}
\Delta F_i \approx \frac{G m \Delta M_i}{D_i^2}
\end{equation}
Adding them up gives an approximation:
\begin{equation}
F \approx \sum_{i=1}^N \Delta F_i = \sum_{i=1}^N \frac{G m \Delta M_i}{D_i^2}
\end{equation}
As $N$ grows, this gets closer to the true gravitational force.


You will be asked to approximate the gravitational force between a wire and a point mass using MATLAB code.

\begin{problem}
    The following curve is a simple helix broken into 10 pieces with a linear density (increases with time):

    \begin{center}
    \includegraphics{gravity_wire_pieces_2.png}
    \end{center}

    Notice that the colors indicate highest gravity on the segments closest to the point mass. This makes sense because \wordChoice{\choice[correct]{gravity decreases as the square distance increases}\choice{gravity increases as the square distance increases}}.

    Even though density increases linearly with time, which of the following statements is true about the relationship between mass and distance with respect to gravity on each segment?
    \begin{multipleChoice}
        \choice{Segments with higher mass always exert more gravity.}
        \choice{Segments that are closer always exert more gravity.}
        \choice[correct]{Gravity increases linearly with mass and shrinks quadratically with distance.}
        \choice{Gravity increases quadratically with mass and shrinks linearly with distance.}
        \choice{Mass and distance have no effect on gravity.}
    \end{multipleChoice}
\end{problem}

You will use MATLAB code to recreate the approximations yourselves, using Newton's Law of gravity on each segment. 

As you do so, the course functions will break up the wire and provide you the information as needed, you just need to apply Newton's law

$$F = G * m * \Delta M / D^2$$

to each segment and add them up.

Now let's see a different example that alters the features of the wire.

\begin{problem}
    The following wire is again the simple helix, but this time the density cubicly increases with time:

    \begin{center}
    \includegraphics{gravity_wire_pieces_flipped.png}
    \end{center}

    Notice that the gravity is now highest at the farthest segments from the point mass. 

    Select the following statements that best explain this phenomenon:
    \begin{selectAll}
        \choice[correct]{The cubic increase in density makes the mass outweigh the quadratic shrinking of gravity with distance.}
        \choice[correct]{The segments closer to the point mass have less mass, so they exert less gravity.}
        \choice[correct]{Gravity increases with distance in this case.}
        \choice{The point mass is actually repelled by the wire segments.}
    \end{selectAll}
   
\end{problem}


\subsection*{Refining Approximation: Increasing N}
Hopefully unsurprising by now, we suspect that increasing N will give us better approximations. This should be evident physically, the segments better approximate a point mass, meaning we're getting closer to treating each particle along the wire as if it were a point mass (and hence adding up many accurate instances of Newton's law).

You will be asked to recreate the approximations of gravity at higher N values in MATLAB. You should see results such as the following approximations of gravity with the cubic density helix wire:

\begin{enumerate}
    \item First $N=5$:
    \begin{center}
    \includegraphics{gravity_wire_pieces_flipped_n_1.png}
    \end{center}

    \item Then $N=50$:
    \begin{center}
    \includegraphics{gravity_wire_flipped_n_2.png}
    \end{center}

    \item Then, $N=100$:
    \begin{center}
    \includegraphics{gravity_wire_flipped_n_3.png}
    \end{center}

    \item Finally, $N=200$:
    \begin{center}
    \includegraphics{gravity_wire_flipped_n_4.png}
    \end{center}
\end{enumerate}

\subsection*{Step 5: Exact Model Using Calculus}
The main difference between finite sum approximations and the exact sum from integration is the limit. Definite integrals and the Fundamental Theorem of Calculus give us tools for making exact calculations. To use calculus efficiently, we need to rewrite the integral until we get a form that makes calculations on a single variable that we can run continuously through an interval.

So far, our sums have taken the form:
\begin{equation}
F = \sum_i \Delta F_i = \sum_i \frac{G m \Delta M_i}{D_i^2}
\end{equation}

Translating this to integrals, we have:
\begin{equation}
F = \int dF
\end{equation}

tells us that we get the exact force from adding up all of the little differential forces.

From Newton's law we get $dF = \frac{G m dM}{D_M^2}$ (each little force is Newton's law applied to the little mass  at a distance ), we can compute the total sum by the integral

\begin{equation}
F = G m \int \frac{dM}{D_M^2}
\end{equation}

The problem is that we can't use the fundamental theorem of calculus yet because we don't have a good interval of masses to integrate over, nor a good way to represent the distances $D_M$. What we do have is the formula for both the curve $r$ and the density $\rho$ in terms of time, and a way to represent the arc length of a small segment $ds$ in terms of the curve's velocity.

So, your job is to finish representing the integral for your curve and density and point in a way that you can use the FTC, and then use MATLAB's symbolic package to compute the integral!

\end{document}
