\documentclass{ximera}

\title{Final Project Help: Gravity Between a Wire and a Point Mass}
\author{Zack Reed}

\begin{document}
\begin{abstract}
This page will help you as you work on your final project, exploring how to model the gravitational attraction between a wire and a point mass. We'll begin with the idealized point-mass formula, then extend it to piecewise and continuous approximations using calculus.
\end{abstract}
\maketitle

\section*{Introduction}

Welcome to the introduction to the ``gravity between a wire and a point'' mini project!  
As with all integration problems, we'll begin with a simple situation where multiplication suffices, and then build up toward an integral model that accounts for continuous variation.

\section*{Final Project - Gravity Between a Wire and a Point Mass}

\subsection*{Introduction}
Welcome to your final project topic focusing on gravitational attraction! As before with all instances of using integration in applications, we will:
- first introduce the basic model for the phenomenon (gravity),
- discuss why variation in some aspect of the phenomenon forces us to not use the basic model
- break up the objects involved into many small pieces on which we can use the basic model
- add up the instances of the basic model in the small pieces to get an approximation
- use integrals as the limiting value of the approximations to get an exact measure

\subsection*{Basic Model: Point Masses and Gravity}
In physics, the gravitational force between two \textbf{point masses} $m_1$ and $m_2$ is given by Newton's Law of Gravitation:
\begin{equation}
\vec{F} = G \frac{m_1 m_2}{||\vec{r}||^2} \hat{r}
\end{equation}
where $G$ is the universal gravitational constant, $\vec{r}$ is the vector between the two masses, and $\hat{r}$ is the unit vector pointing in the direction of $\vec{r}$.

In single-variable cases this can be simplified to:
\begin{equation}
F = G \frac{m_1 m_2}{r^2}
\end{equation}
where $r$ is the distance between the points.

\youtube{animation_placeholder}

You'll notice that the inverse square $\frac{1}{r^2}$ determines the gravity force in the case of two points separated by a single variable. We will explore the gravity between a point and a wire, which means moving away from a single-variable case and into a vector scenario.

\subsection*{Challenges to the Basic Model: Point Mass and a Wire}
We want to calculate the force of gravity between a wire floating in space and a nearby point mass. If the wire were itself a lumped point mass and was uniformly symmetric, the situation would be easy:
- Compute the distance from the point mass to the wire’s center
- Use Newton’s formula

Unfortunately, many wires don't have exploitable symmetry, and so gravity acts between the wire and the point mass differently depending on where you are along the wire.
- Parts of the wire are closer, exerting stronger forces.
- Other parts are farther, exerting weaker forces.
- Forces don’t all point in the same direction.

\youtube{animation_placeholder}

\textbf{Reflection Question:}
- The color of each piece signifies the gravity approximation between the wire segment and the point mass, using the midpoint of the segment for a measure of distance and the mass of the small segment. Given that the wire is uniformly dense, explain (using Newton's Law of Gravity) the color gradient shown in the plot.

\subsection*{Gravity in Pieces: Building the Global Model}
In this final project, you're going to approximate (and then exactly measure) the force of gravity between different wires and a point mass. To ease our way into the calculations, we're going to work with a \textbf{simplified setup} for gravity:
\begin{equation}
F = G \frac{m_1 m_2}{D^2}
\end{equation}
For each segment of the wire, we calculate the distance between the midpoint of the wire and the point mass, treating the mass of that segment as $m_2$ in the formula.

\subsection*{Step 1: Approximate with Pieces}
Suppose the wire has total mass $M$ and length $L$. If we cut it into $N$ pieces, each piece has approximate mass $\Delta M = \frac{M}{N}$. We treat each piece of wire as if it were a point mass and use Newton's formula to approximate the gravitational attraction between the wire piece and the point mass. If $D$ represents the distance between the segment piece and the point mass, we use Newton's law to measure the small addition of force for that section:
\begin{equation}
\Delta F_i \approx \frac{G m \Delta M_i}{D_i^2}
\end{equation}
Adding them up gives an approximation:
\begin{equation}
F \approx \sum_{i=1}^N \Delta F_i = \sum_{i=1}^N \frac{G m \Delta M_i}{D_i^2}
\end{equation}
As $N$ grows, this gets closer to the true gravitational force.

\textbf{MATLAB code:}
\begin{verbatim}
verify_setup('Name')
syms t
% r = ...
% rho = ...
% P = ...
% m = ...
% N = ...
plot_wire_gravity_pieces(r, N, rho, P, m)
\end{verbatim}

\textbf{Reflection Questions:}
- Why is there error for any gravity calculation for any specific N? What assumptions are we making when we break the wire into pieces in this way that technically break the assumptions of Newton's Law?
- Why does increasing N reduce the error of the approximation?

\subsection*{Step 2: Recreate the Approximation}
The \texttt{break\_into\_pieces} function is helpful here. It will break the wire into N pieces and give you lists of the lengths of the wire segments as well as their densities. You'll use those lists in your calculations.

\textbf{MATLAB code:}
\begin{verbatim}
G = 6.67e-11;
% m = ...
% N = ...
[~, lengths, densities, ~, ~] = break_into_pieces(r, N, rho)
masses = ...
distances = compute_approximate_distances(r, N, rho, P)
forces = G * m * masses ./ (distances.^2)
totalForce = sum(forces)
\end{verbatim}

\textbf{Reflection Questions:}
- If you continued making this same calculation for different N, particularly as N grows quite large, how would you expect the resulting gravity calculations to behave?
- Write out symbolically, for your specific function, what sum is being calculated to measure the gravity for a particular (general) N.

\subsection*{Step 3: Examining changes to the setup}
Now let's make some changes to make sure we understand the process. Let's see what happens if we change the density to be drastically increasing throughout the length of the wire, like $\rho = \frac{1}{2} + t^3$.

\textbf{MATLAB code:}
\begin{verbatim}
rho_test = .5 + t^3
plot_wire_gravity_pieces(r, 10, rho_test, P, m)
\end{verbatim}

\textbf{Reflection Question:}
- The color distribution likely shows the brighter colors (higher values) late in the curve. Explain this.

\subsection*{Step 4: Refining Approximation: Increasing N}
Increasing $N$ gives better approximations. The segments better approximate a point mass, meaning we're getting closer to treating each particle along the wire as if it were a point mass and hence adding up many accurate instances of Newton's law.

\textbf{MATLAB code:}
\begin{verbatim}
Ns = []
plot_multiple_gravity_approximations(r, Ns, rho, P, m)
for N = Ns
    % manual calculations here
end
\end{verbatim}

\textbf{Reflection Questions:}
- If you continued making this same calculation for different N, particularly as N grows quite large, how would you expect the resulting gravity calculations to behave?
- Write out symbolically, for your specific function, what sum is being calculated to measure the gravity for a particular (general) N.

\subsection*{Step 5: Exact Model Using Calculus}
The main difference between finite sum approximations and the exact sum from integration is the limit. Definite integrals and the Fundamental Theorem of Calculus give us tools for making exact calculations. To use calculus efficiently, we need to rewrite the integral until we get a form that makes calculations on a single variable that we can run continuously through an interval.

So far, our sums have taken the form:
\begin{equation}
F = \sum_i \Delta F_i = \sum_i \frac{G m \Delta M_i}{D_i^2}
\end{equation}
Translating this to integrals, we have:
\begin{equation}
F = \int dF
\end{equation}
From Newton's law, $dF = \frac{G m dM}{D_M^2}$, so we can compute the total sum by the integral:
\begin{equation}
F = G m \int \frac{dM}{D_M^2}
\end{equation}
The problem is that we can't use the fundamental theorem of calculus yet because we don't have a good interval of masses to integrate over, nor a good way to represent the distances $D_M$. What we do have is the formula for both the curve $r$ and the density $\rho$ in terms of time, and a way to represent the arc length of a small segment $ds$ in terms of the curve's velocity.

\textbf{Reflection Questions:}
- Explain the formula $G m \int \frac{dM}{D_M^2}$. Does it actually represent anything?
- Why, in our current setup, do we need to make changes to actually compute the integral value?
- What would it take to be able to compute the integral as is, rather than doing further variable changes?
- Write your specific integral using your vector curve and density function. How is this new integral different or the same from the original integral $F = \int dF$?

\subsection*{Finishing the Calculations}
Once you have the integral ready, change the definitions of the following functions to match your setup and compute the final result! You will need to verify that your integral calculations match numerical approximations using the methods in this project.

\textbf{MATLAB code:}
\begin{verbatim}
syms t
f = sqrt(exp(t+1) - cos(t))
a = 0
b = 1
F_exact = int(f, a, b)
double(F_exact)
\end{verbatim}

\end{document}
