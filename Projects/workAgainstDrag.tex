\documentclass{ximera}

\title{Final Project Help: Work Done Against Drag}
\author{Zack Reed}

\begin{document}
\begin{abstract}
In this project you'll explore how to model the work required to overcome drag forces when an object moves along a curve. We'll start with a simple ``ideal'' situation that can be modeled using multiplication, and then build up to an integration-based approach that adds up contributions across the motion.
\end{abstract}
\maketitle

\section*{Final Project Walkthrough - Work Done Against Drag}

\subsection*{Introduction}
Welcome to the "Work Done Against Drag" final project!
For this final project we'll explore how to model the \textbf{work required to overcome drag forces when an object moves along a curve}. As with all integration problems, we'll begin with a simple "ideal" situation that can be measured using multiplication, and then build up to an integration-based measurement approach that "adds up tiny pieces" of approximations across an object's motion.

\subsection*{Context: Drag on a Moving Object}
In physics, drag is a force that resists motion in a fluid (like water or air). Drag is always directed in the opposite direction of motion.

\textbf{Basic Model 1: Drag Force}
The basic model for drag is the ideal context of an object moving through a fluid that doesn't vary much as the object moves along, and the object moves at a constant velocity in a straight direction.
Under these conditions, the drag force $F_D$ is modeled as:
\begin{equation}
F_D = \rho \cdot C_D \cdot A \cdot \frac{v^2}{2}
\end{equation}
where $\rho$ is the density of the fluid, $C_D$ is a drag coefficient, $A$ is the reference area, and $v$ is the velocity of the object.

In our 3D context, forces and velocities are vectors, so the above equation is specifically about the magnitudes of those vectors:
\begin{equation}
||\vec{F_D}|| = \rho \cdot C_D \cdot A \cdot \frac{||\vec{v}||^2}{2}
\end{equation}
where $\vec{F_D}$ and $\vec{v}$ are the drag force and velocity vectors.

\textbf{Basic Model 2: Work (Or Energy used)}
In an ideal context, we use "work" to quantify how much energy is used to do some action, like moving through a fluid or lifting an object a certain amount. With "work", there is some force exerted across some distance (displacement).
If the force $\vec{F}$ is constant across a displacement vector $\vec{D}$, work is measured by the dot product:
\begin{equation}
W = \vec{F} \cdot \vec{D} = \sum_i F_i D_i = ||\vec{F}|| \cdot ||\vec{D}|| \cos(\theta)
\end{equation}
where $\theta$ is the smallest angle between the vectors.

We're going to imagine a spherical object flying through the air in a tight pattern (so the altitude and measures like air density are basically fixed).

\youtube{kVS_PuLWiR4}

In this example, $\rho, C_D, A,$ and $v$ are all constant. So, $\vec{F}$ is constant across the displacement vector at a magnitude of $0.92\,N$. Since drag is in the opposite direction of motion, the angle between displacement and drag will always be $\pi$, so for the displacement $\vec{D} = [6-2\pi, 2\pi, 2\pi]$:
\begin{equation}
W = ||\vec{F}|| \cdot ||\vec{D}|| \cdot \cos(\pi) \approx 0.92 \times 8.89 \times (-1) = -8.179\,J
\end{equation}
where $J$ is Joules.

\subsection*{Challenges to the Basic Model}
If drag were constant, computing work would be simple:
- $\rho$ is the fluid density
- $C_D$ is the drag coefficient
- $A$ is the reference area
- $||v||$ is the speed of the object
- Multiply the constant drag force magnitude by the total displacement

Unfortunately, there are confounding factors: drag depends on velocity, which can change along a curve. Faster sections produce stronger drag, slower sections produce weaker drag, and the drag vector changes direction with the velocity. $C_D$ and $\rho$ can also change during a flight.

The following is an animation of an object moving along a simple circular curve with the drag forces and velocities changing throughout the motion:

\begin{center}
\youtube{L4HAzbdjI_g}
\end{center}

Because of situations as above, we cannot always treat drag as one constant force being applied over the entire curve.

\subsection*{Solution: Add up small bits of work along the curve}
As with all integration problems, we solve the "multiplying doesn't work" problem by:

\begin{enumerate}
\item breaking the curve into small pieces,
\item computing the work on each small piece,
\item adding up all of the small contributions to get total work.
\end{enumerate}

For most curves there is no single displacement vector, so instead you have to treat movement along the curves as movement along tiny displacement vectors.

\subsection*{Step 1: Approximate with Pieces}
Suppose an object moves along a curve $r(t)$. If we cut the interval into $N$ pieces, we can take the velocity $v_i$ from the curve at the middle of the interval and use the time change to get an approximate displacement vector $\Delta r_i$. If any elements of the drag equation change along the curve, we can also track their values to get a sample force magnitude at each midpoint segment:
\begin{equation}
||\vec{F_i}|| = \rho_i C_{D_i} A \frac{||\vec{v_i}||^2}{2}
\end{equation}

As an example of this, suppose that the object is drastically moving through different altitudes, the air density will be heavier at lower altitudes than at higher altitiudes. 

This will affect $\rho$, which then also affects $C_D$ in turn.

The work on piece $i$ is approximated by

$$\Delta W_i \approx \vec{F_i} \cdot \Delta r_i$$

Adding them up gives an approximation:

$$W \approx \sum_{i=1}^N \Delta W_i$$

As ($N$) grows, this approximation converges to the exact work done against drag.

\begin{example}

    In the project livescript you are instructed to enter your custom curve $r$ along with the relevant parameters for drag.

    You are then given course functions to plot an approximation of the work done against drag along your curve.

    Such an example plot is shown below:

    \begin{center}
    \includegraphics{complex_drag_1.png}
    \end{center}

    You see a more complicated wire that is broken up into segments (much like in the mass examples from Module 3). Each segment is colored by air density which is acting as a proxy for the possible complexity of factors in the drag equation.Small arrows along the curve show the drag force vectors at each segment.

    In the live script you can zoom and rotate the figure, here you see a zoomed out image. 

\end{example}

\begin{remark}
    The livescript walks you through the steps to recreate the approximation of work using midpoint Riemann Sums, and you should end up with the same final work calculation shown in your personal plot.

    Below are the important contextual formulas that carry out this Riemann Sum approximation.

    \begin{enumerate}
        \item First, note that along each piece work against drag is approximated by the dot product:
        $$\Delta W_i \approx \vec{F_i} \cdot \Delta r_i$$
        where $\vec{F_i}$ is the drag force vector at piece $i$ and $\Delta r_i$ is the displacement vector along piece $i$.
        \item The drag force vector magnitude is given by the drag equation:
        $$||\vec{F_i}|| = \rho_i C_{D_i} A \frac{||\vec{v_i}||^2}{2}$$
        where $\rho_i$ is the fluid density at piece $i$, $C_{D_i}$ is the drag coefficient at piece $i$, $A$ is the reference area, and $||\vec{v_i}||$ is the speed of the object at piece $i$.
        \item The total work approximation is given by the sum of the piecewise work calculations:
        $$W \approx \sum_{i=1}^N \Delta W_i$$
    \end{enumerate}
\end{remark}

\subsection*{Refining Approximation - Increasing N}

Hopefully unsurprising by now, we suspect that increasing N will give us better approximations. This should be evident physically:
\begin{enumerate}
\item Movement along small segments of the curve behaves more like straight displacements,
\item which means the approximations from both the drag equation and the work computations are more accurate over small steps.
\end{enumerate}

In the livescript you are given the tools to recreate your approximations with increasing values of $N$ and observe patterns.

\begin{example}

The resulting plots and calcualtions should look like the following examples, which are taken from different values of $N$ for the curve 

$$r= [ (2.316 + 0.592\cos(4t))cos(7t), (2.316 + 0.592\cos(4t))sin(7t), 0.592sin(4t)]$$

The density, $C_D$, and $A$ values used are:

$$\rho = 1.045 + 0.484(t/(2\pi))^2 $$

$$C_D = .6536$$

$$A=7.7294$$

First, at $N=5$, we get:

\begin{center}
\includegraphics{complex_drag_low_n.png}
\end{center}

At $N=30$, we get:

\begin{center}
\includegraphics{complex_drag_n_2.png}
\end{center}

At $N=100$, we get:

\begin{center}
\includegraphics{complex_drag_n_3.png}
\end{center}

At $N=150$, we get:
\begin{center}
\includegraphics{complex_drag_n_4.png}
\end{center}

At a value $N=100000$, you get a work calculation of $-9.1266\times 10^4$ Joules, which is quite similar to the calculation from $N=150$. 

\end{example}

\begin{remark}
    Remember that at each $N$, the process is exactly the same:
    \begin{enumerate}
        \item Break the curve into $N$ pieces,
        \item Compute the drag force vector and displacement vector at each piece,
        \item Compute the work on each piece using the dot product,
        \item Add up all the piecewise work calculations to get total work.
    \end{enumerate}

    As $N$ increases, the approximations get better and better. This leads us to the limit, where we can use the Fundamental Theorem of Calculus to get an exact value.
\end{remark}


\subsection*{Exact Model Using Calculus}

The main difference between finite sum approximations and the exact sum from integration is the limit. Definite integrals and the Fundamental Theorem of Calculus give us tools for making exact calculations. To use calculus efficiently, we need to rewrite the integral until we get a form that makes calculations on a single variable that we can run continuously through an interval.

So far, our sums have taken the form:
\begin{equation}
W \approx \sum_{i=1}^N \Delta W_i, \quad \Delta W_i \approx \vec{F_i} \cdot \Delta r_i
\end{equation}
where $||\vec{F_i}|| = \rho_i C_{D_i} A \frac{||\vec{v_i}||^2}{2}$ and $\Delta r_i$ is a displacement vector.
Translating this into integrals, we have:
\begin{equation}
W = \int dW
\end{equation}
The work differential is the dot product $dW = \vec{F_D} \cdot d\vec{r}$ between the drag force and displacement across a small segment of the object's path.

\subsection*{Your Turn: Using Integrals to get an Exact Measure}

You'll finish the project by using an integral to generate an exact measure of work done against drag with a new curve

$$r= [ C_1 \cos(t), C_1 \sin(t), C_2 t ]$$

where $C_1$ and $C_2$ are constants provided to you, as are the density function and the time interval $[0, 2\pi]$.

You will want to take into account the following information to help you set up your integral:

So far, we know that the work differential is the dot product $dW = \vec{F_D} \cdot d\vec{r}$ between the drag force and displacement across a small segment of the object's path.

Within this small path segment, $\vec{F_D}$ the magnitue of  doesn't change, so we just need to determine how to compute the dot product along the differential displacements $\vec{dr}$. This will be your job, to go the rest of the way.

Remember that you can determine the dot product by summing the component-products of two vectors, or by the formula $\vec{u}\cdot\vec{v} = ||\vec{u}|| \, ||\vec{v}|| \cos \theta$, where $\theta$ is the smallest angle between the vectors. Either approach will help, particularly if you remember how the drag force relates to the direction of motion.

\end{document}
