\documentclass{ximera}

\title{Warm Up Project Help: Mass of a Wire}
\author{Zack Reed}

\begin{document}
\begin{abstract}
These resources will help you as you work on your warm up project, exploring how we compute the mass of a wire with variable density using integration along a curve.
\end{abstract}
\maketitle

\section*{Introduction}
Welcome to your first project walkthrough!
This walkthrough will give you a good glimpse of some of the main ways we want to think about differentiating and integrating along curves when modeling phenomena using mathematics! 

While you will need to complete the project in your live script document, and should become familiar with working in the live scripts, we have also provided the animations, some explanations, and some practice materials on this page for further help, along with video walkthroughs of the introduction.

Here is the first video walkthrough, covering introductory ideas through the confirmation of constant density of the first wire:

\begin{center}
\youtube{Kpc1k3sBxc4}
\end{center}

\section*{Density: Basic Model}
\subsection*{Basic Model: Mass = Density $\times$ Length}
As with all things integral, you start with a basic model (typically involving multiplication) for quantifying something about a mathematical object. In this case, the basic model is:
$$
\text{mass} = \text{density} \times \text{length}
$$
Then, some variation in an object forces us to use integration instead of multiplication, adding up lots of small approximate pieces formed from the basic model.

Many wires you come across are \textbf{uniformly dense}. That is, no matter how big or small of a piece of wire, the ratio between the mass of the wire and the length of the wire will always be constant (i.e. a constant density).



% Animation placeholder
\youtube{fKjpdnlIlIE?si=y2fj0AKYGHi9gxK2}

Use the slider to run through the animation after it finishes. You'll note that there is a constant ratio between the mass and the length no matter what the segment size is.

\begin{problem}
Remember that the basic model for density is $\text{mass} = \text{density} \times \text{length}$, so density is mass divided by length.

The first entry is a fraction where the numerator is the mass of the full wire. Fill in the missing length and the density measure up to 2 decimal places.

Then refer to the animation to fill in the density measures at the remaining scales. This will verify that across these instances the wire density was indeed constant (slight differences are just numerical error).

\begin{enumerate}
\item $\frac{17.772}{\answer{8.886}}=\answer[tolerance=0.1]{2}$
\item $\frac{\answer{.355}}{\answer{0.178}}=\answer[tolerance=0.01]{1.99}$
\item $\frac{\answer{.178}}{\answer{0.089}}=\answer[tolerance=0.01]{2.00}$
\end{enumerate}


\end{problem}

The next video covers the activity through the first approximation of mass by Riemann Sums:

\begin{center}
\youtube{ghGjOJ1TeUU}
\end{center}


\subsection*{Variable Density}

Now consider the following animation that tracks the mass and length of small sections of a wire that has \emph[{variable density}], meaning the density itself changes depending on the location of the wire. 

% Animation placeholder
\begin{center}
\youtube{FIFZBi8JPFU?si=VcA5Q-NM_qZlT_Ob}
\end{center}

The formula $\rho = 2 + t^2$ should provide the density at any location $t$ along the wire. 

\begin{problem}

As before, slide through the animation and determine what the density is at for times closest to $t=\pi$ and $t=2\pi$ (at each of the scales) by dividing the mass by the length of the segments at those times.

For $t=\pi$:

\begin{enumerate}
\item In the animation, the closest time to $t=\pi$ at the largest scale is $t=\answer{3.166}$. At this time, the density of the small segment is $\frac{\answer[tolerance=.01]{5.342}}{.444}=\answer[tolerance=0.01]{12.0315}$
\item In the animation, the closest time to $t=\pi$ at the middle scale is $t=\answer{3.167}$. At this time, the density of the small segment is $\frac{\answer[tolerance=.01]{2.138}}{.178}=\answer[tolerance=0.01]{12.0112}$
\item In the animation, the closest time to $t=\pi$ at the smallest scale is $t=\answer{3.167}$. At this time, the density of the small segment is $\frac{\answer[tolerance=.01]{1.069}}{.089}=\answer[tolerance=0.01]{12.0112}$
\end{enumerate}

The value of the density function at $t=\answer[tolerance=.01]{3.167}$ is $\rho = 2 + (\answer[tolerance=.01]{3.167})^2 = \answer[tolerance=0.01]{12.0299}$.

%now for t=2pi. The largest scale time is 6.126 length .444 mass 17.562, middle scale is 6.220, .178, 7.232, smallest scale is 6.252, .089, 3.651

For $t=2\pi$:

\begin{enumerate}
\item In the animation, the closest time to $t=2\pi$ at the largest scale is $t=\answer{6.126}$. At this time, the density of the small segment is $\frac{\answer[tolerance=.01]{17.562}}{.444}=\answer[tolerance=0.01]{39.5541}$
\item In the animation, the closest time to $t=2\pi$ at the middle scale is $t=\answer{6.220}$. At this time, the density of the small segment is $\frac{\answer[tolerance=.01]{7.232}}{.178}=\answer[tolerance=0.01]{40.6292}$
\item In the animation, the closest time to $t=2\pi$ at the smallest scale is $t=\answer{6.252}$. At this time, the density of the small segment is $\frac{\answer[tolerance=.01]{3.651}}{.089}=\answer[tolerance=0.01]{41.0225}$
\end{enumerate}

The value of the density function at $t=\answer[tolerance=.01]{6.283}$ is $\rho = 2 + (\answer[tolerance=.01]{6.283})^2 = \answer[tolerance=0.1]{41.4761}$.

Within a small error, the density calculations from the animation and from the density function should agree. 

The animation broke the wire up into pieces of a fixed length, used the density at the time representing the midpoint of that piece, and then calculated the mass of that piece by multiplying the density by the length.

The error in the calculations comes from (select all):

\begin{selectAll}
\choice[correct]{Rounding errors in the decimal approximations of the time values}
\choice[correct]{Using a fixed density for each piece}
\end{selectAll}

which means that the basic model of mass = density $\times$ length \wordChoice{\choice[correct]{does not work}\choice{works}} for wires with variable density. 
\end{problem}

\section*{Visualize the Approximate Pieces}
You are provided with course functions that break up the wire into pieces and use the midpoint values of the pieces to approxiamte mass. As the example curve (not your own), you should see a plot such as the followoing:

\begin{center}
\includegraphics{20_seg_wire.png}
\end{center}

You are also provided with course functions that will both plot the wire and provide the approximate measure of wire mass based on the approximation, such as the following plot with an approximation measure of $132.354$ grams:

\begin{center}
\includegraphics{20_mass_approx.png}
\end{center}

\begin{example}
You are asked to recreate this approximation yourself in the live script using MATLAB code.

The code breaks down the process by giving you the lists of segment lengths and the midpoint densities, only requiring that you multiply the individual entries together and then sum them to get a total mass approximation.


Remember that mass is approximated by the product $M = \rho \cdot L$, ``density times the length''. If we had the lengths and densities of our wire after breaking it up into $N$ pieces, we could just multiply those densities and lengths to get the mass of each individual piece.

You're also asked to write out a formula symbolically to describe what MATLAB is doing in the code.

Remember that the general sum is written as:
$$
\sum_{i=1}^N \Delta M_i$$

where $\Delta M_i$ is the mass of the $i$th piece (like, the $5th$ piece, the $100th$ piece, etc). Be sure to also use $\rho_i$ as the density of the $i$th piece and $\Delta L_i$ as the length of the $i$th piece. 

Youre symbols should represent that we are adding up $N$ instances of the basic model at work, density times length.
\end{example}

\section*{Refining the Approximation}

The next video details the refinement of the approximation process at smaller scales:

\begin{center}
\youtube{ZVBcRqfqNog}
\end{center}

\begin{example}


You are then asked to recreate this process for increasing values of $N$, which refines the approximation. At each stage you should multiply the densities and the lengths togther and sum the results.

This should yield approximations such as the following:

First, $N=50$:

\begin{center}
\includegraphics{50_mass_approx.png}
\end{center}

Then $N=100$:
\begin{center}
\includegraphics{100_mass_approx.png}
\end{center}

Then $N=150$:
\begin{center}
\includegraphics{150_mass_approx.png}
\end{center}

You are then asked to just recreate the calculations for very large $N$ values, such as the following:

\begin{enumerate}
  \item $N=1000$ gives an approximation of $134.7034$
  \item $N=10000$ gives an approximation of $134.7035$
  \item $N=100000$ also gives an approximation of $134.7035$
\end{enumerate}

This does not mean that we found the exact mass, it simply means we surpassed the default precision of MATLAB.

\end{example}

\section*{Generating Exact Measures: Integration}
Let's see if we can use integrals to get a more exact measure, and double check that it's close to our approximations!

The key with integrating is that we're still adding up small bits of mass to get an exact measure of mass.

The small bit of mass are the differential bits whose ``total sum'' is really the limit of the finite sums.

This final video covers the integration process and sets up your work in the project:

\begin{center}
\youtube{ya4HOuEXlDw}
\end{center}

\subsection*{Symbolizing}
Let's translate our finite approximations into integrals to get an idea of how we want to set up the integral for computation.

\begin{enumerate}

\item The approximation equation $M\approx \Sigma\Delta M$ becomes the integral equation $M=\int dM$, ``add up tiny bits of mass to get the total mass''.

\item The small bit of mass comes from the density*length products on a small bit of arc length along the curve, $dL$, so our equation for mass becomes $M=\int dM=\int \rho dL$. ``The total mass comes from adding up small density \* length products along the wire''.
\end{enumerate}


The main trick when computing integrals is using the Fundamental Theorem of Calculus (FTC):
\begin{itemize}
  \item The FTC is usable for differential products of the form $dF = f \, dt$ so that we can make use of antiderivatives to find explicit formulas for $F$.
  \item Since the curve and density are defined by time, without doing some kind of substitution we can't immediately use the FTC when we integrate with respect to Mass or Length.
  \item Luckily, we know how to differentially represent the arc length differential from a curve's velocity magnitude:
\end{itemize}
$$
dL = \|\vec{v}\| \, dt
$$
So, with one final re-write of our equations, we get
$$
M = \int dM = \int \rho dL = \int_a^b \rho(t) \|\vec{v}\| \, dt
$$

You are provided with MATLAB code to carry out these calculations for the provided curve

$$\vec{r} = [\cos(t), \sin(t), t]$$

and density function $\rho = 2 + t^2$. This should yield an exact mass of $\frac{4\pi\sqrt{2}(2\pi^2+3)}{3}\approx 134.7035$ grams.

\section*{Your Turn!}

You'll complete the project first by recreating the finite approximation with a complicated curve, such as 

$$\vec{r} = [ (2.871 + 0.603\cos(2t))\cos(5t), (2.871 + 0.603\cos(2t))\sin(5t), 0.603\sin(2t)]$$

which in pieces would look something like this:

\begin{center}
\includegraphics{mass_pieces_complicated.png}
\end{center}

You will then compute the exact mass using integration along a simpler curve. The process should closely follow the examples laid out in the livescript.

Good luck!

\end{document}
