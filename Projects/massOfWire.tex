\documentclass{ximera}

\title{Warm Up Project Help: Mass of a Wire}
\author{Zack Reed}

\begin{document}
\begin{abstract}
These resources will help you as you work on your warm up project, exploring how we compute the mass of a wire with variable density using integration along a curve.
\end{abstract}
\maketitle

\section*{Introduction}
Welcome to your first project walkthrough!
This walkthrough will give you a good glimpse of some of the main ways we want to think about differentiating and integrating along curves when modeling phenomena using mathematics! 

While you will need to complete the project in your live script document, and should become familiar with working in the live scripts, we have also provided the animations, some explanations, and some practice materials on this page for further help, along with video walkthroughs of the introduction.

\section*{Density: Basic Model}
\subsection*{Basic Model: Mass = Density $\times$ Length}
As with all things integral, you start with a basic model (typically involving multiplication) for quantifying something about a mathematical object. In this case, the basic model is:
$$
\text{mass} = \text{density} \times \text{length}
$$
Then, some variation in an object forces us to use integration instead of multiplication, adding up lots of small approximate pieces formed from the basic model.

Many wires you come across are \textbf{uniformly dense}. That is, no matter how big or small of a piece of wire, the ratio between the mass of the wire and the length of the wire will always be constant (i.e. a constant density).



% Animation placeholder
\youtube{fKjpdnlIlIE?si=y2fj0AKYGHi9gxK2}

Use the slider to run through the animation after it finishes. You'll note that there is a constant ratio between the mass and the length no matter what the segment size is.

\begin{problem}
Remember that the basic model for density is $\text{mass} = \text{density} \times \text{length}$, so density is mass divided by length.

The first entry is a fraction where the numerator is the mass of the full wire. Fill in the missing length and the density measure up to 2 decimal places.

Then refer to the animation to fill in the density measures at the remaining scales. This will verify that across these instances the wire density was indeed constant (slight differences are just numerical error).

\begin{enumerate}
\item $\frac{17.772}{\answer{8.886}}=\answer[tolerance=0.1]{2}$
\item $\frac{\answer{.335}}{\answer{0.178}}=\answer[tolerance=0.01]{1.99}$
\item $\frac{\answer{.178}}{\answer{0.089}}=\answer[tolerance=0.01]{2.00}$
\end{enumerate}


\end{problem}


\subsection*{Variable Density}
% Animation placeholder
% \youtube{FIFZBi8JPFU?si=VcA5Q-NM_qZlT_Ob}

You should see that the mass changes as you move along the curve. This means that the density of this wire is not uniform.

The formula $\rho = 2 + t^2$ should provide the density at any location $t$ along the wire. As before, slide through the animation and determine what the density is at $t=\pi$ and $t=2\pi$ by dividing the mass by the length of the segments at those times.

Within a small error, the density calculations from the animation and from the density function should agree.

\subsection*{Application Walkthrough 1: Mass of a Wire}

\subsection*{Introduction}
Welcome to your first project walkthrough! This walkthrough will give you a good glimpse of some of the main ways we want to think about differentiating and integrating along curves when modeling phenomena using mathematics! This will also give you a rough template and introduce you to the tools that you can use for your mini project, due at the end of this module.

\subsection*{Use MATLAB As an Interactive Tool, not "Code"}
You'll notice that all of the projects are housed within MATLAB Live Scripts. This is primarily to use MATLAB as an environment in which you can watch animations, view and interact with plots, use applets, read formulas, and engage with your custom formulas and functions.

While you will be asked to interact in some sections by "writing code", the code you write will mostly be making standard mathematical calculations using basic operations: addition ($+$), subtraction ($-$), multiplication ($\cdot$), division ($/$), and exponentiation ($^2$). For the most part, the rest of your work in MATLAB will be defining terms (like $N$, mass, density), selecting the "Run Section" button to activate the MATLAB code, or using other calculus-oriented commands that you have practiced already.

\subsection*{Challenges to the Basic Model: Variable Density}
We want to calculate the mass of a wire just using multiplication, $M = \rho \cdot L$, where $M$ is the mass, $\rho$ is the uniform density, and $L$ is the length. Unfortunately, we can't use this basic model for many wires because:
\begin{itemize}
  \item Many wires have strange shapes and we don't want to stretch them out to measure their length.
  \item Not all wires have uniform densities, so there is no one density that will work in the basic model.
\end{itemize}
So, how do we compute the total length and total mass? We integrate to "add up small mass pieces" together.

\section*{Step 1: Visualize the Approximate Pieces}
Let's visualize the approximation process first. The function \textbf{plot curve pieces} will:
\begin{itemize}
  \item Take in a wire $\vec{r}$ as the first argument,
  \item A number of pieces $N$ as the second argument,
  \item A density function $\rho$ as the third argument,
  \item Show a plot of the wire pieces color-coded by density.
\end{itemize}

% Animation placeholder
% \youtube{Wire Pieces Color Animation}

This sets up a \textit{midpoint approximation} of the mass. The density at the midpoint of the segment is treated as the uniform density of the entire wire.

The function \textbf{plot curve mass pieces} will also use the same arguments $(r, N, \rho)$ but the plot will tell you the mass approximated by the midpoint sum.

If everything is set up correctly, a 20-piece sample midpoint approximation of the mass should measure $134.354$ grams.

\section*{Step 2: Approximate the Wire Mass}
Remember that mass is approximated by the product $M = \rho \cdot L$, "density times the length". If we had the lengths and densities of our wire after breaking it up into $N$ pieces, we could just multiply those densities and lengths to get the mass of each individual piece.

This is exactly what the course function \textbf{break into pieces} does!

You should see two lists (arrays) called \textbf{lengths} and \textbf{densities}. The numbers in each are the lengths of the broken wire and the densities of the broken wire. The $i$th element of \textbf{lengths} is the length of the $i$th segment, and the $i$th element of \textbf{densities} is the density at the midpoint of the $i$th segment.

You should see a 20-length list of masses whose entries are the products of the corresponding entries in lengths and densities.

\section*{Reflection Questions}
\begin{itemize}
  \item Why for the given $\vec{r}$ and $\rho$ do we need to break the wire into pieces?
  \item What cannot be exactly calculated by the basic model $\text{mass} = \text{density} \times \text{length}$ in the setup of our wire?
  \item Why can we not just use a single length or a single density value?
\end{itemize}

\section*{Reflection Questions}
\begin{itemize}
  \item Write out symbolically a formula to describe what MATLAB is doing with the line . Use $\Delta M_i$ as the "$i$th" mass, $\rho_i$ as the "$i$th" density, and $\Delta L_i$ as the length of the "$i$th" piece.
  \item If we took a 200 or 2000-piece approximation, what about our calculation method would change?
\end{itemize}

% Animation placeholder
% \youtube{Mass Approximation Animation}

\subsection*{Symbolizing: Midpoint Sum Approximation}
To symbolically write out our approximation:
% \begin{align*}
% M &\approx \sum_{i} \Delta M_i \\
%   &= \sum_{i} \rho_i \Delta L_i
% \end{align*}

\subsection*{Refining the Approximation Using Smaller Scales}
This approximation works at any scale, and approximations get better as we use smaller scales (increasing $N$).

\section*{Reflection Questions}
\begin{itemize}
  \item What is happening to the mass approximation as $N$ increases?
  \item Do you think the approximation is becoming more or less accurate? Why or why not?
  \item What is happening to the measurement accuracy of each piece as $N$ increases? Explain.
\end{itemize}

\subsection*{Very Small Scales}
Notice that the last two $N$-values gave the same approximation. This does not mean we found the exact mass, it simply means we surpassed the default precision of MATLAB.

\section*{Reflection Questions}
\begin{itemize}
  \item Why is this section called "very small scales" but we're growing $N$ to have very large values?
  \item Why is a small scale important for the approximation?
\end{itemize}

\section*{Generating Exact Measures: Integration}
Let's see if we can use integrals to get a more exact measure, and double check that it's close to our approximations!

The key with integrating is that we're still adding up small bits of mass to get an exact measure of mass:
% \begin{align*}
% M &\approx \sum \Delta M \\
% M &= \int dM
% \end{align*}

The small bit of mass comes from the density-length products on a small bit of arc length along the curve, $dL$, so our equation for mass becomes:
$$
M = \int \rho \, dL
$$

The main trick when computing integrals is using the Fundamental Theorem of Calculus (FTC):
\begin{itemize}
  \item The FTC is usable for differential products of the form $dF = f \, dt$ so that we can make use of antiderivatives to find explicit formulas for $F$.
  \item Since the curve and density are defined by time, without doing some kind of substitution we can't immediately use the FTC when we integrate with respect to Mass or Length.
  \item Luckily, we know how to differentially represent the arc length differential from a curve's velocity magnitude:
\end{itemize}
$$
dL = \|\vec{v}\| \, dt
$$
So, with one final re-write of our equations, we get
$$
M = \int_a^b \rho(t) \|\vec{v}\| \, dt
$$

\subsection*{Building and Computing the Integral}
You should see a number (and not a function or other entity).

Sometimes, $M$ isn't given as an exact number, it just writes out the integral and the formula is a bit unwieldy. That's okay! Many functions don't have nice integrals (and often it's impossible to find antiderivatives).

Hopefully, you get the same approximation (or very close) as we did with our high $N$ approximations!

\section*{Your Turn!}

\subsection*{Part 1: A Complicated Curve}
Go through the steps from above for yourself, first using your own curve provided by \textbf{verify setup} in the beginning and then using a new curve from the provided constants $C_1$ and $C_2$.

Now manually compute the approximation using the same calculations that we did earlier in the script. Be sure to name all of your variables appropriately.

\section*{Reflection Questions}
\begin{itemize}
  \item Describe what would be difficult about computing the mass of the provided curve by hand.
  \item Are we even guaranteed that all masses can be computed by-hand using antiderivatives? Discuss why or give a counter example.
\end{itemize}

\subsection*{Part 2: A Simpler Curve for Integration}
Now you'll do this process again, but with a simpler function that more easily integrates for verification.
The function will be:
$$
r = [C_1 \cos(t), C_1 \sin(t), C_2 t]
$$
with $C_1$ and $C_2$ taken from your \textbf{verify setup} information.

Now use the same summation methods from earlier in the project to approximate the mass of the wire.

Finally, use integration to verify the approximation using calculus. Feel free to use decimals to three decimal places for the coefficients of the functions. Remember to use a trig identity to greatly simplify your integral before computing it. Also, remember that the wire runs between $t=0$ and $t=2\pi$.

\section*{Reflection Questions}
\begin{itemize}
  \item What was hard and what was easy about this process?
  \item Explain any major or minor errors between the numerical approximation and the result found from integration.
  \item Describe in detail, including diagrams as necessary, the process of measuring the mass of a wire with variable density.
\end{itemize}







\end{document}
