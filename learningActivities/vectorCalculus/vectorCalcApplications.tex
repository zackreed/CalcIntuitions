\documentclass{ximera}
\title{Vector Function Applications — Problems with Worked Solutions (interactive)}
\author{Prepared for Calc III}
\begin{document}
\begin{abstract}
Four worked applications using vector-valued functions. Each solution is presented step-by-step, and final intermediate/final results are provided as interactive inputs (\texttt{\textbackslash answer\{\}}) so students confirm the computations as they read.
\end{abstract}

\maketitle

%%%%%%%%%%%%%%%%%%%%%%%%%%%%%%%%%%%%%%%%%%%%%%%%%%%%%%%%%%%%
\section*{Kinematics}

\begin{problem}
A particle moves in space along the path
\[
\vec{r}(t) = \langle t,\, \sin t,\, \cos t\rangle.
\]

Find the velocity vector $\vec{v}(t)$.

Find the acceleration vector $\vec{a}(t)$.

Compute the speed at $t=\pi/2$.

Find the total distance traveled on $[0,2\pi]$.
\end{problem}

\begin{solution}
We differentiate componentwise.

\[
\vec{v}(t)=\vec r'(t)
= \langle \,\frac{d}{dt}t,\; \frac{d}{dt}\sin t,\; \frac{d}{dt}\cos t \,\rangle
= \langle 1,\; \cos t,\; -\sin t\rangle.
\]
Enter the velocity vector:
\[
\vec{v}(t)=\answer{\langle 1,\;\cos t,\; -\sin t\rangle}.
\]

Differentiate again for acceleration:
\[
\vec{a}(t)=\vec v'(t)=\langle 0,\; -\sin t,\; -\cos t\rangle.
\]
Enter the acceleration vector:
\[
\vec{a}(t)=\answer{\langle 0,\; -\sin t,\; -\cos t\rangle}.
\]

Speed is the norm of velocity:
\[
\|\vec v(t)\|=\sqrt{1^2+\cos^2 t+(-\sin t)^2}
=\sqrt{1+\cos^2 t+\sin^2 t}=\sqrt{1+1}=\sqrt{2}.
\]
So the speed (constant) is
\[
\|\vec v(\tfrac{\pi}{2})\|=\answer{\sqrt{2}}.
\]

Distance traveled on $[0,2\pi]$ is the arc length:
\[
\int_0^{2\pi} \|\vec v(t)\|\,dt = \int_0^{2\pi} \sqrt{2}\,dt
= 2\pi\sqrt{2}.
\]
Enter the total distance:
\[
\text{Distance}=\answer{2\pi\sqrt{2}}.
\]

\bigskip
\begin{multipleChoice}
  \choice{The speed is not constant; it depends on \(t\).}
  \choice[correct]{The speed is constant and equals \(\sqrt{2}\).}
  \choice{The speed is constant and equals \(1\).}
\end{multipleChoice}
\end{solution}

%%%%%%%%%%%%%%%%%%%%%%%%%%%%%%%%%%%%%%%%%%%%%%%%%%%%%%%%%%%%
\section*{Center of Mass of a Quarter-Circle Wire (variable density)}

\begin{problem}
A thin wire is bent into a quarter-circle of radius $2$, lying in the first quadrant (center at origin). Its linear density at a point is proportional to the $x$-coordinate (i.e. $\rho(x,y)=x$). Find the coordinates of the center of mass.
\end{problem}

\begin{solution}
Parameterize the quarter-circle by angle:
\[
\vec r(\theta)=\langle 2\cos\theta,\;2\sin\theta\rangle,\qquad 0\le\theta\le\frac{\pi}{2}.
\]
Enter the parameterization:
\[
\vec r(\theta)=\answer{\langle 2\cos\theta,\;2\sin\theta\rangle}.
\]

The density expressed as a function of $\theta$ is
\[
\rho(\theta)=x(\theta)=2\cos\theta.
\]
The differential arc length is
\[
ds=|\vec r'(\theta)|\,d\theta = (2)\,d\theta.
\]

\textbf{Total mass:}
\[
M=\int_0^{\pi/2} \rho(\theta)\,ds
=\int_0^{\pi/2} (2\cos\theta)\cdot 2\,d\theta
=4\int_0^{\pi/2}\cos\theta\,d\theta
=4(\sin\theta)\Big|_0^{\pi/2}=4.
\]
Enter the mass:
\[
M=\answer{4}.
\]

\textbf{Moment about the $y$-axis} (for $\bar x$):
\[
\int x\,dm = \int_0^{\pi/2} x(\theta)\,\rho(\theta)\,ds
= \int_0^{\pi/2} (2\cos\theta)\cdot(2\cos\theta)\cdot 2\,d\theta
=8\int_0^{\pi/2}\cos^2\theta\,d\theta.
\]
Use $\cos^2\theta=\frac{1+\cos 2\theta}{2}$:
\[
8\int_0^{\pi/2}\cos^2\theta\,d\theta
=8\cdot\frac{1}{2}\int_0^{\pi/2}(1+\cos 2\theta)\,d\theta
=4\Big[\theta+\tfrac{1}{2}\sin 2\theta\Big]_0^{\pi/2}
=4\cdot\frac{\pi}{2}=2\pi.
\]
So
\[
\bar x=\frac{1}{M}\int x\,dm=\frac{2\pi}{4}=\frac{\pi}{2}.
\]
Enter $\bar x$:
\[
\bar x=\answer{\tfrac{\pi}{2}}.
\]

\textbf{Moment about the $x$-axis} (for $\bar y$):
\[
\int y\,dm = \int_0^{\pi/2} (2\sin\theta)(2\cos\theta)(2)\,d\theta
=8\int_0^{\pi/2}\sin\theta\cos\theta\,d\theta.
\]
But $\int_0^{\pi/2}\sin\theta\cos\theta\,d\theta = \tfrac12$, so numerator = $8\cdot\tfrac12=4$.
Thus
\[
\bar y=\frac{4}{M}=\frac{4}{4}=1.
\]
Enter $\bar y$:
\[
\bar y=\answer{1}.
\]

Therefore the center of mass is
\[
(\bar x,\bar y)=\answer{\left(\tfrac{\pi}{2},\,1\right)}.
\]

\bigskip
\begin{multipleChoice}
  \choice{The center lies at the geometric centroid of the quarter circle.}
  \choice[correct]{The center is shifted toward the positive $x$-side because density depends on $x$.}
  \choice{The center lies on the circle itself.}
\end{multipleChoice}
\end{solution}

%%%%%%%%%%%%%%%%%%%%%%%%%%%%%%%%%%%%%%%%%%%%%%%%%%%%%%%%%%%%
\section*{Variable Density Rod}

\begin{problem}
A rod lies on the $x$-axis from $x=0$ to $x=3$ and has linear density $\rho(x)=1+x^2$ (mass per unit length). Find the rod's total mass and its center of mass coordinate $\bar x$.
\end{problem}

\begin{solution}
\textbf{Total mass:}
\[
M=\int_0^3 \rho(x)\,dx = \int_0^3 (1+x^2)\,dx
= \Big[x+\tfrac{x^3}{3}\Big]_0^3 = 3 + \tfrac{27}{3} = 3+9 = 12.
\]
Enter the mass:
\[
M=\answer{12}.
\]

\textbf{Moment about the origin (first moment):}
\[
\int_0^3 x\,\rho(x)\,dx = \int_0^3 x(1+x^2)\,dx
= \int_0^3 (x + x^3)\,dx
= \Big[\tfrac{x^2}{2} + \tfrac{x^4}{4}\Big]_0^3
= \tfrac{9}{2} + \tfrac{81}{4} = \tfrac{18}{4} + \tfrac{81}{4} = \tfrac{99}{4}.
\]

Thus the center of mass coordinate is
\[
\bar x = \frac{1}{M}\int_0^3 x\rho(x)\,dx = \frac{99/4}{12} = \frac{99}{48} = \frac{33}{16}.
\]
Enter $\bar x$:
\[
\bar x=\answer{\tfrac{33}{16}}.
\]
\end{solution}

%%%%%%%%%%%%%%%%%%%%%%%%%%%%%%%%%%%%%%%%%%%%%%%%%%%%%%%%%%%%
\section*{Economics — Revenue Maximization}

\begin{problem}
A firm's demand satisfies \(q(p)=100-2p\). Revenue is \(R(p)=p\cdot q(p)\).

Express \(R(p)\) as a function of \(p\).

Compute \(R'(p)\).

Solve \(R'(p)=0\) for the critical price.

Use the second derivative test to classify the critical point.

Compute the maximum revenue.
\end{problem}

\begin{solution}
\textbf{Revenue as a single-variable function:}
\[
R(p)=p(100-2p)=100p-2p^2.
\]
Enter \(R(p)\):
\[
R(p)=\answer{100p-2p^2}.
\]

Differentiate:
\[
R'(p)=100-4p.
\]
Enter \(R'(p)\):
\[
R'(p)=\answer{100-4p}.
\]

Solve \(R'(p)=0\Rightarrow 100-4p=0\Rightarrow p=25\).
Enter the critical price:
\[
p=\answer{25}.
\]

Second derivative:
\[
R''(p)=-4.
\]
Enter \(R''(p)\):
\[
R''(p)=\answer{-4}.
\]
\begin{multipleChoice}
  \choice{Since \(R''(25)>0\), \(p=25\) is a local minimum.}
  \choice[correct]{Since \(R''(25)<0\), \(p=25\) is a local maximum.}
  \choice{The second derivative test is inconclusive here.}
\end{multipleChoice}

Finally compute the revenue at \(p=25\):
\[
R(25)=100(25)-2(25)^2 = 2500 - 1250 = 1250.
\]
Enter the maximum revenue:
\[
R_{\max}=\answer{1250}.
\]
\end{solution}

\end{document}