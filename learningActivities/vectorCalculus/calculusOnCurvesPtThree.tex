\documentclass{ximera}

\title{Calculus on Curves: Integration and Arc Length}
\author{Zack Reed}

\begin{document}
\begin{abstract}
In this activity we explore integration of vector functions, focusing on reconstructing position from velocity and calculating arc length.
\end{abstract}
\maketitle

\section*{Integrating Vector Functions}

\subsection*{Reconstructing Position from Velocity}

In single-variable calculus, we learned that integration adds up small changes made by rate-time products $r\cdot dt$. Because of the Fundamental Theorem of Calculus, if we know the rate of change, we can find the total change amount by integrating.

\begin{problem}
Let's review: In single-variable calculus, if a rocket has velocity $v(t)$ over time interval $[a,b]$, the total change in height is:

\begin{multipleChoice}
    \choice{$v(b) - v(a)$}
    \choice[correct]{$\int_a^b v(t) \, dt$}
    \choice{$v(b) \cdot (b-a)$}
    \choice{$\frac{dv}{dt}$}
\end{multipleChoice}

\begin{center}
\youtube{Q0IChRh3dKU}
\end{center}

\begin{feedback}
We integrate velocity to find displacement! Small bits of distance $dD = v(t) \cdot dt$ add up to give total change: $\Delta D = \int_a^b v(t) \, dt$.
\end{feedback}
\end{problem}

\begin{problem}
Let's visualize this with a rocket example.

\begin{expandable}{stuff}{GeoGebra Instructions}
    Alter the ``t='' slider to view the rocket's height and velocity simultaneously. Notice how velocity determines the rate at which height accumulates.
\end{expandable}

\begin{center}
\geogebra{ysumvptx}{750}{467}
\end{center}

The rocket's velocity determines its motion, and total height accumulates over time. For instance, a height of $1090.2$ meters over $5.57$ seconds comes from: $1090.2 = \int_0^{5.57} v(t) \, dt$.

After exploring the applet, select the true statements:

\begin{enumerate}
    \item First about the velocity: 
    
    \begin{selectAll}
    \choice[correct]{The height of the velocity curve at time $t$ gives the rocket's speed at that time}
    \choice[correct]{The integral $\int_{T_1}^{T_2} v(t) \, dt$ gives the change in height between times $T_1$ and $T_2$}
    \choice{The integral $\int_{T_1}^{T_2} v(t) \, dt$ gives the length along the rocket's graph between times $T_1$ and $T_2$}
    \choice{The area under the height function from $T_1$ to $T_2$ gives the rocket's height change from $T_1$ to $T_2$}
    \choice[correct]{The area under the velocity curve from $T_1$ to $T_2$ gives the rocket's height change from $T_1$ to $T_2$}
    \choice{The velocity vector on the height function indicates the direction of motion}
    \end{selectAll}

    \item Now about the height:
    
    \begin{selectAll}
    \choice{The height graph gives the rocket's trajectory as it travels through the air}
    \choice[correct]{The height function's derivative $h'(t)$ gives the rocket's velocity at time $t$}
    \choice{The Fundamental Theorem of Calculus says that the change in velocity, $v(T_2)-v(T_1)$, equals the integral $\int_{T_1}^{T_2} h(t) \, dt$}
    \choice[correct]{The Fundamental Theorem of Calculus says that the change in height, $h(T_2)-h(T_1)$, equals the integral $\int_{T_1}^{T_2} v(t) \, dt$}
    \choice[correct]{If $\Delta$ means ``net change'', then $\Delta h = \int_{T_1}^{T_2} v(t) \, dt$}
    \end{selectAll}


\end{enumerate}

We will recreate these same calcualtions and interpretations but on vector-valued functions.


\begin{feedback}
Watch how the area under the velocity curve corresponds to the height gained. This is the fundamental theorem of calculus in action!
\end{feedback}
\end{problem}

\subsection*{Vector Integration: Component by Component}

\begin{definition}
For a vector velocity function $\vec{v}(t)=\langle x'(t), y'(t), z'(t)\rangle$, the change in position over $[a,b]$ is:
$$\vec{r}(b)-\vec{r}(a)= \int_a^b \vec{v}(t)\, dt=\left\langle \int_a^b x'(t)\, dt, \int_a^b y'(t)\, dt, \int_a^b z'(t)\, dt\right\rangle$$

We integrate each component separately!
\end{definition}

\begin{problem}
Key insight: When we integrate a vector function, what do we get?

\begin{multipleChoice}
    \choice{A scalar (number)}
    \choice[correct]{A vector (displacement)}
    \choice{The arc length}
    \choice{The speed}
\end{multipleChoice}

\begin{feedback}
Unlike scalar integration which gives a number, vector integration gives us a displacement vector showing the overall change in position!
\end{feedback}
\end{problem}

\begin{problem}
Let's practice! If $\vec{v}(t) = \langle 2, 3t, 4t^2 \rangle$, find the displacement from $t=0$ to $t=2$.

$\int_0^2 \vec{v}(t) \, dt = \left\langle \int_0^2 \answer{2} \, dt, \int_0^2 \answer{3t} \, dt, \int_0^2 \answer{4t^2} \, dt \right\rangle$

$= \langle \answer{4}, \answer{6}, \answer{32/3} \rangle$

\begin{feedback}
Remember: $\int_0^2 2 \, dt = 2t \big|_0^2 = 4$, $\int_0^2 3t \, dt = \frac{3t^2}{2} \big|_0^2 = 6$, and $\int_0^2 4t^2 \, dt = \frac{4t^3}{3} \big|_0^2 = \frac{32}{3}$.
\end{feedback}
\end{problem}

\begin{problem}
Now visualize this process in 3D.

\begin{expandable}{stuff}{GeoGebra Instructions}
    Alter the ``T='' slider to watch $\vec{r}$ build progressively by integrating each component. Observe how each coordinate changes independently.
\end{expandable}

\begin{center}
\geogebra{excy8qtq}{730}{510}
\end{center}

After exploring, select the true statements:
\begin{selectAll}
    \choice[correct]{Each component is integrated separately}
    \choice[correct]{The result is a displacement vector}
    \choice{The result tells us the arc length traveled}
    \choice[correct]{The process reconstructs position from velocity}
\end{selectAll}
\end{problem}

\section*{Arc Length}

\subsection*{Distance vs. Displacement}

While integrating components gives us displacement (a vector), we often want to know the actual distance traveled along the curve—the arc length.

\begin{problem}
Think about the difference between displacement and distance:

If you walk 3 meters east, then 4 meters north, your displacement is:
\begin{multipleChoice}
    \choice{7 meters}
    \choice[correct]{5 meters (by Pythagorean theorem)}
    \choice{3 meters}
    \choice{4 meters}
\end{multipleChoice}

But the actual distance you walked is:
\begin{multipleChoice}
    \choice{5 meters}
    \choice[correct]{7 meters}
    \choice{3 meters}
    \choice{4 meters}
\end{multipleChoice}

\begin{feedback}
Displacement measures the straight-line change from start to finish. Distance measures the actual path length traveled. For curves, we use arc length to measure distance!
\end{feedback}
\end{problem}

\subsection*{Building the Arc Length Formula}

We've already learned about arc length in Calc II! Now we extend it to higher dimensions using the Pythagorean theorem.

\begin{problem}
Let's review the 2D case first. Explore the visualization below.

\begin{expandable}{stuff}{GeoGebra Instructions}
    Alter the ``x='' slider to see how small segments of the curve can be approximated by straight line segments. Notice the Pythagorean calculation.
\end{expandable}

\begin{center}
\geogebra{pd8zaw8v}{786}{584}
\end{center}

After exploring, what represents a small bit of arc length?
\begin{multipleChoice}
    \choice{$dx + dy$}
    \choice[correct]{$ds = \sqrt{(dx)^2 + (dy)^2}$}
    \choice{$dx \cdot dy$}
    \choice{$\frac{dy}{dx}$}
\end{multipleChoice}

\begin{feedback}
The Pythagorean theorem gives us the length of the small segment: $ds = \sqrt{(dx)^2 + (dy)^2}$!
\end{feedback}
\end{problem}

\begin{definition}
For a 2D curve with coordinate functions $x(t)$ and $y(t)$, a small arc length element is:
$$ds=\sqrt{\left(\frac{dx}{dt}\right)^2+\left(\frac{dy}{dt}\right)^2} \, dt=\sqrt{x'(t)^2+y'(t)^2} \, dt = |\vec{v}(t)| \, dt$$

The total arc length from $t=a$ to $t=b$ is:
$$L = \int_a^b ds = \int_a^b |\vec{v}(t)| \, dt = \int_a^b \sqrt{x'(t)^2+y'(t)^2} \, dt$$
\end{definition}

\begin{problem}
Key insight: What's the difference between $\int_a^b \vec{v}(t) \, dt$ and $\int_a^b |\vec{v}(t)| \, dt$?

$\int_a^b \vec{v}(t) \, dt$ gives:
\begin{multipleChoice}
    \choice{Arc length}
    \choice[correct]{Displacement (a vector)}
    \choice{Speed}
    \choice{Distance}
\end{multipleChoice}

$\int_a^b |\vec{v}(t)| \, dt$ gives:
\begin{multipleChoice}
    \choice{Displacement (a vector)}
    \choice[correct]{Arc length (distance traveled)}
    \choice{Velocity}
    \choice{Acceleration}
\end{multipleChoice}

\begin{feedback}
Without the magnitude bars, we integrate the vector to get displacement. With the magnitude bars, we integrate speed to get total distance (arc length)!
\end{feedback}
\end{problem}

\subsection*{Arc Length in 3D}

Now let's extend this idea to 3D curves.

\begin{center}
\youtube{QjahEvlCD8E}
\end{center}

\begin{definition}
For a 3D curve $\vec{r}(t) = \langle x(t), y(t), z(t) \rangle$:
$$L = \int_a^b |\vec{v}(t)| \, dt = \int_a^b \sqrt{x'(t)^2+y'(t)^2+z'(t)^2} \, dt$$
\end{definition}

\begin{problem}
After exploring the following applet, what statements are true about arc length in 3D (Select All that Apply):

\begin{expandable}{stuff}{GeoGebra Instructions}
    Alter the ``t='' slider to see the arc length approximation at various points. Click, drag, and scroll to rotate and zoom. Select ``Reset Zoom'' or ``Zoom to Arc Length'' buttons for different views. Check ``Show x-y hypotenuse'' for more detail.
\end{expandable}

\begin{center}
\geogebra{nrnm8k7p}{780}{438}
\end{center}

\begin{selectAll}
    \choice[correct]{Small arc length elements use the Pythagorean theorem in 3D}
    \choice[correct]{We integrate speed over time to get arc length}
    \choice{Arc length is always equal to displacement magnitude}
    \choice[correct]{Arc length measures actual distance traveled along the curve}
\end{selectAll}
\end{problem}

\begin{problem}
Let's compute an arc length! For $\vec{r}(t) = \langle 3t, 4t, 0 \rangle$ from $t=0$ to $t=2$:

First, find $\vec{v}(t) = \langle \answer{3}, \answer{4}, \answer{0} \rangle$

The speed is $|\vec{v}(t)| = \sqrt{9 + 16 + 0} = \answer{5}$

Since speed is constant, the arc length is simply:
$L = \int_0^2 5 \, dt = \answer{10}$

\begin{feedback}
When speed is constant, arc length is just speed times time! This curve is actually a straight line in the $xy$-plane moving at constant speed.
\end{feedback}
\end{problem}

\begin{problem}
Now for the helix $\vec{r}(t) = \langle \cos(t), \sin(t), t \rangle$ from $t=0$ to $t=2\pi$:

$\vec{v}(t) = \langle \answer{-\sin(t)}, \answer{\cos(t)}, \answer{1} \rangle$

$|\vec{v}(t)| = \sqrt{\sin^2(t) + \cos^2(t) + 1} = \answer{\sqrt{2}}$

The arc length for one complete turn is:
$L = \int_0^{2\pi} \sqrt{2} \, dt = \answer{2\pi\sqrt{2}}$ (or $\answer[tolerance=0.1]{8.886}$)

\begin{feedback}
The helix also has constant speed! Even though it's curving and rising, the total speed $\sqrt{2}$ remains constant.
\end{feedback}
\end{problem}



\section*{A Comprehensive Example}

First, let's check in with Torty and Harry one more time to examine how far the racers traveled along the track.

\begin{center}
\youtube{rO7p2Vel8-o}
\end{center}

\begin{problem}
Let's walk through a comprehensive example. Consider $\vec{r}(t) = \langle t, t^3/3, \sqrt{2}t^2/2 \rangle$ from $t=0$ to $t=1$.

\textbf{Part A: Find the displacement}

$\int_0^1 \vec{v}(t) \, dt = \int_0^1 \langle \answer{1}, \answer{t^2}, \answer{\sqrt{2}t} \rangle \, dt$
$= \langle \answer{1}, \answer{1/3}, \answer{\sqrt{2}/2} \rangle$

\textbf{Part B: Find the arc length}

$|\vec{v}(t)| = \sqrt{1 + t^4 + 2t^2} = \sqrt{\answer{1 + 2t^2 + t^4}}$

$L = \int_0^1 \sqrt{1 + 2t^2 + t^4} \, dt = \int_0^1 (t^2 + 1) \, dt = \left.\left(\frac{t^3}{3} + t\right)\right|_0^1 = \answer{4/3} \approx 1.33$

\textbf{Part C: Compare}

The magnitude of the displacement is $|\langle 1, 1/3, \sqrt{2}/2 \rangle| = \sqrt{1+(1/3)^2+(\sqrt{2}/2)^2} = \answer{\sqrt{1 + 1/9 + 1/2}} \approx 1.15$
Notice that the arc length ($\approx 1.33$) is slightly larger than the displacement magnitude ($\approx 1.15$). This makes sense because:
\begin{multipleChoice}
    \choice{Arc length should always be smaller}
    \choice[correct]{The curve isn't perfectly straight, so the path is slightly longer}
    \choice{We made a calculation error}
    \choice{They should be exactly equal}
\end{multipleChoice}

\begin{feedback}
The displacement gives the straight-line distance from start to finish. The arc length gives the actual distance traveled along the (slightly curved) path. The arc length is always greater than or equal to the displacement magnitude!
\end{feedback}
\end{problem}


%%%% DO TORTY AND HARRY STUFF %%%%

\end{document}