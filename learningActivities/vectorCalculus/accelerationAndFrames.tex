\documentclass{ximera}

\title{Activity: Curvature, Acceleration, and TNB Frames}
\author{Zack Reed}

\begin{document}
\begin{abstract}
In this activity we explore how curves bend in space through curvature, develop the TNB frame for analyzing motion, and understand acceleration in moving reference frames.
\end{abstract}
\maketitle

\section*{Introduction: From Single-Variable to Vector Calculus}

We are continuing our study on how curves turn and bend. Let's first connect this to ideas you've already mastered in single-variable calculus.

\subsection*{Review: Concavity and Second Derivatives}

We first examined bending in MATH 241 when we considered a function's second derivative. Thinking about the graph of a function $f$:
\begin{itemize}
    \item The first derivative $f'$ tells us the slopes of tangent lines (whether $f$ increases or decreases)
    \item The second derivative $f''$ tells us about concavity (how the function bends)
\end{itemize}

If $f''(x)>0$, the tangent line slopes increase near $x$ (concave up). If $f''(x)<0$, the tangent line slopes decrease near $x$ (concave down).

\begin{problem}
Before exploring the GeoGebra applet below, let's check your understanding of concavity.

For a function $f$ where $f''(x)  > 0$, what happens to the tangent line slopes as you move from left to right?
\begin{multipleChoice}
    \choice[correct]{The slopes increase (lines get steeper upward)}
    \choice{The slopes decrease (lines get less steep)}
    \choice{The slopes remain constant}
    \choice{The slopes alternate between increasing and decreasing}
\end{multipleChoice}

\begin{feedback}
When $f''(x)  > 0$, we say $f$ is concave up. This means the tangent lines are rotating counterclockwise as you move left to right, so their slopes increase.
\end{feedback}
\end{problem}

\begin{problem}
Now explore the interactive visualization below to see concavity in action.

\begin{expandable}{stuff}{GeoGebra Instructions}
    Read the text on screen and advance the animation by clicking the buttons (in red or black text) to view new text, tangent lines, and portions of the graph. Use the reset button at the bottom left to start over.
\end{expandable}

\begin{center}
\geogebra{uervaqau}{757}{611}
\end{center}

After exploring the applet, identify what you observed:
\begin{selectAll}
    \choice[correct]{When $f''(x)  > 0$, the graph bends upward}
    \choice[correct]{When $f''(x) < 0$, the graph bends downward}
    \choice[correct]{Tangent line slopes change as you move along the curve}
    \choice{The second derivative measures the slope of the tangent line}
\end{selectAll}

\begin{feedback}
The second derivative measures how the first derivative (slope) changes, not the slope itself. This rate of change of slope is what we call concavity.
\end{feedback}
\end{problem}

\section*{Task One: Extending to Space Curves}

\subsection*{The Challenge: Measuring Bending in Space}

When we move from graphs of functions to curves in space, we face a new challenge. In single-variable calculus, we only needed to worry about upward or downward bending. But curves in 3D space can bend in any direction!

\begin{problem}
Consider a particle moving along a curve $\mathbf{r}(t) = \langle \cos(t), \sin(t), t \rangle$ (a helix). As the particle moves, what aspects of its motion might we want to measure?

\begin{selectAll}
    \choice[correct]{The direction the particle is heading (tangent direction)}
    \choice[correct]{How sharply the curve is turning}
    \choice[correct]{The direction in which the curve is bending}
    \choice{Only the particle's speed}
\end{selectAll}

\begin{feedback}
Just like in single-variable calculus, we care about both the direction of motion (first derivative idea) and how the curve bends (second derivative idea). But now these concepts are vectors in space!
\end{feedback}
\end{problem}

\subsection*{Why Arc Length?}

As a particle moves along a smooth curve in the plane or space, we want to define the curvature of the curve independent of how fast the particle is moving. A particle could race along a gentle curve or crawl along a sharp turn—the geometry of the curve itself doesn't change.

\begin{problem}
Why should we use arc length $s$ instead of time $t$ to measure curvature?

\begin{multipleChoice}
    \choice{Arc length is easier to calculate than time}
    \choice[correct]{Arc length measures the geometry of the curve independent of speed}
    \choice{Time doesn't exist for curves in space}
    \choice{Arc length makes the math simpler}
\end{multipleChoice}

\begin{feedback}
Arc length parameterization gives us a "speed-independent" way to describe curves. Two particles moving at different speeds along the same curve experience the same curvature at each point—curvature is a geometric property of the curve itself, not of the motion.
\end{feedback}
\end{problem}

\subsection*{Building the Unit Tangent Vector}

We first define a speed-independent tangent vector as $\mathbf{T}=\frac{d\mathbf{r}}{ds}$. Using the chain rule, we can relate this to our time-based parametrization:

\[\mathbf{T}=\frac{d\mathbf{r}}{ds}=\frac{d\mathbf{r}}{dt}\frac{dt}{ds}=\frac{\vec{v}}{|\vec{v}|}\]

\begin{problem}
Let's understand what $\mathbf{T}$ represents. If $\mathbf{r}(t) = \langle 2t, 3t^2, t^3 \rangle$, compute $\mathbf{T}$ at $t=1$.

First, find $\vec{v} = \mathbf{r}'(t) = \langle \answer{2}, \answer{6t}, \answer{3t^2} \rangle$

At $t=1$: $\vec{v}(1) = \langle \answer{2}, \answer{6}, \answer{3} \rangle$

The magnitude is $|\vec{v}(1)| = \sqrt{4 + 36 + 9} = \answer{7}$

Therefore, $\mathbf{T}(1) = \langle \answer{2/7}, \answer{6/7}, \answer{3/7} \rangle$

\begin{feedback}
The unit tangent vector $\mathbf{T}$ always has length 1 and points in the direction of motion. Since it's a unit vector, its length remains constant and only its direction changes as the particle moves along the curve.
\end{feedback}
\end{problem}

\section*{Task Two: Curvature and the Normal Vector}

\subsection*{Measuring How the Curve Bends}

If we differentiate $\mathbf{T}$ with respect to arc length, we measure how the tangent vectors change as you move along the curve. This is analogous to the concavity measurements from $f''$ in single-variable calculus.

\begin{definition}
The derivative $\frac{d\mathbf{T}}{ds}$ tells us:
\begin{itemize}
    \item The \textbf{direction} in which the curve bends
    \item The \textbf{amount} the curve is bending
\end{itemize}

We capture these with two key quantities:
\begin{itemize}
    \item \textbf{Curvature:} $\kappa=\left|\frac{d\mathbf{T}}{ds}\right|$ (how much the curve bends)
    \item \textbf{Normal vector:} $\mathbf{N}=\frac{1}{\kappa}\frac{d\mathbf{T}}{ds}$ (direction of bending)
\end{itemize}
\end{definition}

\begin{problem}
Before diving into calculations, let's build intuition. Which of the following curves would have the largest curvature?

\begin{multipleChoice}
    \choice{A straight line}
    \choice{A circle with radius 10 units}
    \choice[correct]{A circle with radius 1 unit}
    \choice{A gentle wave}
\end{multipleChoice}

\begin{feedback}
Curvature measures how sharply a curve bends. A smaller circle bends more sharply than a larger circle, so it has greater curvature. A straight line doesn't bend at all, so its curvature is zero.

In fact, for a circle of radius $r$, the curvature is $\kappa = \frac{1}{r}$. Smaller radius means larger curvature!
\end{feedback}
\end{problem}

\begin{problem}
Let's compute curvature for a simple example. Consider a circle of radius $r$ parameterized by $\mathbf{r}(t) = \langle r\cos(t), r\sin(t) \rangle$.

First, find the velocity: $\vec{v}(t) = \langle \answer{-r\sin(t)}, \answer{r\cos(t)} \rangle$

The speed is $|\vec{v}(t)| = \answer{r}$

The unit tangent vector is $\mathbf{T}(t) = \langle \answer{-\sin(t)}, \answer{\cos(t)} \rangle$

Now differentiate with respect to $t$: $\frac{d\mathbf{T}}{dt} = \langle \answer{-\cos(t)}, \answer{-\sin(t)} \rangle$

To find $\frac{d\mathbf{T}}{ds}$, we use: $\frac{d\mathbf{T}}{ds} = \frac{d\mathbf{T}}{dt} \cdot \frac{dt}{ds} = \frac{1}{|\vec{v}|} \frac{d\mathbf{T}}{dt}$

Therefore: $\frac{d\mathbf{T}}{ds} = \frac{1}{r}\langle -\cos(t), -\sin(t) \rangle$

The curvature is $\kappa = \left|\frac{d\mathbf{T}}{ds}\right| = \frac{1}{r}\sqrt{\cos^2(t) + \sin^2(t)} = \answer{1/r}$

\begin{feedback}
Notice that the curvature of a circle is constant everywhere on the circle and equals $\frac{1}{r}$. This confirms our intuition: smaller circles bend more sharply (higher curvature).
\end{feedback}
\end{problem}

\subsection*{The Principal Unit Normal Vector}

The normal vector $\mathbf{N}$ points in the direction the curve is bending. Since $\mathbf{T}$ is a unit vector, $\mathbf{N}$ is perpendicular to $\mathbf{T}$.

\begin{problem}
For the circle example above, find the principal unit normal vector at a general point.

We have $\frac{d\mathbf{T}}{ds} = \frac{1}{r}\langle -\cos(t), -\sin(t) \rangle$ and $\kappa = \frac{1}{r}$

The normal vector is: $\mathbf{N} = \frac{1}{\kappa}\frac{d\mathbf{T}}{ds} = \langle \answer{-\cos(t)}, \answer{-\sin(t)} \rangle$

\begin{feedback}
Notice that $\mathbf{N}$ points toward the center of the circle! This makes geometric sense: the circle is bending toward its center at every point.

You can verify that $\mathbf{T} \cdot \mathbf{N} = (-\sin(t))(-\cos(t)) + (\cos(t))(-\sin(t)) = 0$, confirming they are perpendicular.
\end{feedback}
\end{problem}

\section*{Task Three: The TNB Frame}

\subsection*{Building a Moving Coordinate System}

For curves in 3D space, we can construct a complete coordinate system that moves along with a particle. This is called the \textbf{TNB frame} or \textbf{Frenet frame}.

\begin{definition}
The TNB frame consists of three mutually perpendicular unit vectors:
\begin{itemize}
    \item $\mathbf{T}$: Unit tangent vector (direction of motion)
    \item $\mathbf{N}$: Principal unit normal vector (direction of bending)
    \item $\mathbf{B}$: Binormal vector, defined as $\mathbf{B} = \mathbf{T} \times \mathbf{N}$
\end{itemize}
\end{definition}

\begin{problem}
What properties should the binormal vector $\mathbf{B}$ have?

\begin{selectAll}
    \choice[correct]{$\mathbf{B}$ is perpendicular to $\mathbf{T}$}
    \choice[correct]{$\mathbf{B}$ is perpendicular to $\mathbf{N}$}
    \choice[correct]{$\mathbf{B}$ is a unit vector}
    \choice{$\mathbf{B}$ points in the direction of motion}
\end{selectAll}

\begin{feedback}
Since $\mathbf{B} = \mathbf{T} \times \mathbf{N}$ and both $\mathbf{T}$ and $\mathbf{N}$ are unit vectors that are perpendicular to each other, $\mathbf{B}$ is automatically perpendicular to both and has length 1.

The binormal vector points in the direction perpendicular to the plane containing $\mathbf{T}$ and $\mathbf{N}$ (called the osculating plane).
\end{feedback}
\end{problem}

\begin{problem}
For the helix $\mathbf{r}(t) = \langle \cos(t), \sin(t), t \rangle$ at $t = 0$:

First, $\vec{v}(0) = \langle \answer{-\sin(0)}, \answer{\cos(0)}, \answer{1} \rangle = \langle 0, 1, 1 \rangle$

The speed is $|\vec{v}(0)| = \answer{\sqrt{2}}$

So $\mathbf{T}(0) = \langle \answer{0}, \answer{1/\sqrt{2}}, \answer{1/\sqrt{2}} \rangle$

(For this problem, you can assume that $\mathbf{N}(0) = \langle -1, 0, 0 \rangle$)

The binormal vector is:
\[\mathbf{B}(0) = \mathbf{T}(0) \times \mathbf{N}(0) = \begin{vmatrix} \mathbf{i} & \mathbf{j} & \mathbf{k} \\ 0 & 1/\sqrt{2} & 1/\sqrt{2} \\ -1 & 0 & 0 \end{vmatrix}\]

$\mathbf{B}(0) = \langle \answer{0}, \answer{1/\sqrt{2}}, \answer{-1/\sqrt{2}} \rangle$

\begin{feedback}
The TNB frame gives us a coordinate system that travels along with the particle. At each point, $\mathbf{T}$ points forward, $\mathbf{N}$ points toward where the curve is bending, and $\mathbf{B}$ points "out of the plane" of the curve.
\end{feedback}
\end{problem}

\section*{Task Four: Acceleration in the TNB Frame}

\subsection*{Decomposing Acceleration}

One of the most powerful applications of the TNB frame is understanding acceleration. We can decompose the acceleration vector into components along $\mathbf{T}$ and $\mathbf{N}$.

\begin{definition}
The acceleration vector can be written as:
\[\mathbf{a} = a_T \mathbf{T} + a_N \mathbf{N}\]

where:
\begin{itemize}
    \item $a_T = \frac{d|\vec{v}|}{dt}$ is the \textbf{tangential component} (change in speed)
    \item $a_N = \kappa |\vec{v}|^2$ is the \textbf{normal component} (change in direction)
\end{itemize}
\end{definition}

\begin{problem}
Let's build intuition about these components. When a car goes around a curve at constant speed:

What is the tangential component of acceleration?
\begin{multipleChoice}
    \choice[correct]{Zero (speed isn't changing)}
    \choice{Positive (car is speeding up)}
    \choice{Negative (car is slowing down)}
\end{multipleChoice}

What about the normal component?
\begin{multipleChoice}
    \choice{Zero (no acceleration)}
    \choice[correct]{Non-zero (direction is changing)}
    \choice{Undefined}
\end{multipleChoice}

\begin{feedback}
When speed is constant, $a_T = 0$. But if the car is turning (changing direction), there must be acceleration toward the center of the turn. This is the normal component $a_N$, which depends on both the curvature of the path and the speed.

This is why you feel pushed to the side when a car turns—that's the normal acceleration!
\end{feedback}
\end{problem}

\begin{problem}
Consider a particle moving along $\mathbf{r}(t) = \langle t, t^2 \rangle$ at $t = 1$.

Velocity: $\vec{v}(t) = \langle \answer{1}, \answer{2t} \rangle$, so $\vec{v}(1) = \langle 1, 2 \rangle$

Speed: $|\vec{v}(1)| = \answer{\sqrt{5}}$

Acceleration: $\mathbf{a}(t) = \langle \answer{0}, \answer{2} \rangle$, so $\mathbf{a}(1) = \langle 0, 2 \rangle$

To find $a_T$, we compute $\frac{d|\vec{v}|}{dt}$:
\[|\vec{v}(t)| = \sqrt{1 + 4t^2}\]
\[\frac{d|\vec{v}|}{dt} = \frac{4t}{\sqrt{1+4t^2}}\]

At $t=1$: $a_T = \frac{4}{\sqrt{5}} = \answer[tolerance=0.01]{1.789}$

To find $a_N$, we can use: $|\mathbf{a}|^2 = a_T^2 + a_N^2$

$|\mathbf{a}(1)| = \answer{2}$

$a_N^2 = 4 - \frac{16}{5} = \frac{4}{5}$

$a_N = \answer[tolerance=0.01]{0.894}$

\begin{feedback}
The tangential component tells us the particle is speeding up (positive $a_T$). The normal component tells us the path is curving. Together, these give us complete information about the acceleration.
\end{feedback}
\end{problem}

\section*{Summary and Key Takeaways}

\begin{problem}
Let's review the key concepts. Match each quantity to its meaning:

The curvature $\kappa$ measures:
\begin{multipleChoice}
    \choice{How fast a particle is moving}
    \choice[correct]{How sharply a curve is bending}
    \choice{The direction of motion}
    \choice{The speed of the particle}
\end{multipleChoice}

The unit tangent vector $\mathbf{T}$ points:
\begin{multipleChoice}
    \choice{Toward the center of curvature}
    \choice[correct]{In the direction of motion}
    \choice{Perpendicular to the curve}
    \choice{Opposite to the velocity}
\end{multipleChoice}

The principal unit normal $\mathbf{N}$ points:
\begin{multipleChoice}
    \choice{In the direction of motion}
    \choice[correct]{In the direction the curve is bending}
    \choice{Perpendicular to the osculating plane}
    \choice{Opposite to acceleration}
\end{multipleChoice}

The tangential acceleration component $a_T$ measures:
\begin{multipleChoice}
    \choice{Change in direction}
    \choice[correct]{Change in speed}
    \choice{Curvature of the path}
    \choice{Distance traveled}
\end{multipleChoice}
\end{problem}

\begin{problem}
True or False: Select all TRUE statements.

\begin{selectAll}
    \choice[correct]{A straight line has zero curvature}
    \choice[correct]{The TNB frame provides a moving coordinate system along a curve}
    \choice{The binormal vector is always constant along a curve}
    \choice[correct]{Normal acceleration exists whenever direction is changing}
    \choice[correct]{At constant speed, tangential acceleration is zero}
    \choice{Curvature depends on how fast a particle moves along the curve}
\end{selectAll}

\begin{feedback}
Great work! Remember that curvature is a geometric property of the curve itself, independent of how fast something moves along it. The TNB frame captures the geometry of spatial curves, extending the ideas of derivatives and concavity from single-variable calculus to three dimensions.
\end{feedback}
\end{problem}

\end{document}