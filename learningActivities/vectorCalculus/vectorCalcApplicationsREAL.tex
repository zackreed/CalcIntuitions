\documentclass{ximera}
% \input{../../xmpreamble.tex}

\title{Vector Calculus Applications}
\author{Zack Reed}

\begin{document}
\begin{abstract}
In this activity, we apply vector calculus concepts to real-world problems, showing how to treat instances of vector functions as vectors and use them for calculations such as area, distances, and projections.
\end{abstract}
\maketitle

\section*{Main Idea: Treating Vector Functions as Vectors}

In previous activities, we've learned how vectors represent quantities with both magnitude and direction, and how vector arithmetic, dot products, and cross products allow us to combine and analyze these quantities. Now, we'll see how vector functions—functions that output vectors—can be used in calculations just like individual vectors. This opens up a wide range of applications, including finding areas, distances, and projections in physical and geometric contexts.

\section*{Applications: Calculations with Vector Functions}

\subsection*{Example 1: Distance from a Point to a Plane}
Suppose we have a point $P = [2, 3, 4]$ and a plane given by $2x - y + 2z = 7$. Find the distance from $P$ to the plane.

\begin{problem}
The distance from $P$ to the plane is given by:
\[
d = \frac{|2(2) - 3 + 2(4) - 7|}{\sqrt{2^2 + (-1)^2 + 2^2}} = \answer{2}
\]
\begin{feedback}
Recall the formula for the distance from a point $[x_0, y_0, z_0]$ to the plane $ax + by + cz = d$:
\[
d = \frac{|ax_0 + by_0 + cz_0 - d|}{\sqrt{a^2 + b^2 + c^2}}
\]
\end{feedback}
\end{problem}

\subsection*{Example 2: Area Enclosed by a Vector Function}
Suppose a curve $\vec{r}(t) = [\cos t, \sin t]$ for $t$ in $[0, 2\pi]$ traces out a circle. Find the area enclosed by the curve.

\begin{problem}
The area enclosed by the curve is:
\[
A = \int_0^{2\pi} \frac{1}{2} |\vec{r}(t) \times \vec{r}'(t)| dt = \answer{\pi}
\]
\begin{feedback}
For a planar curve, the area can be found using the cross product of $\vec{r}(t)$ and its derivative $\vec{r}'(t)$:
\[
A = \frac{1}{2} \int |\vec{r}(t) \times \vec{r}'(t)| dt
\]
For the unit circle, this gives $\pi$.
\end{feedback}
\end{problem}

\subsection*{Example 3: Work Done by a Force Along a Path}
Suppose a force $\vec{F}(t) = [2t, 3]$ acts on a particle moving along $\vec{r}(t) = [t, t^2]$ for $t$ in $[0, 1]$. Find the total work done.

\begin{problem}
The work done is:
\[
W = \int_0^1 \vec{F}(t) \cdot \vec{r}'(t) dt = \answer{4}
\]
\begin{feedback}
Recall that work is the dot product of force and displacement:
\[
W = \int \vec{F}(t) \cdot \vec{r}'(t) dt
\]
For $\vec{r}'(t) = [1, 2t]$, $\vec{F}(t) \cdot \vec{r}'(t) = 2t \cdot 1 + 3 \cdot 2t = 2t + 6t = 8t$. Integrate $8t$ from $0$ to $1$ to get $4$.
\end{feedback}
\end{problem}

\section*{Practice: Calculations with Vector Functions}

\subsection*{Problem 1: Vector Projection}
Let $\vec{a} = [3, 4]$ and $\vec{b} = [5, 0]$. Find the projection of $\vec{a}$ onto $\vec{b}$.

\begin{problem}
The projection is:
\[
\text{proj}_{\vec{b}}(\vec{a}) = \frac{\vec{a} \cdot \vec{b}}{||\vec{b}||^2} \vec{b} = \answer{[3, 0]}
\]
\begin{feedback}
The projection formula is:
\[
\text{proj}_{\vec{b}}(\vec{a}) = \frac{\vec{a} \cdot \vec{b}}{||\vec{b}||^2} \vec{b}
\]
\end{feedback}
\end{problem}

\subsection*{Problem 2: Cross Product and Area}
Let $\vec{u} = [1, 2, 3]$ and $\vec{v} = [4, 5, 6]$. Compute $\vec{u} \times \vec{v}$ and its magnitude.

\begin{problem}
$\vec{u} \times \vec{v} = [\answer{-3}, \answer{6}, \answer{-3}]$ with magnitude $|\vec{u} \times \vec{v}| = \answer[tolerance=.1]{7.35}$
\begin{feedback}
Use the cross product formula:
\[
\vec{u} \times \vec{v} = [u_2 v_3 - u_3 v_2, u_3 v_1 - u_1 v_3, u_1 v_2 - u_2 v_1]
\]
\end{feedback}
\end{problem}

\subsection*{Problem 3: Dot Product and Angle}
Let $\vec{u} = [1, 0]$ and $\vec{v} = [0, 1]$. What is the dot product and the angle between them?

\begin{problem}
$\vec{u} \cdot \vec{v} = \answer{0}$, so the angle is $\answer{90}$ degrees.
\begin{feedback}
The dot product is zero, so the vectors are orthogonal (perpendicular).
\end{feedback}
\end{problem}

\section*{Challenge: Vector Calculus in Context}

\subsection*{Problem 4: Distance Between Skew Lines}
Find the distance between the lines $\vec{r}_1(t) = [1, 2, 3] + t[1, 0, -1]$ and $\vec{r}_2(s) = [2, 0, 1] + s[0, 1, 2]$.

\begin{problem}
The distance between the lines is:
\[
d = \frac{|([2,0,1]-[1,2,3]) \cdot ([1,0,-1] \times [0,1,2])|}{||[1,0,-1] \times [0,1,2]||} = \answer{3}
\]
\begin{feedback}
The formula for the distance between skew lines uses the cross product of their direction vectors and the vector between points on the lines.
\end{feedback}
\end{problem}

\end{document}