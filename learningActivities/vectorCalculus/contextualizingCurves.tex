\documentclass{ximera}

\title{Activity: Torty and Harry Rematch}
\author{Zack Reed}

\begin{document}
\begin{abstract}
This activity focuses on applying vector calculus concepts including arc length, curvature, speed, acceleration, and partial derivatives in the context of a race between Torty and Harry on a hilly track.
\end{abstract}
\maketitle

\section*{Introduction}

The goal of this activity is to not just hone your calculative skills related to vector and multivariable functions but to engage you in interpretation as well and to build some intuition about what the various differential and integral calculations mean in vector contexts.

This activity focuses on a race between Torty and Harry. Harry decided that he could gain an advantage by racing along a hilly region in the mountains. The hilly region and track are visualized in the provided GeoGebra applet.

Harry has become quite adjusted to racing along inclines and declines. His path in the provided GeoGebra applet is defined by the curve $h(t)=\langle \cos(t), \sin(t), \cos(t)\sin^2(t)\rangle$, for which two laps around the course takes $4 \pi$ seconds.

In their rematch, Torty is less adjusted to running along hilly regions and tires after bursts of speed. His path in the provided GeoGebra applet is defined by the curve $\tau (t)=\langle \cos(t+\sin(t)), \sin(t+\sin(t)), \cos(t+\sin(t))\sin^2(t+\sin(t))\rangle$.

\begin{expandable}{stuff}{GeoGebra Instructions}
    You may alter the ``t='' slider to view Torty and Harry's positions, velocities, and accelerations at any time during the race. You may also use the ``Show Torty's Curve'' and ``Show Harry's Curve'' checkboxes to view either of the racer's curve in isolation.
\end{expandable}

\begin{center}
\geogebra{yzxdk5uw}{844}{629}
\end{center}

\subsection*{Getting Started: Understanding the Race}

Before we dive into calculations, let's build intuition about this race by exploring the GeoGebra applet.

\begin{problem}
    Using the GeoGebra applet, observe the race and answer the following questions based on your intuition:
    
    \begin{enumerate}
        \item At time $t=0$, where are both racers located?
        \begin{multipleChoice}
            \choice[correct]{Both start at approximately $(1,0,0)$}
            \choice{Torty starts ahead of Harry}
            \choice{Harry starts ahead of Torty}
            \choice{Both start at the origin}
        \end{multipleChoice}
        
        \item As you move the slider forward, what do you observe about their paths?
        \begin{selectAll}
            \choice[correct]{Both racers follow similar circular paths}
            \choice[correct]{The paths include vertical (z-direction) changes}
            \choice{The paths are perfect circles in a flat plane}
            \choice[correct]{Torty's path appears slightly different from Harry's}
        \end{selectAll}
        
        \item Based on visual observation, who appears to complete one lap first?
        \begin{multipleChoice}
            \choice[correct]{Harry completes a lap first}
            \choice{Torty completes a lap first}
            \choice{They complete laps at the same time}
        \end{multipleChoice}
    \end{enumerate}
    
    \begin{feedback}
        Use the slider to trace through the race. Pay attention to when each racer returns to their starting position. Harry's curve is smoother, while Torty's curve includes periods where he slows down and speeds up (due to the $\sin(t)$ term added to his parameter).
    \end{feedback}
\end{problem}

\section*{Task One: Analyzing the Race Track}

\subsection*{Understanding Arc Length}

The length of a curve in space is fundamental to understanding this race. Let's explore this concept.

\begin{problem}
    Consider Harry's path $h(t)=\langle \cos(t), \sin(t), \cos(t)\sin^2(t)\rangle$ for one complete lap from $t=0$ to $t=2\pi$.
    
    First, let's think about what affects the arc length of a curve:
    \begin{multipleChoice}
        \choice{Only the horizontal ($x$ and $y$) components matter for arc length}
        \choice[correct]{All three components ($x$, $y$, and $z$) contribute to arc length}
        \choice{Only the $z$ component affects arc length}
    \end{multipleChoice}
    
    \begin{feedback}
        Arc length in 3D space depends on all three dimensions. The formula for arc length is $L=\int_a^b ||h'(t)||dt=\int_a^b\sqrt{(x'(t))^2+(y'(t))^2+(z'(t))^2}dt$.
    \end{feedback}
\end{problem}

\begin{problem}
    To determine the physical length of the racetrack, can we use only one of the racers' curves?
    \begin{multipleChoice}
        \choice[correct]{Yes, either curve gives the track length since they follow the same physical track}
        \choice{No, we must average both curves to get the true length}
        \choice{No, we must use both curves and take the longer one}
    \end{multipleChoice}
    
    \begin{feedback}
        Both racers follow the same physical track in space - they just traverse it at different rates (parametrized differently). Computing the arc length of either curve for one complete lap gives the track length.
        
        Using computational tools (MATLAB, Python, etc.), you should find that one lap of the track is approximately $\answer[tolerance=0.1]{6.4}$ units long.
    \end{feedback}
\end{problem}

\subsection*{Speed and Velocity}

Now let's understand the difference between the racers' speeds.

\begin{problem}
    The velocity vector of Harry at time $t$ is given by $h'(t)=\frac{d}{dt}\langle \cos(t), \sin(t), \cos(t)\sin^2(t)\rangle$.
    
    What is $h'(t)$?
    \[h'(t)=\langle \answer{-\sin(t)}, \answer{\cos(t)}, \answer{-\sin^3(t)+2\cos^2(t)\sin(t)}\rangle\]
    
    The speed of Harry at time $t$ is the magnitude of the velocity vector: $||h'(t)||=\sqrt{(\sin(t))^2+(\cos(t))^2+(-\sin^3(t)+2\cos^2(t)\sin(t))^2}$.
    
    \begin{feedback}
        Remember that velocity is the derivative of position, and speed is the magnitude of velocity. Speed tells us how fast the racer is moving, while velocity also tells us the direction.
    \end{feedback}
\end{problem}

\begin{problem}
    Who wins the race depicted in the GeoGebra applet (two complete laps, ending at $t=4\pi$)?
    
    Based on your observation of the applet:
    \begin{multipleChoice}
        \choice[correct]{Harry wins the race}
        \choice{Torty wins the race}
        \choice{They tie}
    \end{multipleChoice}
    
    To verify this mathematically, we need to find when each racer completes two laps. Harry completes two laps at $t=\answer{4\pi}$ seconds (by definition).
    
    For Torty, we need to solve for when $\tau(t)$ returns to the starting position after two complete laps. Due to the $t+\sin(t)$ parametrization, Torty takes longer than $4\pi$ seconds.
    
    \begin{feedback}
        The key insight is that Torty's parameter $t+\sin(t)$ causes him to slow down and speed up periodically. While both racers cover the same track length, Torty's varying speed means he takes more time overall. Computing $\int_0^T ||\tau'(t)||dt$ and finding when this equals twice the lap length confirms Harry wins.
    \end{feedback}
\end{problem}

\subsection*{When Racers Meet}

\begin{problem}
    Examining the curves $h(t)=\langle \cos(t), \sin(t), \cos(t)\sin^2(t)\rangle$ and $\tau(t)=\langle \cos(t+\sin(t)), \sin(t+\sin(t)), \cos(t+\sin(t))\sin^2(t+\sin(t))\rangle$, we can determine when they meet.
    
    Notice that $\tau(t)=h(t+\sin(t))$. This means Torty and Harry meet when:
    \begin{multipleChoice}
        \choice{$t=0$ only}
        \choice[correct]{$\sin(t)=0$, which occurs at $t=0, \pi, 2\pi, 3\pi, 4\pi$}
        \choice{They never meet}
        \choice{$\cos(t)=0$}
    \end{multipleChoice}
    
    \begin{feedback}
        When $\sin(t)=0$, we have $\tau(t)=h(t+0)=h(t)$, so they are at the same position. This happens at multiples of $\pi$.
    \end{feedback}
\end{problem}

\begin{problem}
    At $t=\pi$ (one of the times they meet), who has greater speed?
    
    Harry's speed at $t=\pi$ is $||h'(\pi)||=\answer[tolerance=0.1]{1}$.
    
    For Torty, we compute $||\tau'(\pi)||$. Using the chain rule: $\tau'(t)=h'(t+\sin(t))\cdot(1+\cos(t))$.
    
    At $t=\pi$: $\tau'(\pi)=h'(\pi)\cdot(1+\cos(\pi))=h'(\pi)\cdot(1-1)=\answer{0}$.
    
    Therefore, at $t=\pi$:
    \begin{multipleChoice}
        \choice[correct]{Harry has greater speed (Torty is momentarily stopped)}
        \choice{Torty has greater speed}
        \choice{They have equal speeds}
    \end{multipleChoice}
    
    \begin{feedback}
        The factor $(1+\cos(t))$ in Torty's velocity causes him to stop completely when $\cos(t)=-1$ (at $t=\pi, 3\pi$, etc.). This is why Torty appears to pause at certain points in the race!
    \end{feedback}
\end{problem}

\section*{Task Two: Torty's Strategic Sprint}

After the first race on this hilly track, Torty and Harry want a rematch. Torty uses his rest periods to store up enough ``running energy'' to double his speed during the last stretch of the race. However, doubling his speed consumes significant energy.

Torty estimates that, while doubling his speed, for every small unit of distance ($ds$) over which his speed is doubled, he uses up $s^{1.3}\cdot ds$ units of energy, where $s$ is his current speed.

\subsection*{Understanding Energy Expenditure}

\begin{problem}
    First, let's understand the energy model. If Torty doubles his speed over a curve segment, the energy required is:
    \[RE=\int_{\text{segment}} s^{1.3}\, ds\]
    
    What does $s$ represent in this integral?
    \begin{multipleChoice}
        \choice{The arc length parameter}
        \choice[correct]{Torty's speed (magnitude of velocity) at each point}
        \choice{The time parameter}
        \choice{The distance traveled}
    \end{multipleChoice}
    
    What does $ds$ represent?
    \begin{multipleChoice}
        \choice{A small change in speed}
        \choice{A small change in time}
        \choice[correct]{A small arc length element}
        \choice{A small change in position}
    \end{multipleChoice}
    
    \begin{feedback}
        This integral sums up energy costs over the arc length of the curve. Remember that $ds=||r'(t)||dt$, so we can convert this to an integral over time: $RE=\int s^{1.3}||r'(t)||dt=\int ||r'(t)||^{2.3}dt$.
    \end{feedback}
\end{problem}

\begin{problem}
    Let's build intuition: If Torty were to double his speed over the entire race (two laps, approximately $\answer[tolerance=0.1]{12.8}$ units of distance), would this require:
    \begin{multipleChoice}
        \choice{Less energy than doubling speed for just the final lap}
        \choice[correct]{More energy than doubling speed for just the final lap}
        \choice{The same energy regardless of when he speeds up}
    \end{multipleChoice}
    
    \begin{feedback}
        The energy required depends on both the speed $s$ and the distance over which the speed is doubled. Since Torty's speed varies throughout the race (remember he stops at certain points!), the total energy depends on the specific curve segment he chooses.
    \end{feedback}
\end{problem}

\begin{problem}
    Computing the energy integrals requires numerical methods. Using computational tools:
    
    The energy required for Torty to double his speed over the entire race (two laps) is approximately $RE_{\text{total}}=\answer[tolerance=1]{100}$ energy units.
    
    The energy required to double speed over just the final lap is approximately $RE_{\text{final lap}}=\answer[tolerance=1]{50}$ energy units.
    
    \begin{feedback}
        You can compute these using: $RE=\int_0^{T} ||\tau'(t)||^{2.3}dt$ where $T$ is the appropriate time interval. Use numerical integration in MATLAB, Python, or similar tools.
    \end{feedback}
\end{problem}

\subsection*{Strategic Timing}

\begin{problem}
    Torty knows he stops running from exhaustion after expending $RE$ equal to half of the energy it would take to double his speed over the entire race.
    
    How much energy can Torty expend before exhaustion? $RE_{\text{max}}=\answer[tolerance=1]{50}$ energy units.
    
    Torty wants to time his sprint so that:
    \begin{selectAll}
        \choice[correct]{He uses exactly his maximum available energy}
        \choice[correct]{He finishes the race before Harry}
        \choice{He starts sprinting as early as possible}
        \choice[correct]{He completes two full laps}
    \end{selectAll}
    
    \begin{feedback}
        Torty needs to find time $t_0$ such that $\int_{t_0}^{T_{\text{finish}}} ||\tau'(t)||^{2.3}dt=RE_{\text{max}}$ and the total distance covered equals two laps. This requires numerical solving.
    \end{feedback}
\end{problem}

\begin{problem}
    Using computational tools to solve for the optimal sprint time:
    
    Torty should start doubling his speed at approximately $t=\answer[tolerance=0.5]{6.5}$ seconds.
    
    With this strategy, Torty finishes the race at approximately $t=\answer[tolerance=0.5]{12}$ seconds.
    
    Since Harry finishes at $t=4\pi\approx 12.57$ seconds, Torty wins by approximately $\answer[tolerance=0.5]{0.57}$ seconds!
    
    \begin{feedback}
        This problem demonstrates the power of calculus for optimization. By carefully choosing when to expend energy, Torty can overcome his initial disadvantage. The solution requires balancing arc length integrals (for distance) with energy integrals (for the constraint).
    \end{feedback}
\end{problem}

\end{document}