\documentclass{ximera}

\title{Polar Coordinates and Multivariable Sets}
\author{Zack Reed}

\begin{document}
\begin{abstract}
In this activity we explore sets in two and three dimensions, learn about polar coordinates as an alternative coordinate system, and develop skills for working with multivariable functions.
\end{abstract}
\maketitle

\section*{Why Change Coordinate Systems?}

For most of your mathematical career, you've used the Cartesian coordinate system with horizontal $x$-axis and vertical $y$-axis. Points are represented as coordinate pairs $(x,y)$.

\begin{center}
\geogebra{vhyaeqmy}{739}{613}
\end{center}

\begin{problem}
    Using the GeoGebra applet above, drag point $P$ to the location $(3,2)$.
    
    The point $P=(3,2)$ means you move $\answer{3}$ units \wordChoice{\choice[correct]{right}\choice{left}} and $\answer{2}$ units \wordChoice{\choice[correct]{up}\choice{down}} from the origin.
\end{problem}

\section*{Describing Sets in Cartesian Coordinates}

A \emph{set} is a collection of points $P$. We can describe sets using conditions on $x$ and $y$.

\begin{center}
\geogebra{sucycy28}{689}{563}
\end{center}

\begin{problem}
    Using the GeoGebra applet above, explore the different sets by selecting the checkboxes.
    
    \begin{enumerate}
        \item The rectangular region is described by points $(x,y)$ where $\answer{0} \leq x \leq \answer{3}$ and $\answer{1} \leq y \leq \answer{3}$.
        
        \item The circle is described by the equation $x^2 + y^2 = \answer{1}$.
        
        \item Which set would be most difficult to describe using only conditions on $x$ and $y$?
        \begin{multipleChoice}
            \choice{The rectangular region}
            \choice{The circle}
            \choice{The region between $y=x$ and $y=x^2$}
            \choice[correct]{The spiral region (Set 4)}
        \end{multipleChoice}
    \end{enumerate}
    
    \begin{feedback}
        The spiral region is very difficult to describe using Cartesian coordinates! This motivates finding a different coordinate system.
    \end{feedback}
\end{problem}

\section*{Introducing Polar Coordinates}

Instead of describing a point by its horizontal and vertical distances from the origin, we can describe it by:
\begin{itemize}
    \item How far it is from the origin: the radius $r$
    \item What angle we rotate from the positive $x$-axis: the angle $\theta$
\end{itemize}

This gives us polar coordinates $(r,\theta)$.

\begin{center}
\geogebra{r9g6amrn}{691}{563}
\end{center}

\begin{problem}
    Using the polar coordinate GeoGebra applet above, drag point $P$ to explore how $r$ and $\theta$ change.
    
    \begin{enumerate}
        \item When point $P$ is at Cartesian coordinates $(1,0)$, the polar coordinates are $(r,\theta) = (\answer{1},\answer{0})$.
        
        \item When point $P$ is at Cartesian coordinates $(0,1)$, the polar coordinates are $(r,\theta) = (\answer{1},\answer[tolerance=0.01]{\pi/2})$.
        
        \item As you move $P$ further from the origin, the value of $r$ \wordChoice{\choice[correct]{increases}\choice{decreases}\choice{stays the same}}.
        
        \item As you move $P$ counterclockwise around the origin (at constant distance), the value of $\theta$ \wordChoice{\choice[correct]{increases}\choice{decreases}\choice{stays the same}}.
    \end{enumerate}
\end{problem}

\begin{problem}
    Now let's describe that mysterious fourth set from earlier!
    
    In polar coordinates, the spiral region can be described as:
    \begin{multipleChoice}
        \choice{$r = 2$ and $0 \leq \theta \leq 2\pi$}
        \choice[correct]{$r \leq 2\theta + 1$ and $0 \leq \theta \leq 2\pi$}
        \choice{$\theta = 2r + 1$ and $0 \leq r \leq 2\pi$}
        \choice{$r^2 + \theta^2 = 1$}
    \end{multipleChoice}
    
    \begin{feedback}
        The condition $r \leq 2\theta + 1$ means the radius grows linearly with the angle, creating a spiral! This is much simpler than trying to describe this region using $x$ and $y$.
    \end{feedback}
\end{problem}

\section*{Converting Between Coordinate Systems}

\begin{problem}
    The relationship between Cartesian $(x,y)$ and polar $(r,\theta)$ coordinates comes from right triangle trigonometry.
    
    Given polar coordinates $(r,\theta)$, the Cartesian coordinates are:
    \begin{align*}
        x &= \answer{r\cos(\theta)} \\
        y &= \answer{r\sin(\theta)}
    \end{align*}
    
    Given Cartesian coordinates $(x,y)$, the polar coordinates are:
    \begin{align*}
        r &= \sqrt{x^2 + \answer{y^2}} \\
        \theta &= \arctan\left(\frac{\answer{y}}{\answer{x}}\right) \text{ (with adjustments for quadrant)}
    \end{align*}
    
    \begin{feedback}
        These formulas come directly from the definitions of sine, cosine, and the Pythagorean theorem!
    \end{feedback}
\end{problem}

\begin{problem}
    Let's practice converting between coordinate systems.
    
    \begin{enumerate}
        \item Convert the polar coordinates $(r,\theta) = (2, \pi/4)$ to Cartesian coordinates.
        
        $x = \answer[tolerance=0.01]{\sqrt{2}}$ and $y = \answer[tolerance=0.01]{\sqrt{2}}$
        
        \item Convert the Cartesian coordinates $(x,y) = (3,4)$ to polar coordinates.
        
        $r = \answer{5}$ and $\theta = \answer[tolerance=0.01]{\arctan(4/3)}$ radians
        
        \item Convert the polar coordinates $(r,\theta) = (5, 0)$ to Cartesian coordinates.
        
        $x = \answer{5}$ and $y = \answer{0}$
    \end{enumerate}
\end{problem}

\section*{Working in Three Dimensions}

Now we extend our thinking to three dimensions, where we have three variables: $x$, $y$, and $z$.

\begin{problem}
    A key strategy for understanding 3D relationships is to \emph{fix one variable and see what happens with the other two}.
    
    This creates a \emph{cross-section} or \emph{slice} of the 3D object.
\end{problem}

\begin{center}
\geogebra{xypfhkea}{732}{506}
\end{center}

\begin{problem}
    Using the GeoGebra applet above, experiment with the $z$ slider.
    
    \begin{enumerate}
        \item When $z = 0$, you see a slice of the surface in the $x$-$y$ plane. This slice is a \wordChoice{\choice[correct]{circle}\choice{parabola}\choice{line}\choice{point}}.
        
        \item As you increase $z$, the slices \wordChoice{\choice{get larger}\choice[correct]{get smaller}\choice{stay the same size}}.
        
        \item What 3D shape does this surface represent?
        \begin{multipleChoice}
            \choice{A cylinder}
            \choice[correct]{A cone}
            \choice{A sphere}
            \choice{A paraboloid}
        \end{multipleChoice}
    \end{enumerate}
    
    \begin{feedback}
        By looking at slices, we can understand complex 3D shapes by breaking them down into familiar 2D curves!
    \end{feedback}
\end{problem}

\begin{center}
\geogebra{rf4mpv5m}{732}{506}
\end{center}

\begin{problem}
    Using the second GeoGebra applet, explore slices in different directions.
    
    \begin{enumerate}
        \item Select "Show x-Slice" and move the slider. When we fix $x$ and look at the $y$-$z$ plane, we see the relationship between \wordChoice{\choice{$x$ and $y$}\choice{$x$ and $z$}\choice[correct]{$y$ and $z$}}.
        
        \item Select "Show y-Slice" and move the slider. When we fix $y$ and look at the $x$-$z$ plane, we see the relationship between \wordChoice{\choice{$x$ and $y$}\choice[correct]{$x$ and $z$}\choice{$y$ and $z$}}.
        
        \item By rotating the view to look directly at each slice, you reduce the problem to a familiar \wordChoice{\choice[correct]{2-dimensional}\choice{1-dimensional}\choice{4-dimensional}} perspective.
    \end{enumerate}
\end{problem}

\section*{Summary and Reflection}

\begin{problem}
    Reflect on what you've learned:
    
    \begin{enumerate}
        \item Why might you choose polar coordinates over Cartesian coordinates?
        \begin{selectAll}
            \choice[correct]{To describe circular or spiral patterns more easily}
            \choice{To make all problems simpler}
            \choice[correct]{When the natural description involves distances and angles}
            \choice{Because they're always better than Cartesian coordinates}
        \end{selectAll}
        
        \item The key strategy for understanding 3D relationships is:
        \begin{multipleChoice}
            \choice{Always convert to polar coordinates}
            \choice{Memorize formulas for every possible surface}
            \choice[correct]{Fix one variable at a time and analyze 2D slices}
            \choice{Avoid 3D problems whenever possible}
        \end{multipleChoice}
    \end{enumerate}
\end{problem}

\end{document}