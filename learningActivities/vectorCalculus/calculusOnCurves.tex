\documentclass{ximera}

\title{Activity: Calculus on Curves}
\author{Zack Reed}

\begin{document}
\begin{abstract}
In this activity we explore vector functions, tangent vectors, integration of vector functions, and arc length calculations for curves in space. We'll build intuition through interactive problems and visualizations.
\end{abstract}
\maketitle

\section*{Introduction: Vector Functions}

In single-variable calculus, functions took a number as input and produced a number as output. Now we'll work with functions that take a number (time) as input and produce a vector (position) as output.

\begin{definition}
A \textbf{vector function} uses separate functions of time to define the coordinates of a position vector:
$$\vec{r}(t) = \langle x(t), y(t), z(t) \rangle = \langle f(t), g(t), h(t) \rangle, \quad \text{for } t \in [a,b]$$

We can work with $\vec{r}(t)$ by working individually with each of its component functions.
\end{definition}

\begin{problem}
Before exploring visualizations, let's check understanding. If $\vec{r}(t) = \langle 2t, t^2, 3 \rangle$, what does this vector function represent?

\begin{multipleChoice}
    \choice{A single point in space}
    \choice[correct]{A curve in 3D space traced out as $t$ varies}
    \choice{Three separate unrelated functions}
    \choice{A plane in 3D space}
\end{multipleChoice}

\begin{feedback}
As $t$ varies, the position vector $\vec{r}(t)$ traces out a path (curve) through 3D space. Think of it like a particle moving through space over time!
\end{feedback}
\end{problem}

\begin{problem}
For the same function $\vec{r}(t) = \langle 2t, t^2, 3 \rangle$, find the position at specific times:

At $t=0$: $\vec{r}(0) = \langle \answer{0}, \answer{0}, \answer{3} \rangle$

At $t=1$: $\vec{r}(1) = \langle \answer{2}, \answer{1}, \answer{3} \rangle$

At $t=2$: $\vec{r}(2) = \langle \answer{4}, \answer{4}, \answer{3} \rangle$

\begin{feedback}
Notice that the $z$-coordinate stays constant at 3, while the $x$ and $y$ coordinates change. This means the curve lies entirely in the plane $z=3$!
\end{feedback}
\end{problem}

\subsection*{Visualizing a Classic Curve: The Helix}

Now let's explore a famous 3D curve: the helix $\vec{r}(t)=\langle \cos(t), \sin(t), t\rangle$.

\begin{problem}
Before viewing the applet, predict what this curve looks like:

The $x$ and $y$ components $\langle \cos(t), \sin(t) \rangle$ describe:
\begin{multipleChoice}
    \choice{A straight line}
    \choice[correct]{A circle in the $xy$-plane}
    \choice{A parabola}
    \choice{A spiral}
\end{multipleChoice}

As $t$ increases, the $z$-coordinate:
\begin{multipleChoice}
    \choice{Stays constant}
    \choice[correct]{Increases linearly}
    \choice{Decreases}
    \choice{Oscillates}
\end{multipleChoice}

Combining these, the curve should look like:
\begin{multipleChoice}
    \choice{A flat circle}
    \choice[correct]{A spring or corkscrew shape}
    \choice{A straight line}
    \choice{A sphere}
\end{multipleChoice}

\begin{feedback}
Excellent prediction! The curve circles around (from the $\cos$ and $\sin$) while simultaneously rising (from the $t$ in the $z$-component), creating a helix shape.
\end{feedback}
\end{problem}

\begin{problem}
Now explore the visualization to confirm your prediction.

\begin{expandable}{stuff}{GeoGebra Instructions}
    Rotate the 3D view by clicking and dragging. Use your scroll wheel to zoom. Observe how the curve spirals upward.
\end{expandable}

\begin{center}
\geogebra{yxed8hcu}{730}{510}
\end{center}

After exploring, verify your observations:
\begin{selectAll}
    \choice[correct]{The curve spirals upward}
    \choice[correct]{The curve repeats its circular pattern in each horizontal slice}
    \choice{The curve stays in a single plane}
    \choice[correct]{The curve rises at a constant rate}
\end{selectAll}
\end{problem}

\section*{Task One: Tangent Vectors and Velocity}

\subsection*{From Single-Variable to Vector Derivatives}

Because we can work with each component separately, differentiating vector functions is straightforward: differentiate each component!

\begin{problem}
Let's review the single-variable case first. In Calc I and II, we measured instantaneous rate of change by taking derivatives.

For a function $f(x)$, the derivative $f'(x)$ tells us:
\begin{selectAll}
    \choice[correct]{The instantaneous rate of change}
    \choice[correct]{The slope of the tangent line}
    \choice{The area under the curve}
    \choice[correct]{How fast the function is changing}
\end{selectAll}

\begin{feedback}
The derivative captures instantaneous rate of change. Now we'll extend this idea to vector functions!
\end{feedback}
\end{problem}

\begin{problem}
Before viewing the next applet, let's explore the 2D case conceptually.

\begin{expandable}{stuff}{GeoGebra Instructions}
    Alter the "X=" slider to see the tangent line shift along the curve. Notice the approximating horizontal and vertical components represented by $dx$ and $df$.
\end{expandable}

\begin{center}
\geogebra{gyh4uhzy}{786}{584}
\end{center}

What does the tangent line represent?
\begin{multipleChoice}
    \choice{The curve itself}
    \choice[correct]{The direction and rate of change at a specific point}
    \choice{The average rate of change}
    \choice{The total change over an interval}
\end{multipleChoice}
\end{problem}

\subsection*{Computing Velocity Vectors}

\begin{definition}
For a vector function $\vec{r}(t) = \langle x(t), y(t), z(t) \rangle$, the \textbf{velocity vector} (or \textbf{tangent vector}) is:
$$\vec{v}(t) = \vec{r}'(t) = \left\langle \frac{dx}{dt}, \frac{dy}{dt}, \frac{dz}{dt} \right\rangle = \langle x'(t), y'(t), z'(t) \rangle$$

This vector is tangent to the curve and points in the direction of motion.
\end{definition}

\begin{problem}
Let's compute some velocity vectors. If $\vec{r}(t) = \langle t^2, 3t, 2t^3 \rangle$, find $\vec{v}(t)$.

$\vec{v}(t) = \langle \answer{2t}, \answer{3}, \answer{6t^2} \rangle$

At $t=1$: $\vec{v}(1) = \langle \answer{2}, \answer{3}, \answer{6} \rangle$

At $t=2$: $\vec{v}(2) = \langle \answer{4}, \answer{3}, \answer{24} \rangle$

\begin{feedback}
Notice how the velocity vector changes as $t$ changes! At $t=2$, the particle is moving much faster in the $z$-direction than at $t=1$.
\end{feedback}
\end{problem}

\begin{problem}
For the helix $\vec{r}(t) = \langle \cos(t), \sin(t), t \rangle$, compute the velocity vector.

$\vec{v}(t) = \langle \answer{-\sin(t)}, \answer{\cos(t)}, \answer{1} \rangle$

\begin{feedback}
The $z$-component of velocity is constantly 1, meaning the helix rises at a constant rate. The $x$ and $y$ components oscillate, creating the circular motion.
\end{feedback}
\end{problem}

\begin{problem}
Now visualize how the velocity vector moves along the curve in 3D.

\begin{expandable}{stuff}{GeoGebra Instructions}
    Alter the "t=" slider to view the velocity vector at different points. Click and drag to rotate your view. Notice how the vector is always tangent to the curve.
\end{expandable}

\begin{center}
\geogebra{zr86834p}{730}{487}
\end{center}

After exploring the visualization, answer:
\begin{selectAll}
    \choice[correct]{The velocity vector is always tangent to the curve}
    \choice[correct]{The velocity vector's direction changes as the particle moves}
    \choice{The velocity vector is always the same length}
    \choice[correct]{The velocity vector points in the direction of motion}
\end{selectAll}
\end{problem}

\section*{Task Two: Integrating Vector Functions}

\subsection*{Reconstructing Position from Velocity}

In single-variable calculus, we learned that integration is the reverse of differentiation. If we know the rate of change, we can find the total change by integrating.

\begin{problem}
Let's review: In single-variable calculus, if a rocket has velocity $v(t)$ over time interval $[a,b]$, the total change in height is:

\begin{multipleChoice}
    \choice{$v(b) - v(a)$}
    \choice[correct]{$\int_a^b v(t) \, dt$}
    \choice{$v(b) \cdot (b-a)$}
    \choice{$\frac{dv}{dt}$}
\end{multipleChoice}

\begin{feedback}
We integrate velocity to find displacement! Small bits of distance $dD = v(t) \cdot dt$ add up to give total change: $\Delta D = \int_a^b v(t) \, dt$.
\end{feedback}
\end{problem}

\begin{problem}
Let's visualize this with a rocket example.

\begin{expandable}{stuff}{GeoGebra Instructions}
    Alter the "t=" slider to view the rocket's height and velocity simultaneously. Notice how velocity determines the rate at which height accumulates.
\end{expandable}

\begin{center}
\geogebra{ysumvptx}{750}{467}
\end{center}

The rocket's velocity determines its motion, and total height accumulates over time. For instance, a height of $1090.2$ meters over $5.57$ seconds comes from: $1090.2 = \int_0^{5.57} v(t) \, dt$.

\begin{feedback}
Watch how the area under the velocity curve corresponds to the height gained. This is the fundamental theorem of calculus in action!
\end{feedback}
\end{problem}

\subsection*{Vector Integration: Component by Component}

\begin{definition}
For a vector velocity function $\vec{v}(t)=\langle x'(t), y'(t), z'(t)\rangle$, the change in position over $[a,b]$ is:
$$\vec{r}(b)-\vec{r}(a)= \int_a^b \vec{v}(t)\, dt=\left\langle \int_a^b x'(t)\, dt, \int_a^b y'(t)\, dt, \int_a^b z'(t)\, dt\right\rangle$$

We integrate each component separately!
\end{definition}

\begin{problem}
Key insight: When we integrate a vector function, what do we get?

\begin{multipleChoice}
    \choice{A scalar (number)}
    \choice[correct]{A vector (displacement)}
    \choice{The arc length}
    \choice{The speed}
\end{multipleChoice}

\begin{feedback}
Unlike scalar integration which gives a number, vector integration gives us a displacement vector showing the overall change in position!
\end{feedback}
\end{problem}

\begin{problem}
Let's practice! If $\vec{v}(t) = \langle 2, 3t, 4t^2 \rangle$, find the displacement from $t=0$ to $t=2$.

$\int_0^2 \vec{v}(t) \, dt = \left\langle \int_0^2 \answer{2} \, dt, \int_0^2 \answer{3t} \, dt, \int_0^2 \answer{4t^2} \, dt \right\rangle$

$= \langle \answer{4}, \answer{6}, \answer{32/3} \rangle$

\begin{feedback}
Remember: $\int_0^2 2 \, dt = 2t \big|_0^2 = 4$, $\int_0^2 3t \, dt = \frac{3t^2}{2} \big|_0^2 = 6$, and $\int_0^2 4t^2 \, dt = \frac{4t^3}{3} \big|_0^2 = \frac{32}{3}$.
\end{feedback}
\end{problem}

\begin{problem}
Now visualize this process in 3D.

\begin{expandable}{stuff}{GeoGebra Instructions}
    Alter the "T=" slider to watch $\vec{r}$ build progressively by integrating each component. Observe how each coordinate changes independently.
\end{expandable}

\begin{center}
\geogebra{excy8qtq}{730}{510}
\end{center}

After exploring, select the true statements:
\begin{selectAll}
    \choice[correct]{Each component is integrated separately}
    \choice[correct]{The result is a displacement vector}
    \choice{The result tells us the arc length traveled}
    \choice[correct]{The process reconstructs position from velocity}
\end{selectAll}
\end{problem}

\section*{Task Three: Arc Length}

\subsection*{Distance vs. Displacement}

While integrating components gives us displacement (a vector), we often want to know the actual distance traveled along the curve—the arc length.

\begin{problem}
Think about the difference between displacement and distance:

If you walk 3 meters east, then 4 meters north, your displacement is:
\begin{multipleChoice}
    \choice{7 meters}
    \choice[correct]{5 meters (by Pythagorean theorem)}
    \choice{3 meters}
    \choice{4 meters}
\end{multipleChoice}

But the actual distance you walked is:
\begin{multipleChoice}
    \choice{5 meters}
    \choice[correct]{7 meters}
    \choice{3 meters}
    \choice{4 meters}
\end{multipleChoice}

\begin{feedback}
Displacement measures the straight-line change from start to finish. Distance measures the actual path length traveled. For curves, we use arc length to measure distance!
\end{feedback}
\end{problem}

\subsection*{Building the Arc Length Formula}

We've already learned about arc length in Calc II! Now we extend it to higher dimensions using the Pythagorean theorem.

\begin{problem}
Let's review the 2D case first. Explore the visualization below.

\begin{expandable}{stuff}{GeoGebra Instructions}
    Alter the "x=" slider to see how small segments of the curve can be approximated by straight line segments. Notice the Pythagorean calculation.
\end{expandable}

\begin{center}
\geogebra{pd8zaw8v}{786}{584}
\end{center}

After exploring, what represents a small bit of arc length?
\begin{multipleChoice}
    \choice{$dx + dy$}
    \choice[correct]{$ds = \sqrt{(dx)^2 + (dy)^2}$}
    \choice{$dx \cdot dy$}
    \choice{$\frac{dy}{dx}$}
\end{multipleChoice}

\begin{feedback}
The Pythagorean theorem gives us the length of the small segment: $ds = \sqrt{(dx)^2 + (dy)^2}$!
\end{feedback}
\end{problem}

\begin{definition}
For a 2D curve with coordinate functions $x(t)$ and $y(t)$, a small arc length element is:
$$ds=\sqrt{\left(\frac{dx}{dt}\right)^2+\left(\frac{dy}{dt}\right)^2} \, dt=\sqrt{x'(t)^2+y'(t)^2} \, dt = |\vec{v}(t)| \, dt$$

The total arc length from $t=a$ to $t=b$ is:
$$L = \int_a^b ds = \int_a^b |\vec{v}(t)| \, dt = \int_a^b \sqrt{x'(t)^2+y'(t)^2} \, dt$$
\end{definition}

\begin{problem}
Key insight: What's the difference between $\int_a^b \vec{v}(t) \, dt$ and $\int_a^b |\vec{v}(t)| \, dt$?

$\int_a^b \vec{v}(t) \, dt$ gives:
\begin{multipleChoice}
    \choice{Arc length}
    \choice[correct]{Displacement (a vector)}
    \choice{Speed}
    \choice{Distance}
\end{multipleChoice}

$\int_a^b |\vec{v}(t)| \, dt$ gives:
\begin{multipleChoice}
    \choice{Displacement (a vector)}
    \choice[correct]{Arc length (distance traveled)}
    \choice{Velocity}
    \choice{Acceleration}
\end{multipleChoice}

\begin{feedback}
Without the magnitude bars, we integrate the vector to get displacement. With the magnitude bars, we integrate speed to get total distance (arc length)!
\end{feedback}
\end{problem}

\subsection*{Arc Length in 3D}

The same principle extends beautifully to 3D—we just add the $z$-component!

\begin{definition}
For a 3D curve $\vec{r}(t) = \langle x(t), y(t), z(t) \rangle$:
$$L = \int_a^b |\vec{v}(t)| \, dt = \int_a^b \sqrt{x'(t)^2+y'(t)^2+z'(t)^2} \, dt$$
\end{definition}

\begin{problem}
Let's compute an arc length! For $\vec{r}(t) = \langle 3t, 4t, 0 \rangle$ from $t=0$ to $t=2$:

First, find $\vec{v}(t) = \langle \answer{3}, \answer{4}, \answer{0} \rangle$

The speed is $|\vec{v}(t)| = \sqrt{9 + 16 + 0} = \answer{5}$

Since speed is constant, the arc length is simply:
$L = \int_0^2 5 \, dt = \answer{10}$

\begin{feedback}
When speed is constant, arc length is just speed times time! This curve is actually a straight line in the $xy$-plane moving at constant speed.
\end{feedback}
\end{problem}

\begin{problem}
Now for the helix $\vec{r}(t) = \langle \cos(t), \sin(t), t \rangle$ from $t=0$ to $t=2\pi$:

$\vec{v}(t) = \langle \answer{-\sin(t)}, \answer{\cos(t)}, \answer{1} \rangle$

$|\vec{v}(t)| = \sqrt{\sin^2(t) + \cos^2(t) + 1} = \answer{\sqrt{2}}$

The arc length for one complete turn is:
$L = \int_0^{2\pi} \sqrt{2} \, dt = \answer{2\pi\sqrt{2}}$ (or $\answer[tolerance=0.1]{8.886}$)

\begin{feedback}
The helix also has constant speed! Even though it's curving and rising, the total speed $\sqrt{2}$ remains constant.
\end{feedback}
\end{problem}

\begin{problem}
Visualize the 3D arc length calculation.

\begin{expandable}{stuff}{GeoGebra Instructions}
    Alter the "t=" slider to see the arc length approximation at various points. Click, drag, and scroll to rotate and zoom. Select "Reset Zoom" or "Zoom to Arc Length" buttons for different views. Check "Show x-y hypotenuse" for more detail.
\end{expandable}

\begin{center}
\geogebra{nrnm8k7p}{780}{438}
\end{center}

After exploring, verify your understanding:
\begin{selectAll}
    \choice[correct]{Small arc length elements use the Pythagorean theorem in 3D}
    \choice[correct]{We integrate speed over time to get arc length}
    \choice{Arc length is always equal to displacement magnitude}
    \choice[correct]{Arc length measures actual distance traveled along the curve}
\end{selectAll}
\end{problem}

\section*{Summary: Two Types of Vector Integration}

We now have powerful tools for analyzing curves!

\begin{problem}
Let's review the two main ways we integrate vector functions:

\textbf{Type 1: Displacement (Vector Result)}
$$\vec{r}(b)-\vec{r}(a)= \int_a^b \vec{v}(t)\, dt=\left\langle \int_a^b x'(t)\, dt, \int_a^b y'(t)\, dt, \int_a^b z'(t)\, dt\right\rangle$$

This gives us:
\begin{multipleChoice}
    \choice{The distance traveled}
    \choice[correct]{The net change in position (displacement vector)}
    \choice{The speed}
    \choice{The arc length}
\end{multipleChoice}

\textbf{Type 2: Arc Length (Scalar Result)}
$$L = \int_a^b |\vec{v}(t)|\, dt=\int_a^b\sqrt{x'(t)^2+y'(t)^2+z'(t)^2}\, dt$$

This gives us:
\begin{multipleChoice}
    \choice{The displacement vector}
    \choice[correct]{The total distance traveled along the curve}
    \choice{The velocity}
    \choice{The final position}
\end{multipleChoice}

\begin{feedback}
These are fundamentally different! One integrates velocity (vector) to get displacement (vector). The other integrates speed (scalar) to get distance (scalar).
\end{feedback}
\end{problem}

\begin{problem}
Let's solidify understanding with a comprehensive example. Consider $\vec{r}(t) = \langle t, t^2, 2t \rangle$ from $t=0$ to $t=1$.

\textbf{Part A: Find the displacement}

$\int_0^1 \vec{v}(t) \, dt = \int_0^1 \langle \answer{1}, \answer{2t}, \answer{2} \rangle \, dt$

$= \langle \answer{1}, \answer{1}, \answer{2} \rangle$

\textbf{Part B: Find the arc length}

$|\vec{v}(t)| = \sqrt{1 + 4t^2 + 4} = \sqrt{\answer{5 + 4t^2}}$

The arc length requires numerical integration (this integral doesn't have a simple closed form):
$L = \int_0^1 \sqrt{5 + 4t^2} \, dt \approx \answer[tolerance=0.1]{2.46}$

\textbf{Part C: Compare}

The magnitude of the displacement is $|\langle 1, 1, 2 \rangle| = \sqrt{1+1+4} = \answer{\sqrt{6}} \approx 2.45$

Notice that the arc length ($\approx 2.46$) is slightly larger than the displacement magnitude ($\approx 2.45$). This makes sense because:
\begin{multipleChoice}
    \choice{Arc length should always be smaller}
    \choice[correct]{The curve isn't perfectly straight, so the path is slightly longer}
    \choice{We made a calculation error}
    \choice{They should be exactly equal}
\end{multipleChoice}

\begin{feedback}
The displacement gives the straight-line distance from start to finish. The arc length gives the actual distance traveled along the (slightly curved) path. The arc length is always greater than or equal to the displacement magnitude!
\end{feedback}
\end{problem}

\begin{problem}
Final check: Select all TRUE statements about vector calculus on curves.

\begin{selectAll}
    \choice[correct]{Velocity is the derivative of position}
    \choice[correct]{Velocity vectors are tangent to the curve}
    \choice[correct]{Integrating velocity gives displacement}
    \choice[correct]{Integrating speed gives arc length}
    \choice{Displacement and arc length are always equal}
    \choice[correct]{Arc length uses the Pythagorean theorem}
    \choice[correct]{We can work with vector functions component by component}
    \choice{The magnitude of displacement can exceed arc length}
\end{selectAll}

\begin{feedback}
Excellent! You've mastered the fundamentals of calculus on curves. These concepts—tangent vectors, integration, and arc length—are essential for understanding motion in space and will be the foundation for more advanced topics like curvature and TNB frames.
\end{feedback}
\end{problem}

\end{document}