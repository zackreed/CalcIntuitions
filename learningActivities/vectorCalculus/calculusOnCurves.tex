\documentclass{ximera}

\title{Course Notes: Curves and Tangent Vectors}
\author{YOUR-NAME-HERE}

\begin{document}
\begin{abstract}
These notes introduce vector functions, tangent vectors, integration of vector functions, and arc length calculations for curves in two and three dimensions.
\end{abstract}
\maketitle

\section{Course Notes: Curves and Tangent Vectors}

As suggested in the overview, a vector function uses three separate functions of time to comprise the coordinates of a vector. These functions usually take the form

$$x=f(t), \quad y=g(t), \quad z=h(t), \quad \text{for } t \in [a,b].$$

We will use the notation $\langle x(t),y(t),z(t)\rangle=\langle f(t),g(t),h(t)\rangle$ interchangeably. This construction effectively allows us to work with $\vec{r}(t)$ by working individually with each of its component functions, which is a benefit that we won't quite be able to make use of for more complicated functions in later modules.

You may again use the following GeoGebra application to view the three-dimensional curve defined by the vector function $\vec{r}(t)=\langle \cos(t), \sin(t), t\rangle$.

\begin{center}
\geogebra{yxed8hcu}{730}{510}
\end{center}

For now, however, because we use the outputs of $f, g,$ and $h$ in tandem for the same values of time, we can break apart $\vec{r}(t)$ into its components, tweak the components (for instance, by differentiating or integrating), and then put the new components back together again.

One way that we use the components of a vector function productively is by differentiating. In 241 and 242, we measured the instantaneous rate of change of a function $f$ by taking the limit of its average rate of change over increasingly small intervals of time. This gave us the derivative function $f'(t)=\frac{df}{dt}$, from which we could build a line tangent to the curve at any particular time, as is depicted in the following GeoGebra application.

You may alter the "X=" slider to see the shifting tangent line (represented as a vector now) and the approximating horizontal and vertical shifts represented by $df$ and $dx$.

\begin{center}
\geogebra{gyh4uhzy}{786}{584}
\end{center}

Now, with the advent of our new vector functions, we take the derivative of each component to yield a tangent vector. You may again alter the "t=" slider to view the vector given by $\vec{v}(t)=\langle x'(t), y'(t), z'(t)\rangle$.

\begin{center}
\geogebra{zr86834p}{730}{487}
\end{center}

As in the single-variable case, the vector given by $\vec{v}(t)$ lies tangent to the curve. We call the vectors given by $\vec{v}(t)$ \emph{tangent vectors} or \emph{velocity vectors}.

\section{Course Notes: Integrating Vector Functions}

When given information about a function's rate of change, we can reconstruct changes in the original function by adding up the rate-time products in a definite integral. More specific to motion, if we know that an object is moving in a single direction with velocity $v(t)$, then over a small interval of time ($dt$) we can describe how far the object has moved by the product $dD=v(t)\cdot dt$. This lets us calculate a total change in distance using definite integrals, $\Delta D=\int dD=\int v(t)\cdot dt$. Informally, we can think of this as adding up the small distances generated by each small product $dD=v(t)\cdot dt$, akin to what is seen in the following GeoGebra application.

You may alter the "t=" slider to simultaneously view the rocket's height and velocity at time $t$.

\begin{center}
\geogebra{ysumvptx}{750}{467}
\end{center}

The rocket's velocity determines its forward motion, and the total height at any one time is the accumulation of distances amassed over previously completed time intervals. For instance, the rocket's total height of $1090.2$ meters was amassed over the 5.57 seconds of its observed flight, which can be calculated by the integral $1090.2=\int_0^{1090.2}dD=\int_0^{5.57}v(t)\cdot dt$.

Now, with vector functions, we have multiple ways of measuring the accumulation of change from the velocity function $\vec{v}(t)=\langle x'(t), y'(t), z'(t)\rangle$. If we care about measuring the total change to a vector $\vec{r}(t)=\langle x(t), y(t), z(t)\rangle$ on the time interval $[a,b]$, we simply measure the total change to each component. If we can compute $\int_a^b x'(t) \cdot dt$, $\int_a^b y'(t) \cdot dt$, and $\int_a^b z'(t) \cdot dt$, then we have measured the amount of change for each coordinate of $\vec{r}(t)$ on the time interval $[a,b]$.

We, therefore can measure the change to $\vec{r}(t)$ by the calculation 
$$\vec{r}(b)-\vec{r}(a)= \int_a^b \vec{v}(t)\cdot dt=\left\langle \int_a^b x'(t)\cdot dt, \int_a^b y'(t)\cdot dt, \int_a^b z'(t)\cdot dt\right\rangle.$$

Notice that it is a vector difference that is measured here, rather than a scalar difference. This can be visualized in the following GeoGebra application. As before, you may alter the "T=" slider to view the progressive building of $\vec{r}$ by integrating each of the component functions to measure a total change in each function.

\begin{center}
\geogebra{excy8qtq}{730}{510}
\end{center}


\section{Course Notes: Arc Length}

While integrating each component function yields a measure of change in the vector's position, we also want to measure the length of the curve $\vec{r}(t)$. Measuring and working with arc length is a crucial component of the ways that we will do calculus on vector functions throughout the remainder of this course.

Luckily, we have already used integrals to measure the lengths of curves in 242, which gives us a very straightforward generalization into more dimensions. Before, we used the linear approximation of functions to allow us use of the Pythagorean calculation of distance. Defining the $(x,y)$ pairs on a graph in terms of coordinate functions $(x(t), y(t))$, we described this small segment of arc length as $ds=\sqrt{\left(\frac{dx}{dt}\right)^2+\left(\frac{dy}{dt}\right)^2}\cdot dt$, as seen in the following GeoGebra application.

You may alter the "x=" slider to view this arc length calculation at various points along the curve.

\begin{center}
\geogebra{pd8zaw8v}{786}{584}
\end{center}

Now, our new language of vector functions and velocities given by the derivatives of coordinate functions $x'(t)$ and $y'(t)$, we can capture a small bit of arc length as $ds=\sqrt{\left(\frac{dx}{dt}\right)^2+\left(\frac{dy}{dt}\right)^2}\cdot dt=\sqrt{x'(t)^2+y'(t)^2}\cdot dt$. We can then find the total arc length by adding up the small pieces of arc length to get a total length $\Delta s=\int ds=\int_a^b\sqrt{\left(\frac{dx}{dt}\right)^2+\left(\frac{dy}{dt}\right)^2}\cdot dt=\int_a^b\sqrt{x'(t)^2+y'(t)^2}\cdot dt$.

These calculations, in effect, measure the size of the velocity vector $|\vec{v}(t)|=|\langle x'(t), y'(t)\rangle |=\sqrt{x'(t)^2+y'(t)^2}$. The need for the differential $dt$ arises when we consider that the approximating lengths $ds$ must be small for the approximation to generate a definite integral in the limit. In other words, $|\vec{v}(t)|$ itself does not give a small bit of arc length, as velocity vectors can be quite large in size. As such, we need to consider small pieces of arc length from the approximation $ds=|\vec{v}(t)|\cdot dt$.

This lets us succinctly measure arc length (in both 2 and 3 dimensions) by integrating the speed-time product $\Delta s=\int ds=\int_a^b|\vec{v}(t)|\cdot dt$. A 3-dimensional representation of this is given in the following application.

You may alter the "t=" slider to view the arc length approximation at various points along the curve. You may click, drag, and scroll to rotate your view and zoom in and out. You may select the "Reset Zoom" and "Zoom to Arc Length" buttons to globally view the curve or to view the arc length approximation up close. You may select the "Show x-y hypotenuse" check box to view the construction in slightly greater detail.

\begin{center}
\geogebra{nrnm8k7p}{780}{438}
\end{center}

If we can construct a small bit of arc length of a curve as $ds=|\vec{v}(t)|\cdot dt$, then we measure the entire arc length by adding up the small bits of arc length over the time interval in which we construct the curve. Hence, $\Delta s=\int ds=\int_a^b|\vec{v}(t)|\cdot dt$.

We thus have two ways to use definite integration on vector functions. We can either measure the change along the curve between two different times from the definite integral 
$$\vec{r}(b)-\vec{r}(a)= \int_a^b \vec{v}(t)\cdot dt=\left\langle \int_a^b x'(t)\cdot dt, \int_a^b y'(t)\cdot dt, \int_a^b z'(t)\cdot dt\right\rangle,$$
or we can measure the length of the curve by taking the size of the velocity vector and scaling it accordingly to leverage the Pythagorean Theorem, giving 
$$\Delta s=\int ds=\int_a^b|\vec{v}(t)|\cdot dt=\int_a^b\sqrt{x'(t)^2+y'(t)^2+z'(t)^2}\cdot dt.$$


\section{Video Resources}

Visit MyLab Math Multimedia Library and select this module's referenced sections of your Pearson book for Instructional Videos, Interactive Figures, and Animations related to the material.

Visit the \href{https://www.youtube.com/playlist?list=PLHXZ9OQGMqxc_CvEy7xBKRQr6I214QJcd}{Calculus III: Multivariable Calculus playlist by Dr. Trefor Bazett}, found on YouTube, for further video resources on the big-picture ideas of multivariable calculus.

\end{document}