\documentclass{ximera}
% \input{../../xmpreamble.tex}

\title{Vector Products: Cross Product}
\author{Zack Reed}

\begin{document}
\begin{abstract}
In this activity we continue our grounded exploration of vectors, focusing on vector multiplication: the cross product.
\end{abstract}
\maketitle

\section*{Multiplication Part 2 - Cross Product}

We've come to the end of the journey for Riley and Taylor, who need to put away their box. They are trying to put away their box using a teeter-totter, but having some trouble because the box is too heavy. 
\begin{center}
\youtube{bLtI_Ydj7YU}
\end{center}

\begin{problem}
    (Referencing the video) Do the boys or the box create a greater torque about the pivot? (If they are both 1m from the pivot.)

    With a weight of $800$N, the box creates a torque of $\answer{800}$N $*\answer{1}$m, making a torque of $\answer{800}$Nm. With a combined weight of $600$N, the boys create a torque of $\answer{600}$N $*\answer{1}$m, making a torque of $\answer{600}$Nm. Thus, the \wordChoice{\choice[correct]{box}\choice{boys}} create(s) a greater torque.
\end{problem}

Now let's see how we extend this idea of torque to less ideal situations.

(video)

\begin{problem}
    (Referencing the video) If the box and boys are each located 1m from the center of the plank, what can the boys do to lift the box about the pivot?

    \begin{selectAll}
        \choice{Nothing, they can't lift the box.}
        \choice[correct]{Adjust the radius lengths by moving the pivot.}
        \choice{Change the angles of the radii with the forces.}
        \choice[correct]{Grab something to increase the weight on their side.}
    \end{selectAll}
\end{problem}

Let's see the first option in action.

(video)

\begin{problem}
    (Referencing the video) Where should they adjust the pivot so that they can lift the box, assuming they and the box stay 1m from the center of the plank?

    Solution: The pivot should be located farther than $\answer[tolerance=.1]{8/7}$ m from their position.
\end{problem}

Now we wrap up the section by looking more generally at the \emph{cross product} of two vectors, which is a third vector whose magnitude represents the torque created by the two vectors.

\begin{center}
 \youtube{ID2dVcps9q8}
\end{center}

\begin{definition}
    As described in the video, the cross product of two vectors $\vec{u}$ and $\vec{v}$, denoted $\vec{u}\times \vec{v}$, is a vector perpendicular to both $\vec{u}$ and $\vec{v}$ with length $|\vec{u}||\vec{v}|\sin(\theta)$, where $\theta$ is the smallest angle between $\vec{u}$ and $\vec{v}$. The direction of the cross product vector is determined by the "right-hand rule."
\end{definition}

\subsection*{Geometric Connections}
We apply the cross product to "multiply" vectors together by returning to an area-based interpretation of multiplication. We interpret the product $ab$ as the area of a parallelogram with height $a$ and base $b$. 

One nice property of areas is that, even when slanted into parallelograms, if $a$ is the height and $b$ is the base length then the area is still given by $ab$.

As seen in the following GeoGebra applet, the area of parallelograms depends on the base height product, where the height is perpendicular to the base. 

\begin{expandable}{stuff}{GeoGebra Instructions}
    You may alter the "$\theta$=" slider and the position of the red point to alter the parallelogram.
\end{expandable}

\begin{center}
\geogebra{hccaztdd}{789}{461}
\end{center}

In each case, the area of the parallelogram remains the product of its base and its height. You might imagine this by moving the triangle (created by the vertical height on the left side) to fit in the empty space on the right of the parallelogram, making a rectangle.

It is important to remember that the cross product must be defined to 3-dimensional vectors, and that it results in a 3-dimensional vector, despite our heavy focus on the magnitude of the vector, $|u||v|\sin(\theta)$.

\begin{definition}
    A quick way to find the cross product vector is with the vector formula below, where $\vec{u}=[u_1,u_2,u_3]$ and $\vec{v}=[v_1,v_2,v_3]$: 

    \[\vec{u}\times \vec{v} = 
    \begin{bmatrix}
    u_2 v_3 - u_3 v_2 \\
    u_3 v_1 - u_1 v_3 \\
    u_1 v_2 - u_2 v_1
    \end{bmatrix}\]
\end{definition}

\begin{remark}
Your textbook will give you a way to compute the cross product using what is called a "determinant." Ignore this method, as it's mostly a trick of notation and can obscure the actual meaning of the determinant. If you're going to memorize something, memorize the formula above.
\end{remark}

Now let's finish with some practice computing cross products. Remember to just use the simple formula above, don't bother with the ``determinant'' method.

\begin{problem}
    Compute the cross product $\vec{u}\times\vec{v}$ for each of the following pairs of vectors. Give both the coordinates of the resulting vector and its magnitude.
    
    \begin{enumerate}
        \item $\vec{u}=[1,0,0]$ and $\vec{v}=[0,1,0]$
        
        $\vec{u}\times\vec{v}=[\answer{0},\answer{0},\answer{1}]$ with magnitude $|\vec{u}\times\vec{v}|=\answer{1}$
        
        \item $\vec{u}=[2,1,0]$ and $\vec{v}=[1,2,0]$
        
        $\vec{u}\times\vec{v}=[\answer{0},\answer{0},\answer{3}]$ with magnitude $|\vec{u}\times\vec{v}|=\answer{3}$
        
        \item $\vec{u}=[1,2,3]$ and $\vec{v}=[4,5,6]$
        
        $\vec{u}\times\vec{v}=[\answer{-3},\answer{6},\answer{-3}]$ with magnitude $|\vec{u}\times\vec{v}|=\answer[tolerance=.1]{7.35}$
        
        % \item $\vec{u}=[3,0,1]$ and $\vec{v}=[1,2,0]$
        
        % $\vec{u}\times\vec{v}=[\answer{-2},\answer{1},\answer{6}]$ with magnitude $|\vec{u}\times\vec{v}|=\answer[tolerance=.1]{6.40}$
        
        % \item $\vec{u}=[2,-1,3]$ and $\vec{v}=[1,4,-2]$
        
        % $\vec{u}\times\vec{v}=[\answer{-10},\answer{7},\answer{9}]$ with magnitude $|\vec{u}\times\vec{v}|=\answer[tolerance=.1]{13.93}$
        
        % \item $\vec{u}=[1,1,1]$ and $\vec{v}=[2,2,2]$
        
        % $\vec{u}\times\vec{v}=[\answer{0},\answer{0},\answer{0}]$ with magnitude $|\vec{u}\times\vec{v}|=\answer{0}$
        
        \begin{feedback}
            When two vectors are parallel (one is a scalar multiple of the other), their cross product is the zero vector.

            Remember to use the formula for the cross product given above, and that the magnitude of the vector can be found either using the norm formula or by finding the torque, $|\vec{u}||\vec{v}|\sin(\theta)$.
        \end{feedback}
    \end{enumerate}
\end{problem}

\end{document}