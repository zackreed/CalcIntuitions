\documentclass{ximera}
% \input{../../xmpreamble.tex}

\title{Activity: Representing and Applying Vectors}
\author{Zack Reed}

\begin{document}
\begin{abstract}
This  activity focuses on applying and interpreting vectors in real-world situations under two-dimensional and three-dimensional systems.
\end{abstract}
\maketitle


\section{Introduction}

This initial \textbf{individual} discussion activity spans two module weeks. You will have two tasks to complete for this discussion, detailed in the Discussion Tasks section. You can view and post to this discussion from either Module 1 or Module 2.

While it is very important to uncover basic geometric intuitions about vectors and how we scale them, combine them, and "multiply" them, being able to interpret and apply vectors to physical, technological, and social contexts is tantamount to the successful application of calculus in STEM. These interpretations and applications can, at times, require careful and systematic visualization and organization of the various vectors contributing to a system and can also require extra assumptions that enable a system to operate in a certain way.

Carefully setting up and labeling a system with multiple interacting vectors can potentially be complicated. Still, a useful strategy is to translate interacting vectors so that their tails meet at the origin. This, at the least, allows for an initial comparison of the magnitudes and directions all interacting with each other.

For instance, the rowing team members depicted in the following image are also moving to shore and being carried along by the river current. While we might represent the velocity vector of the river current in multiple ways, one simple way would be to represent both velocities as extending from the origin. The vector sum representing the boat's resulting velocity vector then also extends from the origin, and we have an easily discernible representation of the boat's trajectory.

% \includegraphics[width=589px]{boat_image.jpg}
% Two campers in a boat floating down the river.

We also must make some assumptions when considering physical systems. Often, we can think of systems as being in balance. For instance, if Tayler is carrying a box above their head, their shoulders and back bear the box's weight, as represented in the following image. The balance of the system can be represented by the assumption that the horizontal and vertical forces each sum to zero: $\sum_iF_{i,x}=0$ and $\sum_iF_{i,y}=0$ ($F_{i,x}$ and $F_{i,y}$ represent the horizontal and vertical force components for the ith force). If Tayler were to lift the box above their head, their arms would have to exert a force greater than the force calculated in equilibrium.

% \includegraphics[width=340px]{person_holding_box.jpg}
% A silhouette of a person holding a box up to their chest.

Similarly, a rotational system in balance has the property that torques sum to zero. If Riley and Cooper are suspended in the air on a seesaw, and the seesaw is not in motion, then we can assume that the opposing torques created by Riley and Cooper balance each other out around the pivot point. We can express this through the equation $\sum_i\tau_i=0$, where $\tau_i$ is any particular torque on the same pivot point.

% \includegraphics[width=454px]{seesaw.jpg}
% Two people on a seesaw.

Notice that $\tau_1$ and $\tau_2$ (Torque 1 and 2) must add to 0 for the seesaw to be in balance and that there is no torque at the pivot point because there is no radius extending towards the pivot force from the radius. Also note that because of the symmetry in the angles on the seesaw and because of the longer radius extending towards Weight 1, Weight 2 must be greater than Weight 1 for the system to be in balance.

\section{``Applied'' Contexts: 2D}

In this activity, you will solve the following three problems that apply and interpret vectors in "real-world" situations under a two-dimensional or three-dimensional system.

\subsection{2D}

The following problems begin in two-dimensional settings, we will then extend them to three dimensions.


\textbf{Problem 1:} You have installed a new solar panel on the roof of your house.

We will situate the $x$-$y$ plane so that the panel runs between the points $(0,3)$ and $(7,0)$. We want to know at what times of day the panel will receive the most sunlight. For simplicity, the curve $x^2+y^2=81$ determines the path from which we will measure the sunlight rays. The following GeoGebra application provides unit vectors that run along the normal lines to this curve at different points. However, you could also use tools from Calculus I to make these vectors (i.e., make normal lines and take unit vectors in the normal direction). These unit vectors represent a single "unit" of sunlight (loosely modeling the notion of sunlight intensity at varying times of day). The depicted rotation starts at noon and ends at 8 p.m.

\begin{center}
\geogebra{hcwcdekx}{739}{480}
\end{center}

Calculate how much of a "unit" of sunlight is absorbed by the panel at three different times along the path of the sun (Hint: absorption is modeled by the orientation of the vectors rather than by distances; determine the orientation of the sun's rays with respect to the orientation of the panel). Then, determine at what rotation(s) the solar panel will receive the most "units" of sunlight. You must justify your choice of rotation(s). (Note: You only need to work with the vectors listed in the applet. You can generalize this if you want to, but the problem only requires you to consider the vectors given as the rotation goes from 0 to 100 and to identify which of those vectors maximizes the sunlight "units").

\textbf{Problem 2:} A stoplight weighing 200 Newtons is hung between two poles that are 10 meters apart. One pole is 8 meters tall, and the other is 11 meters tall. The stoplight is hung by attaching one rope from the top of each pole to a ring at the top of the stoplight. The stoplight is to hang 6 meters off of the ground (measured from the ring).

As seen in the following image, the fixed dimensions of the pole heights and the stoplight height are in blue. The sample stop light location is given as $x$ meters from the left pole, and the resulting ropes have tensions $T_1$ and $T_2$ (all shown in orange). Other possible stoplight locations are depicted as well (that result from different values of $x$).

% \includegraphics[width=481px]{stoplights.jpg}
% Stoplights.

Give the location of the stoplight so that there is essentially equal tension (i.e., tension magnitudes) in both ropes. (Hint: At a particular $x$-value, there is a resulting system of equations by which you can solve for the tensions. You can use the MATLAB code provided in the MATLAB Support section to quickly solve multiple systems of equations and hone in on the desired location.)

\textbf{Problem 3:} A woman is moving to a new house and will spend much of her morning lifting boxes into a truck. She's always heard that she shouldn't "lift with her back" but wants to examine some of the physics at play. While on a break, she decides to compare the forces at play in two opposing lifting techniques.

For the "Proper Technique," she keeps her back essentially straight and vertical and uses her legs to lift the box from the ground. For the "Improper Technique," she keeps her legs straight, rotates downward (i.e., her hip is a pivot point) until reaching the box, and then rotates back up (as seen in the following GeoGebra applet).

You may alter the "Rotation=" slider to view different moments in the lifting motion. You may also click the "View Back Muscle Arc" button to zoom in around the motion of the back muscle. Given are the magnitudes of the displayed vectors and the tail-to-tail angle measures curling from the radial vectors to the weight vectors. The forces $F_1$ and $F_2$ are the weight forces of her torso and the carried box, respectively. The force $F_B$ is the force exerted by her back to enact this motion. The $r$-vectors are the radius vectors between the forces and the pivot point at her hip. The related angle measure moving from $r_B$ to $F_B$ is always $\pi/2$. Notice that the magnitude for $r_2$ changes throughout the rotation.

\begin{center}
\geogebra{hpenkqcp}{741}{559}
\end{center}

Your overall goal is to approximate the work done by her back while lifting with the "Improper Technique." As a warm-up, calculate the force exerted by her back to maintain the upright position represented by 100% rotation (simulating the "Proper Technique"). Then, calculate the force exerted by her back to maintain a 30% rotation (simulating part of the "Improper Technique").

Finally, approximate the work required to lift the box from 0% to 100% rotation in the "Improper Technique." To do this, click the "View Back Muscle Arc" button, then check the "Show Approximate Path" box. Using three random displacement vectors to approximate the rotation arc, re-apply the warm-up technique to calculate the magnitude of the back force at a single point in each displacement. Use this information to calculate the total approximate work done by the "Improper Lifting Technique." (Hint: Find a back force to uniformly apply over each displacement.)

\subsection{3D Extensions}

Now we tackle the same problems, but in three-dimensional settings.


\textbf{Problem 1 (Three-Dimensional Extension):}

You have installed a new solar panel on the roof of your house. We will use the same applet as before but orient the vertical axis as the $z$-axis, and the horizontal will remain the $x$-axis. Vectors in the applet, therefore, should now be interpreted as $\left(x,0,z\right)$ instead of just $\left(x,z\right)$.

In an attempt to compensate for the changing seasons, you are modifying your panel by adjusting its tilt in different directions and then assessing the efficiency of the new positions. The most recent orientation of the panel has the points $\left(0,0,3\right)$, $\left(7,0,0\right)$, and $\left(-4,4,2\right)$ as three of its four corners (making a rectangle). The previous panel was able to reach 100\% efficiency at some time during the day; can the new panel's orientation reach 100\% efficiency? If so, where? If not, what is the maximum efficiency of the panel, and where is it achieved?

\textbf{Problem 2 (Three-Dimensional Extension):}

A stoplight is hung between three poles scattered around an intersection. The tops of the poles are located at the points $\left(1,1,9\right)$, $\left(-1,1,8\right)$, and $\left(1,-1,10\right)$. The stoplight lies at the point $\left(x,y,6\right)$, where $-1\leq x\leq1$ and $x\leq y\leq1$. Assume the ropes are taught at any stoplight location $\left(x,y,6\right)$. At what coordinate $\left(x,y,6\right)$ will the tension magnitudes in each rope be equal (up to a reasonably small error) if the stoplight weighs 200N?

\end{document}