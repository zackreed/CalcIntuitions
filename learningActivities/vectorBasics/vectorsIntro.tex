\documentclass{ximera}
% \input{../../xmpreamble.tex}

\title{Introduction: Vector Basics}
\author{Zack Reed}

\begin{document}
\begin{abstract}
In this activity we ground the basics of vectors within physical processes that we've already encountered in calculus, and will expand the ideas to multiple dimensions.
\end{abstract}
\maketitle

\section*{Grounding Context: Motion, Forces, and Energy}
Hopefully thus far in calculus you have become accustomed to not just memorizing abstract symbols and formulas, but also interpreting the formulas in contexts relevant to engineering or the sciences. We'll continue to develop this practice in multivariable calculus, starting with motion, forces, and energy to give us intuitions for how and why vectors work the way they do.

First, here's a video that reviews motion, forces, and work.
%\youtube{}

Let's check our understanding of the basic ideas.

\begin{problem}
Riley and Taylor push a box across the floor between times $t=1$ and $t=3$, and the box moved from its position $d=5$ ft to a new position at $d=15$ ft. Select all from the following that describe the approximate velocity of the box.
\begin{selectAll}
\choice{$v=\frac{d}{t}$}
\choice[correct]{$v=\frac{10}{2} \text{ feet per second}$}
\choice{$v=\frac{15}{3} \text{ feet per second}$}
\choice[correct]{$v=\frac{\Delta v}{\Delta t}$}
\end{selectAll}
\begin{feedback}
Remember that the point of velocity is that it captures an object \emph{in motion}. That is, the object's distance and time quantities \emph{need to change} for a velocity to be quantified. 

The formula $v=\frac{d}{t}$ does not capture change, it just captures the values.
\end{feedback}
\end{problem}

\begin{problem}
Riley and Taylor push a box across the floor and exert a constant force of $F=10$ Newtons between distances $D=10$ and $D=30$ meters. Select all from the following that describe the approximate work done by Riley and Taylor on the box.

\begin{selectAll}
\choice{$W=F\cdot D$}
\choice[correct]{$10\cdot 20$ Joules}
\choice[correct]{$\Delta W=F\cdot \Delta D$}
\choice{$W=10\cdot 10$}
\choice{$W=10\cdot 30$}
\end{selectAll}
\begin{feedback}
Remember that work is a measure of energy expended, and that energy is expended when a force is applied over a distance. The formula $\Delta W=F\cdot \Delta D$ captures this idea.
\end{feedback}
\end{problem}


\section{Vectors: Magnitude and Direction}

The quantity we use to describe and measure multi-dimensional phenomena is the \emph{vector}. Vectors can be described in many ways, but we are actually quite familiar with the basic idea: vectors are quantities of directed motion. That is, what matters about a vector is how big it is and in what direction it's going.

Describing change is nothing new to calculus. We have, in fact, been working all along with change amounts, particularly when changes occur along line segments. 

Here's a short video going over this transition from directed change in one dimension to directed change in multiple dimensions.

%\youtube{}
\begin{expandable}{example}{Test Expand}
    expandable
\end{expandable}

This kind of change is depicted in the following GeoGebra applet. Click "Animate" to show the vector giving the change from $P$ to $Q$.

\begin{center}
\geogebra{vup8fp9rbw}{790}{434}
\end{center}

What matters about the vector is \emph{how big it is}, which we can measure with the Pythagorean Theorem, and \emph{in what direction it points}, which we can measure with angles.

%NEXT do the geogebra applet again but use angle measure from the infinite trig applet to show the angle of the vector, have them move the points around and log the lengths and angles.
Here is the same vector depicted in the GeoGebra applet, showing the length and the angle measure. 

\begin{center}
\geogebra{zd8jmsz4}{789}{461}
\end{center}

Another way to view a vector describing the change from $P$ to $Q$ is as the motion along the directed line segment connecting $P$ to $Q$, as is depicted in the following GeoGebra application.

\begin{center}
\geogebra{egprdk2x}{789}{461}
\end{center}

In MATH 241 and 242, we used $dx$ to denote a horizontal change (i.e., in $x$, usually in the forward direction) and used $dy$ to represent a vertical change (i.e., in $y$, usually in the upwards direction). When working with lines (tangent or otherwise), we often used a constant rate of change relationship between $x$ and $y$ to our advantage, which was expressed as the relationship $dx=r\cdot dy$.

There is little difference when we transition to talking about vectors in 2 and 3 dimensions; we simply use the notation $\langle x,y\rangle$ (rather than $dx$ and $dy$) to convey how much horizontal and vertical change occurs to move between points. We will still be using $dx$ and $dy$, particularly when we begin making local approximations on multivariable functions and surfaces. For now, we will use the notation $\langle x,y\rangle$ to denote directionality, and will also be considering the size of a change (i.e., the length of the directed line segment over which the change occurs).

We also will begin considering three-dimensional space, which involves coordinating three directions at once, building from our usual $x$ and $y$ directions by implementing a third direction to consider, $z$. Vectors in three dimensions are represented directionally according to how much change occurs in each direction, making vectors representable by the triples $\langle x,y,z\rangle$, as seen in the following GeoGebra application.

\begin{center}
\geogebra{prfu6rmh}{780}{438}
\end{center}

\begin{exercise}
    $3+1=\answer{4}$
\end{exercise}

\begin{multipleChoice}
\choice{$|\vec{u} + \vec{v}| = |\vec{u}| + |\vec{v}|$}
\choice[correct]{$|\vec{u}| = \sqrt{3^2 + 4^2} = 5$}
\choice{$|2\vec{u}| = 2|\vec{u}|^2$}
\choice{$|\vec{u} \cdot \vec{v}| = |\vec{u}| \cdot |\vec{v}|$}
\choice{$|\vec{u}| = 3 + 4 = 7$}
\end{multipleChoice}

\section{Multiplication Part 1 - Dot Product}

With the advent of new objects to work with (i.e., vectors) comes the possibility of new operations to perform. Section 12.2 covered the basics of vector addition and scalar multiplication. Both operations largely followed the conventions of addition and multiplication of numbers. We now raise a very natural question: Can we "multiply" two vectors together, and if so, what is the result of such "multiplication"?

As it turns out, we need to do slightly more than simply multiply the components of two vectors together to yield a quantity that is initially useful for us. There are actually two different ways that we might "multiply" vectors together. One such multiplication is called the "Dot Product" and yields a number as the result of the vector "multiplication." The other is called the "Cross Product," which yields a new vector. We will first focus on the dot product.

If we have two vectors, $\vec u$ and $\vec v$, that we wish to "multiply" together with the dot product, rather than simply multiplying the components of $\vec u$ and $\vec v$ together into a new vector, we add the products of the components of $\vec u$ and $\vec v$ to make a measurement on the two vectors. For instance, if we dot $\vec u=\langle 1,2\rangle$ with $\vec v=\langle 4,1\rangle$, we get the product $\vec u\cdot\vec v=1\cdot 4+2\cdot 1=6$. The dot product allows us to compare two vectors to each other simultaneously.

While there are multiple ways to define, compute, and think about the dot product, one fundamental question that the dot product answers is "How much of one vector travels in the direction of another vector?". If we have two vectors, $\vec u$ and $\vec v$, then the dot product $\vec u\cdot\vec v$ allows us to ask "How much of $\vec u$ is traveling in the direction of $\vec v$?" and vice versa. This process is called "projecting $\vec u$ onto $\vec v$" and "projecting $\vec v$ onto $\vec u$", which is exemplified in the following GeoGebra application. The projection of $\vec u$ onto $\vec v$ is denoted $\text{Proj}_u(v)$, and the projection of $\vec v$ onto $\vec u$ is denoted $\text{Proj}_v(u)$.

You may drag the points $U$ and $V$ at the tips of vectors $\vec u$ and $\vec v$ to alter their location. You may either drag them horizontally or vertically at one time. Clicking on the point lets you choose whether you alter it horizontally or vertically. You may also select the "Show $\text{Proj}_u(v)$" and "Show $\text{Proj}_v(u)$" to see $\vec u$ projected onto $\vec v$ and vice versa. The projections are given in red.

\begin{center}
\geogebra{uvyvxkjz}{683}{539}
\end{center}

The dot product $\vec u\cdot\vec v$ is what primarily determines the size of each projection and answers the fundamental question: "How much of $\vec u$ is traveling in the direction of $\vec v$?". This is intimately related to the angle between the vectors, which will be more fully explored in your reading.

For now, notice that if the vectors share an angle of $\frac{\pi}{2}$, then the vectors do not project onto each other at all (except for at the origin). For this reason, if the angle between $\vec u$ and $\vec v$ is $\frac{\pi}{2}$, then $\vec u\cdot\vec v=0$. Similarly, if the vectors lie on top of each other, then $\vec u\cdot\vec v=|u||v|$, the product of the vectors' magnitudes. This is one indication that the product is inherently tied to the information gained by the original operation of multiplication.

\section{Multiplication Part 2 - Cross Product}

The other way we might "multiply" vectors together is by returning to the area-based interpretation of multiplication. One of the fundamental ways that we interpret the product of two numbers is as the area of a rectangle with height $a$ and base $b$. One nice property of areas is that, even when slanted into parallelograms, quadrangular shapes always have the same area calculation $ab$, where $a$ is the height and $b$ is the base length.

As seen in the following GeoGebra applet, the area of parallelograms depends on the base height product, where the height is perpendicular to the base. You may alter the "$\theta$=" slider and the position of the red point to alter the parallelogram.

\begin{center}
\geogebra{hccaztdd}{789}{461}
\end{center}

In each case, the area of the parallelogram remains the product of its base and its height. You might imagine this by moving the triangle (created by the vertical height on the left side) to fit in the empty space on the right of the parallelogram, making a rectangle.

If we take the lengths of vectors $u$ and $v$ as giving the sides of the parallelogram, then the product $|u||v|\sin(\theta)$ gives the area of the generated parallelogram and thus preserves the area interpretation of the "vector product" that we call the "cross product".

A very useful result of the cross product is a created relationship between $u$, $v$, and $w$, where $w$ is a vector normal to the plane created by $u$ and $v$. Specifically, if we take $n$ to be the unit normal vector to the plane created by $u$ and $v$, then we can make a new vector from the calculation $|u||v|\sin(\theta)n$. We call this vector the "cross product" and denote it $|u||v|\sin(\theta)n=u\times v$. Notice that the magnitude of $u\times v$ is the same as the area of the parallelogram spanned by $u$ and $v$, thus, preserving the area interpretation.



\section{Video Resources}


Visit the Calculus III: Multivariable Calculus playlist by Dr. Trefor Bazett, found on YouTube, for further video resources on the big-picture ideas of multivariable calculus.

\end{document}