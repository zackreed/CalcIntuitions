\documentclass{ximera}
% \input{../../xmpreamble.tex}

\title{Introduction: Vector Basics}
\author{Zack Reed}

\begin{document}
\begin{abstract}
In this activity we ground the basics of vectors within physical processes that we've already encountered in calculus, and will expand the ideas to multiple dimensions.
\end{abstract}
\maketitle

\section*{Grounding Context: Motion, Forces, and Energy}
Hopefully thus far in calculus you have become accustomed to not just memorizing abstract symbols and formulas, but also interpreting the formulas in contexts relevant to engineering or the sciences. We'll continue to develop this practice in multivariable calculus, starting with motion, forces, and energy to give us intuitions for how and why vectors work the way they do.

First, here's a video that reviews motion, forces, and work.

\begin{center}
    \youtube{SDh29BKsZ6U}
\end{center}

Let's check our understanding of the basic ideas.

\begin{problem}
Riley and Taylor push a box across the floor between times $t=1$ and $t=3$, and the box moved from its position $D=5$ ft to a new position at $D=15$ ft. Select all from the following that describe the approximate velocity of the box.
\begin{selectAll}
\choice{$v=\frac{d}{t}$}
\choice[correct]{$v=\frac{10}{2} \text{ feet per second}$}
\choice{$v=\frac{15}{3} \text{ feet per second}$}
\choice[correct]{$v=\frac{\Delta D}{\Delta t}$}
\end{selectAll}
\begin{feedback}
Remember that the point of velocity is that it captures an object \emph{in motion}. That is, the object's distance and time quantities \emph{need to change} for a velocity to be quantified. 

The formula $v=\frac{d}{t}$ does not capture change, it just captures the values.
\end{feedback}
\end{problem}

\begin{problem}
Riley and Taylor push a box across the floor and exert a constant force of $F=10$ Newtons between distances $D=10$ and $D=30$ meters. Select all from the following that describe the approximate work done by Riley and Taylor on the box.

\begin{selectAll}
\choice{$W=F\cdot D$}
\choice[correct]{$W=10\cdot 20$ Joules}
\choice[correct]{$\Delta W=F\cdot \Delta D$}
\choice{$W=10\cdot 10$ Joules}
\choice{$W=10\cdot 30$ Joules}
\end{selectAll}
\begin{feedback}
Remember that work is a measure of energy expended, and that energy is expended when a force is applied over a distance. The formula $\Delta W=F\cdot \Delta D$ captures this idea.
\end{feedback}
\end{problem}


\section*{Vectors: Magnitude and Direction}

The quantity we use to describe and measure multi-dimensional phenomena is the \emph{vector}. 

Vectors can be described in many ways, but we are actually quite familiar with the basic idea: \emph{vectors are quantities of directed motion}. That is, what matters about a vector is {\bf how big it is} and {\bf in what direction it's going}.

\section*{Introducing Vectors: Riley and Taylor Pushing Boxes}

    Let's continue exploring Riley and Taylor's box pushing, but as they vary along a plane.

\begin{problem}
    Let's check out understanding: According to typical conventions in 1-dimensional motion, a displacement of $-5$m is movement \wordChoice{
            \choice[correct]{left by 5 meters}
            \choice{right by 5 meters}
        }
\end{problem}
    
    Now let's think about movement within a 2D plane.

    \begin{center}
        \youtube{PC3smGIa8dE}
    \end{center}

\begin{problem}
    (Refer to the end of the previous video to answer the following question)
    
    Suppose Riley is in space, and is pushing the box at a speed of $5$m/s. When will Riley set the box on the target $10$m away?
        \begin{multipleChoice}
            \choice{Riley will reach the target in $2$ seconds}
            \choice[correct]{Riley cannot reach the target following his current direction}
            \choice{Riley will reach the target in $5$ seconds}
            \choice[correct]{There is not enough information to determine when Riley will reach the target}
        \end{multipleChoice}
    \begin{feedback}
        There are two correct ways of thinking about this. If Riley pushes straight, as oriented in the video, he will never reach the target because his direction is not aimed at the target. 

        Alternatively, you can correctly conclude that there is not enough information to determine when Riley will reach the target, because we don't know if Riley will adjust his direction to aim at the target.
    \end{feedback}
\end{problem}

    Now let's continue

\begin{center}
\youtube{MtL6SCW8xbY}
\end{center}

\begin{problem}
    Suppose Riley is pushing the box horizontally at a speed of $-5$m/s and Taylor is pulling the box vertically at a speed of $2$m/s. When will the box reach teh target, located $100$m left and $40$m up from their location?

    Answer: The box will reach the target in $\answer{20}$ seconds.

    \begin{feedback}
    \begin{center}
        \youtube{youtu.be/WEYsYt6ImZ4}
    \end{center}
    \end{feedback}

\end{problem}

\subsection*{Vectors: Some Practice}

    Now let's learn about the notation and geometry of vectors, and then check our understanding.

\begin{problem}

The vector $\vec{v}=[25, -2]$ represents \begin{multipleChoice}
    \choice{movement $25$ units upward and $2$ units to the left}
    \choice{movement $25$ units to the left and $2$ units downward}
    %finish the problem
    \choice[correct]{movement $25$ units to the right and $2$ units downward}
    \choice{movement $25$ units downward and $2$ units to the right}
\end{multipleChoice}

The vector $\vec{u}=[-3,4]$ is the sum of the vectors $\vec{x}=[-3,\answer{0}]$ and $\vec{y}=[\answer{0},4]$. Symbolically, $\vec{u}=\vec{\answer{x}}+\vec{\answer{y}}$.

\begin{feedback}
Remember that the first component of a vector represents horizontal movement (negative: left, positive: right), and the second component represents vertical movement (negative: down, positive: up).
\end{feedback}

\end{problem}

Even though much of the time vectors will start at the origin, you often want to think about vectors as representing a change between two points. In such cases, the $x$-component of the vector is the amount of horizontal change, and the $y$-component is the amount of vertical change.

\begin{problem}
    Use the following GeoGebra applet to find the vector $\vec{v}$ that represents the change from points $A$ to $B$.

    \begin{expandable}{stuff}{GeoGebra Instructions}
        You can click and drag the points at the end of the vector to change their locations. You can click Animate to show the vector being created. 
    \end{expandable}

    \begin{center}
        \geogebra{jrvzw9gu}{643}{321}
    \end{center}

    \begin{enumerate}
        \item The vector going from $A=(1/2,1)$ to $B=(3/2,2)$ is given by $\vec{v}=[\answer{1},\answer{1}]$.
        \item The vector going from $A=(-1/2,1)$ to $B=(3/2,1/2)$ is given by $\vec{v}=[\answer{2},\answer{-1/2}]$.
    \end{enumerate}
    \begin{feedback}
        Remember that the first component of a vector represents horizontal movement (negative: left, positive: right), and the second component represents vertical movement (negative: down, positive: up).
    \end{feedback}
\end{problem}

The same structure is true for three-dimensional vectors, but now we have a third component that represents movement in the $z$-direction (negative: down, positive: up).

\begin{problem}
    Use the following GeoGebra applet to find the vector $\vec{v}$ that represents the change from points $A$ to $B$. 
    
    \begin{expandable}{stuff}{GeoGebra Instructions}
        Clicking on a point multiple times lets you switch between horizontal and vertical change. 
        
        Right clicking and dragging the right screen lets you alter the view.
    \end{expandable}

    \begin{center}
        \geogebra{dezaean3}{602}{384}
    \end{center}

    \begin{enumerate}
        \item The vector going from $A=(-1,2,2)$ to $B=(1,1,1)$ is given by $\vec{v}=[\answer{2},\answer{-1},\answer{-1}]$.
        \item The vector going from $A=(1,-1,0)$ to $B=(-1,1,-1)$ is given by $\vec{v}=[\answer{-2},\answer{2},\answer{-1}]$.
    \end{enumerate}
    \begin{feedback}
        Remember that the first component of a vector represents horizontal movement (negative: left, positive: right), the second component represents vertical movement (negative: down, positive: up), and the third component represents depth movement (negative: backward, positive: forward).
    \end{feedback}
\end{problem}


\end{document}