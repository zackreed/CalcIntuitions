\documentclass{ximera}

\title{Change of Variables in Multiple Integrals}
\author{Zack Reed}

\begin{document}
\begin{abstract}
In this activity we explore how to change coordinate systems in multiple integrals, extending u-substitution from single-variable calculus to transformations in higher dimensions.
\end{abstract}
\maketitle

\section*{Introduction: Why Change Variables?}

In single-variable calculus, u-substitution transformed difficult integrals into easier ones. For multiple integrals, changing coordinate systems serves the same purpose!

\begin{problem}
Recall u-substitution from Calculus I:
$$\int f(g(x)) g'(x)\,dx = \int f(u)\,du \text{ where } u = g(x)$$

The key insight: \wordChoice{\choice{We make the integral harder}\choice[correct]{We transform to simpler coordinates}\choice{We change the function}}

What's the role of $g'(x)$?
\begin{multipleChoice}
    \choice{It makes the problem harder}
    \choice[correct]{It accounts for how $dx$ transforms to $du$}
    \choice{It's optional}
    \choice{It cancels out}
\end{multipleChoice}

\begin{feedback}
The derivative $g'(x)$ accounts for the ``stretching'' or ``compression'' when we change variables! In multiple dimensions, we'll need something similar.
\end{feedback}
\end{problem}


\section*{Polar Coordinates}

For circular or radial regions, polar coordinates often simplify the integral dramatically.

\begin{problem}
Why use polar coordinates for integration?
\begin{selectAll}
    \choice[correct]{Circular regions have simpler bounds: $0 \leq r \leq R$, $0 \leq \theta \leq 2\pi$}
    \choice{Polar coordinates always make integrals easier}
    \choice[correct]{Functions with $x^2 + y^2$ simplify to $r^2$}
    \choice[correct]{Radial symmetry becomes apparent}
\end{selectAll}

\begin{feedback}
Polar coordinates are ideal when the region or function has circular/radial symmetry!
\end{feedback}
\end{problem}

\subsection*{The Polar Area Element}

\begin{problem}
The key question: What is $dA$ in polar coordinates?

\begin{expandable}{stuff}{GeoGebra Instructions}
    Use the ``Shift Radius'' and ``Shift $\theta$'' sliders to move the area element $dA$ around. Check ``Show Volume'' to see the approximating cylinder. Use ``Zoom to Unit Circle'' and drag to view from above.
\end{expandable}

\begin{center}
\geogebra{uucwvg4g}{740}{481}
\end{center}

The area element $dA$ is bounded between:
\begin{itemize}
    \item Circles of radius $r - \frac{dr}{2}$ and $r + \frac{dr}{2}$
    \item Angles $\theta - \frac{d\theta}{2}$ and $\theta + \frac{d\theta}{2}$
\end{itemize}

The area of this annular sector is approximately:
$$dA = \frac{d\theta}{2}(r+\frac{dr}{2})^2 - \frac{d\theta}{2}(r-\frac{dr}{2})^2$$

After algebra, this simplifies to: $dA = \answer{r} \cdot dr \cdot d\theta$

The extra factor $r$ is crucial! Without it:
\begin{multipleChoice}
    \choice{The integral would be easier}
    \choice[correct]{We'd get the wrong answer}
    \choice{Nothing would change}
\end{multipleChoice}

\begin{feedback}
The polar area element is $dA = r\,dr\,d\theta$. That extra $r$ accounts for the curved geometry!
\end{feedback}
\end{problem}

\begin{problem}
\textbf{Heat Distribution in a Circular Plate}

A circular metal plate of radius 3 meters has temperature $T(r,\theta) = 50 + 20\cos(2\theta)$ degrees Celsius, where $(r,\theta)$ are polar coordinates.

The heat energy density (energy per unit area) is $E(r,\theta) = cT(r,\theta)$ where $c$ is a constant.

\textbf{Construct the differential element carefully:}

In polar coordinates, the area element is: $dA = \answer{r}\,dr\,d\theta$

Why not just $dr\,d\theta$?
\begin{multipleChoice}
    \choice{It's a convention}
    \choice{To make the integral harder}
    \choice[correct]{The ``width'' of an angular slice is $r\,d\theta$, not $d\theta$}
    \choice{It's only true for circles}
\end{multipleChoice}

The small heat energy is: $dE = E(r,\theta) \cdot dA = c(50 + 20\cos(2\theta)) \cdot \answer{r}\,dr\,d\theta$

Total heat energy in the plate:
$$E_{\text{total}} = \int_{\answer{0}}^{\answer{2\pi}} \int_{\answer{0}}^{\answer{3}} c(50 + 20\cos(2\theta)) \cdot r\,dr\,d\theta$$

The temperature varies with $\theta$ but not $r$, meaning:
\begin{multipleChoice}
    \choice{The plate is hotter at the edge}
    \choice[correct]{Temperature varies around the circle but is constant along any ray from the center}
    \choice{The plate is uniformly heated}
    \choice{There's an error in the problem}
\end{multipleChoice}

\begin{feedback}
Great work! You recognized that in polar coordinates, the area element $dA = r\,dr\,d\theta$ accounts for the geometry. The factor of $r$ appears because angular slices get wider as you move away from the origin!
\end{feedback}
\end{problem}

\subsection*{Polar Integral Example}

\begin{problem}
Let's integrate $f(x,y) = xy$ over the unit disk $x^2 + y^2 \leq 1$.

\textbf{In rectangular coordinates (yuck!):}
$$\int_{-1}^1\int_{-\sqrt{1-x^2}}^{\sqrt{1-x^2}} xy\,dy\,dx$$

Those square root bounds are messy!

\textbf{In polar coordinates:}
\begin{itemize}
    \item Region: $0 \leq r \leq \answer{1}$, $0 \leq \theta \leq \answer{2\pi}$
    \item Function: $f(x,y) = xy = (r\cos\theta)(r\sin\theta) = \answer{r^2 \sin\theta \cos\theta}$
    \item Area element: $dA = \answer{r}\,dr\,d\theta$
\end{itemize}

The integral becomes:
$$V = \int_0^{2\pi}\int_0^1 r^2\sin\theta\cos\theta \cdot r\,dr\,d\theta = \int_0^{2\pi}\int_0^1 r^3\sin\theta\cos\theta\,dr\,d\theta$$

\textbf{Inner integral (with respect to $r$):}
$$\int_0^1 r^3\,dr = \left[\frac{r^4}{4}\right]_0^1 = \answer{1/4}$$

\textbf{Outer integral (with respect to $\theta$):}
$$\int_0^{2\pi} \frac{1}{4}\sin\theta\cos\theta\,d\theta = \frac{1}{4}\int_0^{2\pi} \frac{1}{2}\sin(2\theta)\,d\theta$$
$$= \frac{1}{8}\left[-\frac{1}{2}\cos(2\theta)\right]_0^{2\pi} = \frac{1}{8} \cdot 0 = \answer{0}$$

Why is the volume zero?
\begin{multipleChoice}
    \choice{We made a calculation error}
    \choice[correct]{The function $xy$ is negative in two quadrants and positive in two quadrants, so they cancel}
    \choice{The unit disk has zero area}
\end{multipleChoice}

\begin{feedback}
The symmetry of $xy$ means equal positive and negative volumes that cancel! Polar coordinates made this much cleaner to compute.
\end{feedback}
\end{problem}


\section*{Review: Polar Coordinates Revisited}

Let's examine what happens when we transform from rectangular to polar coordinates.

\begin{problem}
The transformation from polar to rectangular coordinates is:
\begin{align*}
x &= r\cos\theta\\
y &= r\sin\theta
\end{align*}

This transformation maps:
\begin{itemize}
    \item The $r\theta$-plane (polar) $\to$ The $xy$-plane (rectangular)
\end{itemize}

What happens to a small rectangle in the $r\theta$-plane?
\begin{multipleChoice}
    \choice{It stays a rectangle}
    \choice[correct]{It becomes a curved shape}
    \choice{It disappears}
    \choice{It becomes a circle}
\end{multipleChoice}

The area element transforms as:
$$dA = dx\,dy = \answer{r}\,dr\,d\theta$$

Why the factor of $r$?
\begin{selectAll}
    \choice[correct]{It accounts for the curved geometry}
    \choice[correct]{It measures how areas change under the transformation}
    \choice{It's arbitrary}
    \choice[correct]{It's the Jacobian determinant for this transformation}
\end{selectAll}

\begin{feedback}
The factor $r$ tells us how much a small area in the $r\theta$-plane expands or contracts when mapped to the $xy$-plane!
\end{feedback}
\end{problem}


\section*{Practice Problems}

\begin{problem}
Compute $\displaystyle\int_0^{\pi/2}\int_0^{\cos\theta} r^2\,dr\,d\theta$ in polar coordinates.

\textbf{Inner integral:}
$$\int_0^{\cos\theta} r^2\,dr = \left[\frac{r^3}{3}\right]_0^{\cos\theta} = \answer{\cos^3\theta / 3}$$

\textbf{Outer integral:}
$$\int_0^{\pi/2} \frac{\cos^3\theta}{3}\,d\theta = \frac{1}{3}\int_0^{\pi/2} \cos^3\theta\,d\theta$$

Using $\cos^3\theta = \cos\theta(1-\sin^2\theta)$ and substitution $u = \sin\theta$:
$$= \frac{1}{3}\left[\sin\theta - \frac{\sin^3\theta}{3}\right]_0^{\pi/2} = \frac{1}{3}\left(1 - \frac{1}{3}\right) = \answer{2/9}$$

\begin{feedback}
Great work! This integral computed the volume of a region bounded by a cosine curve in polar coordinates.
\end{feedback}
\end{problem}

\begin{problem}
Set up (but don't evaluate) the integral for the area of a circle of radius $R$ using polar coordinates.

Area formula: $A = \iint_R dA$

In polar: $A = \int_{\answer{0}}^{\answer{2\pi}}\int_{\answer{0}}^{\answer{R}} \answer{r}\,dr\,d\theta$

Evaluating quickly:
$$A = \int_0^{2\pi} \left[\frac{r^2}{2}\right]_0^R d\theta = \int_0^{2\pi} \frac{R^2}{2}\,d\theta = \frac{R^2}{2} \cdot 2\pi = \answer{\pi R^2}$$

Does this match the familiar formula? \wordChoice{\choice[correct]{Yes!}\choice{No}}

\begin{feedback}
Perfect! Double integrals can compute areas (by integrating $f=1$), volumes (by integrating $f(x,y)$), and many other quantities!
\end{feedback}
\end{problem}


\textbf{Mass of Fluid in a Rotating Container}

A cylindrical container (radius 2 m, height 5 m) rotates, causing the fluid density to vary:
$$\delta(r,z) = 1000 + 50r^2 - 20z \text{ kg/m}^3$$

where $(r, \theta, z)$ are cylindrical coordinates.

In cylindrical coordinates, the volume element is:
$$dV = \answer{r}\,dr\,d\theta\,dz$$

Why does $r$ appear here?
\begin{multipleChoice}
    \choice{It's always there in 3D}
    \choice[correct]{The cylindrical shell at radius $r$ has circumference $2\pi r$, making volume proportional to $r$}
    \choice{To account for rotation}
    \choice{It's a mistake}
\end{multipleChoice}

The small mass element: $dM = \delta(r,z) \cdot dV = (1000 + 50r^2 - 20z) \cdot \answer{r}\,dr\,d\theta\,dz$

The integral setup for total mass:
$$M = \int_{\answer{0}}^{\answer{2\pi}} \int_{\answer{0}}^{\answer{5}} \int_{\answer{0}}^{\answer{2}} (1000 + 50r^2 - 20z) \cdot r\,dr\,dz\,d\theta$$

Notice the density:
\begin{selectAll}
    \choice[correct]{Increases with distance from the axis (centrifugal effect)}
    \choice[correct]{Decreases with height (pressure effect)}
    \choice{Depends on angle $\theta$}
    \choice[correct]{Is independent of angle (axial symmetry)}
\end{selectAll}

\begin{feedback}
Perfect! You've unpacked $dM = \delta \cdot dV$ in cylindrical coordinates, recognizing that $dV = r\,dr\,d\theta\,dz$ reflects the geometry of cylindrical shells. The density function models realistic physical effects in a rotating container!
\end{feedback}
\end{problem}


\section*{Application: Mass of 3D Objects}

\begin{problem}
We'll work with two shapes with variable density:

\textbf{Shape 1 (Cone):} 
\begin{itemize}
    \item Height: 5 meters
    \item Base radius: 4 meters
    \item Density: $\delta = 3 + 2h$ kg/m$^3$, where $h$ is height from tip
\end{itemize}

\textbf{Shape 2 (Sphere):}
\begin{itemize}
    \item Radius: 5 meters
    \item Density: $\delta = 3 + 2\rho$ kg/m$^3$, where $\rho$ is distance from center
\end{itemize}

For the cone, which coordinate system is most natural?
\begin{multipleChoice}
    \choice{Rectangular $(x,y,z)$}
    \choice[correct]{Cylindrical $(r,\theta,z)$}
    \choice{Spherical $(\rho,\phi,\theta)$}
\end{multipleChoice}

For the sphere, which coordinate system is most natural?
\begin{multipleChoice}
    \choice{Rectangular $(x,y,z)$}
    \choice{Cylindrical $(r,\theta,z)$}
    \choice[correct]{Spherical $(\rho,\phi,\theta)$}
\end{multipleChoice}

\begin{feedback}
Choose coordinates that match the geometry! Cones have circular cross-sections (cylindrical), spheres have radial symmetry (spherical).
\end{feedback}
\end{problem}

\begin{center}
\geogebra{ssum9mnc}{885}{543}
\end{center}

\begin{expandable}{stuff}{GeoGebra Instructions}
\begin{itemize}
    \item Check ``Show Cone'' or ``Show Sphere'' to view each shape
    \item Drag the 3D view to rotate; scroll to zoom
    \item Use ``Zoom Out'' for global view, ``Zoom In'' for detail
    \item Adjust sliders to move the volume element $dV$
    \item Observe how $dV$ changes position in cylindrical (cone) or spherical (sphere) coordinates
\end{itemize}
\end{expandable}

\begin{problem}
\textbf{Part A: Finding Mass}

For the cone with density $\delta = 3 + 2z$ (where $z$ is height from tip):

In cylindrical coordinates, the volume element is: $dV = \answer{r}\,dr\,d\theta\,dz$

The mass element is: $dM = \delta \cdot dV = (3 + 2z) \cdot \answer{r}\,dr\,d\theta\,dz$

At height $z$, the cone's radius varies. If the base has radius 4 m at height 5 m, then at height $z$:
$$r_{\text{max}}(z) = \frac{4z}{5}$$

The mass integral setup is:
$$M = \int_0^{\answer{2\pi}} \int_0^{\answer{5}} \int_0^{4z/5} (3+2z) \cdot r\,dr\,dz\,d\theta$$

\begin{feedback}
Setting up the integral requires understanding the geometry and choosing bounds carefully!
\end{feedback}
\end{problem}


\section*{Cylindrical Coordinates}

For regions with cylindrical symmetry (cylinders, cones), cylindrical coordinates simplify the integral.

\begin{definition}
\textbf{Cylindrical coordinates} $(r, \theta, z)$ extend polar coordinates into 3D:
\begin{itemize}
    \item $x = r\cos\theta$
    \item $y = r\sin\theta$
    \item $z = z$
\end{itemize}

The volume element is: $dV = r\,dr\,d\theta\,dz$
\end{definition}

\begin{problem}
Cylindrical coordinates are built from:
\begin{selectAll}
    \choice[correct]{Polar coordinates $(r,\theta)$ in the $xy$-plane}
    \choice[correct]{Height $z$ perpendicular to the plane}
    \choice{Spherical shells}
    \choice[correct]{The familiar extra factor of $r$ from polar integrals}
\end{selectAll}

The volume element $dV = r\,dr\,d\theta\,dz$ comes from:
\begin{multipleChoice}
    \choice{Multiplying all three differentials}
    \choice[correct]{Taking a polar area element $dA = r\,dr\,d\theta$ and extending it by height $dz$}
    \choice{Spherical geometry}
\end{multipleChoice}

\begin{feedback}
Cylindrical coordinates = polar coordinates + vertical height! The volume element is literally $(r\,dr\,d\theta) \times dz$.
\end{feedback}
\end{problem}

\begin{problem}
Explore cylindrical coordinates visually.

\begin{expandable}{stuff}{GeoGebra Instructions}
    Select ``Show Cylinder'' view. Use sliders to change $r$, $\theta$, and $z$. Click ``Zoom to Volume Element'' to see the small piece $dV = r\,dr\,d\theta\,dz$.
\end{expandable}

\begin{center}
\geogebra{xqrs44yg}{750}{675}
\end{center}

As you vary $r$, $\theta$, $z$, observe:
\begin{selectAll}
    \choice[correct]{$r$ controls distance from the $z$-axis}
    \choice[correct]{$\theta$ controls rotation around the $z$-axis}
    \choice[correct]{$z$ controls height}
    \choice{The volume element shape is always a cube}
\end{selectAll}

\begin{feedback}
The volume element in cylindrical coordinates is a tiny curved box that fits the cylindrical geometry!
\end{feedback}
\end{problem}

\begin{problem}
Find the volume of a cylinder of radius $R$ and height $H$.

In cylindrical coordinates:
\begin{itemize}
    \item $0 \leq r \leq \answer{R}$
    \item $0 \leq \theta \leq \answer{2\pi}$
    \item $0 \leq z \leq \answer{H}$
\end{itemize}

$$V = \int_0^{2\pi}\int_0^R\int_0^H r\,dz\,dr\,d\theta$$

\textbf{Inner integral:}
$$\int_0^H r\,dz = r \cdot H = \answer{rH}$$

\textbf{Middle integral:}
$$\int_0^R rH\,dr = H \left[\frac{r^2}{2}\right]_0^R = \answer{HR^2/2}$$

\textbf{Outer integral:}
$$\int_0^{2\pi} \frac{HR^2}{2}\,d\theta = \frac{HR^2}{2} \cdot 2\pi = \answer{\pi R^2 H}$$

Does this match the familiar cylinder volume formula? \wordChoice{\choice[correct]{Yes!}\choice{No}}

\begin{feedback}
Perfect! Cylindrical coordinates make this integral trivial with constant bounds.
\end{feedback}
\end{problem}

\section*{Spherical Coordinates}

For regions with spherical symmetry (spheres, cones), spherical coordinates are ideal.

\begin{definition}
\textbf{Spherical coordinates} $(\rho, \phi, \theta)$ describe a point by:
\begin{itemize}
    \item $\rho$ = distance from origin
    \item $\phi$ = angle down from positive $z$-axis
    \item $\theta$ = angle of rotation around $z$-axis (same as in cylindrical)
\end{itemize}

Conversion formulas:
\begin{itemize}
    \item $x = \rho\sin\phi\cos\theta$
    \item $y = \rho\sin\phi\sin\theta$
    \item $z = \rho\cos\phi$
\end{itemize}

The volume element is: $dV = \rho^2\sin\phi\,d\rho\,d\phi\,d\theta$
\end{definition}

\begin{problem}
The spherical volume element $dV = \rho^2\sin\phi\,d\rho\,d\phi\,d\theta$ has:

An extra factor of $\rho^2$: \wordChoice{\choice[correct]{Yes}\choice{No}}

An extra factor of $\sin\phi$: \wordChoice{\choice[correct]{Yes}\choice{No}}

These factors account for:
\begin{multipleChoice}
    \choice{Mathematical complexity}
    \choice[correct]{The geometry of spherical shells and how they expand with radius}
    \choice{Random constants}
\end{multipleChoice}

\begin{feedback}
The $\rho^2\sin\phi$ factor is crucial! It accounts for how spherical volume elements grow with radius and latitude.
\end{feedback}
\end{problem}

\begin{problem}
Explore spherical coordinates visually.

\begin{expandable}{stuff}{GeoGebra Instructions}
    Select ``Show Sphere'' view. Use sliders for $\rho$, $\phi$, $\theta$. Zoom to see the volume element shape—it's like a tiny curved box on a sphere!
\end{expandable}

\begin{center}
\geogebra{xqrs44yg}{750}{675}
\end{center}

In spherical coordinates:
\begin{selectAll}
    \choice[correct]{$\rho$ measures distance from origin}
    \choice[correct]{$\phi$ measures angle from the positive $z$-axis (colatitude)}
    \choice[correct]{$\theta$ measures rotation around $z$-axis (azimuth)}
    \choice{All angles range from $0$ to $2\pi$}
\end{selectAll}

Typical ranges are:
\begin{itemize}
    \item $\rho \geq 0$
    \item $0 \leq \phi \leq \answer{\pi}$ (from north pole to south pole)
    \item $0 \leq \theta \leq \answer{2\pi}$ (full rotation)
\end{itemize}

\begin{feedback}
Spherical coordinates are perfect for spheres, cones, and any geometry with radial symmetry!
\end{feedback}
\end{problem}

\begin{problem}
Find the volume of a sphere of radius $R$.

In spherical coordinates, the sphere is simply: $0 \leq \rho \leq \answer{R}$

Full angular coverage: $0 \leq \phi \leq \answer{\pi}$, $0 \leq \theta \leq \answer{2\pi}$

$$V = \int_0^{2\pi}\int_0^{\pi}\int_0^R \rho^2\sin\phi\,d\rho\,d\phi\,d\theta$$

\textbf{Inner integral (w.r.t. $\rho$):}
$$\int_0^R \rho^2\,d\rho = \left[\frac{\rho^3}{3}\right]_0^R = \answer{R^3/3}$$

\textbf{Middle integral (w.r.t. $\phi$):}
$$\int_0^{\pi} \frac{R^3}{3}\sin\phi\,d\phi = \frac{R^3}{3}[-\cos\phi]_0^{\pi} = \frac{R^3}{3}[1-(-1)] = \answer{2R^3/3}$$

\textbf{Outer integral (w.r.t. $\theta$):}
$$\int_0^{2\pi} \frac{2R^3}{3}\,d\theta = \frac{2R^3}{3} \cdot 2\pi = \answer{4\pi R^3/3}$$

This is the familiar sphere volume formula: $V = \frac{4}{3}\pi R^3$! \wordChoice{\choice[correct]{Correct!}\choice{Incorrect}}

\begin{feedback}
Spherical coordinates make the sphere volume calculation elegant with simple constant bounds!
\end{feedback}
\end{problem}

\section*{Choosing the Right Coordinate System}

\begin{problem}
Match each region type with the best coordinate system:

For a cylinder aligned with the $z$-axis:
\begin{multipleChoice}
    \choice{Rectangular}
    \choice[correct]{Cylindrical}
    \choice{Spherical}
\end{multipleChoice}

For a sphere centered at the origin:
\begin{multipleChoice}
    \choice{Rectangular}
    \choice{Cylindrical}
    \choice[correct]{Spherical}
\end{multipleChoice}

For a rectangular box:
\begin{multipleChoice}
    \choice[correct]{Rectangular}
    \choice{Cylindrical}
    \choice{Spherical}
\end{multipleChoice}

For a cone:
\begin{multipleChoice}
    \choice{Rectangular}
    \choice{Cylindrical}
    \choice[correct]{Either cylindrical or spherical work well}
\end{multipleChoice}

\begin{feedback}
Choose coordinates that match your region's symmetry! This makes bounds simpler and integrals easier to evaluate.
\end{feedback}
\end{problem}


\end{document}
