\documentclass{ximera}

\title{Triple Integrals}
\author{Zack Reed}

\begin{document}
\begin{abstract}
In this activity we extend double integrals to three dimensions, exploring triple integrals for volume and mass calculations, along with cylindrical and spherical coordinate systems that simplify integration over curved 3D regions.
\end{abstract}
\maketitle

\section*{Introduction: From Double to Triple Integrals}

% APPLETS FOR REFERENCE

% https://www.geogebra.org/m/bqqfyjdm
% https://www.geogebra.org/m/sebfhwzx
% https://www.geogebra.org/m/xqrs44yg
% https://www.geogebra.org/m/hv9rk7hj
% https://www.geogebra.org/m/sgevexqr
% https://www.geogebra.org/m/yukegeqk
% https://www.geogebra.org/m/rwjghwbb

Just as we extended single integrals (summing over intervals) to double integrals (summing over 2D regions), we now extend to \textbf{triple integrals} that sum over 3D volumes.

\begin{problem}
Complete the pattern:
\begin{itemize}
    \item Single integral: $\displaystyle\int_a^b f(x)\,dx$ sums over a \wordChoice{\choice[correct]{1D interval}\choice{2D region}\choice{3D volume}}
    \item Double integral: $\displaystyle\iint_R f(x,y)\,dA$ sums over a \wordChoice{\choice{1D interval}\choice[correct]{2D region}\choice{3D volume}}
    \item Triple integral: $\displaystyle\iiint_E f(x,y,z)\,dV$ sums over a \wordChoice{\choice{1D interval}\choice{2D region}\choice[correct]{3D volume}}
\end{itemize}

What changes at each level?
\begin{selectAll}
    \choice[correct]{The dimension of the domain}
    \choice[correct]{The number of differentials}
    \choice{The fundamental idea of summing small pieces}
    \choice[correct]{The complexity of setting up bounds}
\end{selectAll}

\begin{feedback}
The core concept remains: integrals sum infinitesimal pieces. We're just extending to higher dimensions!
\end{feedback}
\end{problem}

\section*{The Challenge of Visualization}


An important limitation of triple integrals is that we \textbf{cannot easily visualize} the output of $f(x,y,z)$. Moreover, when we increase the input space to $4$ or more dimensions, we also lose the ability to visualize the input domain.

This limitation is why it's important to maintain the big idea, ``adding up pieces''. With all integral problems you figure out how to state the quantity you desire $M$ as a sum of small pieces $dM$. $dM$ will typically involve a product $f$ with some differential element $dV$, and it's up to you to decide into how many dimensions you want to split $dV$ and how you accurately structure the sum accordingly.


\section*{Setting up Triple Integrals}

For triple integrals, we have three variables $x,y,z$ and we integrate with respect to each, iteratively. Like with double integrals, we need to do some work up front to figure out the bounds of integration and the order of integration.

\begin{problem}

Let's integrate the simple function $x+y+z$ over the wedge given in the first octant (where $x,y,z \geq 0$) and below the plane $x+y+z=1$.

%a 3d tikz graph of the wedge in the first octant below the plane x+y+z=1


We'll walk through this step-by-step, but the first step is ignoring the function $x+y+z$ and just figuring out the bounds of integration for the region $E$.
Think about approximating a sphere's volume by filling it with tiny cubes.

As the cube size gets smaller:
\begin{multipleChoice}
    \choice{The approximation gets worse}
    \choice[correct]{The approximation approaches the true volume}
    \choice{The approximation stays constant}
\end{multipleChoice}

The volume element in rectangular coordinates is: $dV = \answer{dx} \cdot \answer{dy} \cdot \answer{dz}$

This represents the volume of an infinitesimal \wordChoice{\choice{sphere}\choice{cylinder}\choice[correct]{rectangular box}\choice{pyramid}}.

\begin{feedback}
Taking the limit as cube size approaches zero gives the exact volume via the triple integral!
\end{feedback}
\end{problem}

\begin{problem}
Compute the volume of the rectangular box $0 \leq x \leq 2$, $0 \leq y \leq 3$, $0 \leq z \leq 4$.

$$V = \int_0^2\int_0^3\int_0^4 dz\,dy\,dx$$

\textbf{Innermost integral:}
$$\int_0^4 dz = [z]_0^4 = \answer{4}$$

\textbf{Middle integral:}
$$\int_0^3 4\,dy = 4[y]_0^3 = \answer{12}$$

\textbf{Outer integral:}
$$\int_0^2 12\,dx = 12[x]_0^2 = \answer{24}$$

Does this match $V = 2 \times 3 \times 4$? \wordChoice{\choice[correct]{Yes!}\choice{No}}

\begin{feedback}
For rectangular boxes, triple integrals give us the familiar length × width × height formula!
\end{feedback}
\end{problem}

\section*{Mass Integrals with Variable Density}

A more interesting application is computing mass when density varies throughout an object.

\begin{problem}
If density $\rho(x,y,z)$ varies throughout a 3D object, the small mass in a volume element is:
$$dM = \answer{\rho(x,y,z)} \cdot dV$$

The total mass is:
$$M = \iiint_E dM = \iiint_E \rho(x,y,z)\,dV$$

This extends which familiar formulas?
\begin{selectAll}
    \choice[correct]{Single-variable: Area = $\int f(x)\,dx$}
    \choice[correct]{Double integral: Volume = $\iint f(x,y)\,dA$}
    \choice{Newton's law $F = ma$}
    \choice[correct]{The pattern: sum of (function value $\times$ infinitesimal element)}
\end{selectAll}

\begin{feedback}
The pattern continues: integrate (function $\times$ differential) over the domain!
\end{feedback}
\end{problem}

\begin{problem}
Visualize variable density using color.

\begin{expandable}{stuff}{GeoGebra Instructions}
    Drag the point $(x,y,z)$ or use sliders to move the small volume element $dV$ around. The color represents density: brighter = less dense, darker = more dense.
\end{expandable}

\begin{center}
\geogebra{yn6dgycn}{740}{481}
\end{center}

The color coding helps us:
\begin{selectAll}
    \choice[correct]{Visualize how $\rho(x,y,z)$ varies spatially}
    \choice[correct]{Understand where mass is concentrated}
    \choice{See the 4D graph of the function}
    \choice[correct]{Build intuition about the integral}
\end{selectAll}

\begin{feedback}
Color is a creative way to represent the function value in 3D space without needing a 4th spatial dimension!
\end{feedback}
\end{problem}

\begin{problem}
Find the mass of a cube $0 \leq x,y,z \leq 1$ with density $\rho(x,y,z) = x + y + z$.

$$M = \int_0^1\int_0^1\int_0^1 (x+y+z)\,dz\,dy\,dx$$

\textbf{Inner integral (w.r.t. $z$):}
$$\int_0^1 (x+y+z)\,dz = \left[(x+y)z + \frac{z^2}{2}\right]_0^1 = (x+y) + \frac{1}{2} = \answer{x + y + 1/2}$$

\textbf{Middle integral (w.r.t. $y$):}
$$\int_0^1 \left(x + y + \frac{1}{2}\right)\,dy = \left[xy + \frac{y^2}{2} + \frac{y}{2}\right]_0^1 = x + \frac{1}{2} + \frac{1}{2} = \answer{x + 1}$$

\textbf{Outer integral (w.r.t. $x$):}
$$\int_0^1 (x+1)\,dx = \left[\frac{x^2}{2} + x\right]_0^1 = \frac{1}{2} + 1 = \answer{3/2}$$

\begin{feedback}
The mass is $3/2$ units. Notice how we evaluated from the inside out, one variable at a time using Fubini's Theorem!
\end{feedback}
\end{problem}


\section*{Summary and Key Formulas}

\begin{problem}
Complete the coordinate system summary:

\textbf{Rectangular:} $dV = \answer{dx\,dy\,dz}$
\begin{itemize}
    \item Best for: \wordChoice{\choice[correct]{rectangular boxes}\choice{spheres}\choice{cylinders}}
\end{itemize}

\textbf{Cylindrical:} $dV = \answer{r}\,dr\,d\theta\,dz$
\begin{itemize}
    \item Best for: \wordChoice{\choice{rectangular boxes}\choice{spheres}\choice[correct]{cylinders and cones}}
\end{itemize}

\textbf{Spherical:} $dV = \answer{\rho^2 \sin\phi}\,d\rho\,d\phi\,d\theta$
\begin{itemize}
    \item Best for: \wordChoice{\choice{rectangular boxes}\choice[correct]{spheres and cones}\choice{cylinders}}
\end{itemize}

\begin{feedback}
Knowing these three coordinate systems and their volume elements is essential for triple integrals!
\end{feedback}
\end{problem}

\begin{problem}
Final check: Select all TRUE statements about triple integrals.

\begin{selectAll}
    \choice[correct]{Triple integrals sum over 3D volumes}
    \choice[correct]{We can compute volumes, masses, and other 3D quantities}
    \choice{We can easily visualize the graph of $f(x,y,z)$}
    \choice[correct]{Fubini's Theorem extends to three iterated integrals}
    \choice[correct]{Cylindrical coordinates extend polar coordinates by adding height}
    \choice[correct]{Spherical coordinates use $(\rho, \phi, \theta)$}
    \choice{The volume element is always $dx\,dy\,dz$}
    \choice[correct]{Choosing coordinates wisely simplifies bounds and integrals}
    \choice[correct]{Color coding helps visualize density functions}
\end{selectAll}

\begin{feedback}
Excellent! Triple integrals extend our integration toolkit to three dimensions, with coordinate systems that match the geometry of the problem!
\end{feedback}
\end{problem}

\end{document}