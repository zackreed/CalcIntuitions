\documentclass{ximera}

\title{Triple Integrals and Coordinate Systems}
\author{Zack Reed}

\begin{document}
\begin{abstract}
In this activity we extend double integrals to three dimensions, exploring triple integrals for volume and mass calculations, along with cylindrical and spherical coordinate systems that simplify integration over curved 3D regions.
\end{abstract}
\maketitle

\section*{Introduction: From Double to Triple Integrals}

Just as we extended single integrals (summing over intervals) to double integrals (summing over 2D regions), we now extend to \textbf{triple integrals} that sum over 3D volumes.

\begin{problem}
Complete the pattern:
\begin{itemize}
    \item Single integral: $\displaystyle\int_a^b f(x)\,dx$ sums over a \wordChoice{\choice[correct]{1D interval}\choice{2D region}\choice{3D volume}}
    \item Double integral: $\displaystyle\iint_R f(x,y)\,dA$ sums over a \wordChoice{\choice{1D interval}\choice[correct]{2D region}\choice{3D volume}}
    \item Triple integral: $\displaystyle\iiint_E f(x,y,z)\,dV$ sums over a \wordChoice{\choice{1D interval}\choice{2D region}\choice[correct]{3D volume}}
\end{itemize}

What changes at each level?
\begin{selectAll}
    \choice[correct]{The dimension of the domain}
    \choice[correct]{The number of differentials}
    \choice{The fundamental idea of summing small pieces}
    \choice[correct]{The complexity of setting up bounds}
\end{selectAll}

\begin{feedback}
The core concept remains: integrals sum infinitesimal pieces. We're just extending to higher dimensions!
\end{feedback}
\end{problem}

\section*{The Challenge of Visualization}

\begin{problem}
An important limitation of triple integrals is that we \textbf{cannot easily visualize} the output of $f(x,y,z)$.

Why is this a problem?
\begin{multipleChoice}
    \choice{We can't compute the integrals}
    \choice{The integrals are always zero}
    \choice[correct]{We can visualize the 3D domain, but adding a 4th dimension for the function value is impossible}
    \choice{Triple integrals don't exist}
\end{multipleChoice}

What can we visualize?
\begin{selectAll}
    \choice[correct]{The 3D domain of integration}
    \choice{The 4D graph of $w = f(x,y,z)$}
    \choice[correct]{Density using color coding}
    \choice[correct]{The volume element $dV$}
\end{selectAll}

\begin{feedback}
We can visualize the 3D domain, but not the function values in a traditional graph. Clever techniques like color-coding density help us understand what's happening!
\end{feedback}
\end{problem}

\section*{Computing Volumes with Triple Integrals}

The simplest application of triple integrals is computing volumes.

\begin{definition}
The \textbf{volume} of a 3D region $E$ is:
$$V = \iiint_E dV = \iiint_E dx\,dy\,dz$$

We're integrating the function $f(x,y,z) = 1$ over the region, which counts up the volume!
\end{definition}

\begin{problem}
Think about approximating a sphere's volume by filling it with tiny cubes.

As the cube size gets smaller:
\begin{multipleChoice}
    \choice{The approximation gets worse}
    \choice[correct]{The approximation approaches the true volume}
    \choice{The approximation stays constant}
\end{multipleChoice}

The volume element in rectangular coordinates is: $dV = \answer{dx} \cdot \answer{dy} \cdot \answer{dz}$

This represents the volume of an infinitesimal \wordChoice{\choice{sphere}\choice{cylinder}\choice[correct]{rectangular box}\choice{pyramid}}.

\begin{feedback}
Taking the limit as cube size approaches zero gives the exact volume via the triple integral!
\end{feedback}
\end{problem}

\begin{problem}
Compute the volume of the rectangular box $0 \leq x \leq 2$, $0 \leq y \leq 3$, $0 \leq z \leq 4$.

$$V = \int_0^2\int_0^3\int_0^4 dz\,dy\,dx$$

\textbf{Innermost integral:}
$$\int_0^4 dz = [z]_0^4 = \answer{4}$$

\textbf{Middle integral:}
$$\int_0^3 4\,dy = 4[y]_0^3 = \answer{12}$$

\textbf{Outer integral:}
$$\int_0^2 12\,dx = 12[x]_0^2 = \answer{24}$$

Does this match $V = 2 \times 3 \times 4$? \wordChoice{\choice[correct]{Yes!}\choice{No}}

\begin{feedback}
For rectangular boxes, triple integrals give us the familiar length × width × height formula!
\end{feedback}
\end{problem}

\section*{Mass Integrals with Variable Density}

A more interesting application is computing mass when density varies throughout an object.

\begin{problem}
If density $\rho(x,y,z)$ varies throughout a 3D object, the small mass in a volume element is:
$$dM = \answer{\rho(x,y,z)} \cdot dV$$

The total mass is:
$$M = \iiint_E dM = \iiint_E \rho(x,y,z)\,dV$$

This extends which familiar formulas?
\begin{selectAll}
    \choice[correct]{Single-variable: Area = $\int f(x)\,dx$}
    \choice[correct]{Double integral: Volume = $\iint f(x,y)\,dA$}
    \choice{Newton's law $F = ma$}
    \choice[correct]{The pattern: sum of (function value × infinitesimal element)}
\end{selectAll}

\begin{feedback}
The pattern continues: integrate (function × differential) over the domain!
\end{feedback}
\end{problem}

\begin{problem}
Visualize variable density using color.

\begin{expandable}{stuff}{GeoGebra Instructions}
    Drag the point $(x,y,z)$ or use sliders to move the small volume element $dV$ around. The color represents density: brighter = less dense, darker = more dense.
\end{expandable}

\begin{center}
\geogebra{yn6dgycn}{740}{481}
\end{center}

The color coding helps us:
\begin{selectAll}
    \choice[correct]{Visualize how $\rho(x,y,z)$ varies spatially}
    \choice[correct]{Understand where mass is concentrated}
    \choice{See the 4D graph of the function}
    \choice[correct]{Build intuition about the integral}
\end{selectAll}

\begin{feedback}
Color is a creative way to represent the function value in 3D space without needing a 4th spatial dimension!
\end{feedback}
\end{problem}

\begin{problem}
Find the mass of a cube $0 \leq x,y,z \leq 1$ with density $\rho(x,y,z) = x + y + z$.

$$M = \int_0^1\int_0^1\int_0^1 (x+y+z)\,dz\,dy\,dx$$

\textbf{Inner integral (w.r.t. $z$):}
$$\int_0^1 (x+y+z)\,dz = \left[(x+y)z + \frac{z^2}{2}\right]_0^1 = (x+y) + \frac{1}{2} = \answer{x + y + 1/2}$$

\textbf{Middle integral (w.r.t. $y$):}
$$\int_0^1 \left(x + y + \frac{1}{2}\right)\,dy = \left[xy + \frac{y^2}{2} + \frac{y}{2}\right]_0^1 = x + \frac{1}{2} + \frac{1}{2} = \answer{x + 1}$$

\textbf{Outer integral (w.r.t. $x$):}
$$\int_0^1 (x+1)\,dx = \left[\frac{x^2}{2} + x\right]_0^1 = \frac{1}{2} + 1 = \answer{3/2}$$

\begin{feedback}
The mass is $3/2$ units. Notice how we evaluated from the inside out, one variable at a time using Fubini's Theorem!
\end{feedback}
\end{problem}

\section*{Cylindrical Coordinates}

For regions with cylindrical symmetry (cylinders, cones), cylindrical coordinates simplify the integral.

\begin{definition}
\textbf{Cylindrical coordinates} $(r, \theta, z)$ extend polar coordinates into 3D:
\begin{itemize}
    \item $x = r\cos\theta$
    \item $y = r\sin\theta$
    \item $z = z$
\end{itemize}

The volume element is: $dV = r\,dr\,d\theta\,dz$
\end{definition}

\begin{problem}
Cylindrical coordinates are built from:
\begin{selectAll}
    \choice[correct]{Polar coordinates $(r,\theta)$ in the $xy$-plane}
    \choice[correct]{Height $z$ perpendicular to the plane}
    \choice{Spherical shells}
    \choice[correct]{The familiar extra factor of $r$ from polar integrals}
\end{selectAll}

The volume element $dV = r\,dr\,d\theta\,dz$ comes from:
\begin{multipleChoice}
    \choice{Multiplying all three differentials}
    \choice[correct]{Taking a polar area element $dA = r\,dr\,d\theta$ and extending it by height $dz$}
    \choice{Spherical geometry}
\end{multipleChoice}

\begin{feedback}
Cylindrical coordinates = polar coordinates + vertical height! The volume element is literally $(r\,dr\,d\theta) \times dz$.
\end{feedback}
\end{problem}

\begin{problem}
Explore cylindrical coordinates visually.

\begin{expandable}{stuff}{GeoGebra Instructions}
    Select "Show Cylinder" view. Use sliders to change $r$, $\theta$, and $z$. Click "Zoom to Volume Element" to see the small piece $dV = r\,dr\,d\theta\,dz$.
\end{expandable}

\begin{center}
\geogebra{xqrs44yg}{750}{675}
\end{center}

As you vary $r$, $\theta$, $z$, observe:
\begin{selectAll}
    \choice[correct]{$r$ controls distance from the $z$-axis}
    \choice[correct]{$\theta$ controls rotation around the $z$-axis}
    \choice[correct]{$z$ controls height}
    \choice{The volume element shape is always a cube}
\end{selectAll}

\begin{feedback}
The volume element in cylindrical coordinates is a tiny curved box that fits the cylindrical geometry!
\end{feedback}
\end{problem}

\begin{problem}
Find the volume of a cylinder of radius $R$ and height $H$.

In cylindrical coordinates:
\begin{itemize}
    \item $0 \leq r \leq \answer{R}$
    \item $0 \leq \theta \leq \answer{2\pi}$
    \item $0 \leq z \leq \answer{H}$
\end{itemize}

$$V = \int_0^{2\pi}\int_0^R\int_0^H r\,dz\,dr\,d\theta$$

\textbf{Inner integral:}
$$\int_0^H r\,dz = r \cdot H = \answer{rH}$$

\textbf{Middle integral:}
$$\int_0^R rH\,dr = H \left[\frac{r^2}{2}\right]_0^R = \answer{HR^2/2}$$

\textbf{Outer integral:}
$$\int_0^{2\pi} \frac{HR^2}{2}\,d\theta = \frac{HR^2}{2} \cdot 2\pi = \answer{\pi R^2 H}$$

Does this match the familiar cylinder volume formula? \wordChoice{\choice[correct]{Yes!}\choice{No}}

\begin{feedback}
Perfect! Cylindrical coordinates make this integral trivial with constant bounds.
\end{feedback}
\end{problem}

\section*{Spherical Coordinates}

For regions with spherical symmetry (spheres, cones), spherical coordinates are ideal.

\begin{definition}
\textbf{Spherical coordinates} $(\rho, \phi, \theta)$ describe a point by:
\begin{itemize}
    \item $\rho$ = distance from origin
    \item $\phi$ = angle down from positive $z$-axis
    \item $\theta$ = angle of rotation around $z$-axis (same as in cylindrical)
\end{itemize}

Conversion formulas:
\begin{itemize}
    \item $x = \rho\sin\phi\cos\theta$
    \item $y = \rho\sin\phi\sin\theta$
    \item $z = \rho\cos\phi$
\end{itemize}

The volume element is: $dV = \rho^2\sin\phi\,d\rho\,d\phi\,d\theta$
\end{definition}

\begin{problem}
The spherical volume element $dV = \rho^2\sin\phi\,d\rho\,d\phi\,d\theta$ has:

An extra factor of $\rho^2$: \wordChoice{\choice[correct]{Yes}\choice{No}}

An extra factor of $\sin\phi$: \wordChoice{\choice[correct]{Yes}\choice{No}}

These factors account for:
\begin{multipleChoice}
    \choice{Mathematical complexity}
    \choice[correct]{The geometry of spherical shells and how they expand with radius}
    \choice{Random constants}
\end{multipleChoice}

\begin{feedback}
The $\rho^2\sin\phi$ factor is crucial! It accounts for how spherical volume elements grow with radius and latitude.
\end{feedback}
\end{problem}

\begin{problem}
Explore spherical coordinates visually.

\begin{expandable}{stuff}{GeoGebra Instructions}
    Select "Show Sphere" view. Use sliders for $\rho$, $\phi$, $\theta$. Zoom to see the volume element shape—it's like a tiny curved box on a sphere!
\end{expandable}

\begin{center}
\geogebra{xqrs44yg}{750}{675}
\end{center}

In spherical coordinates:
\begin{selectAll}
    \choice[correct]{$\rho$ measures distance from origin}
    \choice[correct]{$\phi$ measures angle from the positive $z$-axis (colatitude)}
    \choice[correct]{$\theta$ measures rotation around $z$-axis (azimuth)}
    \choice{All angles range from $0$ to $2\pi$}
\end{selectAll}

Typical ranges are:
\begin{itemize}
    \item $\rho \geq 0$
    \item $0 \leq \phi \leq \answer{\pi}$ (from north pole to south pole)
    \item $0 \leq \theta \leq \answer{2\pi}$ (full rotation)
\end{itemize}

\begin{feedback}
Spherical coordinates are perfect for spheres, cones, and any geometry with radial symmetry!
\end{feedback}
\end{problem}

\begin{problem}
Find the volume of a sphere of radius $R$.

In spherical coordinates, the sphere is simply: $0 \leq \rho \leq \answer{R}$

Full angular coverage: $0 \leq \phi \leq \answer{\pi}$, $0 \leq \theta \leq \answer{2\pi}$

$$V = \int_0^{2\pi}\int_0^{\pi}\int_0^R \rho^2\sin\phi\,d\rho\,d\phi\,d\theta$$

\textbf{Inner integral (w.r.t. $\rho$):}
$$\int_0^R \rho^2\,d\rho = \left[\frac{\rho^3}{3}\right]_0^R = \answer{R^3/3}$$

\textbf{Middle integral (w.r.t. $\phi$):}
$$\int_0^{\pi} \frac{R^3}{3}\sin\phi\,d\phi = \frac{R^3}{3}[-\cos\phi]_0^{\pi} = \frac{R^3}{3}[1-(-1)] = \answer{2R^3/3}$$

\textbf{Outer integral (w.r.t. $\theta$):}
$$\int_0^{2\pi} \frac{2R^3}{3}\,d\theta = \frac{2R^3}{3} \cdot 2\pi = \answer{4\pi R^3/3}$$

This is the familiar sphere volume formula: $V = \frac{4}{3}\pi R^3$! \wordChoice{\choice[correct]{Correct!}\choice{Incorrect}}

\begin{feedback}
Spherical coordinates make the sphere volume calculation elegant with simple constant bounds!
\end{feedback}
\end{problem}

\section*{Choosing the Right Coordinate System}

\begin{problem}
Match each region type with the best coordinate system:

For a cylinder aligned with the $z$-axis:
\begin{multipleChoice}
    \choice{Rectangular}
    \choice[correct]{Cylindrical}
    \choice{Spherical}
\end{multipleChoice}

For a sphere centered at the origin:
\begin{multipleChoice}
    \choice{Rectangular}
    \choice{Cylindrical}
    \choice[correct]{Spherical}
\end{multipleChoice}

For a rectangular box:
\begin{multipleChoice}
    \choice[correct]{Rectangular}
    \choice{Cylindrical}
    \choice{Spherical}
\end{multipleChoice}

For a cone:
\begin{multipleChoice}
    \choice{Rectangular}
    \choice{Cylindrical}
    \choice[correct]{Either cylindrical or spherical work well}
\end{multipleChoice}

\begin{feedback}
Choose coordinates that match your region's symmetry! This makes bounds simpler and integrals easier to evaluate.
\end{feedback}
\end{problem}

\section*{Summary and Key Formulas}

\begin{problem}
Complete the coordinate system summary:

\textbf{Rectangular:} $dV = \answer{dx\,dy\,dz}$
\begin{itemize}
    \item Best for: \wordChoice{\choice[correct]{rectangular boxes}\choice{spheres}\choice{cylinders}}
\end{itemize}

\textbf{Cylindrical:} $dV = \answer{r}\,dr\,d\theta\,dz$
\begin{itemize}
    \item Best for: \wordChoice{\choice{rectangular boxes}\choice{spheres}\choice[correct]{cylinders and cones}}
\end{itemize}

\textbf{Spherical:} $dV = \answer{\rho^2 \sin\phi}\,d\rho\,d\phi\,d\theta$
\begin{itemize}
    \item Best for: \wordChoice{\choice{rectangular boxes}\choice[correct]{spheres and cones}\choice{cylinders}}
\end{itemize}

\begin{feedback}
Knowing these three coordinate systems and their volume elements is essential for triple integrals!
\end{feedback}
\end{problem}

\begin{problem}
Final check: Select all TRUE statements about triple integrals.

\begin{selectAll}
    \choice[correct]{Triple integrals sum over 3D volumes}
    \choice[correct]{We can compute volumes, masses, and other 3D quantities}
    \choice{We can easily visualize the graph of $f(x,y,z)$}
    \choice[correct]{Fubini's Theorem extends to three iterated integrals}
    \choice[correct]{Cylindrical coordinates extend polar coordinates by adding height}
    \choice[correct]{Spherical coordinates use $(\rho, \phi, \theta)$}
    \choice{The volume element is always $dx\,dy\,dz$}
    \choice[correct]{Choosing coordinates wisely simplifies bounds and integrals}
    \choice[correct]{Color coding helps visualize density functions}
\end{selectAll}

\begin{feedback}
Excellent! Triple integrals extend our integration toolkit to three dimensions, with coordinate systems that match the geometry of the problem!
\end{feedback}
\end{problem}

\end{document}