\documentclass{ximera}

\title{Double Integrals and Polar Coordinates}
\author{Zack Reed}

\begin{document}
\begin{abstract}
In this activity we extend single-variable integration to functions of two variables, exploring double integrals over rectangular and general regions, Fubini's Theorem, and polar coordinates for curved regions.
\end{abstract}
\maketitle

\section*{Introduction: From Single to Double Integrals}

In single-variable calculus, the definite integral $\int_a^b f(x)\,dx$ sums up infinitely many small ``bits'' of area under a curve. For multivariable functions, we extend this idea to sum over two-dimensional regions.

\begin{problem}
The fundamental concept remains unchanged:
\begin{center}
\textbf{An integral is the sum of small pieces of a measurement.}
\end{center}

What changes from single to double integrals?
\begin{selectAll}
    \choice[correct]{We sum over a 2D region instead of a 1D interval}
    \choice[correct]{We have two differentials: $dx$ and $dy$}
    \choice{The idea of Riemann sums no longer applies}
    \choice[correct]{We can compute volumes under surfaces}
\end{selectAll}

\begin{feedback}
The core idea is the same! We're just increasing the complexity of how we build the regions over which we sum.
\end{feedback}
\end{problem}

\section*{Motivating Example: Hydrostatic Force on a Dam}

Let's revisit a classic single-variable problem and see how double integrals naturally arise.

\begin{problem}
Consider a rectangular dam that is 12 meters deep and 5 meters wide.

\begin{expandable}{stuff}{GeoGebra Instructions}
    Alter the ``Specific Piece of Area'' slider to view different small area regions at various depths. Observe how the force (purple curve) increases with depth.
\end{expandable}

\begin{center}
\geogebra{xebf6ft3}{733}{594}
\end{center}

Why does the force curve bend upward?
\begin{multipleChoice}
    \choice{The width changes}
    \choice[correct]{Pressure increases with depth}
    \choice{The dam is getting weaker}
    \choice{The water density changes}
\end{multipleChoice}

\begin{feedback}
Pressure at depth $D$ is $P(D) = 9800 \cdot D$ (in Pascals). As we go deeper, pressure increases linearly!
\end{feedback}
\end{problem}

\subsection*{Single-Variable Approach}

\begin{problem}
In the single-variable approach, we fixed the width at 5 meters and varied only the depth.

The small bit of force at depth $D$ is: $dF = P(D) \cdot 5 \cdot dD = \answer{9800 \cdot D \cdot 5} \cdot dD$

The total force becomes:
$$F = \int_0^{12} 9800 \cdot D \cdot 5 \,dD$$

Compute this integral:
$$F = 9800 \cdot 5 \cdot \left[\frac{D^2}{2}\right]_0^{12} = 9800 \cdot 5 \cdot \answer{72} = \answer{3528000} \text{ N}$$

\begin{feedback}
We integrated over a one-dimensional depth line, treating width as constant.
\end{feedback}
\end{problem}

\subsection*{Double Integral Approach}

\begin{problem}
Now let's incorporate the entire 2D area of the dam using \textbf{two} variables: depth $D$ and width $w$.

The small area element is: $dA = \answer{dD \cdot dw}$

The pressure at any point $(D,w)$ depends only on depth: $P(D,w) = \answer{9800 \cdot D}$

The small bit of force is:
$$dF = P(A) \cdot dA = \answer{9800 \cdot D} \cdot dD \cdot dw$$

The double integral becomes:
$$F = \iint_R dF = \int_0^{12}\int_0^5 9800 \cdot D \,dw\,dD$$

Let's evaluate this step by step using Fubini's Theorem!

\textbf{Inner integral (with respect to $w$):}
$$\int_0^5 9800 \cdot D \,dw = 9800 \cdot D \cdot [w]_0^5 = \answer{49000 \cdot D}$$

\textbf{Outer integral (with respect to $D$):}
$$\int_0^{12} 49000 \cdot D \,dD = 49000 \cdot \left[\frac{D^2}{2}\right]_0^{12} = \answer{3528000} \text{ N}$$

Notice we got the same answer! This makes sense because:
\begin{multipleChoice}
    \choice{Double integrals are always equal to single integrals}
    \choice[correct]{We're measuring the same physical quantity, just with different coordinate representations}
    \choice{The force doesn't actually depend on width}
\end{multipleChoice}

\begin{feedback}
The key insight: double integrals have two sum symbols (the two integral signs) and identify the 2D region over which we sum!
\end{feedback}
\end{problem}

\section*{Double Integrals for Volume}

The most common interpretation of double integrals is computing volume under a surface.

\begin{definition}
For a function $f(x,y) \geq 0$ over a region $R$ in the $xy$-plane, the \textbf{double integral}
$$V = \iint_R f(x,y)\,dA = \iint_R f(x,y)\,dx\,dy$$
represents the volume between the surface $z=f(x,y)$ and the region $R$.
\end{definition}

\begin{problem}
Explore double integral approximations visually.

\begin{expandable}{stuff}{GeoGebra Instructions}
    Alter the rectangle dimensions by dragging the top right corner. Change grid fineness with the sliders. Select a specific area element $dA$ to see the corresponding volume approximation $dV = f(a,b) \cdot dA$.
\end{expandable}

\begin{center}
\geogebra{yukegeqk}{755}{628}
\end{center}

As the grid becomes finer (smaller $dA$), the approximation:
\begin{multipleChoice}
    \choice{Gets worse}
    \choice[correct]{Gets better and approaches the true volume}
    \choice{Stays the same}
    \choice{Oscillates}
\end{multipleChoice}

\begin{feedback}
This is exactly like Riemann sums in single-variable calculus! The limit of these approximations as $dA \to 0$ gives the exact double integral.
\end{feedback}
\end{problem}

\section*{Computing Double Integrals: Fubini's Theorem}

\begin{theorem}[Fubini's Theorem]
For a continuous function $f(x,y)$ over a rectangle $R = [a,b] \times [c,d]$:
$$\iint_R f(x,y)\,dA = \int_a^b \int_c^d f(x,y)\,dy\,dx = \int_c^d \int_a^b f(x,y)\,dx\,dy$$

We can compute double integrals as \textbf{iterated integrals}, evaluating one variable at a time!
\end{theorem}

\begin{problem}
Evaluate $\displaystyle\int_0^2\int_1^3 (x^2 + 2y)\,dy\,dx$.

\textbf{Step 1: Inner integral (with respect to $y$)}
$$\int_1^3 (x^2 + 2y)\,dy = \left[x^2 y + y^2\right]_1^3$$
$$= (x^2 \cdot 3 + 9) - (x^2 \cdot 1 + 1) = \answer{2x^2 + 8}$$

\textbf{Step 2: Outer integral (with respect to $x$)}
$$\int_0^2 (2x^2 + 8)\,dx = \left[\frac{2x^3}{3} + 8x\right]_0^2$$
$$= \frac{16}{3} + 16 = \answer{64/3}$$

\begin{feedback}
Fubini's Theorem lets us break a double integral into two single-variable integrals!
\end{feedback}
\end{problem}

\begin{problem}
Now try the other order: $\displaystyle\int_1^3\int_0^2 (x^2 + 2y)\,dx\,dy$.

\textbf{Inner integral (with respect to $x$):}
$$\int_0^2 (x^2 + 2y)\,dx = \left[\frac{x^3}{3} + 2xy\right]_0^2 = \answer{8/3 + 4y}$$

\textbf{Outer integral (with respect to $y$):}
$$\int_1^3 \left(\frac{8}{3} + 4y\right)\,dy = \left[\frac{8y}{3} + 2y^2\right]_1^3$$
$$= \left(\frac{24}{3} + 18\right) - \left(\frac{8}{3} + 2\right) = \answer{64/3}$$

Did you get the same answer? \wordChoice{\choice[correct]{Yes!}\choice{No}}

\begin{feedback}
For rectangular regions, the order of integration doesn't matter—we get the same answer either way!
\end{feedback}
\end{problem}

\section*{Non-Rectangular Regions}

For non-rectangular regions, the bounds of integration become functions.

\begin{problem}
Find the volume under $f(x,y) = x + y$ over the triangular region with vertices $(0,0)$, $(1,0)$, $(1,1)$.

The region can be described as: $0 \leq x \leq 1$ and $0 \leq y \leq \answer{x}$

The double integral is:
$$V = \int_0^1 \int_0^x (x+y)\,dy\,dx$$

\textbf{Inner integral:}
$$\int_0^x (x+y)\,dy = \left[xy + \frac{y^2}{2}\right]_0^x = x^2 + \frac{x^2}{2} = \answer{3x^2/2}$$

\textbf{Outer integral:}
$$\int_0^1 \frac{3x^2}{2}\,dx = \frac{3}{2} \cdot \left[\frac{x^3}{3}\right]_0^1 = \frac{3}{2} \cdot \frac{1}{3} = \answer{1/2}$$

\begin{feedback}
When the region is not rectangular, the bounds of the inner integral often depend on the outer variable!
\end{feedback}
\end{problem}

\section*{Polar Coordinates}

For circular or radial regions, polar coordinates often simplify the integral dramatically.

\begin{problem}
Why use polar coordinates for integration?
\begin{selectAll}
    \choice[correct]{Circular regions have simpler bounds: $0 \leq r \leq R$, $0 \leq \theta \leq 2\pi$}
    \choice{Polar coordinates always make integrals easier}
    \choice[correct]{Functions with $x^2 + y^2$ simplify to $r^2$}
    \choice[correct]{Radial symmetry becomes apparent}
\end{selectAll}

\begin{feedback}
Polar coordinates are ideal when the region or function has circular/radial symmetry!
\end{feedback}
\end{problem}

\subsection*{The Polar Area Element}

\begin{problem}
The key question: What is $dA$ in polar coordinates?

\begin{expandable}{stuff}{GeoGebra Instructions}
    Use the ``Shift Radius'' and ``Shift $\theta$'' sliders to move the area element $dA$ around. Check ``Show Volume'' to see the approximating cylinder. Use ``Zoom to Unit Circle'' and drag to view from above.
\end{expandable}

\begin{center}
\geogebra{uucwvg4g}{740}{481}
\end{center}

The area element $dA$ is bounded between:
\begin{itemize}
    \item Circles of radius $r - \frac{dr}{2}$ and $r + \frac{dr}{2}$
    \item Angles $\theta - \frac{d\theta}{2}$ and $\theta + \frac{d\theta}{2}$
\end{itemize}

The area of this annular sector is approximately:
$$dA = \frac{d\theta}{2}(r+\frac{dr}{2})^2 - \frac{d\theta}{2}(r-\frac{dr}{2})^2$$

After algebra, this simplifies to: $dA = \answer{r} \cdot dr \cdot d\theta$

The extra factor $r$ is crucial! Without it:
\begin{multipleChoice}
    \choice{The integral would be easier}
    \choice[correct]{We'd get the wrong answer}
    \choice{Nothing would change}
\end{multipleChoice}

\begin{feedback}
The polar area element is $dA = r\,dr\,d\theta$. That extra $r$ accounts for the curved geometry!
\end{feedback}
\end{problem}

\subsection*{Polar Integral Example}

\begin{problem}
Let's integrate $f(x,y) = xy$ over the unit disk $x^2 + y^2 \leq 1$.

\textbf{In rectangular coordinates (yuck!):}
$$\int_{-1}^1\int_{-\sqrt{1-x^2}}^{\sqrt{1-x^2}} xy\,dy\,dx$$

Those square root bounds are messy!

\textbf{In polar coordinates:}
\begin{itemize}
    \item Region: $0 \leq r \leq \answer{1}$, $0 \leq \theta \leq \answer{2\pi}$
    \item Function: $f(x,y) = xy = (r\cos\theta)(r\sin\theta) = \answer{r^2 \sin\theta \cos\theta}$
    \item Area element: $dA = \answer{r}\,dr\,d\theta$
\end{itemize}

The integral becomes:
$$V = \int_0^{2\pi}\int_0^1 r^2\sin\theta\cos\theta \cdot r\,dr\,d\theta = \int_0^{2\pi}\int_0^1 r^3\sin\theta\cos\theta\,dr\,d\theta$$

\textbf{Inner integral (with respect to $r$):}
$$\int_0^1 r^3\,dr = \left[\frac{r^4}{4}\right]_0^1 = \answer{1/4}$$

\textbf{Outer integral (with respect to $\theta$):}
$$\int_0^{2\pi} \frac{1}{4}\sin\theta\cos\theta\,d\theta = \frac{1}{4}\int_0^{2\pi} \frac{1}{2}\sin(2\theta)\,d\theta$$
$$= \frac{1}{8}\left[-\frac{1}{2}\cos(2\theta)\right]_0^{2\pi} = \frac{1}{8} \cdot 0 = \answer{0}$$

Why is the volume zero?
\begin{multipleChoice}
    \choice{We made a calculation error}
    \choice[correct]{The function $xy$ is negative in two quadrants and positive in two quadrants, so they cancel}
    \choice{The unit disk has zero area}
\end{multipleChoice}

\begin{feedback}
The symmetry of $xy$ means equal positive and negative volumes that cancel! Polar coordinates made this much cleaner to compute.
\end{feedback}
\end{problem}

\section*{Practice Problems}

\begin{problem}
Compute $\displaystyle\int_0^{\pi/2}\int_0^{\cos\theta} r^2\,dr\,d\theta$ in polar coordinates.

\textbf{Inner integral:}
$$\int_0^{\cos\theta} r^2\,dr = \left[\frac{r^3}{3}\right]_0^{\cos\theta} = \answer{\cos^3\theta / 3}$$

\textbf{Outer integral:}
$$\int_0^{\pi/2} \frac{\cos^3\theta}{3}\,d\theta = \frac{1}{3}\int_0^{\pi/2} \cos^3\theta\,d\theta$$

Using $\cos^3\theta = \cos\theta(1-\sin^2\theta)$ and substitution $u = \sin\theta$:
$$= \frac{1}{3}\left[\sin\theta - \frac{\sin^3\theta}{3}\right]_0^{\pi/2} = \frac{1}{3}\left(1 - \frac{1}{3}\right) = \answer{2/9}$$

\begin{feedback}
Great work! This integral computed the volume of a region bounded by a cosine curve in polar coordinates.
\end{feedback}
\end{problem}

\begin{problem}
Set up (but don't evaluate) the integral for the area of a circle of radius $R$ using polar coordinates.

Area formula: $A = \iint_R dA$

In polar: $A = \int_{\answer{0}}^{\answer{2\pi}}\int_{\answer{0}}^{\answer{R}} \answer{r}\,dr\,d\theta$

Evaluating quickly:
$$A = \int_0^{2\pi} \left[\frac{r^2}{2}\right]_0^R d\theta = \int_0^{2\pi} \frac{R^2}{2}\,d\theta = \frac{R^2}{2} \cdot 2\pi = \answer{\pi R^2}$$

Does this match the familiar formula? \wordChoice{\choice[correct]{Yes!}\choice{No}}

\begin{feedback}
Perfect! Double integrals can compute areas (by integrating $f=1$), volumes (by integrating $f(x,y)$), and many other quantities!
\end{feedback}
\end{problem}

\section*{Summary and Key Concepts}

\begin{problem}
Select all TRUE statements about double integrals:

\begin{selectAll}
    \choice[correct]{Double integrals sum over 2D regions}
    \choice[correct]{Fubini's Theorem allows us to compute them as iterated integrals}
    \choice{The order of integration always matters}
    \choice[correct]{In polar coordinates, $dA = r\,dr\,d\theta$}
    \choice[correct]{Polar coordinates simplify integrals over circular regions}
    \choice{Every double integral represents a volume}
    \choice[correct]{We can integrate to find areas, volumes, masses, and more}
    \choice[correct]{For non-rectangular regions, bounds may depend on the other variable}
\end{selectAll}

\begin{feedback}
Excellent! You've mastered the fundamentals of double integrals—a powerful tool for computing quantities over planar regions!
\end{feedback}
\end{problem}

\end{document}