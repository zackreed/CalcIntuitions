\documentclass{ximera}

\title{Jacobians and Transformations}
\author{Zack Reed}

\begin{document}
\begin{abstract}
In this activity we learn how to compute the Jacobian determinant for coordinate transformations and understand its geometric meaning as a scaling factor for area and volume.
\end{abstract}
\maketitle



\section*{The General Problem}

\begin{problem}
Consider a transformation $T$ from $uv$-coordinates to $xy$-coordinates:
\begin{align*}
x &= x(u,v)\\
y &= y(u,v)
\end{align*}

We want to evaluate:
$$\iint_R f(x,y)\,dx\,dy$$

But the region $R$ is easier to describe in $uv$-coordinates!

The transformation allows us to write:
$$\iint_R f(x,y)\,dx\,dy = \iint_S f(x(u,v), y(u,v)) \cdot \answer{|J|} \,du\,dv$$

where $|J|$ is the \wordChoice{\choice{derivative}\choice{partial derivative}\choice[correct]{Jacobian determinant}\choice{gradient}}.

\begin{feedback}
The Jacobian determinant $|J|$ is the multivariable analog of $g'(x)$ from u-substitution!
\end{feedback}
\end{problem}

\section*{Visualizing Transformations}

\begin{problem}
A transformation $T: (u,v) \mapsto (x,y)$ can:
\begin{selectAll}
    \choice[correct]{Stretch regions}
    \choice[correct]{Rotate regions}
    \choice[correct]{Shear regions}
    \choice[correct]{Change areas}
    \choice{Create area from nothing}
\end{selectAll}

Example transformations:
\begin{enumerate}
    \item \textbf{Scaling:} $x = 2u$, $y = 3v$
    
    A $1 \times 1$ square in the $uv$-plane becomes a $\answer{2} \times \answer{3}$ rectangle in the $xy$-plane.
    
    The area scales by a factor of: $\answer{6}$
    
    \item \textbf{Rotation by $\pi/4$:} $x = \frac{u-v}{\sqrt{2}}$, $y = \frac{u+v}{\sqrt{2}}$
    
    Rotations preserve area: the scaling factor is $\answer{1}$
    
    \item \textbf{Polar:} $x = r\cos\theta$, $y = r\sin\theta$
    
    The scaling factor depends on position and equals: $\answer{r}$
\end{enumerate}

\begin{feedback}
Different transformations affect areas differently! The Jacobian captures exactly how much the area changes.
\end{feedback}
\end{problem}

\section*{Worked Example: Linear Transformation}

\begin{problem}
Consider the transformation:
\begin{align*}
x &= 2u + v\\
y &= u + 3v
\end{align*}

Let's compute the Jacobian step by step (we'll learn the formula in the next activity).

The transformation maps the square $0 \leq u,v \leq 1$ to a parallelogram in the $xy$-plane.

\textbf{Find the corners:}

$(u,v) = (0,0) \mapsto (x,y) = (\answer{0}, \answer{0})$

$(u,v) = (1,0) \mapsto (x,y) = (\answer{2}, \answer{1})$

$(u,v) = (0,1) \mapsto (x,y) = (\answer{1}, \answer{3})$

$(u,v) = (1,1) \mapsto (x,y) = (\answer{3}, \answer{4})$

The Jacobian determinant for this linear transformation is $|J| = 5$.

If we integrate $f(x,y) = 1$ over the parallelogram to find its area:
$$\text{Area} = \iint_{\text{parallelogram}} dx\,dy = \int_0^1\int_0^1 |J|\,du\,dv = \int_0^1\int_0^1 5\,du\,dv = \answer{5}$$

\begin{feedback}
The Jacobian told us that the $1 \times 1$ square (area = 1) in the $uv$-plane maps to a parallelogram with area = 5 in the $xy$-plane!
\end{feedback}
\end{problem}

\section*{Setting Up Integrals with Change of Variables}

Here's an applet for reference

% https://www.geogebra.org/m/bqqfyjdm
% https://www.geogebra.org/m/sebfhwzx
% https://www.geogebra.org/m/xqrs44yg
% https://www.geogebra.org/m/hv9rk7hj
% https://www.geogebra.org/m/sgevexqr
% https://www.geogebra.org/m/yukegeqk
% https://www.geogebra.org/m/rwjghwbb

\begin{problem}
To change variables in a double integral, follow these steps:

\textbf{Step 1:} Identify the transformation
$$x = x(u,v), \quad y = y(u,v)$$

\textbf{Step 2:} Find the Jacobian determinant $|J|$

\textbf{Step 3:} Express the function in new coordinates
$$f(x,y) = f(x(u,v), y(u,v))$$

\textbf{Step 4:} Determine the new bounds (region $S$ in $uv$-plane)

\textbf{Step 5:} Write the transformed integral
$$\iint_R f(x,y)\,dx\,dy = \iint_S f(x(u,v), y(u,v)) |J|\,du\,dv$$

Which step is most critical?
\begin{multipleChoice}
    \choice{Step 1 - identifying the transformation}
    \choice[correct]{Step 2 - computing the Jacobian (all steps are important!)}
    \choice{Step 3 - substituting the function}
    \choice{Step 4 - finding new bounds}
\end{multipleChoice}

\begin{feedback}
Actually, all steps are crucial! Miss any one and the integral will be wrong. But the Jacobian is what makes the integral correct.
\end{feedback}
\end{problem}

\section*{Example: Ellipse to Circle}

\begin{problem}
Integrate $f(x,y) = 1$ over the ellipse $\frac{x^2}{4} + y^2 \leq 1$.

\textbf{Bad approach:} Use rectangular coordinates directly. The bounds are messy!

\textbf{Better approach:} Transform the ellipse to a circle.

Define: $u = \frac{x}{2}$, $v = y$

Then: $x = \answer{2u}$, $y = \answer{v}$

The ellipse becomes: $u^2 + v^2 \leq \answer{1}$ (a unit circle!)

Now we can use polar: $u = r\cos\theta$, $v = r\sin\theta$

Chain the transformations:
\begin{align*}
x &= 2u = 2r\cos\theta\\
y &= v = r\sin\theta
\end{align*}

The combined Jacobian is: $|J| = 2r$ (we'll verify this in the next activity)

The integral becomes:
$$\text{Area} = \int_0^{2\pi}\int_0^1 2r\,dr\,d\theta$$

Evaluate:
$$= \int_0^{2\pi} \left[\answer{r^2}\right]_0^1 d\theta = \int_0^{2\pi} 1\,d\theta = \answer{2\pi}$$

This is the area of an ellipse with semi-axes $a=2, b=1$: Area $= \pi ab = \pi(2)(1) = 2\pi$ \checkmark

\begin{feedback}
By transforming to simpler coordinates, we turned a difficult integral into an easy one!
\end{feedback}
\end{problem}

\section*{Triple Integrals: Cylindrical Coordinates}

\begin{problem}
For cylindrical coordinates:
\begin{align*}
x &= r\cos\theta\\
y &= r\sin\theta\\
z &= z
\end{align*}

The volume element transforms as:
$$dV = dx\,dy\,dz = \answer{r}\,dr\,d\theta\,dz$$

This is really just:
\begin{multipleChoice}
    \choice{A random formula}
    \choice[correct]{Polar in the $xy$-plane, times $dz$ for height}
    \choice{Spherical coordinates}
\end{multipleChoice}

Evaluate: $\displaystyle\iiint_E z\,dV$ where $E$ is the cylinder $x^2 + y^2 \leq 4$, $0 \leq z \leq 3$.

In cylindrical coordinates:
\begin{itemize}
    \item $0 \leq r \leq \answer{2}$
    \item $0 \leq \theta \leq \answer{2\pi}$
    \item $0 \leq z \leq \answer{3}$
\end{itemize}

$$\iiint_E z\,dV = \int_0^{2\pi}\int_0^2\int_0^3 z \cdot r\,dz\,dr\,d\theta$$

Evaluate from inside out:
$$\int_0^3 z\,dz = \left[\frac{z^2}{2}\right]_0^3 = \answer{9/2}$$

$$\int_0^2 \frac{9r}{2}\,dr = \frac{9}{2}\left[\frac{r^2}{2}\right]_0^2 = \frac{9}{2} \cdot 2 = \answer{9}$$

$$\int_0^{2\pi} 9\,d\theta = 9 \cdot 2\pi = \answer{18\pi}$$

\begin{feedback}
Cylindrical coordinates turned this into a straightforward computation with constant bounds!
\end{feedback}
\end{problem}

\section*{Triple Integrals: Spherical Coordinates}

\begin{problem}
For spherical coordinates:
\begin{align*}
x &= \rho\sin\phi\cos\theta\\
y &= \rho\sin\phi\sin\theta\\
z &= \rho\cos\phi
\end{align*}

The volume element is:
$$dV = dx\,dy\,dz = \answer{\rho^2\sin\phi}\,d\rho\,d\phi\,d\theta$$

The Jacobian $\rho^2\sin\phi$ accounts for:
\begin{selectAll}
    \choice[correct]{How spherical shells expand with radius}
    \choice[correct]{The latitude effect (circles are smaller near poles)}
    \choice{Random factors}
    \choice[correct]{The geometry of spherical coordinates}
\end{selectAll}

Find the volume of a cone: $z \geq \sqrt{x^2+y^2}$, $0 \leq z \leq 1$.

In spherical coordinates, $z = \rho\cos\phi$ and $\sqrt{x^2+y^2} = \rho\sin\phi$.

The condition $z \geq \sqrt{x^2+y^2}$ becomes:
$$\rho\cos\phi \geq \rho\sin\phi \Rightarrow \cos\phi \geq \sin\phi$$

This is satisfied when $\phi \leq \answer{\pi/4}$ (the cone angle from the $z$-axis).

The condition $0 \leq z \leq 1$ with $z = \rho\cos\phi$ gives $\rho \leq \frac{1}{\cos\phi}$ (for $\phi \leq \pi/4$).

Bounds:
\begin{itemize}
    \item $0 \leq \rho \leq \frac{1}{\cos\phi}$
    \item $0 \leq \phi \leq \pi/4$
    \item $0 \leq \theta \leq 2\pi$
\end{itemize}

$$V = \int_0^{2\pi}\int_0^{\pi/4}\int_0^{1/\cos\phi} \rho^2\sin\phi\,d\rho\,d\phi\,d\theta$$

This evaluates to $V = \frac{\pi(2-\sqrt{2})}{3}$ (computation omitted).

\begin{feedback}
Spherical coordinates naturally describe cones and spheres with their radial and angular structure!
\end{feedback}
\end{problem}

\section*{When to Change Variables}

\begin{problem}
Consider changing variables when:
\begin{selectAll}
    \choice[correct]{The region has circular, cylindrical, or spherical symmetry}
    \choice[correct]{The function simplifies in other coordinates ($x^2+y^2$ suggests polar)}
    \choice[correct]{The bounds become constants in new coordinates}
    \choice[correct]{An ellipse can be transformed to a circle}
    \choice{You want to make the problem harder}
    \choice[correct]{The Jacobian is easy to compute}
\end{selectAll}

Match the coordinate system to the region:

\textbf{1. Cylinder $x^2+y^2 \leq R^2$, $0 \leq z \leq H$:}
\wordChoice{\choice{Rectangular}\choice[correct]{Cylindrical}\choice{Spherical}}

\textbf{2. Sphere $x^2+y^2+z^2 \leq R^2$:}
\wordChoice{\choice{Rectangular}\choice{Cylindrical}\choice[correct]{Spherical}}

\textbf{3. Rectangle $0 \leq x \leq a$, $0 \leq y \leq b$:}
\wordChoice{\choice[correct]{Rectangular}\choice{Cylindrical}\choice{Spherical}}

\textbf{4. Disk $x^2+y^2 \leq R^2$:}
\wordChoice{\choice{Rectangular}\choice[correct]{Polar}\choice{Spherical}}

\begin{feedback}
Match the coordinate system to the geometry! This is the key to simplifying integrals.
\end{feedback}
\end{problem}

\section*{Converting Between Coordinate Systems}

\begin{problem}
\textbf{From Cylindrical to Spherical Coordinates}

Sometimes a problem is initially set up in one coordinate system but would be simpler in another!

Consider the integral in cylindrical coordinates:
$$I = \int_0^{2\pi} \int_0^{\pi/4} \int_0^{2\cos z} r\sqrt{r^2+z^2}\,dr\,dz\,d\theta$$

The region is complicated in cylindrical coordinates. Let's analyze it:
\begin{itemize}
    \item The bound $0 \leq r \leq 2\cos z$ suggests the radius depends on height
    \item The expression $\sqrt{r^2+z^2}$ appears—this is the distance from the origin!
\end{itemize}

In cylindrical coordinates: $r$ (radial from $z$-axis), $\theta$ (angle), $z$ (height)

In spherical coordinates: $\rho$ (distance from origin), $\phi$ (angle from $z$-axis), $\theta$ (azimuthal angle)

The relationship between systems:
\begin{align*}
r &= \rho\sin\phi\\
z &= \rho\cos\phi\\
\theta &= \theta
\end{align*}

Importantly: $r^2 + z^2 = \rho^2\sin^2\phi + \rho^2\cos^2\phi = \answer{\rho^2}$

So $\sqrt{r^2+z^2} = \answer{\rho}$ (assuming $\rho \geq 0$)

The integrand simplifies: $r\sqrt{r^2+z^2} = (\rho\sin\phi)(\rho) = \answer{\rho^2\sin\phi}$

\textbf{Converting the bounds:}

The upper bound $r = 2\cos z$ becomes:
$$\rho\sin\phi = 2\cos(\rho\cos\phi)$$

Wait, this is messy! Let's reconsider the region description.

Actually, for $0 \leq z \leq \pi/4$ and $0 \leq r \leq 2\cos z$, as $z$ increases from 0 to $\pi/4$:
\begin{itemize}
    \item When $z = 0$: $r$ can be from 0 to $2\cos(0) = 2$
    \item When $z = \pi/4$: $r$ can be from 0 to $2\cos(\pi/4) = \sqrt{2}$
\end{itemize}

In spherical coordinates, $\phi$ is angle from the $z$-axis, so:
\begin{itemize}
    \item $0 \leq \phi \leq \answer{\pi/4}$ (covers the cone-like region)
    \item $0 \leq \rho \leq \answer{2}$ (maximum radius)
    \item $0 \leq \theta \leq \answer{2\pi}$ (full rotation)
\end{itemize}

The volume element transforms: $r\,dr\,dz\,d\theta \to \answer{\rho^2\sin\phi}\,d\rho\,d\phi\,d\theta$

The integral becomes:
$$I = \int_0^{2\pi} \int_0^{\pi/4} \int_0^{2} \rho^2\sin\phi \cdot \rho^2\sin\phi\,d\rho\,d\phi\,d\theta$$

Simplifying: $\rho^2\sin\phi \cdot \rho^2\sin\phi = \answer{\rho^4\sin^2\phi}$

$$I = \int_0^{2\pi} \int_0^{\pi/4} \int_0^{2} \rho^4\sin^2\phi\,d\rho\,d\phi\,d\theta$$

This is much easier to evaluate! The bounds are constants, and the integrand separates nicely.

\begin{feedback}
Converting between coordinate systems requires understanding how the variables relate and how volume elements transform. The key insight was recognizing that $\sqrt{r^2+z^2} = \rho$ in spherical coordinates, which simplified the integrand dramatically!
\end{feedback}
\end{problem}

\begin{problem}
\textbf{Another Example: Ice Cream Cone}

Evaluate the mass of an ``ice cream cone'' region:
\begin{itemize}
    \item Below the hemisphere $z = \sqrt{4 - x^2 - y^2}$ (radius 2, centered at origin)
    \item Above the cone $z = \sqrt{x^2+y^2}$
    \item With density $\delta = z$
\end{itemize}

This is initially in rectangular coordinates. Let's try cylindrical first:

In cylindrical: $x^2+y^2 = r^2$, so:
\begin{itemize}
    \item Hemisphere: $z = \sqrt{4-r^2}$
    \item Cone: $z = r$
\end{itemize}

The cone and hemisphere intersect when:
$$r = \sqrt{4-r^2} \implies r^2 = 4-r^2 \implies 2r^2 = 4 \implies r = \answer{\sqrt{2}}$$

In cylindrical coordinates:
$$M = \int_0^{2\pi} \int_0^{\sqrt{2}} \int_r^{\sqrt{4-r^2}} z \cdot r\,dz\,dr\,d\theta$$

The $z$-integral is: $\int_r^{\sqrt{4-r^2}} z\,dz = \left[\frac{z^2}{2}\right]_r^{\sqrt{4-r^2}} = \frac{4-r^2}{2} - \frac{r^2}{2} = \answer{2-r^2}$

But what if we use spherical coordinates instead?

In spherical coordinates:
\begin{itemize}
    \item The hemisphere is simply: $\rho = \answer{2}$
    \item The cone $z = \sqrt{x^2+y^2}$ becomes: $\rho\cos\phi = \rho\sin\phi$, so $\tan\phi = 1$, giving $\phi = \answer{\pi/4}$
\end{itemize}

The density $\delta = z = \rho\cos\phi$

In spherical:
$$M = \int_0^{2\pi} \int_0^{\pi/4} \int_0^{2} (\rho\cos\phi) \cdot \rho^2\sin\phi\,d\rho\,d\phi\,d\theta$$

$$= \int_0^{2\pi} d\theta \int_0^{\pi/4} \cos\phi\sin\phi\,d\phi \int_0^{2} \rho^3\,d\rho$$

Which coordinate system is easier here?
\begin{multipleChoice}
    \choice{Cylindrical - the bounds are more familiar}
    \choice[correct]{Spherical - the bounds are constants and the integral separates}
    \choice{Rectangular - it's always safest}
    \choice{They're equally difficult}
\end{multipleChoice}

The spherical integral separates into three independent integrals:
\begin{itemize}
    \item $\int_0^{2\pi} d\theta = \answer{2\pi}$
    \item $\int_0^{\pi/4} \cos\phi\sin\phi\,d\phi = \int_0^{\pi/4} \frac{1}{2}\sin(2\phi)\,d\phi = \answer{1/4}$
    \item $\int_0^{2} \rho^3\,d\rho = \left[\frac{\rho^4}{4}\right]_0^2 = \answer{4}$
\end{itemize}

Total mass: $M = 2\pi \cdot \frac{1}{4} \cdot 4 = \answer{2\pi}$

\begin{feedback}
Excellent! The spherical coordinate system matched the geometry perfectly—both the hemisphere and cone had simple descriptions, and the integral separated into three easy pieces. This demonstrates the power of choosing the right coordinate system!
\end{feedback}
\end{problem}

\section*{Summary of Coordinate Transformations}

\begin{problem}
Complete the transformation summary:

\textbf{Polar Coordinates:}
\begin{itemize}
    \item Transformation: $x = r\cos\theta$, $y = r\sin\theta$
    \item Jacobian: $|J| = \answer{r}$
    \item Element: $dA = \answer{r\,dr\,d\theta}$
\end{itemize}

\textbf{Cylindrical Coordinates:}
\begin{itemize}
    \item Transformation: $x = r\cos\theta$, $y = r\sin\theta$, $z = z$
    \item Jacobian: $|J| = \answer{r}$
    \item Element: $dV = \answer{r\,dr\,d\theta\,dz}$
\end{itemize}

\textbf{Spherical Coordinates:}
\begin{itemize}
    \item Transformation: $x = \rho\sin\phi\cos\theta$, $y = \rho\sin\phi\sin\theta$, $z = \rho\cos\phi$
    \item Jacobian: $|J| = \answer{\rho^2\sin\phi}$
    \item Element: $dV = \answer{\rho^2\sin\phi\,d\rho\,d\phi\,d\theta}$
\end{itemize}

The Jacobian in each case tells us:
\begin{multipleChoice}
    \choice{The function value}
    \choice[correct]{How volume elements scale under the transformation}
    \choice{The bounds of integration}
    \choice{The region shape}
\end{multipleChoice}

\begin{feedback}
These standard transformations and their Jacobians should be memorized—they appear constantly in applications!
\end{feedback}
\end{problem}

\section*{Final Practice}

\begin{problem}
Select all TRUE statements about changing variables:

\begin{selectAll}
    \choice[correct]{The Jacobian accounts for how areas/volumes change}
    \choice[correct]{Change of variables extends u-substitution to multiple dimensions}
    \choice{The Jacobian is always 1}
    \choice[correct]{Polar, cylindrical, and spherical are the most common transformations}
    \choice[correct]{Good coordinate choice simplifies both bounds and integrands}
    \choice{You can ignore the Jacobian if the transformation is simple}
    \choice[correct]{The Jacobian can be computed using partial derivatives}
    \choice[correct]{Linear transformations have constant Jacobians}
\end{selectAll}

\begin{feedback}
Perfect! In the next activity, we'll learn exactly how to compute the Jacobian determinant for any transformation.
\end{feedback}
\end{problem}


\section*{Introduction: What is a Jacobian?}

The Jacobian is the multivariable calculus tool that tells us how areas and volumes change under coordinate transformations.

\begin{problem}
In single-variable u-substitution, we have:
$$\int f(g(x)) g'(x)\,dx = \int f(u)\,du$$

The derivative $g'(x)$ measures:
\begin{multipleChoice}
    \choice{The value of the function}
    \choice[correct]{How $dx$ stretches or compresses to become $du$}
    \choice{The area under the curve}
    \choice{Nothing important}
\end{multipleChoice}

For transformations in multiple dimensions, we need to measure how:
\begin{selectAll}
    \choice[correct]{Areas change in 2D}
    \choice[correct]{Volumes change in 3D}
    \choice{Functions change}
    \choice[correct]{Small regions stretch, compress, rotate, or shear}
\end{selectAll}

\begin{feedback}
The Jacobian determinant is the multidimensional analog of the derivative $g'(x)$!
\end{feedback}
\end{problem}

\section*{The Jacobian Matrix}

\begin{definition}
For a transformation from $(u,v)$ to $(x,y)$:
\begin{align*}
x &= x(u,v)\\
y &= y(u,v)
\end{align*}

The \textbf{Jacobian matrix} is:
$$\mathbf{J} = \begin{bmatrix} 
\frac{\partial x}{\partial u} & \frac{\partial x}{\partial v} \\[8pt]
\frac{\partial y}{\partial u} & \frac{\partial y}{\partial v}
\end{bmatrix}$$

This matrix contains all first-order partial derivatives of the transformation.
\end{definition}

\begin{problem}
The Jacobian matrix tells us how:
\begin{selectAll}
    \choice[correct]{Small displacements in the $uv$-plane map to the $xy$-plane}
    \choice[correct]{The transformation behaves locally (at each point)}
    \choice{The function values change}
    \choice[correct]{Vectors transform under the change of coordinates}
\end{selectAll}

For a transformation $(u,v) \to (x,y)$, the Jacobian matrix has:
\begin{itemize}
    \item $\answer{2}$ rows
    \item $\answer{2}$ columns
    \item $\answer{4}$ entries (partial derivatives)
\end{itemize}

\begin{feedback}
The Jacobian matrix is a linear approximation to the transformation at each point!
\end{feedback}
\end{problem}

\section*{The Jacobian Determinant}

\begin{definition}
The \textbf{Jacobian determinant} (or just the ``Jacobian'') is:
$$J = \det(\mathbf{J}) = \frac{\partial(x,y)}{\partial(u,v)} = \frac{\partial x}{\partial u}\frac{\partial y}{\partial v} - \frac{\partial x}{\partial v}\frac{\partial y}{\partial u}$$

The absolute value $|J|$ gives the area scaling factor.
\end{definition}

\begin{problem}
For a $2 \times 2$ matrix $\begin{bmatrix} a & b \\ c & d \end{bmatrix}$, the determinant is:

$$\det = ad - bc = \answer{ad - bc}$$

This is the formula we use for the Jacobian determinant!

Why do we use $|J|$ (absolute value) in integrals?
\begin{multipleChoice}
    \choice{Because determinants are always negative}
    \choice{To make the math easier}
    \choice[correct]{Because we want the magnitude of scaling, not the sign (which indicates orientation)}
    \choice{It's a random convention}
\end{multipleChoice}

\begin{feedback}
The absolute value ensures we get a positive area scaling factor. The sign of $J$ tells us about orientation (which way the transformation ``flips'' things).
\end{feedback}
\end{problem}

\begin{problem}
\textbf{Electric Charge on an Ellipse}

An elliptical region defined by $\frac{x^2}{9} + \frac{y^2}{4} \leq 1$ has surface charge density:
$$\sigma(x,y) = \frac{5}{1 + x^2} \text{ charge per square meter}$$

\textbf{Strategy: Use a transformation!}

Let $u = \frac{x}{3}$ and $v = \frac{y}{2}$. Then the ellipse transforms to the unit circle: $u^2 + v^2 \leq 1$.

From the transformations: $x = \answer{3u}$ and $y = \answer{2v}$

The Jacobian is:
$$J = \frac{\partial(x,y)}{\partial(u,v)} = \begin{vmatrix} \frac{\partial x}{\partial u} & \frac{\partial x}{\partial v} \\ \frac{\partial y}{\partial u} & \frac{\partial y}{\partial v} \end{vmatrix} = \begin{vmatrix} \answer{3} & \answer{0} \\ \answer{0} & \answer{2} \end{vmatrix} = \answer{6}$$

The differential element transforms: $dA = dx\,dy = |J|\,du\,dv = \answer{6}\,du\,dv$

The charge density in new coordinates: $\sigma(x,y) = \frac{5}{1+x^2} = \frac{5}{1+(3u)^2} = \frac{5}{1+9u^2}$

The small charge element: $dQ = \sigma(x,y)\,dA = \frac{5}{1+9u^2} \cdot \answer{6}\,du\,dv$

Total charge (using polar coordinates $u = r\cos\theta$, $v = r\sin\theta$ in the unit circle):
$$Q = \int_0^{2\pi} \int_0^1 \frac{5}{1+9r^2\cos^2\theta} \cdot 6 \cdot \answer{r}\,dr\,d\theta$$

The key steps were:
\begin{selectAll}
    \choice[correct]{Transform the ellipse to a circle}
    \choice[correct]{Compute the Jacobian for area scaling}
    \choice[correct]{Express the charge density in new coordinates}
    \choice[correct]{Use polar coordinates in the unit circle}
    \choice{Ignore the geometry}
\end{selectAll}

\begin{feedback}
Outstanding! You've combined coordinate transformations with careful differential construction. The ellipse problem became manageable by transforming to a circle, but you had to track how $dA$ changes (via the Jacobian) and how the charge density function transforms!
\end{feedback}
\end{problem}


\section*{Computing Jacobians: Step by Step}

\begin{problem}
Let's compute the Jacobian for polar coordinates:
\begin{align*}
x &= r\cos\theta\\
y &= r\sin\theta
\end{align*}

\textbf{Step 1: Compute partial derivatives}

$$\frac{\partial x}{\partial r} = \answer{\cos\theta}$$

$$\frac{\partial x}{\partial \theta} = \answer{-r\sin\theta}$$

$$\frac{\partial y}{\partial r} = \answer{\sin\theta}$$

$$\frac{\partial y}{\partial \theta} = \answer{r\cos\theta}$$

\textbf{Step 2: Form the Jacobian matrix}

$$\mathbf{J} = \begin{bmatrix}
\cos\theta & -r\sin\theta\\
\sin\theta & r\cos\theta
\end{bmatrix}$$

\textbf{Step 3: Compute the determinant}

$$J = (\cos\theta)(r\cos\theta) - (-r\sin\theta)(\sin\theta)$$
$$= r\cos^2\theta + r\sin^2\theta = r(\cos^2\theta + \sin^2\theta) = \answer{r}$$

Therefore, $|J| = r$ (assuming $r \geq 0$).

This confirms: $dA = dx\,dy = \answer{r}\,dr\,d\theta$

\begin{feedback}
We derived the polar area element from first principles using the Jacobian!
\end{feedback}
\end{problem}

\section*{Example: Linear Transformation}

\begin{problem}
Compute the Jacobian for:
\begin{align*}
x &= 3u - 2v\\
y &= u + 4v
\end{align*}

\textbf{Partial derivatives:}

$$\frac{\partial x}{\partial u} = \answer{3}, \quad \frac{\partial x}{\partial v} = \answer{-2}$$

$$\frac{\partial y}{\partial u} = \answer{1}, \quad \frac{\partial y}{\partial v} = \answer{4}$$

\textbf{Jacobian determinant:}

$$J = (3)(4) - (-2)(1) = 12 + 2 = \answer{14}$$

Since this is a linear transformation, $|J| = \answer{14}$ is constant everywhere.

This means:
\begin{itemize}
    \item A region with area $A$ in the $uv$-plane maps to area $\answer{14A}$ in the $xy$-plane
    \item The unit square ($1 \times 1$) maps to a parallelogram with area $\answer{14}$
\end{itemize}

\begin{feedback}
Linear transformations have constant Jacobians! The transformation uniformly scales all areas by the same factor.
\end{feedback}
\end{problem}

\section*{Example: Nonlinear Transformation}

\begin{problem}
Compute the Jacobian for:
\begin{align*}
x &= u^2 - v^2\\
y &= 2uv
\end{align*}

(This is related to the complex function $z \mapsto z^2$!)

\textbf{Partial derivatives:}

$$\frac{\partial x}{\partial u} = \answer{2u}, \quad \frac{\partial x}{\partial v} = \answer{-2v}$$

$$\frac{\partial y}{\partial u} = \answer{2v}, \quad \frac{\partial y}{\partial v} = \answer{2u}$$

\textbf{Jacobian determinant:}

$$J = (2u)(2u) - (-2v)(2v) = 4u^2 + 4v^2 = \answer{4(u^2 + v^2)}$$

Therefore: $|J| = 4(u^2 + v^2)$

This Jacobian:
\begin{multipleChoice}
    \choice{Is constant}
    \choice[correct]{Depends on position $(u,v)$}
    \choice{Is always zero}
    \choice{Is always one}
\end{multipleChoice}

Where is the scaling factor largest?
\begin{multipleChoice}
    \choice{At the origin}
    \choice[correct]{Far from the origin (large $u^2+v^2$)}
    \choice{It's the same everywhere}
\end{multipleChoice}

\begin{feedback}
Nonlinear transformations have variable Jacobians—the scaling changes from point to point!
\end{feedback}
\end{problem}

\section*{Geometric Interpretation}

\begin{problem}
The Jacobian determinant $|J|$ represents:

The factor by which \wordChoice{\choice{length}\choice[correct]{area}\choice{volume}\choice{angle}} is scaled locally.

Consider a small rectangle in the $uv$-plane with area $\Delta u \cdot \Delta v$.

After transformation to the $xy$-plane, the area becomes approximately:
$$\Delta A \approx \answer{|J|} \cdot \Delta u \cdot \Delta v$$

As $\Delta u, \Delta v \to 0$, this becomes exact:
$$dA = dx\,dy = |J|\,du\,dv$$

The Jacobian captures:
\begin{selectAll}
    \choice[correct]{Local stretching}
    \choice[correct]{Local compression}
    \choice{Global behavior}
    \choice[correct]{How small areas transform}
    \choice{The function values}
\end{selectAll}

\begin{feedback}
Think of the Jacobian as the "magnification factor" for infinitesimal areas under the transformation!
\end{feedback}
\end{problem}

\section*{Three Dimensions: Triple Integrals}

\begin{definition}
For a transformation from $(u,v,w)$ to $(x,y,z)$, the Jacobian matrix is:
$$\mathbf{J} = \begin{bmatrix}
\frac{\partial x}{\partial u} & \frac{\partial x}{\partial v} & \frac{\partial x}{\partial w}\\[8pt]
\frac{\partial y}{\partial u} & \frac{\partial y}{\partial v} & \frac{\partial y}{\partial w}\\[8pt]
\frac{\partial z}{\partial u} & \frac{\partial z}{\partial v} & \frac{\partial z}{\partial w}
\end{bmatrix}$$

The Jacobian determinant is computed using the $3 \times 3$ determinant formula.
\end{definition}

\begin{problem}
For a $3 \times 3$ matrix, the determinant calculation is more complex.

The transformation rule for triple integrals is:
$$\iiint_R f(x,y,z)\,dx\,dy\,dz = \iiint_S f(x(u,v,w), y(u,v,w), z(u,v,w)) \cdot \answer{|J|}\,du\,dv\,dw$$

The Jacobian $|J|$ measures:
\begin{multipleChoice}
    \choice{Area scaling}
    \choice[correct]{Volume scaling}
    \choice{Length scaling}
    \choice{Angle changes}
\end{multipleChoice}

\begin{feedback}
In 3D, the Jacobian tells us how volumes (not areas) scale under the transformation!
\end{feedback}
\end{problem}

\section*{Cylindrical Coordinates Jacobian}

\begin{problem}
Verify the Jacobian for cylindrical coordinates:
\begin{align*}
x &= r\cos\theta\\
y &= r\sin\theta\\
z &= z
\end{align*}

\textbf{Jacobian matrix:}
$$\mathbf{J} = \begin{bmatrix}
\frac{\partial x}{\partial r} & \frac{\partial x}{\partial \theta} & \frac{\partial x}{\partial z}\\[8pt]
\frac{\partial y}{\partial r} & \frac{\partial y}{\partial \theta} & \frac{\partial y}{\partial z}\\[8pt]
\frac{\partial z}{\partial r} & \frac{\partial z}{\partial \theta} & \frac{\partial z}{\partial z}
\end{bmatrix} = \begin{bmatrix}
\answer{\cos\theta} & \answer{-r\sin\theta} & \answer{0}\\[8pt]
\answer{\sin\theta} & \answer{r\cos\theta} & \answer{0}\\[8pt]
\answer{0} & \answer{0} & \answer{1}
\end{bmatrix}$$

Computing the $3 \times 3$ determinant (expand along the third row for simplicity):
$$J = 1 \cdot \begin{vmatrix} \cos\theta & -r\sin\theta \\ \sin\theta & r\cos\theta \end{vmatrix}$$

We already know this $2 \times 2$ determinant from polar coordinates:
$$J = 1 \cdot r = \answer{r}$$

Therefore: $dV = dx\,dy\,dz = \answer{r}\,dr\,d\theta\,dz$

This makes sense because:
\begin{multipleChoice}
    \choice{It's a random result}
    \choice[correct]{Cylindrical = polar ($r\,dr\,d\theta$) in the $xy$-plane times height ($dz$)}
    \choice{All Jacobians equal $r$}
\end{multipleChoice}

\begin{feedback}
The cylindrical Jacobian is just the polar Jacobian, extended by the trivial $z$ direction!
\end{feedback}
\end{problem}

\section*{Spherical Coordinates Jacobian}

\begin{problem}
For spherical coordinates:
\begin{align*}
x &= \rho\sin\phi\cos\theta\\
y &= \rho\sin\phi\sin\theta\\
z &= \rho\cos\phi
\end{align*}

The Jacobian matrix is:
$$\mathbf{J} = \begin{bmatrix}
\sin\phi\cos\theta & \rho\cos\phi\cos\theta & -\rho\sin\phi\sin\theta\\
\sin\phi\sin\theta & \rho\cos\phi\sin\theta & \rho\sin\phi\cos\theta\\
\cos\phi & -\rho\sin\phi & 0
\end{bmatrix}$$

Computing this $3 \times 3$ determinant (details omitted) gives:
$$J = \rho^2\sin\phi$$

Therefore: $dV = dx\,dy\,dz = \answer{\rho^2\sin\phi}\,d\rho\,d\phi\,d\theta$

The factors in the Jacobian represent:
\begin{selectAll}
    \choice[correct]{$\rho^2$: shells expand quadratically with radius}
    \choice[correct]{$\sin\phi$: circles shrink near the poles}
    \choice{Random mathematical constants}
    \choice[correct]{Geometric properties of the sphere}
\end{selectAll}

\begin{feedback}
The spherical Jacobian $\rho^2\sin\phi$ captures the beautiful geometry of spherical coordinates!
\end{feedback}
\end{problem}

\section*{Using Jacobians in Practice}

\begin{problem}
To evaluate $\displaystyle\iint_R e^{(x-y)/(x+y)}\,dA$ where $R$ is the square with vertices $(0,0)$, $(1,1)$, $(2,0)$, $(0,2)$.

This region is a rotated square. Let's try:
\begin{align*}
u &= x + y\\
v &= x - y
\end{align*}

Solve for $x$ and $y$:
$$x = \frac{u+v}{2} = \answer{(u+v)/2}$$
$$y = \frac{u-v}{2} = \answer{(u-v)/2}$$

\textbf{Compute the Jacobian:}

$$\frac{\partial x}{\partial u} = \answer{1/2}, \quad \frac{\partial x}{\partial v} = \answer{1/2}$$

$$\frac{\partial y}{\partial u} = \answer{1/2}, \quad \frac{\partial y}{\partial v} = \answer{-1/2}$$

$$J = \left(\frac{1}{2}\right)\left(-\frac{1}{2}\right) - \left(\frac{1}{2}\right)\left(\frac{1}{2}\right) = -\frac{1}{4} - \frac{1}{4} = -\frac{1}{2}$$

Therefore: $|J| = \answer{1/2}$

The function becomes:
$$e^{(x-y)/(x+y)} = e^{v/u}$$

The vertices transform:
\begin{itemize}
    \item $(0,0) \to (u,v) = (0,0)$
    \item $(1,1) \to (u,v) = (\answer{2}, \answer{0})$
    \item $(2,0) \to (u,v) = (\answer{2}, \answer{2})$
    \item $(0,2) \to (u,v) = (\answer{2}, \answer{-2})$
\end{itemize}

The rotated square in the $xy$-plane maps to a rectangle in the $uv$-plane!

The integral becomes:
$$\iint_R e^{(x-y)/(x+y)}\,dA = \int_{-2}^{2}\int_0^2 e^{v/u} \cdot \frac{1}{2}\,du\,dv$$

Much easier to set up with nice rectangular bounds!

\begin{feedback}
Choosing the right transformation can turn a difficult region into a simple rectangle!
\end{feedback}
\end{problem}

\section*{When is the Jacobian Zero?}

\begin{problem}
If $J = 0$ at a point, the transformation is:
\begin{multipleChoice}
    \choice{Perfect}
    \choice{One-to-one}
    \choice[correct]{Degenerate (collapses dimensions)}
    \choice{Conformal}
\end{multipleChoice}

Example: $x = u^2$, $y = u^2$

This maps the entire $uv$-plane onto:
\begin{multipleChoice}
    \choice{The entire $xy$-plane}
    \choice{A circle}
    \choice[correct]{A line ($y = x$)}
    \choice{A parabola}
\end{multipleChoice}

The Jacobian is:
$$J = \frac{\partial x}{\partial u}\frac{\partial y}{\partial v} - \frac{\partial x}{\partial v}\frac{\partial y}{\partial u} = (2u)(0) - (0)(2u) = \answer{0}$$

When $J = 0$:
\begin{selectAll}
    \choice[correct]{The transformation is not one-to-one}
    \choice[correct]{Area is collapsed}
    \choice{We can still use the transformation}
    \choice[correct]{We need to be careful about the domain}
\end{selectAll}

\begin{feedback}
A zero Jacobian indicates that the transformation collapses higher-dimensional regions onto lower-dimensional sets!
\end{feedback}
\end{problem}

\section*{Inverse Transformations}

\begin{problem}
If we have $x = x(u,v)$ and $y = y(u,v)$, sometimes we can invert to get $u = u(x,y)$ and $v = v(x,y)$.

The Jacobians of a transformation and its inverse are related:
$$J_{uv \to xy} \cdot J_{xy \to uv} = \answer{1}$$

Or: $J_{xy \to uv} = \frac{1}{J_{uv \to xy}}$

For polar coordinates:
\begin{itemize}
    \item Forward: $(r,\theta) \to (x,y)$ has $J = r$
    \item Inverse: $(x,y) \to (r,\theta)$ has $J = \answer{1/r}$
\end{itemize}

This relationship means:
\begin{multipleChoice}
    \choice{Both Jacobians are always 1}
    \choice[correct]{If one transformation expands, the inverse contracts}
    \choice{Jacobians are independent}
\end{multipleChoice}

\begin{feedback}
The inverse transformation "undoes" the scaling—if one expands by factor $r$, the other contracts by factor $1/r$!
\end{feedback}
\end{problem}

\section*{Summary and Key Formulas}

\begin{problem}
Complete the Jacobian summary:

\textbf{Definition (2D):}
$$J = \frac{\partial(x,y)}{\partial(u,v)} = \frac{\partial x}{\partial u}\frac{\partial y}{\partial v} - \answer{\frac{\partial x}{\partial v}\frac{\partial y}{\partial u}}$$

\textbf{In integrals:}
$$\iint_R f(x,y)\,dx\,dy = \iint_S f(x(u,v), y(u,v)) \cdot \answer{|J|}\,du\,dv$$

\textbf{Standard Jacobians:}
\begin{itemize}
    \item Polar: $|J| = \answer{r}$
    \item Cylindrical: $|J| = \answer{r}$
    \item Spherical: $|J| = \answer{\rho^2\sin\phi}$
\end{itemize}

\textbf{The Jacobian measures:}
\begin{selectAll}
    \choice[correct]{How areas scale (in 2D)}
    \choice[correct]{How volumes scale (in 3D)}
    \choice[correct]{Local magnification factor}
    \choice{Function values}
    \choice[correct]{The determinant of the transformation's derivative matrix}
\end{selectAll}

\begin{feedback}
The Jacobian determinant is essential for any change of variables in multiple integrals!
\end{feedback}
\end{problem}

\section*{Final Check}

\begin{problem}
Select all TRUE statements about Jacobians:

\begin{selectAll}
    \choice[correct]{The Jacobian matrix contains all first partial derivatives}
    \choice[correct]{The Jacobian determinant gives the scaling factor}
    \choice{The Jacobian is always positive}
    \choice[correct]{We use $|J|$ in integrals to ensure positive scaling}
    \choice[correct]{Linear transformations have constant Jacobians}
    \choice{The Jacobian is the same as the gradient}
    \choice[correct]{A zero Jacobian indicates a degenerate transformation}
    \choice[correct]{Polar, cylindrical, and spherical Jacobians should be memorized}
    \choice[correct]{The Jacobian extends the concept of $g'(x)$ from u-substitution}
\end{selectAll}

\begin{feedback}
Perfect! You now understand how to compute and use Jacobians for any coordinate transformation in multiple integrals!
\end{feedback}
\end{problem}

\end{document}
