\documentclass{ximera}

\title{Jacobians and Transformations}
\author{Zack Reed}

\begin{document}
\begin{abstract}
In this activity we learn how to compute the Jacobian determinant for coordinate transformations and understand its geometric meaning as a scaling factor for area and volume.
\end{abstract}
\maketitle

\section*{Introduction: What is a Jacobian?}

The Jacobian is the multivariable calculus tool that tells us how areas and volumes change under coordinate transformations.

\begin{problem}
In single-variable u-substitution, we have:
$$\int f(g(x)) g'(x)\,dx = \int f(u)\,du$$

The derivative $g'(x)$ measures:
\begin{multipleChoice}
    \choice{The value of the function}
    \choice[correct]{How $dx$ stretches or compresses to become $du$}
    \choice{The area under the curve}
    \choice{Nothing important}
\end{multipleChoice}

For transformations in multiple dimensions, we need to measure how:
\begin{selectAll}
    \choice[correct]{Areas change in 2D}
    \choice[correct]{Volumes change in 3D}
    \choice{Functions change}
    \choice[correct]{Small regions stretch, compress, rotate, or shear}
\end{selectAll}

\begin{feedback}
The Jacobian determinant is the multidimensional analog of the derivative $g'(x)$!
\end{feedback}
\end{problem}

\section*{The Jacobian Matrix}

\begin{definition}
For a transformation from $(u,v)$ to $(x,y)$:
\begin{align*}
x &= x(u,v)\\
y &= y(u,v)
\end{align*}

The \textbf{Jacobian matrix} is:
$$\mathbf{J} = \begin{bmatrix} 
\frac{\partial x}{\partial u} & \frac{\partial x}{\partial v} \\[8pt]
\frac{\partial y}{\partial u} & \frac{\partial y}{\partial v}
\end{bmatrix}$$

This matrix contains all first-order partial derivatives of the transformation.
\end{definition}

\begin{problem}
The Jacobian matrix tells us how:
\begin{selectAll}
    \choice[correct]{Small displacements in the $uv$-plane map to the $xy$-plane}
    \choice[correct]{The transformation behaves locally (at each point)}
    \choice{The function values change}
    \choice[correct]{Vectors transform under the change of coordinates}
\end{selectAll}

For a transformation $(u,v) \to (x,y)$, the Jacobian matrix has:
\begin{itemize}
    \item $\answer{2}$ rows
    \item $\answer{2}$ columns
    \item $\answer{4}$ entries (partial derivatives)
\end{itemize}

\begin{feedback}
The Jacobian matrix is a linear approximation to the transformation at each point!
\end{feedback}
\end{problem}

\section*{The Jacobian Determinant}

\begin{definition}
The \textbf{Jacobian determinant} (or just "the Jacobian") is:
$$J = \det(\mathbf{J}) = \frac{\partial(x,y)}{\partial(u,v)} = \frac{\partial x}{\partial u}\frac{\partial y}{\partial v} - \frac{\partial x}{\partial v}\frac{\partial y}{\partial u}$$

The absolute value $|J|$ gives the area scaling factor.
\end{definition}

\begin{problem}
For a $2 \times 2$ matrix $\begin{bmatrix} a & b \\ c & d \end{bmatrix}$, the determinant is:

$$\det = ad - bc = \answer{ad - bc}$$

This is the formula we use for the Jacobian determinant!

Why do we use $|J|$ (absolute value) in integrals?
\begin{multipleChoice}
    \choice{Because determinants are always negative}
    \choice{To make the math easier}
    \choice[correct]{Because we want the magnitude of scaling, not the sign (which indicates orientation)}
    \choice{It's a random convention}
\end{multipleChoice}

\begin{feedback}
The absolute value ensures we get a positive area scaling factor. The sign of $J$ tells us about orientation (which way the transformation "flips" things).
\end{feedback}
\end{problem}

\section*{Computing Jacobians: Step by Step}

\begin{problem}
Let's compute the Jacobian for polar coordinates:
\begin{align*}
x &= r\cos\theta\\
y &= r\sin\theta
\end{align*}

\textbf{Step 1: Compute partial derivatives}

$$\frac{\partial x}{\partial r} = \answer{\cos\theta}$$

$$\frac{\partial x}{\partial \theta} = \answer{-r\sin\theta}$$

$$\frac{\partial y}{\partial r} = \answer{\sin\theta}$$

$$\frac{\partial y}{\partial \theta} = \answer{r\cos\theta}$$

\textbf{Step 2: Form the Jacobian matrix}

$$\mathbf{J} = \begin{bmatrix}
\cos\theta & -r\sin\theta\\
\sin\theta & r\cos\theta
\end{bmatrix}$$

\textbf{Step 3: Compute the determinant}

$$J = (\cos\theta)(r\cos\theta) - (-r\sin\theta)(\sin\theta)$$
$$= r\cos^2\theta + r\sin^2\theta = r(\cos^2\theta + \sin^2\theta) = \answer{r}$$

Therefore, $|J| = r$ (assuming $r \geq 0$).

This confirms: $dA = dx\,dy = \answer{r}\,dr\,d\theta$

\begin{feedback}
We derived the polar area element from first principles using the Jacobian!
\end{feedback}
\end{problem}

\section*{Example: Linear Transformation}

\begin{problem}
Compute the Jacobian for:
\begin{align*}
x &= 3u - 2v\\
y &= u + 4v
\end{align*}

\textbf{Partial derivatives:}

$$\frac{\partial x}{\partial u} = \answer{3}, \quad \frac{\partial x}{\partial v} = \answer{-2}$$

$$\frac{\partial y}{\partial u} = \answer{1}, \quad \frac{\partial y}{\partial v} = \answer{4}$$

\textbf{Jacobian determinant:}

$$J = (3)(4) - (-2)(1) = 12 + 2 = \answer{14}$$

Since this is a linear transformation, $|J| = \answer{14}$ is constant everywhere.

This means:
\begin{itemize}
    \item A region with area $A$ in the $uv$-plane maps to area $\answer{14A}$ in the $xy$-plane
    \item The unit square ($1 \times 1$) maps to a parallelogram with area $\answer{14}$
\end{itemize}

\begin{feedback}
Linear transformations have constant Jacobians! The transformation uniformly scales all areas by the same factor.
\end{feedback}
\end{problem}

\section*{Example: Nonlinear Transformation}

\begin{problem}
Compute the Jacobian for:
\begin{align*}
x &= u^2 - v^2\\
y &= 2uv
\end{align*}

(This is related to the complex function $z \mapsto z^2$!)

\textbf{Partial derivatives:}

$$\frac{\partial x}{\partial u} = \answer{2u}, \quad \frac{\partial x}{\partial v} = \answer{-2v}$$

$$\frac{\partial y}{\partial u} = \answer{2v}, \quad \frac{\partial y}{\partial v} = \answer{2u}$$

\textbf{Jacobian determinant:}

$$J = (2u)(2u) - (-2v)(2v) = 4u^2 + 4v^2 = \answer{4(u^2 + v^2)}$$

Therefore: $|J| = 4(u^2 + v^2)$

This Jacobian:
\begin{multipleChoice}
    \choice{Is constant}
    \choice[correct]{Depends on position $(u,v)$}
    \choice{Is always zero}
    \choice{Is always one}
\end{multipleChoice}

Where is the scaling factor largest?
\begin{multipleChoice}
    \choice{At the origin}
    \choice[correct]{Far from the origin (large $u^2+v^2$)}
    \choice{It's the same everywhere}
\end{multipleChoice}

\begin{feedback}
Nonlinear transformations have variable Jacobians—the scaling changes from point to point!
\end{feedback}
\end{problem}

\section*{Geometric Interpretation}

\begin{problem}
The Jacobian determinant $|J|$ represents:

The factor by which \wordChoice{\choice{length}\choice[correct]{area}\choice{volume}\choice{angle}} is scaled locally.

Consider a small rectangle in the $uv$-plane with area $\Delta u \cdot \Delta v$.

After transformation to the $xy$-plane, the area becomes approximately:
$$\Delta A \approx \answer{|J|} \cdot \Delta u \cdot \Delta v$$

As $\Delta u, \Delta v \to 0$, this becomes exact:
$$dA = dx\,dy = |J|\,du\,dv$$

The Jacobian captures:
\begin{selectAll}
    \choice[correct]{Local stretching}
    \choice[correct]{Local compression}
    \choice{Global behavior}
    \choice[correct]{How small areas transform}
    \choice{The function values}
\end{selectAll}

\begin{feedback}
Think of the Jacobian as the "magnification factor" for infinitesimal areas under the transformation!
\end{feedback}
\end{problem}

\section*{Three Dimensions: Triple Integrals}

\begin{definition}
For a transformation from $(u,v,w)$ to $(x,y,z)$, the Jacobian matrix is:
$$\mathbf{J} = \begin{bmatrix}
\frac{\partial x}{\partial u} & \frac{\partial x}{\partial v} & \frac{\partial x}{\partial w}\\[8pt]
\frac{\partial y}{\partial u} & \frac{\partial y}{\partial v} & \frac{\partial y}{\partial w}\\[8pt]
\frac{\partial z}{\partial u} & \frac{\partial z}{\partial v} & \frac{\partial z}{\partial w}
\end{bmatrix}$$

The Jacobian determinant is computed using the $3 \times 3$ determinant formula.
\end{definition}

\begin{problem}
For a $3 \times 3$ matrix, the determinant calculation is more complex.

The transformation rule for triple integrals is:
$$\iiint_R f(x,y,z)\,dx\,dy\,dz = \iiint_S f(x(u,v,w), y(u,v,w), z(u,v,w)) \cdot \answer{|J|}\,du\,dv\,dw$$

The Jacobian $|J|$ measures:
\begin{multipleChoice}
    \choice{Area scaling}
    \choice[correct]{Volume scaling}
    \choice{Length scaling}
    \choice{Angle changes}
\end{multipleChoice}

\begin{feedback}
In 3D, the Jacobian tells us how volumes (not areas) scale under the transformation!
\end{feedback}
\end{problem}

\section*{Cylindrical Coordinates Jacobian}

\begin{problem}
Verify the Jacobian for cylindrical coordinates:
\begin{align*}
x &= r\cos\theta\\
y &= r\sin\theta\\
z &= z
\end{align*}

\textbf{Jacobian matrix:}
$$\mathbf{J} = \begin{bmatrix}
\frac{\partial x}{\partial r} & \frac{\partial x}{\partial \theta} & \frac{\partial x}{\partial z}\\[8pt]
\frac{\partial y}{\partial r} & \frac{\partial y}{\partial \theta} & \frac{\partial y}{\partial z}\\[8pt]
\frac{\partial z}{\partial r} & \frac{\partial z}{\partial \theta} & \frac{\partial z}{\partial z}
\end{bmatrix} = \begin{bmatrix}
\answer{\cos\theta} & \answer{-r\sin\theta} & \answer{0}\\[8pt]
\answer{\sin\theta} & \answer{r\cos\theta} & \answer{0}\\[8pt]
\answer{0} & \answer{0} & \answer{1}
\end{bmatrix}$$

Computing the $3 \times 3$ determinant (expand along the third row for simplicity):
$$J = 1 \cdot \begin{vmatrix} \cos\theta & -r\sin\theta \\ \sin\theta & r\cos\theta \end{vmatrix}$$

We already know this $2 \times 2$ determinant from polar coordinates:
$$J = 1 \cdot r = \answer{r}$$

Therefore: $dV = dx\,dy\,dz = \answer{r}\,dr\,d\theta\,dz$

This makes sense because:
\begin{multipleChoice}
    \choice{It's a random result}
    \choice[correct]{Cylindrical = polar ($r\,dr\,d\theta$) in the $xy$-plane times height ($dz$)}
    \choice{All Jacobians equal $r$}
\end{multipleChoice}

\begin{feedback}
The cylindrical Jacobian is just the polar Jacobian, extended by the trivial $z$ direction!
\end{feedback}
\end{problem}

\section*{Spherical Coordinates Jacobian}

\begin{problem}
For spherical coordinates:
\begin{align*}
x &= \rho\sin\phi\cos\theta\\
y &= \rho\sin\phi\sin\theta\\
z &= \rho\cos\phi
\end{align*}

The Jacobian matrix is:
$$\mathbf{J} = \begin{bmatrix}
\sin\phi\cos\theta & \rho\cos\phi\cos\theta & -\rho\sin\phi\sin\theta\\
\sin\phi\sin\theta & \rho\cos\phi\sin\theta & \rho\sin\phi\cos\theta\\
\cos\phi & -\rho\sin\phi & 0
\end{bmatrix}$$

Computing this $3 \times 3$ determinant (details omitted) gives:
$$J = \rho^2\sin\phi$$

Therefore: $dV = dx\,dy\,dz = \answer{\rho^2\sin\phi}\,d\rho\,d\phi\,d\theta$

The factors in the Jacobian represent:
\begin{selectAll}
    \choice[correct]{$\rho^2$: shells expand quadratically with radius}
    \choice[correct]{$\sin\phi$: circles shrink near the poles}
    \choice{Random mathematical constants}
    \choice[correct]{Geometric properties of the sphere}
\end{selectAll}

\begin{feedback}
The spherical Jacobian $\rho^2\sin\phi$ captures the beautiful geometry of spherical coordinates!
\end{feedback}
\end{problem}

\section*{Using Jacobians in Practice}

\begin{problem}
To evaluate $\displaystyle\iint_R e^{(x-y)/(x+y)}\,dA$ where $R$ is the square with vertices $(0,0)$, $(1,1)$, $(2,0)$, $(0,2)$.

This region is a rotated square. Let's try:
\begin{align*}
u &= x + y\\
v &= x - y
\end{align*}

Solve for $x$ and $y$:
$$x = \frac{u+v}{2} = \answer{(u+v)/2}$$
$$y = \frac{u-v}{2} = \answer{(u-v)/2}$$

\textbf{Compute the Jacobian:}

$$\frac{\partial x}{\partial u} = \answer{1/2}, \quad \frac{\partial x}{\partial v} = \answer{1/2}$$

$$\frac{\partial y}{\partial u} = \answer{1/2}, \quad \frac{\partial y}{\partial v} = \answer{-1/2}$$

$$J = \left(\frac{1}{2}\right)\left(-\frac{1}{2}\right) - \left(\frac{1}{2}\right)\left(\frac{1}{2}\right) = -\frac{1}{4} - \frac{1}{4} = -\frac{1}{2}$$

Therefore: $|J| = \answer{1/2}$

The function becomes:
$$e^{(x-y)/(x+y)} = e^{v/u}$$

The vertices transform:
\begin{itemize}
    \item $(0,0) \to (u,v) = (0,0)$
    \item $(1,1) \to (u,v) = (\answer{2}, \answer{0})$
    \item $(2,0) \to (u,v) = (\answer{2}, \answer{2})$
    \item $(0,2) \to (u,v) = (\answer{2}, \answer{-2})$
\end{itemize}

The rotated square in the $xy$-plane maps to a rectangle in the $uv$-plane!

The integral becomes:
$$\iint_R e^{(x-y)/(x+y)}\,dA = \int_{-2}^{2}\int_0^2 e^{v/u} \cdot \frac{1}{2}\,du\,dv$$

Much easier to set up with nice rectangular bounds!

\begin{feedback}
Choosing the right transformation can turn a difficult region into a simple rectangle!
\end{feedback}
\end{problem}

\section*{When is the Jacobian Zero?}

\begin{problem}
If $J = 0$ at a point, the transformation is:
\begin{multipleChoice}
    \choice{Perfect}
    \choice{One-to-one}
    \choice[correct]{Degenerate (collapses dimensions)}
    \choice{Conformal}
\end{multipleChoice}

Example: $x = u^2$, $y = u^2$

This maps the entire $uv$-plane onto:
\begin{multipleChoice}
    \choice{The entire $xy$-plane}
    \choice{A circle}
    \choice[correct]{A line ($y = x$)}
    \choice{A parabola}
\end{multipleChoice}

The Jacobian is:
$$J = \frac{\partial x}{\partial u}\frac{\partial y}{\partial v} - \frac{\partial x}{\partial v}\frac{\partial y}{\partial u} = (2u)(0) - (0)(2u) = \answer{0}$$

When $J = 0$:
\begin{selectAll}
    \choice[correct]{The transformation is not one-to-one}
    \choice[correct]{Area is collapsed}
    \choice{We can still use the transformation}
    \choice[correct]{We need to be careful about the domain}
\end{selectAll}

\begin{feedback}
A zero Jacobian indicates that the transformation collapses higher-dimensional regions onto lower-dimensional sets!
\end{feedback}
\end{problem}

\section*{Inverse Transformations}

\begin{problem}
If we have $x = x(u,v)$ and $y = y(u,v)$, sometimes we can invert to get $u = u(x,y)$ and $v = v(x,y)$.

The Jacobians of a transformation and its inverse are related:
$$J_{uv \to xy} \cdot J_{xy \to uv} = \answer{1}$$

Or: $J_{xy \to uv} = \frac{1}{J_{uv \to xy}}$

For polar coordinates:
\begin{itemize}
    \item Forward: $(r,\theta) \to (x,y)$ has $J = r$
    \item Inverse: $(x,y) \to (r,\theta)$ has $J = \answer{1/r}$
\end{itemize}

This relationship means:
\begin{multipleChoice}
    \choice{Both Jacobians are always 1}
    \choice[correct]{If one transformation expands, the inverse contracts}
    \choice{Jacobians are independent}
\end{multipleChoice}

\begin{feedback}
The inverse transformation "undoes" the scaling—if one expands by factor $r$, the other contracts by factor $1/r$!
\end{feedback}
\end{problem}

\section*{Summary and Key Formulas}

\begin{problem}
Complete the Jacobian summary:

\textbf{Definition (2D):}
$$J = \frac{\partial(x,y)}{\partial(u,v)} = \frac{\partial x}{\partial u}\frac{\partial y}{\partial v} - \answer{\frac{\partial x}{\partial v}\frac{\partial y}{\partial u}}$$

\textbf{In integrals:}
$$\iint_R f(x,y)\,dx\,dy = \iint_S f(x(u,v), y(u,v)) \cdot \answer{|J|}\,du\,dv$$

\textbf{Standard Jacobians:}
\begin{itemize}
    \item Polar: $|J| = \answer{r}$
    \item Cylindrical: $|J| = \answer{r}$
    \item Spherical: $|J| = \answer{\rho^2\sin\phi}$
\end{itemize}

\textbf{The Jacobian measures:}
\begin{selectAll}
    \choice[correct]{How areas scale (in 2D)}
    \choice[correct]{How volumes scale (in 3D)}
    \choice[correct]{Local magnification factor}
    \choice{Function values}
    \choice[correct]{The determinant of the transformation's derivative matrix}
\end{selectAll}

\begin{feedback}
The Jacobian determinant is essential for any change of variables in multiple integrals!
\end{feedback}
\end{problem}

\section*{Final Check}

\begin{problem}
Select all TRUE statements about Jacobians:

\begin{selectAll}
    \choice[correct]{The Jacobian matrix contains all first partial derivatives}
    \choice[correct]{The Jacobian determinant gives the scaling factor}
    \choice{The Jacobian is always positive}
    \choice[correct]{We use $|J|$ in integrals to ensure positive scaling}
    \choice[correct]{Linear transformations have constant Jacobians}
    \choice{The Jacobian is the same as the gradient}
    \choice[correct]{A zero Jacobian indicates a degenerate transformation}
    \choice[correct]{Polar, cylindrical, and spherical Jacobians should be memorized}
    \choice[correct]{The Jacobian extends the concept of $g'(x)$ from u-substitution}
\end{selectAll}

\begin{feedback}
Perfect! You now understand how to compute and use Jacobians for any coordinate transformation in multiple integrals!
\end{feedback}
\end{problem}

\end{document}
