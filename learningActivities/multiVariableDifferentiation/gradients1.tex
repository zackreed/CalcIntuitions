\documentclass{ximera}

\title{Tangent Planes and Differentials}
\author{Zack Reed}

\begin{document}
\begin{abstract}
In this activity we extend local linearity from single-variable calculus to multivariable functions through partial derivatives, directional derivatives, tangent planes, and the gradient vector.
\end{abstract}
\maketitle

\section*{Introduction: Local Linearity Revisited}

One of the central ideas from Calculus I was \textbf{local linearity}. For a function $f(x)$, if its derivative $f'(a)$ exists, then we can treat $f$ \textit{as if it were a line} with slope $f'(a)$ if we don't move too far from the point $(a, f(a))$.

The key idea was that \emph{treating a function like a line} significantly reduced the complexity of the function's changes down to simple multiplication: 

$$dy=f'(a) \cdot dx,$$

where $dy$ is a tiny ``differential'' (meaning locally linear) change in the output of the function and $dx$ is a tiny ``differential'' change in the input. In this context, ``differential'' is a special keyword for linear, but linear only in a tiny window around the point $(a,f(a))$.

Here, we extend this idea to multivariable functions, but instead of treating a function like a line, we will treat it like a plane in a small region near a point on the surface $(a,b,f(a,b))$.

\begin{problem}
Let's review local linearity from single-variable calculus.

\begin{expandable}{stuff}{GeoGebra Instructions}
    Alter the ``a='' slider to view the local linearity of $f$ at different points. Check or uncheck the ``Show Local Linearity'' and ``Show Tangent Line'' boxes to control what you view.
\end{expandable}

\begin{center}
\geogebra{msnbuymg}{786}{584}
\end{center}

Answer the following questions:
\begin{enumerate}

\item At the point $(1,.297)$, the function \wordChoice{\choice{is a line of slope $-.463$}\choice[correct]{can be treated like a line with slope $-.463$}\choice{is not differentiable}}. This means that if you move a tiny amount $dx$ away from $x=1$, the change to $y$ \wordChoice{\choice{is exactly}\choice[correct]{is approximately}} $dy=$\wordChoice{\choice[correct]{$-.463\cdot dx$}\choice{$-.463$}\choice{$f(1+dx)$}}.


\item The ``tangent line'' represents \wordChoice{\choice{the actual function}\choice[correct]{the linear approximation of the function at that point}\choice{the derivative function}}.

\end{enumerate}

\begin{feedback}
The tangent line gives us a simple linear approximation: $f(x) \approx f(a) + f'(a)(x-a)$ near $x=a$.
\end{feedback}
\end{problem}

\section*{Extending to Multiple Variables}

Now with multivariable functions, we need to account for variation in at least two variables simultaneously.

We concluded the last module looking at the partial derivatives $\frac{\partial f}{\partial x}$ and $\frac{\partial f}{\partial y}$, which measure the rate of change of $f$ in the $x$-direction and $y$-direction, respectively.

We hinted that these two derivatives were all we needed to build more complex derivatives in other directions because any direction vector $\vec{v}$ can be broken down into $x$ and $y$ components. We now build on this idea and also discuss how we use the partial derivatives $\frac{\partial f}{\partial x}$ and $\frac{\partial f}{\partial y}$ to build the tangent plane approximation of $f$ near a point.

\section*{Partial Derivatives}

First, let's review and more formally define partial derivatives.

\begin{definition}
The \textbf{partial derivative} of $f(x,y)$ with respect to $x$ at $(a,b)$ is:
$$\frac{\partial f}{\partial x}(a,b) = \lim_{h \to 0} \frac{f(a+h,b) - f(a,b)}{h}$$

This measures the rate of change of $f$ in the $x$-direction while holding $y$ constant.

Similarly, the partial derivative with respect to $y$ is:
$$\frac{\partial f}{\partial y}(a,b) = \lim_{h \to 0} \frac{f(a,b+h) - f(a,b)}{h}$$
\end{definition}

\begin{problem}
Let's compute some partial derivatives. For $f(x,y) = x^2y + 3y^2$:

The partial derivative with respect to $x$ is:
$$\frac{\partial f}{\partial x} = \answer{2xy}$$

The partial derivative with respect to $y$ is:
$$\frac{\partial f}{\partial y} = \answer{x^2 + 6y}$$

At the point $(2,1)$:
$$\frac{\partial f}{\partial x}(2,1) = \answer{4}$$
$$\frac{\partial f}{\partial y}(2,1) = \answer{10}$$

\begin{feedback}
When computing $\frac{\partial f}{\partial x}$, treat $y$ as a constant. When computing $\frac{\partial f}{\partial y}$, treat $x$ as a constant. It's like single-variable calculus, but focusing on one variable at a time!
\end{feedback}
\end{problem}

\begin{problem}
Compute partial derivatives for $g(x,y) = e^{xy} + \sin(x) + y^3$:

$$\frac{\partial g}{\partial x} = \answer{ye^{xy} + \cos(x)}$$

$$\frac{\partial g}{\partial y} = \answer{xe^{xy} + 3y^2}$$

\begin{feedback}
For $\frac{\partial g}{\partial x}$: treat $y$ as constant, so $e^{xy}$ differentiates using chain rule to give $ye^{xy}$.
\end{feedback}
\end{problem}

\subsection*{Tangent Planes}

The truest extension of tangent lines is the \textbf{tangent plane}. Instead of just treating a function as if it were a line near the point $(a,f(a))$, we treat the function as if it were a plane near $(a,b,f(a,b))$. This maintains the idea of \emph{local linearity} since planes (and hyperplanes in higher dimensions) are linear versions of surfaces (and hypersurfaces in higher dimensions).

\begin{problem}
In the following GeoGebra applet you can adjust the point $(x,y)$ in the domain and see the resulting tangent plane approximation to the surface.

\begin{expandable}{stuff}{GeoGebra Instructions}
    Drag the point $(x,y)$ on the right screen to alter the location of the tangency point (you may need to hide the surface).
\end{expandable}

\begin{center}
\geogebra{pkdrny3n}{741}{581}
\end{center}

Notice that if you zoom in near the point, moving along the plane is is not much different from moving along the surface itself!

Recalling our understanding of planes from prior modules, and using $dx$, $dy$, and $dz$ to represent tiny changes in $x$, $y$, and $z$ respectively, the tangent plane should be expressed in the following way:

$$A\cdot dx + B \cdot dy + C \cdot dz = 0$$

If the tangent plane is to linearly approximate the surface $z = f(x,y)$ just like the tangent line approximated $y = f(x)$, which of the following must be true (Select All that Apply):
\begin{selectAll}
    \choice[correct]{Small changes to the height $\Delta z$ should not differ much from $dz$ if you don't move too far from $(a,b)$}
    \choice[correct]{$A$ should be related to the rate of change of $z$ with respect to $x$}
    \choice[correct]{$B$ should be related to the rate of change of $z$ with respect to $y$}
    \choice{We can choose any values for $A$, $B$, and $C$ as long as the point $(a,b,f(a,b))$ is on the plane}
    \choice{Any change to the height $\Delta z$ along the surface should equal $dz$ exactly}
    \choice[correct]{The tangent plane approximates the surface near the point of tangency}
    \choice{The tangent plane always has the same slope}
    \choice[correct]{The tangent plane ``tips'' differently at different points}

\end{selectAll}


\begin{feedback}
Remember that the whole idea is \emph{local} linearity, so small changes should be approximately equal, not exactly equal. Also, since we can move away from $(a,b)$ in both $x$ and $y$ directions, both $A$ and $B$ must relate to rates of change in those directions.
\end{feedback}
\end{problem}

\section*{The Total Differential and Tangent Planes}

In single-variable calculus, motion along a linear approximation was determined by $dy = \frac{df}{dx} \cdot dx$.

For multivariable functions, we have the \textbf{total differential}, which combines the effects of changes in both $x$ and $y$. Since movement in the domain is entirely characterized by two-dimensions, we only need rates of change in those two directions, which are given by the partial derivatives.

\begin{definition}
The \textbf{total differential} of $f(x,y)$ at the point $(a,b)$ is:

$$dz = \frac{\partial f}{\partial x} \cdot dx + \frac{\partial f}{\partial y} \cdot dy$$

This determines the linear approximation of changes in $f$ when moving a tiny amount $dx$ in the $x$-direction and $dy$ in the $y$-direction.

\end{definition}


The idea of the total differential formalizes what was discussed at the end of the last module. If you need a review, examine the following GeoGebra application to visually see that movement in $x$, then movement in $y$, using the partial derivatives as the underlying rates of change, both contribute to the total height change ($dz$).

\begin{expandable}{stuff}{GeoGebra Instructions}
 In this applet, you can move the black point on the left screen to change $(x,y)$. This changes where you're taking the derivative of $f$. You can also move the green dot at the end of the vector to change the direction in which you're taking the derivative. As you do so, note that a different curve is created on the 3D surface on the right.

For any given direction, if you select the ``Zoom to Differential'' button, the right screen will move close to the point on the surface and you can see the directional derivative vector. You can also check the ``Show Directoinal Derivative'' and ``Show Differential Build for dz'' boxes to see how the directional derivative is computed geometrically.
\end{expandable}

\begin{center}
\geogebra{gqa2xchz}{856}{440}
\end{center}

The new information we're adding this module is that the construction of $dz$ actually generates a tangent plane, made of all the vectors possibly constructed from $\frac{\partial f}{\partial x}\cdot dx+\frac{\partial f}{\partial y}\cdot dy$.

\begin{problem}
Find the tangent plane for $f(x,y) = x^2 + 2xy + y^2$ at the point $(1,2)$:

First compute partial derivatives:
$$\frac{\partial f}{\partial x} = \answer{2x + 2y}$$
$$\frac{\partial f}{\partial y} = \answer{2x + 2y}$$

At $(1,2)$:
$$\frac{\partial f}{\partial x}(1,2) = \answer{6}$$
$$\frac{\partial f}{\partial y}(1,2) = \answer{6}$$

The total differential at $(1,2)$ is:
$$dz = \answer{6} \cdot dx + \answer{6} \cdot dy$$

If we move $dx = 0.1$ and $dy = 0.05$, the approximate change in $f$ is:
$$dz \approx 6(0.1) + 6(0.05) = \answer{0.9}$$

\begin{feedback}
The tangent plane provides a linear approximation: $f(x,y) \approx f(a,b) + \frac{\partial f}{\partial x}(a,b)(x-a) + \frac{\partial f}{\partial y}(a,b)(y-b)$.
\end{feedback}
\end{problem}

\begin{remark}
    You'll notice that the equation for the total differential also defines the equation for a tangent plane, but its format is slightly different. The following equation for a tangent plane uses notation a little closer to what you'll remember from past modules.


    The \textbf{tangent plane} to the surface $z=f(x,y)$ at the point $(a,b,f(a,b))$ is given by:
    $$z = f(a,b) + \frac{\partial f}{\partial x}(x-a) + \frac{\partial f}{\partial y}(y-b)$$

    Notice that the small changes $dx$, $dy$, and $dz$ are just the differences $(x-a)$, $(y-b)$, and $(z-f(a,b))$, so the two equations (for differential and for tangent plane) are really saying the same thing.

\end{remark}

\begin{problem}
Let's find a tangent plane equation. For $f(x,y) = x^2 + y^2$ at the point $(1,2)$:

\textbf{Step 1: Find the partial derivatives}
$$\frac{\partial f}{\partial x} = \answer{2x}$$
$$\frac{\partial f}{\partial y} = \answer{2y}$$

\textbf{Step 2: Evaluate at $(1,2)$}
$$f(1,2) = 1^2 + 2^2 = \answer{5}$$
$$\frac{\partial f}{\partial x}(1,2) = \answer{2}$$
$$\frac{\partial f}{\partial y}(1,2) = \answer{4}$$

\textbf{Step 3: Write the tangent plane equation}
$$z = 5 + 2(x-1) + 4(y-2)$$

Simplifying:
$$z = 5 + 2x - 2 + 4y - 8 = 2x + 4y - \answer{5}$$

\begin{feedback}
The coefficients $2$ and $4$ tell us the slope of the plane in the $x$ and $y$ directions respectively!
\end{feedback}
\end{problem}

\begin{problem}
Find the tangent plane to $g(x,y) = xy + y^2$ at $(2,1)$:

First, compute partial derivatives:
$$\frac{\partial g}{\partial x} = \answer{y}$$
$$\frac{\partial g}{\partial y} = \answer{x + 2y}$$

At $(2,1)$:
$$g(2,1) = 2(1) + 1^2 = \answer{3}$$
$$\frac{\partial g}{\partial x}(2,1) = \answer{1}$$
$$\frac{\partial g}{\partial y}(2,1) = \answer{4}$$

The tangent plane equation is:
$$z = 3 + 1(x-2) + 4(y-1)$$

Simplified form:
$$z = x + 4y - \answer{3}$$

\begin{feedback}
Great! You can now find tangent plane equations by using both partial derivatives together.
\end{feedback}
\end{problem}


\end{document}
