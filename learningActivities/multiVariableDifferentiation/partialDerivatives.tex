\documentclass{ximera}
% \input{../../xmpreamble.tex}

\title{Activity: Partial Derivatives}
\author{Zack Reed}

\begin{document}
\begin{abstract}
In this activity we extend our knowledge of derivatives to functions of multiple variables, exploring partial derivatives, tangent planes, directional derivatives, and gradients.
\end{abstract}
\maketitle

\section*{Introduction: From Single-Variable to Multivariable Derivatives}

We've studied multivariable functions $z = f(x, y)$ where we input two values and get one output. Now we ask: how do these functions change?

\subsection*{Review: Single-Variable Derivatives}

\begin{problem}
Recall that for a single-variable function $y = f(x)$, the derivative $f'(a)$ represents:

\begin{selectAll}
    \choice[correct]{The instantaneous rate of change at $x = a$}
    \choice[correct]{The slope of the tangent line at the point $(a, f(a))$}
    \choice{The area under the curve}
    \choice[correct]{How much $y$ changes per unit change in $x$ near $x = a$}
\end{selectAll}

\begin{feedback}
The derivative captures how the function changes locally. We'll extend this concept to multiple variables!
\end{feedback}
\end{problem}

\subsection*{The Challenge with Multiple Variables}

\begin{problem}
For a multivariable function $z = f(x, y)$, both $x$ and $y$ can change. If we want to find the "rate of change," what challenge do we face?

\begin{multipleChoice}
    \choice{The function doesn't change}
    \choice{We can't use calculus on multivariable functions}
    \choice[correct]{The function can change in infinitely many directions}
    \choice{We need more than one derivative}
\end{multipleChoice}

\begin{feedback}
Unlike the single-variable case where we only move left or right along a curve, in multiple variables we can move in any direction on the $xy$-plane! We need a strategy to handle this complexity.
\end{feedback}
\end{problem}

\section*{Task One: Understanding Partial Derivatives}

The key insight: we can study change in one variable at a time by holding the other variables constant.

\begin{definition}
For a function $f(x, y)$, the \textbf{partial derivative with respect to $x$} at the point $(a, b)$ is:
$$f_x(a,b)=\frac{\partial f}{\partial x}\bigg|_{(x,y)=(a,b)}=\lim_{\Delta x\rightarrow 0}\frac{f(a+\Delta x,b)-f(a,b)}{\Delta x}$$

Similarly, the \textbf{partial derivative with respect to $y$} is:
$$f_y(a,b)=\frac{\partial f}{\partial y}\bigg|_{(x,y)=(a,b)}=\lim_{\Delta y\rightarrow 0}\frac{f(a,b+\Delta y)-f(a,b)}{\Delta y}$$

Notice we only vary one variable at a time!
\end{definition}

\begin{problem}
Before computing, predict: For $f(x, y) = x^2 + y^2$, which statement is true?

\begin{multipleChoice}
    \choice{$f_x$ and $f_y$ are always equal}
    \choice[correct]{$f_x$ depends on $x$ but not $y$; $f_y$ depends on $y$ but not $x$}
    \choice{$f_x$ depends on both $x$ and $y$}
    \choice{$f_x$ is always zero}
\end{multipleChoice}

\begin{feedback}
When taking $\frac{\partial f}{\partial x}$, we treat $y$ as a constant. So $\frac{\partial}{\partial x}(x^2 + y^2) = 2x + 0 = 2x$.
\end{feedback}
\end{problem}

\begin{problem}
Now let's explore partial derivatives visually with this interactive applet.

\begin{expandable}{stuff}{GeoGebra Instructions}
    Use the buttons on the top left to visualize the tangent lines with slopes $f_x$ or $f_y$ at different points $(x,y)$. You may zoom in around the point on the surface to view the tangent line. You may also hide the surface to isolate the curves found along the $x$ or $y$ directions.
\end{expandable}

\begin{center}
\geogebra{gsdw8ufk}{741}{572}
\end{center}

After exploring, answer: What does $f_x(a, b)$ represent geometrically?

\begin{multipleChoice}
    \choice{The slope of the surface in all directions}
    \choice[correct]{The slope of the curve when we slice the surface parallel to the $x$-axis}
    \choice{The slope of the curve when we slice the surface parallel to the $y$-axis}
    \choice{The height of the surface at $(a, b)$}
\end{multipleChoice}

\begin{feedback}
$f_x(a,b)$ is the slope of the tangent line when we move only in the $x$-direction, holding $y$ constant at $b$. Similarly, $f_y(a,b)$ is the slope when moving only in the $y$-direction!
\end{feedback}
\end{problem}

\subsection*{Computing Partial Derivatives}

The great news: we use the same derivative rules from single-variable calculus!

\begin{problem}
For $f(x, y) = x^2 + 3xy + y^2$, compute the partial derivatives:

$f_x(x, y) = \answer{2x + 3y}$

$f_y(x, y) = \answer{3x + 2y}$

At the point $(1, 2)$:
$f_x(1, 2) = \answer{8}$

$f_y(1, 2) = \answer{7}$

\begin{feedback}
For $f_x$: treat $y$ as a constant, differentiate with respect to $x$.
For $f_y$: treat $x$ as a constant, differentiate with respect to $y$.
\end{feedback}
\end{problem}

\begin{problem}
For $g(x, y) = e^{xy}$, compute:

$g_x(x, y) = \answer{ye^{xy}}$

$g_y(x, y) = \answer{xe^{xy}}$

At $(0, 5)$:
$g_x(0, 5) = \answer{5}$

$g_y(0, 5) = \answer{0}$

\begin{feedback}
Remember: $\frac{\partial}{\partial x}(e^{xy}) = e^{xy} \cdot \frac{\partial}{\partial x}(xy) = e^{xy} \cdot y$ (using chain rule, treating $y$ as constant).
\end{feedback}
\end{problem}

\begin{problem}
For $h(x, y) = \sin(x)\cos(y)$, compute:

$h_x(x, y) = \answer{\cos(x)\cos(y)}$

$h_y(x, y) = \answer{-\sin(x)\sin(y)}$

\begin{feedback}
Product rule still applies! But only differentiate with respect to the variable you're focusing on.
\end{feedback}
\end{problem}

\section*{Task Two: The Multivariable Chain Rule}

When we compose multivariable functions with paths, we need to account for how all variables change together.

\subsection*{Motivation: Related Rates Revisited}

\begin{problem}
Consider the volume of a cylinder: $V = \pi r^2 h$. If both $r$ and $h$ change with time, what affects $\frac{dV}{dt}$?

\begin{selectAll}
    \choice[correct]{How fast the radius is changing: $\frac{dr}{dt}$}
    \choice[correct]{How fast the height is changing: $\frac{dh}{dt}$}
    \choice[correct]{The partial derivative $\frac{\partial V}{\partial r}$}
    \choice[correct]{The partial derivative $\frac{\partial V}{\partial h}$}
    \choice{Only the current values of $r$ and $h$}
\end{selectAll}

\begin{feedback}
Both variables contribute to the total change! We need to account for variation in each direction.
\end{feedback}
\end{problem}

\begin{definition}
The \textbf{multivariable chain rule}: If $z = f(x, y)$ where $x = x(t)$ and $y = y(t)$ are functions of $t$, then:
$$\frac{dz}{dt}=\frac{\partial f}{\partial x}\cdot \frac{dx}{dt}+\frac{\partial f}{\partial y}\cdot \frac{dy}{dt}$$

Or using subscript notation:
$$\frac{df}{dt}=f_x(x,y)\cdot \frac{dx}{dt}+f_y(x,y)\cdot \frac{dy}{dt}$$
\end{definition}

\begin{problem}
For the cylinder volume $V(r, h) = \pi r^2 h$, compute the partial derivatives:

$\frac{\partial V}{\partial r} = \answer{2\pi rh}$

$\frac{\partial V}{\partial h} = \answer{\pi r^2}$

If $r = 3$ cm and $h = 5$ cm, and $\frac{dr}{dt} = 0.2$ cm/s and $\frac{dh}{dt} = 0.5$ cm/s, then:

$\frac{dV}{dt} = \frac{\partial V}{\partial r}\cdot \frac{dr}{dt}+\frac{\partial V}{\partial h}\cdot \frac{dh}{dt}$

$= 2\pi(\answer{3})(\answer{5})(0.2) + \pi(\answer{3})^2(0.5)$

$= \answer[tolerance=0.1]{32.67}$ cm³/s

\begin{feedback}
Each partial derivative tells us how sensitive $V$ is to changes in that variable. We multiply by the rate of change of that variable, then sum the contributions!
\end{feedback}
\end{problem}

\begin{problem}
For $f(x, y) = x^2y$ where $x(t) = t^2$ and $y(t) = t^3$, find $\frac{df}{dt}$ at $t = 1$.

First compute partial derivatives:
$f_x(x, y) = \answer{2xy}$

$f_y(x, y) = \answer{x^2}$

Then compute rates of change:
$\frac{dx}{dt} = \answer{2t}$

$\frac{dy}{dt} = \answer{3t^2}$

At $t = 1$: $x(1) = \answer{1}$, $y(1) = \answer{1}$, $\frac{dx}{dt}\bigg|_{t=1} = \answer{2}$, $\frac{dy}{dt}\bigg|_{t=1} = \answer{3}$

So: $\frac{df}{dt}\bigg|_{t=1} = f_x(1, 1) \cdot 2 + f_y(1, 1) \cdot 3 = \answer{2}(2) + \answer{1}(3) = \answer{7}$

\begin{feedback}
The chain rule lets us track how $f$ changes as we move along a path in the $xy$-plane!
\end{feedback}
\end{problem}

\section*{Task Three: Tangent Planes and Linearization}

Just as the tangent line approximates a single-variable function, the tangent plane approximates a multivariable function.

\begin{problem}
For a single-variable function $y = f(x)$, the tangent line at $x = a$ is:
$$y = f(a) + f'(a)(x - a)$$

For a multivariable function $z = f(x, y)$, predict the equation of the tangent plane at $(a, b)$:

\begin{multipleChoice}
    \choice{$z = f(a, b) + f_x(a, b)(x - a)$}
    \choice{$z = f(a, b) + f_y(a, b)(y - b)$}
    \choice[correct]{$z = f(a, b) + f_x(a, b)(x - a) + f_y(a, b)(y - b)$}
    \choice{$z = f_x(a, b) + f_y(a, b)$}
\end{multipleChoice}

\begin{feedback}
We need both partial derivatives to capture how the function changes in all directions on the plane!
\end{feedback}
\end{problem}

\begin{problem}
For $f(x, y) = x^2 + y^2$, find the equation of the tangent plane at $(1, 2)$.

First compute:
$f(1, 2) = \answer{5}$

$f_x(x, y) = \answer{2x}$, so $f_x(1, 2) = \answer{2}$

$f_y(x, y) = \answer{2y}$, so $f_y(1, 2) = \answer{4}$

Tangent plane: $z = \answer{5} + \answer{2}(x - 1) + \answer{4}(y - 2)$

Simplified: $z = 2x + 4y - \answer{5}$

\begin{feedback}
The tangent plane is the best linear approximation to the surface near the point!
\end{feedback}
\end{problem}

\section*{Task Four: Directional Derivatives and Gradients}

Partial derivatives tell us about change in the $x$ or $y$ directions. What about other directions?

\begin{problem}
Now explore directional derivatives with this powerful interactive applet.

\begin{expandable}{stuff}{GeoGebra Instructions}
    You can view the tangent plane at point $(a,b,f(a,b))$. The vector $\vec u$ determines the direction of the derivative—drag its tip on the right screen. Alter point $(a,b)$ to change locations. Check "Show Directional Derivative" to visualize the differentials $dz$ and $ds$ that determine the slope. Check "Show Differential Build" to see how $ds$ breaks into $dx$ and $dy$.
\end{expandable}

\begin{center}
\geogebra{y3bnsnbv}{740}{612}
\end{center}

After exploring, answer: How is the directional derivative related to partial derivatives?

\begin{multipleChoice}
    \choice{It's completely independent of partial derivatives}
    \choice[correct]{It's a weighted combination: $f_x \cdot \frac{dx}{ds} + f_y \cdot \frac{dy}{ds}$}
    \choice{It equals $f_x + f_y$}
    \choice{It only depends on $f_x$}
\end{multipleChoice}

\begin{feedback}
The directional derivative combines both partial derivatives, weighted by how much we move in each direction!
\end{feedback}
\end{problem}

\begin{definition}
The \textbf{directional derivative} of $f$ at $(a, b)$ in the direction of unit vector $\vec{u} = \langle u_1, u_2 \rangle$ is:
$$\frac{df}{ds} = f_x(a,b) \cdot u_1 + f_y(a,b) \cdot u_2$$

This is the dot product: $\frac{df}{ds} = \langle f_x(a,b), f_y(a,b) \rangle \cdot \vec{u}$
\end{definition}

\begin{problem}
For $f(x, y) = x^2 + 2y^2$, compute the directional derivative at $(1, 1)$ in the direction $\vec{u} = \langle \frac{1}{\sqrt{2}}, \frac{1}{\sqrt{2}} \rangle$.

First find partial derivatives:
$f_x(1, 1) = \answer{2}$

$f_y(1, 1) = \answer{4}$

Then: $\frac{df}{ds} = \answer{2} \cdot \frac{1}{\sqrt{2}} + \answer{4} \cdot \frac{1}{\sqrt{2}} = \frac{\answer{6}}{\sqrt{2}} = \answer[tolerance=0.1]{4.24}$

\begin{feedback}
The directional derivative tells us the instantaneous rate of change as we move in the direction of $\vec{u}$!
\end{feedback}
\end{problem}

\subsection*{The Gradient Vector}

\begin{problem}
Now explore the gradient with the same applet. Check the "Show Gradient" box and uncheck others.

\begin{center}
\geogebra{y3bnsnbv}{740}{612}
\end{center}

As you drag $\vec{u}$ around, when is the directional derivative largest?

\begin{multipleChoice}
    \choice{When $\vec{u}$ is perpendicular to $\nabla f$}
    \choice[correct]{When $\vec{u}$ points in the same direction as $\nabla f$}
    \choice{When $\vec{u}$ points opposite to $\nabla f$}
    \choice{The directional derivative is always the same}
\end{multipleChoice}

\begin{feedback}
The gradient $\nabla f = \langle f_x, f_y \rangle$ points in the direction of steepest increase!
\end{feedback}
\end{problem}

\begin{definition}
The \textbf{gradient} of $f$ at $(a, b)$ is the vector:
$$\nabla f(a, b) = \langle f_x(a, b), f_y(a, b) \rangle$$

Key properties:
\begin{itemize}
    \item $\nabla f$ points in the direction of steepest increase
    \item $||\nabla f||$ gives the maximum rate of increase
    \item $-\nabla f$ points in the direction of steepest decrease
    \item The directional derivative is: $\frac{df}{ds} = \nabla f \cdot \vec{u}$
\end{itemize}
\end{definition}

\begin{problem}
For $f(x, y) = x^2 - y^2$, find the gradient at $(2, 1)$.

$f_x(x, y) = \answer{2x}$, so $f_x(2, 1) = \answer{4}$

$f_y(x, y) = \answer{-2y}$, so $f_y(2, 1) = \answer{-2}$

Therefore: $\nabla f(2, 1) = \langle \answer{4}, \answer{-2} \rangle$

The magnitude is: $||\nabla f(2, 1)|| = \sqrt{(\answer{4})^2 + (\answer{-2})^2} = \answer[tolerance=0.1]{4.47}$

\begin{feedback}
This gradient vector points toward $(4, -2)$ direction—that's where $f$ increases most rapidly from the point $(2, 1)$!
\end{feedback}
\end{problem}

\begin{problem}
In which direction from $(2, 1)$ does $f(x, y) = x^2 - y^2$ decrease most rapidly?

\begin{multipleChoice}
    \choice{In the direction $\langle 4, -2 \rangle$}
    \choice[correct]{In the direction $\langle -4, 2 \rangle$ (opposite the gradient)}
    \choice{In the direction $\langle 2, 4 \rangle$}
    \choice{In the direction $\langle 0, 1 \rangle$}
\end{multipleChoice}

\begin{feedback}
The steepest descent is always opposite the gradient: $-\nabla f$!
\end{feedback}
\end{problem}

\section*{Summary}

\begin{problem}
Select all TRUE statements about partial derivatives and gradients:

\begin{selectAll}
    \choice[correct]{Partial derivatives measure change in one variable at a time}
    \choice[correct]{The tangent plane requires both $f_x$ and $f_y$}
    \choice[correct]{The gradient points in the direction of steepest increase}
    \choice[correct]{The directional derivative is $\nabla f \cdot \vec{u}$}
    \choice{The gradient is always zero}
    \choice[correct]{The chain rule accounts for change in all variables}
\end{selectAll}

\begin{feedback}
You've mastered the fundamentals of multivariable calculus! Next, we'll use these tools to find maxima and minima of multivariable functions.
\end{feedback}
\end{problem}

\section*{Video Resources}

Visit the \href{https://www.youtube.com/playlist?list=PLHXZ9OQGMqxc_CvEy7xBKRQr6I214QJcd}{Calculus III: Multivariable Calculus playlist by Dr. Trefor Bazett}, found on YouTube, for further video resources on the big-picture ideas of multivariable calculus.

\end{document}