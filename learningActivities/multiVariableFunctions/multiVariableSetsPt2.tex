\documentclass{ximera}

\title{Slicing Three-Variable Sets}
\author{Zack Reed}

\begin{document}
\begin{abstract}
In this activity we explore sets in two and three dimensions and develop visual intuitions for multivariable sets.
\end{abstract}
\maketitle


\section*{Working in Three Dimensions}

Now we extend our thinking to three dimensions, where we have three variables: $x$, $y$, and $z$.

\begin{problem}
    A key strategy for understanding 3D relationships is to \emph{fix one variable and see what happens with the other two}.
    
    This creates a \emph{cross-section} or \emph{slice} of the 3D object.
\end{problem}

\begin{center}
\geogebra{xypfhkea}{732}{506}
\end{center}

\begin{problem}
    Using the GeoGebra applet above, experiment with the $z$ slider.
    
    \begin{enumerate}
        \item When $z = 0$, you see a slice of the surface in the $x$-$y$ plane. This slice is a \wordChoice{\choice[correct]{ellipse}\choice{parabola}\choice{line}\choice{circle}}.
        
        \item As you increase $z$ from $-1.25$ to $1.25$ the slices \wordChoice{\choice{get larger}\choice{get smaller}\choice{stay the same size}\choice[correct]{get larger then smaller}\choice{get smaller then larger}}.
        
        \item The point $(1,0,0)$ is \wordChoice{\choice[correct]{within}\choice{on}\choice{outside of}} the surface because $1^2 + 3 \cdot 0^2 +3 \cdot 0^2$ is \wordChoice{\choice[correct]{less than}\choice{equal to}\choice{greater than}} $4$.
        
        \item The point $(1,1,1)$ is \wordChoice{\choice{within}\choice{on}\choice[correct]{outside of}} the surface because $1^2 + 3 \cdot 1^2 +3 \cdot 1^2$ is \wordChoice{\choice{less than}\choice{equal to}\choice[correct]{greater than}} $4$.
        
        \item The point $(1,1,0)$ is \wordChoice{\choice{within}\choice[correct]{on}\choice{outside of}} the surface because $1^2 + 3 \cdot 1^2 +3 \cdot 0^2$ is \wordChoice{\choice{less than}\choice[correct]{equal to}\choice{greater than}} $4$.
        
        \item What 3D shape does this surface represent?
        \begin{multipleChoice}
            \choice{A cylinder}
            \choice{A cone}
            \choice{A sphere}
            \choice{A paraboloid}
            \choice[correct]{An ellipsoid}
        \end{multipleChoice}
    \end{enumerate}
    
    \begin{feedback}
        By looking at slices, we can understand complex 3D shapes by breaking them down into familiar 2D curves!
    \end{feedback}
\end{problem}

\begin{center}
\geogebra{rf4mpv5m}{732}{506}
\end{center}

\begin{problem}
    Using the second GeoGebra applet, explore slices in different directions.
    
    \begin{enumerate}
        \item Select ``Show x-Slice'' and move the slider. When we fix $x$ and look at the $y$-$z$ plane, we see the relationship between \wordChoice{\choice{$x$ and $y$}\choice{$x$ and $z$}\choice[correct]{$y$ and $z$}}.
        
        \item Select ``Show y-Slice'' and move the slider. When we fix $y$ and look at the $x$-$z$ plane, we see the relationship between \wordChoice{\choice{$x$ and $y$}\choice[correct]{$x$ and $z$}\choice{$y$ and $z$}}.
        
        \item By rotating the view to look directly at each slice, you reduce the problem to a familiar \wordChoice{\choice[correct]{2-dimensional}\choice{1-dimensional}\choice{4-dimensional}} perspective.
    \end{enumerate}
\end{problem}


\end{document}