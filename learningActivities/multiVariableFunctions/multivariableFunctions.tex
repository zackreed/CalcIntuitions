\documentclass{ximera}
% \input{../../xmpreamble.tex}

\title{Activity: Introduction to Multivariable Functions}
\author{Zack Reed}

\begin{document}
\begin{abstract}
In this activity we explore functions with multiple variable inputs and single-variable outputs. We'll build intuition through real-world examples and interactive visualizations.
\end{abstract}
\maketitle

\section*{Introduction: From One Variable to Many}

In single-variable calculus, you worked with functions that took a single number as input and produced a single number as output. Now we expand to functions that take \emph{multiple} numbers as inputs!

\begin{problem}
Before diving in, let's reflect on what you already know. In single-variable calculus, which of the following is true about a function $f(x)$?

\begin{selectAll}
    \choice[correct]{$f$ takes one input (the variable $x$)}
    \choice[correct]{$f$ produces one output}
    \choice{$f$ can have multiple independent variables}
    \choice[correct]{The graph of $f$ is a curve in 2D}
\end{selectAll}

\begin{feedback}
Single-variable functions map one input to one output, and their graphs are curves in the $xy$-plane. Now we'll extend this to functions with multiple inputs!
\end{feedback}
\end{problem}

\section*{Task One: Real-World Examples of Multivariable Functions}

\subsection*{Understanding Multiple Inputs}

\begin{definition}
A \textbf{multivariable function} takes multiple independent variables as input and produces a single output. We write this as:
$$z = f(x, y) \quad \text{or} \quad w = f(x, y, z)$$

The inputs are ordered pairs, triples, or $n$-tuples of real numbers, and the output is a real number.
\end{definition}

Let's explore real-world examples to build intuition.

\subsection*{Example 1: Volume of a Cylinder}

\begin{problem}
Consider the volume of a right circular cylinder. What information do you need to calculate the volume?

\begin{selectAll}
    \choice[correct]{The radius $r$ of the base}
    \choice[correct]{The height $h$ of the cylinder}
    \choice{The surface area}
    \choice{Only the radius (height doesn't matter)}
\end{selectAll}

The volume formula is $V = \pi r^2 h$, which we can write as a function: $V(r, h) = \answer{\pi r^2 h}$

This is a function of \wordChoice{\choice{one variable}\choice[correct]{two variables}\choice{three variables}}.

\begin{feedback}
Both radius $r$ and height $h$ are needed to determine the volume. Changing either one changes the volume, so $V$ depends on both variables!
\end{feedback}
\end{problem}

\begin{problem}
Let's practice evaluating this function. If a cylinder has radius $r = 3$ cm and height $h = 5$ cm:

$V(3, 5) = \pi \cdot (\answer{3})^2 \cdot \answer{5} = \answer{45}\pi$ cubic cm

If we double the radius but keep the height the same:

$V(6, 5) = \pi \cdot (\answer{6})^2 \cdot \answer{5} = \answer{180}\pi$ cubic cm

Notice that doubling the radius \wordChoice{\choice{doubles the volume}\choice{triples the volume}\choice[correct]{quadruples the volume}} because radius is squared in the formula!

\begin{feedback}
When you evaluate a multivariable function, you substitute values for all the input variables. Here, $V(3,5)$ means "evaluate $V$ when $r=3$ and $h=5$."
\end{feedback}
\end{problem}

\begin{problem}
Now explore the interactive visualization of the cylinder volume function.

\begin{expandable}{stuff}{GeoGebra Instructions}
    Use the sliders to adjust the radius and height. Watch how the cylinder's volume changes. Notice how the function depends on both variables simultaneously.
\end{expandable}

\begin{center}
\geogebra{vss2d9yc}{733}{520}
\end{center}

After exploring, answer: If you fix the height at $h=4$ and vary only the radius, the volume:
\begin{multipleChoice}
    \choice{Increases linearly with radius}
    \choice[correct]{Increases quadratically with radius}
    \choice{Increases cubically with radius}
    \choice{Remains constant}
\end{multipleChoice}

If you fix the radius at $r=2$ and vary only the height, the volume:
\begin{multipleChoice}
    \choice[correct]{Increases linearly with height}
    \choice{Increases quadratically with height}
    \choice{Increases cubically with height}
    \choice{Remains constant}
\end{multipleChoice}

\begin{feedback}
The volume is proportional to $r^2$ (quadratic in $r$) but proportional to $h$ (linear in $h$). This shows how the function responds differently to changes in different variables!
\end{feedback}
\end{problem}

\subsection*{Example 2: Speed of Sound in Seawater}

\begin{problem}
The speed of sound through seawater is primarily a function of salinity $S$ (salt concentration) and temperature $T$.

Before looking up the formula, predict: As temperature increases, the speed of sound in water:
\begin{multipleChoice}
    \choice[correct]{Increases (sound travels faster in warmer water)}
    \choice{Decreases (sound travels slower in warmer water)}
    \choice{Stays the same}
\end{multipleChoice}

As salinity increases, the speed of sound in water:
\begin{multipleChoice}
    \choice[correct]{Increases (sound travels faster in saltier water)}
    \choice{Decreases (sound travels slower in saltier water)}
    \choice{Stays the same}
\end{multipleChoice}

\begin{feedback}
Both higher temperature and higher salinity increase the speed of sound in seawater! This is important for sonar and underwater acoustics. We can model this with a function $c(S, T)$ where $c$ is the speed of sound.
\end{feedback}
\end{problem}

\subsection*{Example 3: Mortgage Payments}

\begin{problem}
The monthly payment on a home mortgage depends on three variables:
\begin{itemize}
    \item $P$: the principal borrowed (loan amount)
    \item $i$: the annual interest rate
    \item $t$: the term of the loan (in years)
\end{itemize}

This is a function $M(P, i, t)$ of \wordChoice{\choice{one variable}\choice{two variables}\choice[correct]{three variables}}.

Predict: If you increase the principal $P$ while keeping interest rate and term fixed, the monthly payment:
\begin{multipleChoice}
    \choice[correct]{Increases}
    \choice{Decreases}
    \choice{Stays the same}
\end{multipleChoice}

If you increase the term $t$ (taking longer to pay off the loan) while keeping principal and interest fixed, the monthly payment:
\begin{multipleChoice}
    \choice{Increases}
    \choice[correct]{Decreases}
    \choice{Stays the same}
\end{multipleChoice}

\begin{feedback}
Borrowing more money increases your payment, but spreading payments over a longer time decreases the monthly amount (though you pay more total interest!). This shows how multivariable functions model complex real-world relationships.
\end{feedback}
\end{problem}

\section*{Task Two: Notation and Evaluation}

\subsection*{Function Notation}

\begin{problem}
Let's establish notation. For a function $f(x, y) = x^2 + 3xy - y^2$:

Evaluate at the point $(2, 1)$:
$f(2, 1) = (\answer{2})^2 + 3(\answer{2})(\answer{1}) - (\answer{1})^2 = 4 + 6 - 1 = \answer{9}$

Evaluate at the point $(1, 3)$:
$f(1, 3) = (\answer{1})^2 + 3(\answer{1})(\answer{3}) - (\answer{3})^2 = 1 + 9 - 9 = \answer{1}$

Evaluate at the point $(-1, 2)$:
$f(-1, 2) = (\answer{-1})^2 + 3(\answer{-1})(\answer{2}) - (\answer{2})^2 = 1 - 6 - 4 = \answer{-9}$

\begin{feedback}
To evaluate $f(x,y)$ at a point $(a,b)$, substitute $x=a$ and $y=b$ into the formula. The order matters: the first coordinate is for $x$, the second for $y$!
\end{feedback}
\end{problem}

\begin{problem}
Now try with a three-variable function. Let $g(x, y, z) = xy + yz + xz$.

Evaluate at the point $(1, 2, 3)$:
$g(1, 2, 3) = (\answer{1})(\answer{2}) + (\answer{2})(\answer{3}) + (\answer{1})(\answer{3})$
$= 2 + 6 + 3 = \answer{11}$

Evaluate at the point $(0, 5, 10)$:
$g(0, 5, 10) = (\answer{0})(\answer{5}) + (\answer{5})(\answer{10}) + (\answer{0})(\answer{10})$
$= 0 + 50 + 0 = \answer{50}$

\begin{feedback}
For three-variable functions, you need an ordered triple $(x, y, z)$ to specify a point in the domain. Each variable gets its own value!
\end{feedback}
\end{problem}

\subsection*{Domain and Range}

\begin{problem}
Just like single-variable functions, multivariable functions have domains and ranges.

For $f(x, y) = \sqrt{x + y}$, which points are in the domain?

\begin{selectAll}
    \choice[correct]{$(5, 3)$ because $5 + 3 = 8 \geq 0$}
    \choice[correct]{$(0, 0)$ because $0 + 0 = 0 \geq 0$}
    \choice{$(-5, 3)$ because $-5 + 3 = -2 < 0$}
    \choice[correct]{$(-2, 5)$ because $-2 + 5 = 3 \geq 0$}
    \choice{$(-10, -1)$ because $-10 + (-1) = -11 < 0$}
\end{selectAll}

The domain consists of all points $(x, y)$ where:
\begin{multipleChoice}
    \choice{$x \geq 0$ and $y \geq 0$}
    \choice[correct]{$x + y \geq 0$}
    \choice{$x \geq 0$ or $y \geq 0$}
    \choice{All real numbers}
\end{multipleChoice}

\begin{feedback}
We need $x + y \geq 0$ to take the square root. This means the domain is a half-plane in the $xy$-plane, not just the first quadrant!
\end{feedback}
\end{problem}

\section*{Task Three: Visualizing Multivariable Functions}

\subsection*{The Challenge of Visualization}

\begin{problem}
In single-variable calculus, we graphed $y = f(x)$ as a curve in 2D (the $xy$-plane). For two-variable functions $z = f(x,y)$, where do we graph them?

\begin{multipleChoice}
    \choice{In the $xy$-plane (2D)}
    \choice[correct]{In 3D space (need $x$, $y$, and $z$ axes)}
    \choice{In 4D space}
    \choice{We cannot visualize them}
\end{multipleChoice}

The graph of $z = f(x, y)$ is:
\begin{multipleChoice}
    \choice{A curve in 3D}
    \choice[correct]{A surface in 3D}
    \choice{A point in 3D}
    \choice{A volume in 3D}
\end{multipleChoice}

\begin{feedback}
For each point $(x, y)$ in the domain, we get a height $z = f(x, y)$. This creates a surface in 3D space! Think of it like a landscape where height depends on your location.
\end{feedback}
\end{problem}

\subsection*{Common Function Surfaces}

\begin{problem}
Consider the function $f(x, y) = x^2 + y^2$. 

At the point $(0, 0)$: $f(0, 0) = \answer{0}$

At the point $(1, 0)$: $f(1, 0) = \answer{1}$

At the point $(0, 1)$: $f(0, 1) = \answer{1}$

At the point $(1, 1)$: $f(1, 1) = \answer{2}$

At the point $(2, 2)$: $f(2, 2) = \answer{8}$

What shape do you think this surface resembles?
\begin{multipleChoice}
    \choice{A plane}
    \choice{A saddle}
    \choice[correct]{A paraboloid (bowl shape) opening upward}
    \choice{A sphere}
\end{multipleChoice}

\begin{feedback}
Since $f(x,y) = x^2 + y^2$ is always non-negative and increases as you move away from the origin, it forms a bowl shape (paraboloid) with its minimum at the origin!
\end{feedback}
\end{problem}

\begin{problem}
Now consider $g(x, y) = x^2 - y^2$.

At the point $(0, 0)$: $g(0, 0) = \answer{0}$

At the point $(1, 0)$: $g(1, 0) = \answer{1}$

At the point $(0, 1)$: $g(0, 1) = \answer{-1}$

At the point $(1, 1)$: $g(1, 1) = \answer{0}$

Notice something interesting: $g$ is positive along the $x$-axis but negative along the $y$-axis. This creates:
\begin{multipleChoice}
    \choice{A bowl shape}
    \choice[correct]{A saddle shape (curves up in one direction, down in another)}
    \choice{A plane}
    \choice{A cone}
\end{multipleChoice}

\begin{feedback}
Saddle points are fascinating! The function curves upward in the $x$-direction but downward in the $y$-direction, creating a saddle shape. This will be important when we study critical points!
\end{feedback}
\end{problem}

\section*{Task Four: Extending to Higher Dimensions}

\subsection*{Functions of Three or More Variables}

\begin{problem}
Can we have functions with even more variables?

\begin{selectAll}
    \choice[correct]{Yes, we can have $f(x, y, z)$ (three variables)}
    \choice[correct]{Yes, we can have $f(x_1, x_2, \ldots, x_n)$ ($n$ variables)}
    \choice[correct]{Yes, though visualization becomes harder}
    \choice{No, three variables is the maximum}
\end{selectAll}

For a function $f(x, y, z, w)$ of four variables, the domain consists of:
\begin{multipleChoice}
    \choice{Points in 3D space}
    \choice[correct]{Ordered 4-tuples $(x, y, z, w)$ of real numbers}
    \choice{Surfaces in 3D}
    \choice{Volumes in 3D}
\end{multipleChoice}

\begin{feedback}
There's no limit to the number of variables! In machine learning, functions often have thousands or millions of variables. We can't visualize them geometrically, but the mathematical concepts remain the same.
\end{feedback}
\end{problem}

\begin{problem}
Let $h(x, y, z) = x + 2y + 3z$. Evaluate:

$h(1, 2, 3) = \answer{1} + 2(\answer{2}) + 3(\answer{3}) = 1 + 4 + 9 = \answer{14}$

$h(0, 0, 5) = \answer{0} + 2(\answer{0}) + 3(\answer{5}) = \answer{15}$

$h(-1, 3, 2) = \answer{-1} + 2(\answer{3}) + 3(\answer{2}) = -1 + 6 + 6 = \answer{11}$

\begin{feedback}
The process is the same regardless of how many variables you have: substitute each variable with its corresponding value and calculate!
\end{feedback}
\end{problem}

\section*{Summary and Looking Ahead}

\begin{problem}
Let's review the key concepts. Select all TRUE statements about multivariable functions:

\begin{selectAll}
    \choice[correct]{Multivariable functions take multiple inputs}
    \choice[correct]{The domain consists of ordered pairs, triples, or $n$-tuples}
    \choice[correct]{The output (range) is still a single real number}
    \choice[correct]{The graph of $z = f(x,y)$ is a surface in 3D}
    \choice{We cannot do calculus on multivariable functions}
    \choice[correct]{Real-world quantities often depend on multiple variables}
    \choice{Multivariable functions are too complex to be useful}
    \choice[correct]{We evaluate them by substituting values for all variables}
\end{selectAll}

\begin{feedback}
Excellent! Multivariable functions extend single-variable calculus to more realistic scenarios where outcomes depend on multiple factors. Next, we'll learn how to take derivatives of these functions (partial derivatives) and integrate them!
\end{feedback}
\end{problem}

\begin{problem}
Final reflection: Why do we need multivariable functions?

\begin{multipleChoice}
    \choice{They're harder, so they're more impressive}
    \choice{Single-variable functions are boring}
    \choice[correct]{Real-world phenomena usually depend on multiple factors simultaneously}
    \choice{They're not actually useful}
\end{multipleChoice}

\begin{feedback}
Most real-world phenomena depend on multiple variables! Temperature depends on location (3 variables), profit depends on production costs and sales prices (2+ variables), and flight paths depend on position and velocity (6 variables). Multivariable calculus is essential for engineering, physics, economics, and data science.

We will extend differential calculus to functions of several variables, exploring how derivatives work when functions have multiple inputs. Their derivatives are more varied and interesting because of the different ways variables can interact!
\end{feedback}
\end{problem}

\end{document}
