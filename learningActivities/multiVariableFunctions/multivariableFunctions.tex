\documentclass{ximera}
% \input{../../xmpreamble.tex}

\title{Activity: Introduction to Multivariable Functions}
\author{Zack Reed}

\begin{document}
\begin{abstract}
In this activity we explore functions with multiple variable inputs and single-variable outputs. We'll build intuition through real-world examples and interactive visualizations.
\end{abstract}
\maketitle

\section*{Introduction: From Vector Functions to Multivariable Functions}

After analyzing vector functions (where we input one variable and output a vector), we now examine scalar functions that have multiple input variables. These \emph{multivariate} (or \emph{multivariable}) functions are fundamental to conceptualizing the complexities of the world around us. We will extend our experience with single-variable scalar functions to account for more complex relationships between variables.

\subsection*{Review: What is a Function?}

Recall that we previously defined functions as relationships between variables that have an input-output (or dependent-independent) relationship. Let's review this carefully before extending to multiple variables.

\begin{definition}
For a single-variable function $y = f(x)$, at each $x$-value we have a \textbf{single} assigned $y$-value. We say:
\begin{itemize}
    \item $x$ is the \textbf{input} (independent variable)
    \item $y = f(x)$ is the \textbf{output} (dependent variable)
    \item The graph comprises points $(x, y) = (x, f(x))$
\end{itemize}

The variable $y$ \emph{depends on} $x$ because you can alter the $y$-value by altering the $x$-value.
\end{definition}

\begin{problem}
Let's make sure we understand the single-variable case first. For a function $y = f(x)$, which statements are true?

\begin{selectAll}
    \choice[correct]{Each $x$-value produces exactly one $y$-value}
    \choice{Each $y$-value comes from exactly one $x$-value}
    \choice[correct]{Multiple $x$-values can produce the same $y$-value}
    \choice[correct]{$y$ depends on $x$}
    \choice[correct]{Changing $x$ changes $y$}
\end{selectAll}

\begin{feedback}
The key property of a function is that each input produces exactly one output. However, multiple inputs can produce the same output! For example, $f(x) = x^2$ gives $f(-2) = f(2) = 4$.
\end{feedback}
\end{problem}

\begin{problem}
Now let's visualize this concept with an interactive example.

\begin{expandable}{stuff}{GeoGebra Instructions}
    Alter the "x=" slider to view different points on the graph of $f(x) = \sin(x) + \frac{x}{5}$. Notice how the input $x$ gives the horizontal component and the output $f(x)$ gives the vertical component of each point $(x,y)$.
\end{expandable}

\begin{center}
\geogebra{mvpqg764}{739}{456}
\end{center}

After exploring the applet, answer: As you move the slider, what primarily controls the change in $y$?

\begin{multipleChoice}
    \choice{The $y$-value changes randomly}
    \choice[correct]{The $y$-value changes because we slide the $x$-value}
    \choice{The $y$-value stays constant}
    \choice{Both $x$ and $y$ are independent}
\end{multipleChoice}

\begin{feedback}
This is the essence of a function: we change $y$ by first changing $x$. The output depends on the input!
\end{feedback}
\end{problem}

\subsection*{Extending to Multiple Inputs}

Now we create relationships between three or more variables and require that one variable \emph{depends on} the other variables, making a function in the same way we defined single-variable functions.

\begin{problem}
Before seeing the definition, predict: If we have a function of two variables $x$ and $y$, how should we think about the output?

\begin{multipleChoice}
    \choice{The output is a pair of numbers}
    \choice[correct]{The output is a single number that depends on both $x$ and $y$}
    \choice{The output is a vector}
    \choice{The output is a curve}
\end{multipleChoice}

\begin{feedback}
Just like single-variable functions produce a single output from one input, multivariable functions produce a single output from multiple inputs!
\end{feedback}
\end{problem}

\begin{definition}
For a function $z = f(x, y)$ of two variables:
\begin{itemize}
    \item We can freely change both $x$ and $y$ (independent variables)
    \item For each pair $(x, y)$ we get a \textbf{single} $z$-value: $z = f(x, y)$ (dependent variable)
    \item The graph comprises points $(x, y, z) = (x, y, f(x, y))$
    \item Any one $z$-value might be associated with multiple $(x, y)$ pairs
\end{itemize}

The values of $f$ can be visually interpreted as a height (hence $z$) defined at any point on the $xy$-plane.
\end{definition}

\begin{problem}
Now visualize a multivariable function with this interactive applet.

\begin{expandable}{stuff}{GeoGebra Instructions}
    The left screen shows the $xy$-plane, and the right screen shows the surface $z = f(x,y)$. Click and drag the point $(x,y)$ on either screen. You can also click and drag to rotate the 3D view and scroll to zoom. Use "Reset 3D Zoom" to return to the global view.
\end{expandable}

\begin{center}
\geogebra{krfy4jsw}{751}{652}
\end{center}

After exploring, identify what you observed:

\begin{selectAll}
    \choice[correct]{Both $x$ and $y$ can be changed independently}
    \choice[correct]{Changing either $x$ or $y$ changes the $z$-value}
    \choice{Only one variable can change at a time}
    \choice[correct]{The graph is a surface in 3D, not a curve}
    \choice[correct]{Each $(x,y)$ pair produces exactly one $z$-value}
\end{selectAll}

\begin{feedback}
Just as in the single-variable case, we have a dependent-independent relationship. The primary difference is that we now have two separate variables, $x$ and $y$, that can both be altered to change the value of $z = f(x,y)$.

This forms the foundation of the rest of our study of multivariable calculus!
\end{feedback}
\end{problem}

\section*{Task One: Real-World Examples}

Let's explore concrete examples of multivariable functions.

\subsection*{Example 1: Volume of a Cylinder}

\begin{problem}
The volume formula is $V = \pi r^2 h$, a function: $V(r, h) = \answer{\pi r^2 h}$

If a cylinder has radius $r = 3$ cm and height $h = 5$ cm:
$V(3, 5) = \pi \cdot (\answer{3})^2 \cdot \answer{5} = \answer{45}\pi$ cubic cm

If we double the radius: $V(6, 5) = \answer{180}\pi$ cubic cm

Doubling the radius \wordChoice{\choice{doubles the volume}\choice{triples the volume}\choice[correct]{quadruples the volume}} because radius is squared.

\begin{feedback}
To evaluate $V(3,5)$, substitute $r=3$ and $h=5$. Both variables are needed to determine volume.
\end{feedback}
\end{problem}

\begin{problem}
Now explore the interactive visualization of the cylinder volume function.

\begin{expandable}{stuff}{GeoGebra Instructions}
    Use the sliders to adjust the radius and height. Watch how the cylinder's volume changes. Notice how the function depends on both variables simultaneously.
\end{expandable}

\begin{center}
\geogebra{vss2d9yc}{733}{520}
\end{center}

After exploring, answer: If you fix the height at $h=4$ and vary only the radius, the volume:
\begin{multipleChoice}
    \choice{Increases linearly with radius}
    \choice[correct]{Increases quadratically with radius}
    \choice{Increases cubically with radius}
    \choice{Remains constant}
\end{multipleChoice}

If you fix the radius at $r=2$ and vary only the height, the volume:
\begin{multipleChoice}
    \choice[correct]{Increases linearly with height}
    \choice{Increases quadratically with height}
    \choice{Increases cubically with height}
    \choice{Remains constant}
\end{multipleChoice}

\begin{feedback}
The volume is proportional to $r^2$ (quadratic in $r$) but proportional to $h$ (linear in $h$). This shows how the function responds differently to changes in different variables!
\end{feedback}
\end{problem}

\subsection*{Example 2: Speed of Sound in Seawater}

\begin{problem}
The speed of sound $c(S, T)$ depends on salinity $S$ and temperature $T$.

As temperature increases, sound speed:
\begin{multipleChoice}
    \choice[correct]{Increases (faster in warmer water)}
    \choice{Decreases (slower in warmer water)}
    \choice{Stays the same}
\end{multipleChoice}

As salinity increases, sound speed:
\begin{multipleChoice}
    \choice[correct]{Increases (faster in saltier water)}
    \choice{Decreases (slower in saltier water)}
    \choice{Stays the same}
\end{multipleChoice}

\begin{feedback}
Both higher temperature and salinity increase sound speed—important for sonar and underwater acoustics!
\end{feedback}
\end{problem}

\subsection*{Example 3: Mortgage Payments}

\begin{problem}
Monthly payment $M(P, i, t)$ depends on principal $P$, interest rate $i$, and loan term $t$.

If you increase principal (keeping others fixed), the payment:
\begin{multipleChoice}
    \choice[correct]{Increases}
    \choice{Decreases}
\end{multipleChoice}

If you increase the term (longer payoff), the monthly payment:
\begin{multipleChoice}
    \choice{Increases}
    \choice[correct]{Decreases}
\end{multipleChoice}

\begin{feedback}
Borrowing more increases payments, but longer terms decrease monthly amounts (though total interest paid increases).
\end{feedback}
\end{problem}

\section*{Task Two: Notation and Evaluation}

\begin{problem}
For $f(x, y) = x^2 + 3xy - y^2$, evaluate:

At $(2, 1)$: $f(2, 1) = (\answer{2})^2 + 3(\answer{2})(\answer{1}) - (\answer{1})^2 = \answer{9}$

At $(-1, 2)$: $f(-1, 2) = (\answer{-1})^2 + 3(\answer{-1})(\answer{2}) - (\answer{2})^2 = \answer{-9}$

\begin{feedback}
To evaluate $f(x,y)$ at $(a,b)$, substitute $x=a$ and $y=b$. Order matters!
\end{feedback}
\end{problem}

\begin{problem}
For three variables, let $g(x, y, z) = xy + yz + xz$. Evaluate:

At $(1, 2, 3)$: $g(1, 2, 3) = (\answer{1})(\answer{2}) + (\answer{2})(\answer{3}) + (\answer{1})(\answer{3}) = \answer{11}$

At $(0, 5, 10)$: $g(0, 5, 10) = \answer{50}$

\begin{feedback}
For three-variable functions, use ordered triples $(x, y, z)$.
\end{feedback}
\end{problem}

\begin{problem}
For $f(x, y) = \sqrt{x + y}$, which points are in the domain?

\begin{selectAll}
    \choice[correct]{$(5, 3)$ because $5 + 3 = 8 \geq 0$}
    \choice[correct]{$(0, 0)$ because $0 + 0 = 0 \geq 0$}
    \choice{$(-5, 3)$ because $-5 + 3 = -2 < 0$}
    \choice[correct]{$(-2, 5)$ because $-2 + 5 = 3 \geq 0$}
\end{selectAll}

The domain is all points $(x, y)$ where:
\begin{multipleChoice}
    \choice{$x \geq 0$ and $y \geq 0$}
    \choice[correct]{$x + y \geq 0$}
    \choice{$x \geq 0$ or $y \geq 0$}
\end{multipleChoice}

\begin{feedback}
We need $x + y \geq 0$ for the square root. This is a half-plane, not just the first quadrant!
\end{feedback}
\end{problem}

\section*{Task Three: Visualizing Functions}

\begin{problem}
For two-variable functions $z = f(x,y)$, the graph is:
\begin{multipleChoice}
    \choice{A curve in 2D}
    \choice[correct]{A surface in 3D}
    \choice{A volume in 3D}
\end{multipleChoice}

\begin{feedback}
Each point $(x, y)$ gives a height $z = f(x, y)$, creating a surface—like a landscape!
\end{feedback}
\end{problem}

\begin{problem}
Consider $f(x, y) = x^2 + y^2$. Evaluate at key points: 
$(0, 0)$: $f(0, 0) = \answer{0}$;  $(1, 1)$: $f(1, 1) = \answer{2}$;  $(2, 2)$: $f(2, 2) = \answer{8}$

This surface resembles:
\begin{multipleChoice}
    \choice{A plane}
    \choice{A saddle}
    \choice[correct]{A paraboloid (bowl) opening upward}
    \choice{A sphere}
\end{multipleChoice}

\begin{feedback}
Since $f(x,y) = x^2 + y^2 \geq 0$ and increases away from the origin, it forms a bowl (paraboloid).
\end{feedback}
\end{problem}

\begin{problem}
Now consider $g(x, y) = x^2 - y^2$. Evaluate:

$(0, 0)$: $g(0, 0) = \answer{0}$;  $(1, 0)$: $g(1, 0) = \answer{1}$;  $(0, 1)$: $g(0, 1) = \answer{-1}$

Notice: $g$ is positive along the $x$-axis but negative along the $y$-axis. This creates:
\begin{multipleChoice}
    \choice{A bowl}
    \choice[correct]{A saddle (curves up in one direction, down in another)}
    \choice{A plane}
\end{multipleChoice}

\begin{feedback}
Saddle points will be important when studying critical points!
\end{feedback}
\end{problem}

\section*{Task Four: Higher Dimensions}

\begin{problem}
Functions can have any number of variables! For $f(x, y, z, w)$ of four variables, the domain consists of:
\begin{multipleChoice}
    \choice{Points in 3D space}
    \choice[correct]{Ordered 4-tuples $(x, y, z, w)$}
    \choice{Surfaces in 3D}
\end{multipleChoice}

\begin{feedback}
In machine learning, functions often have thousands of variables. We can't visualize them, but the math works the same!
\end{feedback}
\end{problem}

\begin{problem}
For $h(x, y, z) = x + 2y + 3z$, evaluate:

$h(1, 2, 3) = \answer{1} + 2(\answer{2}) + 3(\answer{3}) = \answer{14}$

$h(-1, 3, 2) = \answer{-1} + 2(\answer{3}) + 3(\answer{2}) = \answer{11}$

\begin{feedback}
Same process: substitute and calculate!
\end{feedback}
\end{problem}

\section*{Summary}

\begin{problem}
Select all TRUE statements:

\begin{selectAll}
    \choice[correct]{Multivariable functions take multiple inputs, produce one output}
    \choice[correct]{The graph of $z = f(x,y)$ is a surface in 3D}
    \choice[correct]{Real-world quantities often depend on multiple variables}
    \choice[correct]{We evaluate by substituting values for all variables}
    \choice{Curvature depends on how fast a particle moves}
\end{selectAll}

\begin{feedback}
Next, we'll learn partial derivatives to analyze how multivariable functions change!
\end{feedback}
\end{problem}

\end{document}
