\documentclass{ximera}

\title{Application Walkthrough 1: Mass of a Wire}
\author{Zack Reed}

\begin{document}
\begin{abstract}
This walkthrough gives a first look at how we use derivatives and integrals along curves to model physical quantities. We will focus on computing the mass of a wire by starting from a basic multiplicative model and then moving to piecewise and integral models.
\end{abstract}
\maketitle

\section*{Introduction}

Welcome to your first project walkthrough!
This will give you a template and introduce tools you can reuse for your mini project at the end of the module.

\section*{Context: Mass of a Wire}

We will treat vector curves in space as skinny wires. Our goal is to compute a wire's mass using derivatives and integrals, and to gut-check with numerical approximations.

\subsection*{Basic Model: Mass = Density $\times$ Length}

We always start with a basic multiplicative model. Here:
\[
M = \rho \cdot L.
\]
A wire has \emph{uniform density} if the ratio mass/length is constant for every piece.

Let's look at a simple wire with constant density and animate how the mass-length ratio stays constant across scales.

\begin{expandable}{code}{Show MATLAB Code}
\texttt{syms t}\\
\texttt{r=[\;sin(t),\;cos(t),\;t\;];}\\
\texttt{rho=2}\\
\texttt{animate\_segment\_cases(r,\;rho)}
\end{expandable}

\section*{Checking Constant Mass}

Use the slider in the animation after it finishes. You should see the mass-length ratio remain constant across segment sizes.

\begin{expandable}{code}{Show MATLAB Code}
\texttt{40.1387/20.193}
\end{expandable}

\begin{problem}
Compute the density from the total mass and length: $\dfrac{40.1387}{20.193} = \answer{}$.  
Now, for each displayed segment, compute $\dfrac{\text{mass of segment}}{\text{length of segment}}$ and record a few values here: , , . Do they agree (up to rounding)?
\end{problem}

\section*{Variable Density}

Now consider a wire with density increasing along the curve: $\rho(t)=2+t^2$.

\begin{expandable}{code}{Show MATLAB Code}
\texttt{rho=2+t\^{}2}\\
\texttt{animate\_segment\_cases(r,\;rho)}
\end{expandable}

You should see the segment mass change as you move along the wire, indicating non-uniform density.

We can evaluate $\rho$ at specific times using \texttt{subs}:

\begin{expandable}{code}{Show MATLAB Code}
\texttt{subs(rho,\;t,\;2)}
\end{expandable}

\begin{problem}
Compute $\rho(\pi)=\answer{}$ and $\rho(2\pi)=\answer{}$ using either hand calculation or \texttt{subs}. Compare with the animation's mass/length ratios at those times. Do they match up to small numerical error?
\end{problem}

\section*{Why the Basic Model Breaks}

We would like $M=\rho\cdot L$, but:
\begin{itemize}
\item Curves can be complicated, so getting $L$ directly may be nontrivial.
\item Density may not be uniform, so there is no single $\rho$ for the whole wire.
\end{itemize}
We therefore move to integration to ``add up small bits of mass.''

\section*{Step 1: Visualize the Approximate Pieces}

The helper \texttt{plot\_curve\_pieces}:
\begin{itemize}
\item takes a curve \(r\), a number of pieces \(N\), and a density \(\rho\),
\item shows the wire in \(N\) pieces, color-coded by density at each midpoint (a midpoint approximation).
\end{itemize}

\begin{expandable}{code}{Show MATLAB Code}
\texttt{plot\_curve\_pieces(r,\;20,\;rho)}
\end{expandable}

\section*{Midpoint Mass Plot}

\texttt{plot\_curve\_mass\_pieces} uses the same inputs but also reports the midpoint-sum mass approximation in the title.

\begin{expandable}{code}{Show MATLAB Code}
\texttt{N=20}\\
\texttt{plot\_curve\_mass\_pieces(r,\;N,\;rho)}
\end{expandable}

\begin{problem}
What approximate mass is reported by the 20-piece midpoint plot? 
\end{problem}

\section*{Step 2: Build the Approximation Manually}

Recall $M \approx \sum_i \rho_i\,\Delta L_i$. If we know the segment lengths and midpoint densities, we can multiply and sum.

\begin{expandable}{code}{Show MATLAB Code}
\texttt{[\textasciitilde,\;lengths,\;densities,\;\textasciitilde,\;\textasciitilde]\;=\;break\_into\_pieces(r,\;N,\;rho)}\\
\texttt{masses\;=\;densities\;.\*\;lengths}\\
\texttt{mass\_approx\;=\;sum(masses)}
\end{expandable}

\begin{problem}
Record your \texttt{lengths} and \texttt{densities} (first three entries):  
\texttt{lengths}: , ,  \quad
\texttt{densities}: , , \\
Compute \texttt{mass\_approx} for \(N=20\): .
\end{problem}

Compare with the plot:

\begin{expandable}{code}{Show MATLAB Code}
\texttt{plot\_curve\_mass\_pieces(r,\;N,\;rho)}
\end{expandable}

\begin{problem}
Do the two approximations match (within small numerical differences)? 
\end{problem}

\section*{Refining the Approximation}

As \(N\) increases, midpoint approximations typically improve.

\begin{expandable}{code}{Show MATLAB Code}
\texttt{N=50}\\
\texttt{[\textasciitilde,\;lengths,\;densities,\;\textasciitilde,\;\textasciitilde]\;=\;break\_into\_pieces(r,\;N,\;rho);}\\
\texttt{masses\;=\;densities\;.\*\;lengths;}\\
\texttt{mass\_approx\;=\;sum(masses)}\\
\texttt{plot\_curve\_mass\_pieces(r,\;N,\;rho);}\\
\\
\texttt{N=100}\\
\texttt{[\textasciitilde,\;lengths,\;densities,\;\textasciitilde,\;\textasciitilde]\;=\;break\_into\_pieces(r,\;N,\;rho);}\\
\texttt{masses\;=\;densities\;.\*\;lengths;}\\
\texttt{mass\_approx\;=\;sum(masses)}\\
\texttt{plot\_curve\_mass\_pieces(r,\;N,\;rho);}\\
\\
\texttt{N=150}\\
\texttt{[\textasciitilde,\;lengths,\;densities,\;\textasciitilde,\;\textasciitilde]\;=\;break\_into\_pieces(r,\;N,\;rho);}\\
\texttt{masses\;=\;densities\;.\*\;lengths;}\\
\texttt{mass\_approx\;=\;sum(masses)}\\
\texttt{plot\_curve\_mass\_pieces(r,\;N,\;rho);}
\end{expandable}

\begin{problem}
Record \texttt{mass\_approx} for \(N=50\), \(100\), and \(150\): , , . What trend do you observe?
\end{problem}

\section*{Very Small Scales}

\begin{expandable}{code}{Show MATLAB Code}
\texttt{N=1000}\\
\texttt{[\textasciitilde,\;L1,\;R1,\;\textasciitilde,\;\textasciitilde]\;=\;break\_into\_pieces(r,\;N,\;rho);}\\
\texttt{sum(R1.\*\;L1)}\\
\\
\texttt{N=10000}\\
\texttt{[\textasciitilde,\;L2,\;R2,\;\textasciitilde,\;\textasciitilde]\;=\;break\_into\_pieces(r,\;N,\;rho);}\\
\texttt{sum(R2.\*\;L2)}\\
\\
\texttt{N=100000}\\
\texttt{[\textasciitilde,\;L3,\;R3,\;\textasciitilde,\;\textasciitilde]\;=\;break\_into\_pieces(r,\;N,\;rho);}\\
\texttt{sum(R3.\*\;L3)}
\end{expandable}

\begin{problem}
Record the three large-\(N\) approximations: , , . If the last two agree to machine precision, what does that suggest? 
\end{problem}

\section*{From Sums to an Exact Integral}

We translate the midpoint sum to an integral. The small mass is
\[
dM = \rho\,dL.
\]
Using arc length $dL = \|\vec v(t)\|\,dt$ with $\vec v(t)=\vec r\,'(t)$, we get
\[
M = \int_a^b \rho(t)\,\|\vec v(t)\|\,dt.
\]

\subsection*{Build and Compute the Integral}

\begin{expandable}{code}{Show MATLAB Code}
\texttt{v\;=\;diff(r)}\\
\texttt{v\_mag\;=\;norm(v)}\\
\texttt{dM\;=\;rho\;*\;v\_mag}\\
\texttt{M\;=\;int(dM,\;0,\;2*pi)}\\
\texttt{int\_approx\;=\;double(M)}
\end{expandable}

\begin{problem}
Report the exact integral expression \(M\): \\
Report the numerical value \texttt{double(M)}: \\
How does it compare with your high-\(N\) approximations? 
\end{problem}

\section*{Your Turn: Use Your Own Curve}

Follow the same steps for the curve you plan to use in the mini project.

\begin{expandable}{code}{Show MATLAB Code}
\texttt{\% Define your curve r as [x(t), y(t), z(t)]}\\
\texttt{\% r = [\;...\;,\;...\;,\;...\;];}\\
\\
\texttt{\% Define your density rho(t)}\\
\texttt{\% rho = ...;}\\
\\
\texttt{\% Choose a piece count (e.g., N=50)}\\
\texttt{\% N = 50;}\\
\\
\texttt{\% Visualize pieces}\\
\texttt{\% plot\_curve\_pieces(r,\;N,\;rho)}
\end{expandable}

Now reproduce the midpoint approximation:

\begin{expandable}{code}{Show MATLAB Code}
\texttt{\% [\textasciitilde,\;lengths,\;densities,\;\textasciitilde,\;\textasciitilde]\;=\;break\_into\_pieces(r,\;N,\;rho)}\\
\texttt{\% masses = densities .\* lengths;}\\
\texttt{\% mass\_approx = sum(masses);}
\end{expandable}

And verify by integration:

\begin{expandable}{code}{Show MATLAB Code}
\texttt{\% v = diff(r);}\\
\texttt{\% v\_mag = norm(v);}\\
\texttt{\% dM = rho * v\_mag;}\\
\texttt{\% M = int(dM,\;a,\;b); \quad double(M)}
\end{expandable}

\begin{problem}
Record your final midpoint approximation and integral approximation:  and . Do they agree within a reasonable tolerance? 
\end{problem}

\section*{Conclusion}

You have:
\begin{itemize}
\item used midpoint approximations to estimate mass as a sum of \(\rho \cdot \Delta L\),
\item refined the approximation by increasing \(N\),
\item derived and computed the exact integral \(M=\int \rho(t)\,\|\vec r\,'(t)\|\,dt\),
\item compared numerical and exact approaches for validation.
\end{itemize}

\end{document}
