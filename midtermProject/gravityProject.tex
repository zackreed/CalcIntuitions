\documentclass{ximera}

\title{Mini Project Walkthrough: Gravity Between a Wire and a Point Mass}
\author{Zack Reed}

\begin{document}
\begin{abstract}
In this project we'll explore how to model the gravitational attraction between a wire and a point mass. We'll begin with the idealized point-mass formula, then extend it to piecewise and continuous approximations using calculus.
\end{abstract}
\maketitle

\section*{Introduction}

Welcome to the introduction to the ``gravity between a wire and a point'' mini project!  
As with all integration problems, we'll begin with a simple situation where multiplication suffices, and then build up toward an integral model that accounts for continuous variation.

\section*{Context: Point Masses and Gravity}

Newton's Law of Gravitation states:
\[
F = G \frac{m_1 m_2}{r^2},
\]
where
\begin{itemize}
\item $G$ is the universal gravitational constant,
\item $r$ is the distance between the two point masses.
\end{itemize}

This force points along the line connecting the two masses.

\begin{expandable}{code}{Show MATLAB Code}
\texttt{animate\_two\_masses()}
\end{expandable}

\begin{problem}
Verify the formula by computing the gravitational force between $m_1=\answer{}$ and $m_2=\answer{}$ separated by $r=\answer{}$ meters.  
Round your answer to 3 decimal places.
\end{problem}

\subsection*{Challenges: A Wire Instead of a Point}

If the wire were treated as a single lumped point mass, we could compute the distance from the point mass to the wire's center and apply Newton's law.  
But in reality:
\begin{itemize}
\item Some parts of the wire are closer and exert stronger forces,
\item Other parts are farther and exert weaker forces,
\item Forces do not all point in the same direction.
\end{itemize}

\begin{expandable}{code}{Show MATLAB Code}
\texttt{plot\_wire\_gravity\_pieces()}
\end{expandable}

\begin{problem}
Observe the vector directions in the plot. How do the closer and farther segments differ? 
\end{problem}

\section*{Step 1: Approximate with Pieces}

Suppose the wire has total mass $M$ and length $L$. If we cut it into $N$ pieces, each piece has mass
\[
\Delta M = \frac{M}{N}.
\]

For a point mass $m$ at distance $D_i$ from piece $i$, the gravitational contribution is
\[
\Delta F_i \approx \frac{G m \Delta M_i}{D_i^2}.
\]

Summing gives
\[
F \approx \sum_{i=1}^N \Delta F_i.
\]

\begin{expandable}{code}{Show MATLAB Code}
\texttt{syms t}\\
\texttt{\% enter your wire's formula, e.g. r=[sin(t),cos(t),t]}\\
\texttt{\% enter your density, e.g. rho=1+t}\\
\texttt{\% enter point mass location, e.g. P=[1,1,0]}\\
\texttt{\% enter point mass m, e.g. m=10}\\
\texttt{\% enter N, e.g. N=10}
\end{expandable}

\begin{expandable}{code}{Show MATLAB Code}
\texttt{plot\_wire\_gravity\_pieces(r,N,rho,P,m)}
\end{expandable}

\begin{problem}
Compute the approximate gravitational force reported in the plot with $N=\answer{}$.  
Now increase $N$ to $50$ or $100$. What happens to the estimate? 
\end{problem}

\section*{Step 2: Recreate the Approximation}

\begin{expandable}{code}{Show MATLAB Code}
\texttt{m=10}\\
\texttt{G=6.67e-11;}\\
\texttt{N=10}\\
\texttt{[\~, lengths, densities, \~, \~] = break\_into\_pieces(r,N,rho)}
\end{expandable}

\begin{problem}
Use elementwise multiplication to compute the segment masses:  
\[
\texttt{masses = lengths .* densities}
\]  
Write down the resulting list: 
\end{problem}

\begin{expandable}{code}{Show MATLAB Code}
\texttt{distances = compute\_approximate\_distances(r,N,rho,P)}
\end{expandable}

\begin{problem}
Using Newton's law, compute the list of forces:

What is the total force? 
\end{problem}

\begin{expandable}{code}{Show MATLAB Code}
\texttt{sum(forces)}
\end{expandable}

Compare with:

\begin{expandable}{code}{Show MATLAB Code}
\texttt{plot\_wire\_gravity\_pieces(r,N,rho,P,m)}
\end{expandable}

\begin{problem}
Do your manual results match the plot output? 
\end{problem}

\section*{Step 3: Examining Changes}

Try a new density:
\[
\rho(t) = \tfrac{1}{2}+t^3.
\]

\begin{expandable}{code}{Show MATLAB Code}
\begin{verbatim}
rho_test = .5+t^3
plot_wire_gravity_pieces(r,10,rho_test,P,m)
\end{verbatim}
\end{expandable}

\begin{problem}
Explain why the regions of higher force shifted later along the curve using Newton's law. 
\end{problem}

\section*{Step 4: Refining Approximation}

\begin{expandable}{code}{Show MATLAB Code}
\texttt{Ns=[5 10 20 50 100]}\\
\texttt{plot\_multiple\_gravity\_approximations(r,Ns,rho,P,m)}
\end{expandable}

\begin{problem}
What trend do you observe as $N$ increases? 
\end{problem}

Now replicate with a for loop:

\begin{expandable}{code}{Show MATLAB Code}
\texttt{for N=Ns}\\
\quad \texttt{\% manual calculations}\\
\texttt{end}
\end{expandable}

\begin{problem}
Pick a large $N$ (say $N=200$). What is your approximation? 
\end{problem}

\section*{Step 5: Exact Model Using Calculus}

From the sum
\[
F \approx \sum_i \frac{G m \Delta M_i}{D_i^2},
\]
we move to the exact model:
\[
F = Gm \int \frac{dM}{D^2}.
\]

\begin{problem}
Rewrite this integral in terms of $t$, $\rho(t)$, and $\vec{r}(t)$ for your setup. 
\end{problem}

\section*{Step 6: Symbolic Computation in MATLAB}

\begin{expandable}{code}{Show MATLAB Code}
\texttt{syms t}\\
\texttt{f = sqrt(exp(t+1)-cos(t))}\\
\texttt{a=0; b=1;}\\
\texttt{F\_exact = int(f,a,b)}\\
\texttt{double(F\_exact)}
\end{expandable}

\begin{problem}
Compute the exact gravitational force for your curve and compare with approximations. Do the values match the trend? 
\end{problem}

\section*{Conclusion}

We have explored how to model gravity between a wire and a point mass:
\begin{itemize}
\item Point-mass Newtonian formula,
\item Piecewise approximations treating wire segments as point masses,
\item Refinements as $N$ increases,
\item Exact solution via calculus and symbolic computation.
\end{itemize}

\end{document}
