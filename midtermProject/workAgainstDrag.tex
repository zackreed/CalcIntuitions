\documentclass{ximera}

\title{Mini Project Walkthrough: Work Done Against Drag}
\author{Zack Reed}

\begin{document}
\begin{abstract}
In this project we'll explore how to model the work required to overcome drag forces when an object moves along a curve. We'll start with a simple ``ideal'' situation that can be modeled using multiplication, and then build up to an integration-based approach that adds up contributions across the motion.
\end{abstract}
\maketitle

\section*{Introduction}

Welcome to the introduction to the ``Work Done Against Drag'' mini project!  
As with all integration problems, we will begin with a simple situation where multiplication suffices, and then build up toward an integral model that accounts for continuous variation.

\section*{Context: Drag on a Moving Object}

In physics, drag is a force that resists motion in a fluid (like air or water). Drag is always directed opposite the motion.

\subsection*{Basic Model 1: Drag Force}

In an idealized setting with constant velocity and unchanging fluid properties, the drag force is given by
\[
F_D = \rho \cdot C_D \cdot A \cdot \frac{v^2}{2},
\]
where
\begin{itemize}
\item $\rho$ is the density of the fluid,
\item $C_D$ is a drag coefficient determined by the fluid and the object,
\item $A$ is the reference area of the object,
\item $v$ is the velocity of the object.
\end{itemize}

In 3D contexts, these are vector magnitudes:
\[
\|\vec{F_D}\| = \rho \cdot C_D \cdot A \cdot \frac{\|\vec{v}\|^2}{2}.
\]

\begin{problem}
Suppose $\rho = \answer{}$, $C_D = \answer{}$, $A = \answer{}$, and $v = \answer{}$.  
Compute the magnitude of the drag force.
\end{problem}

\subsection*{Basic Model 2: Work (Energy Used)}

Work is defined as force applied along a displacement:
\[
W = \vec{F}\cdot\vec{D} = \|\vec{F}\|\|\vec{D}\|\cos(\theta).
\]

\begin{problem}
If $\|\vec{F}\| = \answer{}$ N, $\|\vec{D}\| = \answer{}$ m, and $\theta = \pi$, compute the work $W$.  
Give your answer in Joules.
\end{problem}

\begin{expandable}{code}{Show MATLAB Code}
\texttt{animate\_drag\_motion()}
\end{expandable}

The code above produces a simple animation of drag along a curve with constant parameters.

\begin{problem}
Verify that the drag magnitude is approximately $0.92\,N$ and the displacement vector is 
$\vec{D} = [6-2\pi,\, 2\pi,\, 2\pi]$.  
Compute
\[
W = \|\vec{F}\|\|\vec{D}\|\cos(\pi) \approx \answer{} \ \text{J}.
\]
\end{problem}

\subsection*{Challenges to the Basic Model}

Drag is not always constant:
\begin{itemize}
\item Faster motion produces stronger drag,
\item Slower motion produces weaker drag,
\item Direction of drag changes with velocity,
\item Fluid density and $C_D$ can vary with altitude or speed.
\end{itemize}

\begin{expandable}{code}{Show MATLAB Code}
\texttt{syms t}\\
\texttt{r=[sin(t), cos(t), sin(t)]}\\
\texttt{animate\_drag\_motion(r)}
\end{expandable}

\begin{problem}
Run the animation. What changes about the velocity and drag vectors as the particle moves along this new curve?
\end{problem}

\section*{Step 1: Approximate with Pieces}

Suppose an object moves along a curve $r(t)$. Break the interval into $N$ pieces, approximate velocity at each midpoint, and compute displacement vectors $\Delta r_i$. The drag magnitude at each piece is
\[
\|\vec{F_i}\| = \rho_i C_{D_i} A \frac{\|v_i\|^2}{2}.
\]

Work on piece $i$ is approximated by
\[
\Delta W_i \approx \vec{F_i}\cdot\Delta r_i,
\]
and summing gives
\[
W \approx \sum_{i=1}^N \Delta W_i.
\]

\begin{expandable}{code}{Show MATLAB Code}
\texttt{\% enter your curve's formula}\\
\texttt{rho\_air = ...}\\
\texttt{Cd = ...}\\
\texttt{A = ...}\\
\texttt{N = ...}
\end{expandable}

\begin{expandable}{code}{Show MATLAB Code}
\texttt{plot\_curve\_work\_pieces(r,N,rho\_air,Cd,A)}
\end{expandable}

\begin{problem}
Run the plot. What is the approximate total work reported in the title?
\end{problem}

\section*{Step 2: Recreate the Approximation}

Use the helper function to compute segment quantities:

\begin{expandable}{code}{Show MATLAB Code}
\texttt{[speeds, air\_densities, Cd\_vals] = compute\_drag\_quantities(r,N,rho\_air,Cd)}
\end{expandable}

\begin{problem}
Use the drag formula to compute the drag magnitudes at each segment. Provide the $N$-length list:
\end{problem}

Then compute work with displacement and force vectors:

\begin{expandable}{code}{Show MATLAB Code}
\texttt{[displacement\_vecs, force\_vecs] = compute\_work\_quantities(r,N,drag\_magnitudes)}\\
\texttt{dot(displacement\_vecs,force\_vecs)}\\
\texttt{sum(dot(displacement\_vecs,force\_vecs))}
\end{expandable}

\begin{problem}
What is the total approximate work from your manual calculation?
\end{problem}

\section*{Step 3: Examining Changes}

Try new $\rho$, $C_D$, and $A$:

\begin{expandable}{code}{Show MATLAB Code}
\texttt{rho\_new = ...}\\
\texttt{Cd\_new = ...}\\
\texttt{A\_new = ...}\\
\texttt{plot\_curve\_work\_pieces(r,N,rho\_new,Cd\_new,A\_new)}
\end{expandable}

\begin{problem}
How do the new parameters affect the total work?
\end{problem}

\section*{Step 4: Refining Approximation (Increasing $N$)}

\begin{expandable}{code}{Show MATLAB Code}
\texttt{Ns=[5 10 20 50 100]}\\
\texttt{plot\_multiple\_work\_approximations(r,Ns,rho\_air,Cd,A)}
\end{expandable}

\begin{problem}
Compare the approximations for different $N$. What trend do you observe?
\end{problem}

Use a for loop to replicate the manual calculations:

\begin{expandable}{code}{Show MATLAB Code}
\texttt{for N=Ns}\\
\quad \texttt{\% manual approximation calculations here}\\
\texttt{end}
\end{expandable}

Pick a large $N$:

\begin{expandable}{code}{Show MATLAB Code}
\texttt{N\_big=1000;}
\end{expandable}

\begin{problem}
What is the approximation for $N\_big$?
\end{problem}

\section*{Step 5: Exact Model Using Calculus}

Finite sums become integrals:
\[
W = \int \vec{F_D}\cdot d\vec{r}.
\]

\begin{problem}
Write the integral form of the work done against drag for your chosen curve.
\end{problem}

\section*{Step 6: Symbolic Computation in MATLAB}

\begin{expandable}{code}{Show MATLAB Code}
\texttt{syms t}\\
\texttt{f = sqrt(exp(t+1)-cos(t))}\\
\texttt{a=0}\\
\texttt{b=1}\\
\texttt{F\_exact = int(f,a,b)}\\
\texttt{double(F\_exact)}
\end{expandable}

\begin{problem}
Compare your exact integral result with the approximations from earlier steps. Do they match the trend?
\end{problem}

\section*{Conclusion}

We have explored how to model work against drag:
\begin{itemize}
\item Basic drag model with constant parameters,
\item Piecewise approximations for variable drag,
\item Refinement by increasing the number of pieces,
\item Exact model using integrals and symbolic computation.
\end{itemize}

\end{document}
