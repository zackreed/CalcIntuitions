\documentclass{ximera}
\typeout{Start loading xmPreamble.tex}%

% Add here extra macro's that are loaded automatically by all documents of claas 'ximera' or 'xourse' in this repo

%%
%%  Example:
%%
% \newcommand{\R}{\mathbb{R}

\usepackage{pgfplots}
\pgfplotsset{compat=1.17}
\usepgfplotslibrary{colormaps}


\title{Application Walkthrough 3: Gravity Between a Wire and a Point Mass}
\author{Zack Reed}

\begin{document}
\begin{abstract}
In this walkthrough, we extend our ideas of modeling forces along curves to the case of gravitational attraction. We'll start with the simplest point-mass formula, then see how a distributed mass (a wire) requires approximations and integrals.
\end{abstract}
\maketitle

\section*{Introduction}

Welcome to another project walkthrough!  
Here we will explore how to calculate the \emph{gravitational attraction between a wire and a point mass}.  

As before, we'll begin with an idealized case (two point masses), then show how breaking a wire into small pieces naturally leads to a differential model and an integral.

\section*{Context: Point Masses and Gravity}

The gravitational force between two point masses $m_1$ and $m_2$ is given by Newton's Law:
$$
F = G \frac{m_1 m_2}{r^2}
$$
where $G$ is the universal gravitational constant and $r$ is the distance between the two masses.  

This force vector points along the line connecting the two masses.

\section*{Basic Model: Point Mass vs. Wire}

If the wire were itself a single lumped point mass, the model would be easy:
\begin{itemize}
\item Compute the distance from the point mass to the wire's center,
\item Plug into Newton's formula above.
\end{itemize}

But a real wire has length. Each small piece of the wire:
\begin{itemize}
\item has its own mass,
\item pulls on the point mass with its own direction and magnitude.
\end{itemize}

This means we must break the wire into many small pieces, treat each piece as a mini point mass, and then add the vector contributions together.

\section*{Step 1: Approximate with Pieces}

Suppose the wire has total mass $M$ and length $L$. Cutting it into $N$ pieces gives each piece an approximate mass:
$$
\Delta m = \tfrac{M}{N}
$$

For a piece at midpoint location $r_i$, the gravitational force on a point mass $m$ at position $P$ is:
$$
\Delta F_i \approx G \frac{m \,\Delta m}{\|P-r_i\|^2}\,\hat{u}_i
$$
where $\hat{u}_i = \tfrac{P-r_i}{\|P-r_i\|}$ is the unit vector from the piece to the point mass.  

Adding them up:
$$
F \approx \sum_{i=1}^N \Delta F_i
$$

\begin{expandable}{code}{Show MATLAB Code}
\begin{verbatim}
N = 20;
[midpoints,lengths] = break_into_pieces(r,N);
dm = M/N;
forces = zeros(3,N);

for i = 1:N
    r_i = midpoints(:,i);
    vec = P - r_i;
    dist = norm(vec);
    forces(:,i) = G*m*dm/dist^2 * (vec/dist);
end

F_approx = sum(forces,2)
\end{verbatim}
\end{expandable}

\begin{problem}
Why do closer pieces of the wire contribute more to the force than farther ones? 
\end{problem}

\section*{Step 2: Visualizing the Setup}

We can plot the wire, the point mass, and arrows showing approximate gravitational pulls from each piece.  

\begin{expandable}{code}{Show MATLAB Code}
\begin{verbatim}
plot_wire_gravity_pieces(r,P,N,M,m)
\end{verbatim}
\end{expandable}

This makes clear that contributions vary across the wire.

\section*{Step 3: Exact Model Using Calculus}

To model continuously, let the wire be given by a curve $r(t)$.  
A small element of arc length is $ds$, with linear density $\lambda$:
$$
dm = \lambda \, ds
$$

Each element contributes:
$$
dF = G \frac{m \, dm}{\|P-r(t)\|^2}\,\hat{u}(t)
= G m \lambda \,\frac{P-r(t)}{\|P-r(t)\|^3}\,ds
$$

The total force is the vector integral:
$$
F = G m \int_C \frac{\lambda \,(P-r(t))}{\|P-r(t)\|^3}\,ds
$$

\section*{Step 4: Symbolic / Numerical Computation in MATLAB}

\begin{expandable}{code}{Show MATLAB Code}
\begin{verbatim}
syms t
r = [t; 0; 0];           % wire along x-axis
P = [0; 1; 0];           % point mass at (0,1,0)
lambda = M/L;            % uniform density
dr = diff(r,t);
ds = norm(dr);

vec = P - r;
dist = norm(vec);
dF = G*m*lambda*(vec)/(dist^3) * ds;

F_exact = int(dF,t,0,L);
double(F_exact)
\end{verbatim}
\end{expandable}

\section*{Step 5: Compare Approximations}

\begin{expandable}{code}{Show MATLAB Code}
\begin{verbatim}
for N = [5 10 50 200]
    F_approx = approximate_gravity(r,P,M,m,N);
    disp(F_approx)
end
\end{verbatim}
\end{expandable}

As $N$ increases, the approximations converge to the integral result.

\begin{problem}
Try different values of $N$. How quickly do the approximations get close to the exact integral?  
\end{problem}

\section*{Conclusion}

\begin{itemize}
\item For two point masses, Newton's Law is direct.  
\item For a wire, each little piece must be treated like a mini point mass.  
\item Approximations with finite pieces lead naturally to the continuous vector integral.  
\item MATLAB lets us compare and visualize the two approaches.
\end{itemize}

\end{document}
